\section[%
Расчёт численности промышленно-производственного персонала]{%
РАСЧЕТ ЧИСЛЕННОСТИ \\
ПРОМЫШЛЕННО-ПРОИЗВОДСТВЕННОГО \\
ПЕРСОНАЛА
}
\label{sec:number}

\subsection[%
Расчёт численности основных производственных рабочих
]{%
Расчёт численности основных производственных рабочих
}

Явочная численность производственных рабочих по стандарт-плану
составляет 12 человек.
Списочный состав основных производственных рабочих:
\begin{equation*}
  \text{Ч}_{\text{оп.с}} = 
    \dfrac{C_{\text{пр}}K_{см}}{1-K_{\text{сп}}} =
    \dfrac{12 \cdot 2}{1-0{,}1} =
    26{,}67 \rightarrow 27 \: (\text{чел.}).
\end{equation*}

\subsection[%
Расчёт численности вспомогательных рабочих, ИТР и УП
]{%
Расчёт численности вспомогательных рабочих, ИТР и \\
управленческого персонала
}

Численность транспортных рабочих:
\begin{equation*}
  \text{Ч}_{\text{тр}} = 
  \dfrac{m t_{\text{тр}} \sum^H_{j=1} N_j}{60 F^p_{\text{э}} K_{\text{в}}} =
  \dfrac{8 \cdot 0{,}5 \cdot 109}{480 \cdot 1{,}1} =
  0{,}82 \rightarrow 1 \: (\text{чел.}).
\end{equation*}

С учетом того, что работа производится в две смены, 
всего требуется два траспортных рабочих.

Трудоёмкость слесарных работ по механической и электрической частям соответственно:
{\small
\begin{align*}
  T^{\text{сл}}_{\text{рем}} &= 
  \dfrac{
    n_{\text{к}}t_{\text{к}} + n_{\text{с}}t_{\text{с}} +
    n_{\text{т}}t_{\text{т}} + n_{\text{о}}t_{\text{о}}
  }{
    t_{\text{м.ц}}
  } \cdot \sum^K_{i=1} R_{\text{м}_i} \cdot C_{\text{пр}_i} = \\
  &= 9{,}92
    \cdot
    \big(
      16{,}5 \cdot 4 + 18{,}0 \cdot 2 + 7{,}5 \cdot 3 + 
      6{,}5 \cdot 2 + 5{,}0 \cdot 1
    \big) =
    1413{,}6 \: (\text{мин}). 
  \\
  T^{\text{э.сл}}_{\text{рем}} &= 
  \dfrac{
    n_{\text{к}}t_{\text{к}} + n_{\text{с}}t_{\text{с}} +
    n_{\text{т}}t_{\text{т}} + n_{\text{о}}t_{\text{о}}
  }{
    t_{\text{м.ц}}
  } \cdot \sum^K_{i=1} R_{\text{м}_i} \cdot C_{\text{пр}_i} = \\
  &= 3{,}34
    \cdot
    \big(
      12{,}0 \cdot 4 + 25{,}5 \cdot 2 + 8{,}0 \cdot 3 + 
      8{,}0 \cdot 2 + 4{,}5 \cdot 1
    \big) =
    479{,}29 \: (\text{мин}).
\end{align*}
}

\vspace{-3mm}
Трудоёмкость станочных работ:
{\small
\begin{align*}
  T^{\text{ст}}_{\text{рем}} &= 
  \dfrac{
    n_{\text{к}}t_{\text{к}} + n_{\text{с}}t_{\text{с}} +
    n_{\text{т}}t_{\text{т}} + n_{\text{о}}t_{\text{о}}
  }{
    t_{\text{м.ц}}
  } \cdot \sum^K_{i=1} R_{\text{м}_i} \cdot C_{\text{пр}_i} = \\
  &= 4{,}9
    \cdot
    \big(
      28{,}5 \cdot 4 + 43{,}5 \cdot 2 + 15{,}5 \cdot 3 + 
      14{,}5 \cdot 2 + 9{,}5 \cdot 1
    \big) =
    1401{,}4 \: (\text{мин}).
\end{align*}
}

\vspace{-3mm}
Трудоёмкость прочих работ:
{\small
\begin{align*}
  T^{\text{пр}}_{\text{рем}} &= 
  \dfrac{
    n_{\text{к}}t_{\text{к}} + n_{\text{с}}t_{\text{с}} +
    n_{\text{т}}t_{\text{т}} + n_{\text{о}}t_{\text{о}}
  }{
    t_{\text{м.ц}}
  } \cdot \sum^K_{i=1} R_{\text{м}_i} \cdot C_{\text{пр}_i} = \\
  &= 0{,}984
    \cdot
    \big(
      28{,}5 \cdot 4 + 43{,}5 \cdot 2 + 15{,}5 \cdot 3 + 
      14{,}5 \cdot 2 + 9{,}5 \cdot 1
    \big) =
    281{,}42 \: (\text{мин}).
\end{align*}
}

\vspace{-3mm}
Среднегодовая трудоёмкость по межремонтному обслуживанию 
по слесарным работам (механическая часть и электрическая части соответственно):
{\small
\begin{align*}
  T^{\text{сл}}_{\text{обсл}} &= 
  \dfrac{
    F^p_{\text{э}} K_{см}
  }{
    H^{\text{сл}}_{\text{об}}
  } \cdot \sum^K_{i=1} R_{\text{м}_i} \cdot C_{\text{пр}_i} = \\
  &= \dfrac{480 \cdot 2}{500}
    \cdot
    \big(
      16{,}5 \cdot 4 + 18{,}0 \cdot 2 + 7{,}5 \cdot 3 + 
      6{,}5 \cdot 2 + 5{,}0 \cdot 1
    \big) =
    273{,}6 \: (\text{мин}).
  \\
  T^{\text{э.сл}}_{\text{обсл}} &= 
  \dfrac{
    F^p_{\text{э}} K_{см}
  }{
    H^{\text{э.сл}}_{\text{об}}
  } \cdot \sum^K_{i=1} R_{\text{м}_i} \cdot C_{\text{пр}_i} = \\
  &= \dfrac{480 \cdot 2}{650}
    \cdot
    \big(
      12{,}0 \cdot 4 + 25{,}5 \cdot 2 + 8{,}0 \cdot 3 + 
      8{,}0 \cdot 2 + 4{,}5 \cdot 1
    \big) =
    211{,}94 \: (\text{мин}).
\end{align*}
}

\vspace{-3mm}
Среднегодовая трудоёмкость по межремонтному обслуживанию по
станочным работам:
{\small
\begin{align*}
  T^{\text{ст}}_{\text{обсл}} &= 
  \dfrac{
    F^p_{\text{э}} K_{см}
  }{
    H^{\text{ст}}_{\text{об}}
  } \cdot \sum^K_{i=1} R_{\text{м}_i} \cdot C_{\text{пр}_i} = \\
  &= \dfrac{480 \cdot 2}{1650}
    \cdot
    \big(
      28{,}5 \cdot 4 + 43{,}5 \cdot 2 + 15{,}5 \cdot 3 + 
      14{,}5 \cdot 2 + 9{,}5 \cdot 1
    \big) =
    166{,}4 \: (\text{мин}).
\end{align*}
}

\vspace{-3mm}
Среднегодовая трудоёмкость по межремонтному обслуживанию по
прочим работам:
{\small
\begin{align*}
  T^{\text{пр}}_{\text{обсл}} &= 
  \dfrac{
    F^p_{\text{э}} K_{см}
  }{
    H^{\text{пр}}_{\text{об}}
  } \cdot \sum^K_{i=1} R_{\text{м}_i} \cdot C_{\text{пр}_i} = \\
  &= \dfrac{480 \cdot 2}{1000}
    \cdot
    \big(
      28{,}5 \cdot 4 + 43{,}5 \cdot 2 + 15{,}5 \cdot 3 + 
      14{,}5 \cdot 2 + 9{,}5 \cdot 1
    \big) =
    274{,}56 \: (\text{мин}).
\end{align*}
}

Численность слесарей, станочников и прочих рабочих, необходимых
для выполнения ремонтных работ, составляет:
{\small
\begin{align*}
  \text{Ч}^{\text{сл}}_{\text{рем}} &= 
  \dfrac{
    T^{\text{сл}}_{\text{рем}}
  }{
    F^{p}_{\text{э}} K_{\text{в}}
  } =
  \dfrac{1413{,}6}{480 \cdot 1{,}1} = 
  2{,}68 \rightarrow 3 \: (\text{чел.}), 
  \qquad
  \text{Ч}^{\text{э.сл}}_{\text{рем}} &= 
  \dfrac{
    T^{\text{э.сл}}_{\text{рем}}
  }{
    F^{p}_{\text{э}} K_{\text{в}}
  } =
  \dfrac{479{,}29}{480 \cdot 1{,}1} = 
  0{,}91 \rightarrow 1 \: (\text{чел.}), 
  \\
  \text{Ч}^{\text{ст}}_{\text{рем}} &= 
  \dfrac{
    T^{\text{ст}}_{\text{рем}}
  }{
    F^{p}_{\text{э}} K_{\text{в}}
  } =
  \dfrac{1401{,}40}{480 \cdot 1{,}1} = 
  2{,}65 \rightarrow 3 \: (\text{чел.}), 
  \qquad
  \text{Ч}^{\text{пр}}_{\text{рем}} &= 
  \dfrac{
    T^{\text{пр}}_{\text{рем}}
  }{
    F^{p}_{\text{э}} K_{\text{в}}
  } =
  \dfrac{281{,}42}{480 \cdot 1{,}1} = 
  0{,}53 \rightarrow 1 \: (\text{чел.}). 
\end{align*}
}

\vspace{-3mm}
Численность слесарей, станочников и прочих рабочих
по межремонтному обслуживанию:
{\small
\begin{align*}
  \text{Ч}^{\text{сл}}_{\text{обсл}} &= 
  \dfrac{
    T^{\text{сл}}_{\text{обсл}}
  }{
    F^{p}_{\text{э}} K_{\text{в}}
  } =
  \dfrac{273{,}6}{480 \cdot 1{,}1} = 
  0{,}52 \rightarrow 1 \: (\text{чел.}), 
  \qquad
  \text{Ч}^{\text{э.сл}}_{\text{обсл}} &= 
  \dfrac{
    T^{\text{э.сл}}_{\text{обсл}}
  }{
    F^{p}_{\text{э}} K_{\text{в}}
  } =
  \dfrac{211{,}94}{480 \cdot 1{,}1} = 
  0{,}40 \rightarrow 0 \: (\text{чел.}), 
  \\
  \text{Ч}^{\text{ст}}_{\text{обсл}} &= 
  \dfrac{
    T^{\text{ст}}_{\text{обсл}}
  }{
    F^{p}_{\text{э}} K_{\text{в}}
  } =
  \dfrac{166{,}40}{480 \cdot 1{,}1} = 
  0{,}32 \rightarrow 0 \: (\text{чел.}), 
  \qquad
  \text{Ч}^{\text{пр}}_{\text{обсл}} &= 
  \dfrac{
    T^{\text{пр}}_{\text{обсл}}
  }{
    F^{p}_{\text{э}} K_{\text{в}}
  } =
  \dfrac{274{,}56}{480 \cdot 1{,}1} = 
  0{,}52 \rightarrow 1 \: (\text{чел.}). 
\end{align*}
}

\vspace{-3mm}
Общее число ремонтных рабочих составляет:
\begin{equation*}
  \text{Ч}_{\text{рем}} = 
  \text{Ч}^{\text{сл}}_{\text{рем}} + \text{Ч}^{\text{э.сл}}_{\text{рем}} +
  \text{Ч}^{\text{ст}}_{\text{рем}} + \text{Ч}^{\text{пр}}_{\text{рем}} =
  3 + 1 + 3 + 1 = 8 \: (\text{чел.}).
\end{equation*}

Общее число рабочих по межремонтному обслуживанию составляет:
\begin{equation*}
  \text{Ч}_{\text{обсл}} = 
  \text{Ч}^{\text{сл}}_{\text{обсл}} + \text{Ч}^{\text{э.сл}}_{\text{обсл}} +
  \text{Ч}^{\text{ст}}_{\text{обсл}} + \text{Ч}^{\text{пр}}_{\text{обсл}} =
  1 + 0 + 0 + 1 = 2 \: (\text{чел.}).
\end{equation*}

Общее число рабочих по ремонту и обслуживанию составляет с учетом
двухсменного режима работы составляет 20 чел.

Численность контролеров примем равной 2 чел. 
(на 27 чел. производственных рабочих);
ИТР и управленческого персонала --- 3 чел.;
комплектовщиков и кладовщиков --- 2 чел.;
уборщиков --- 2 чел.;
подсобных и прочих вспомогательных рабочих --- 8 чел.

Результаты расчёта состава промышленно-производственного персонала
приведены в таблице~\ref{tbl:number}.

\begin{table} [h!]
  \caption{
    Состав ППП на предприятии
  }\label{tbl:number}
    \begin{tabular}{| m{9cm} | c | c |}
      \hline
        \parbox{9cm}{
          \smallskip
          \centering Категория работающих
          \smallskip
        }
      & \parbox{2.8cm}{
          \smallskip
          \centering Количество человек
          \smallskip
        }
      & \parbox{3.3cm}{
          \smallskip
          \centering \% от общего количества
          \smallskip
        } \\
      \hline

      1. Основные производственные рабочие & 27 & 40{,}91 \\
      \hline

      2. Вспомогательные рабочие & 36 & 54{,}55 \\

      В том числе: & & \\
      -- обслуживающие оборудование, & 20 & 30{,}30 \\
      -- не обслуживающие оборудование, & 16 & 24{,}24 \\
      \hline

      3. ИТР и управленческий персонал & 3 & 4{,}55 \\
      \hline

      \raggedleft \textbf{Итого} & \textbf{66} & \textbf{100{,}0} \\
      \hline
    \end{tabular}
\end{table}