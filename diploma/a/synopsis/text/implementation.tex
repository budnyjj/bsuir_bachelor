\section[Программная реализация приложения]{%
  ПРОГРАММНАЯ РЕАЛИЗАЦИЯ ПРИЛОЖЕНИЯ
}\label{sec:implementation}

\subsection{Выбор программных средств реализации}

% iOS

Операционная система iOS представляет собой систему, которая работает
на устройствах iPhone, iPad и iPod touch. ОС управляет аппаратным обеспечением
и предоставляет инструменты для разработки нативных приложений.
Нативные приложения представляют собой программы, которые могут быть загружены
из магазинов мобильных приложений (App Store для устройств под управлением
операционной системы iOS) и установлены на мобильное устройство.
Важной особенностью нативных приложений является то,
что разработка ведётся под конкретную платформу, требует
от разработчика специализированных знаний для работы в конкретной
среде программирования.

Комплект средств разработки (англ. Software Development Kit) программного
обеспечения для операционной системы iOS содержит инструменты
и интерфейсы, необходимые для разработки, установки, запуска
и отладки нативных приложений, которые появляются на домашнем экране iOS-устройств.

Саму операционную систему можно рассматривать как набор слоёв,
расположенных на различных логических уровнях. Слоистая архитектура операционной
системы iOS представлена на рисунке~\ref{fig:ios_layers}.
\begin{figure}[h!]
  \centering
  \includegraphics[width=130mm]{fig/ios_layers}
  \caption{Слоистая архитектура \\ операционной системы iOS}
  \label{fig:ios_layers}
\end{figure}

Нижние слои содержат фундаментальные сервисы и технологии.
Слои более высокого уровня опираются на нижние, взаимодействуя с ними,
и обеспечивают доступ к более низкоуровневым сервисам и технологиям.

Слой \textit{Cocoa Touch} содержит ключевые библиотеки для создания iOS-приложений.
Эти фреймворки определяют внешний вид приложений и определяют базовую
инфраструктуру приложений, поддержку ключевых технологий,
таких как многозадачность, сенсорный ввод,
push-уведомления и других высокоуровневых системных сервисов.

Слой \textit{Media} включает в себя графические, аудио- и видеотехнологии,
которые используются для реализации мультимедийных возможностей приложении.

Слой \textit{Core Services} содержит основные системные сервисы. Ключевыми среди
них являются фреймворки Core Foundation и Foundation,
определяющие основные типы, которые используются в приложениях.
Этот слой также включает в себя специфические технологии
для поддержки таких функций, как геолокация, iCloud, сетевые
технологии и другие.

Слой \textit{Core OS} состоит из низкоуровневых функций, на которых базируется
большинство других технологий. Даже если разработчики не используют
эти функции напрямую, они зачастую используются другими библиотеками.
В ситуациях, когда необходимо иметь дело с безопасностью или связью
с аппаратным обеспечением устройства, можно воспользоваться фреймворками этого
слоя.

В процессе создания приложения компания Apple рекомендует отдавать предпочтение
использованию фреймворков более высокого уровня. Библиотеки более высокого
уровня предоставляют собой объектно-ориентированные абстракции
для конструкций более низкого уровня.
Использование библиотек низкого уровня допускается,
если они содержат компоненты, недоступные в фреймворках
более высокого уровня~\cite{ios_core_layers}.

В результате iOS выступает в качестве посредника между низкоуровневым
аппаратным обеспечением устройства и приложениями, которые создают разработчики.
Приложения не взаимодействуют напрямую с аппаратным обеспечением,
они общаются с оборудованием через набор четко определенных системных интерфейсов.
Эти интерфейсы позволяют легко создавать приложения,
которые при этом остаются безопасными для конечного пользователя.

% Xcode & AppCode

Интегрированная среда разработки Xcode --- это пакет средств разработки,
которая используется для создания,
тестирования, отладки и конфигурации iOS-приложений. Среда разработки Xcode
включает в себя приложение Xcode, которое является оболочкой для инструментов
создания приложений, инструменты для отладки и iOS-симулятор.
Для написания и отладки кода используется приложение Xcode,
тестирование происходит путем запуска приложения на iOS симуляторе
или на непосредственно подключенным устройстве под управлением операционной системы iOS.
Для измерения производительности используются инструменты,
которые могут быть запущены из приложения Xcode.

Пользовательский интерфейс среды разработки Xcode представлен
на рисунке~\ref{fig:xcode}.
\begin{figure}[h!]
  \centering
  \includegraphics[width=150mm]{fig/xcode}
  \caption{Пользовательский интерфейс \\ среды разработки Xcode}
  \label{fig:xcode}
\end{figure}

Среда разработки Xcode включает в себя ряд возможностей, среди которых:
\begin{itemize}
  \item встроенный редактор Interface Builder для создания пользовательского
    интерфейса приложения;
  \item расширенная подсветка синтаксиса;
  \item использование официальной документации Apple непосредственно в среде разработки;
  \item система сборки проекта, проверка зависимостей;
  \item компиляция кода с использованием LLVM и Clang;
  \item статический анализатор для контроля поведения приложения и выявления
    потенциальных проблем;
  \item комплексная отладка приложения с использованием дебаггера lldb.
\end{itemize}

Стоит отметить, приложение Xcode не является монополистом на рынке
интегрированных сред разработки программных продуктов для платформы iOS и OS X.
Весной 2013 года компания JetBrains представила IDE для создания приложений для
устройств Apple. По словам разработчиков, AppCode призвана облегчить
повседневную работу программистам, разрабатывающим на Objective-C и Swift
приложения для устройств Apple, таких как Mac, iPhone, iPad, Apple Watch и Apple TV.
Для повышения их продуктивности среда разработки AppCode тесно интегрируется с приложением
Xcode и сосредоточена на обеспечении качества кода за счет удобной
навигации по коду, оптимизированному автодополнению, анализу кода
на лету (с мгновенным исправлением обнаруженных проблем)
и автоматизированному рефакторингу.

Пользовательский интерфейс среды разработки AppCode представлен
на рисунке~\ref{fig:appcode}.
\begin{figure}[h!]
  \centering
  \includegraphics[width=150mm]{fig/appcode}
  \caption{Пользовательский интерфейс \\ среды разработки AppCode}
  \label{fig:appcode}
\end{figure}

Несмотря на большое количество достоинств интегрированной среды разработки AppCode,
не стоит воспринимать её как альтернативу приложению Xcode, так как
IDE от JetBrains не гарантирует полноценную поддержку приложений, написанных с
использованием Xcode. Зачастую разработкам iOS приложений, предпочитающим линейку
продуктов от JetBrains, приходится использовать две среды разработки для одного проекта.
Как правило, в таком случае разработчики используют Xcode для создания
пользовательских интерфейсов с использованием инструмента Interface Builder,
работы с мобильной базой данных CoreData, а написание кода приложения, а также
его отладку производят в интегрированной среде разработки AppCode.

Стоит отметить, что интегрированная среда разработки от JetBrains AppCode
является проприетарным программным обеспечением. Стоимость лицензионной версии AppCode
для индивидуального разработчика составляет приблизительно девяноста
долларов США в год, тогда как IDE Xcode от Apple распространяется через магазин
приложений AppStore на бесплатной основе.

Основными языками программирования, на которых ведётся разработка приложений для
устройств Apple являются Objective-C и Swift.

% Objective-C

Objective-C был придуман Брэдом Коксом в начале 1980-x годов
в качестве модификации языка программирования С в сторону Smalltalk.
Целью Кокса было создание языка, поддерживающего концепцию software IC.
Под этой концепцией понимается возможность собирать программы из готовых
компонент (объектов), подобно тому как сложные электронные устройства могут быть
легко собраны из набора готовых интегральных микросхем (IC, integrated curcuits).
Эта модификация состояла в добавлении новых синтаксических
конструкций и специальном препроцессоре для них (который, проходя по коду
преобразовывал их в обычные вызовы функций С),
а также библиотеке времени выполнения (эти вызовы обрабатывающей).
Таким образом, изначально Objective-C воспринимался как надстройка над C.
В каком-то смысле это так и до сих пор: можно написать программу на чистом С,
а после добавить к ней немного конструкций из Objective-C (при необходимости),
или же наоборот, свободно пользоваться С в программах на Objective-C.

В 1988 компания NeXT (а в последствии Apple)
лицензировала Objective-C и написала для него компилятор и
стандартную библиотеку (SDK).
В 1992 к усовершенствованию языка и компилятора подключились разработчики
проекта GNU в рамках проекта OpenStep. С тех пор компилятор GCC поддерживает Objective-C.
После покупки NeXT, Apple взяля их SDK (компилятор, библиотеки, IDE) за основу
для своих дальнейших разработок. IDE для кода назвали Xcode,
а для GUI --– Interface Builder.
Фреймворк Cocoa, используемый в основном для разработки графического интерфейса пользователя,
на сегодняшний день является наиболее значимой средой разработки программ на Objective-C.

Файлы модулей на языке Objective-C имеют расширение \textit{.m}
(если использовалась смесь С++ и Objective-С, то расширение \textit{.mm}).
Заголовочные файлы --– \textit{.h}. Все создаваемые в Objective-С объекты классов
должны размещаться в динамической памяти. Поэтому особое значение приобретает
тип \textit{id}, который является указателем на объект любого класса (\textit{void *}).
Нулевой указатель именуется константой \textit{nil}.
Таким образом, указатель на любой класс можно привести к типу \textit{id}.
Возникает проблема: как узнать к какому классу относится объект,
скрывающийся под \textit{id}? Это делается благодаря инварианту \textit{isa},
который присутствует в любом объекте класса, унаследовавшего специальный
базовый клас \textit{NSObject} (приставка NS обозначает NeXT Step).
Инвариант \textit{isa} относится к зарезервированному типу \textit{Class}.
Объект такого типа позволяет узнавать имена своего и базового класса,
набор инвариантов класса, а также прототипы всех методов,
которые реализовал этот объект и их адреса (посредством локального списка селекторов).
Все зарезервированные слова Objective-C, отличающиеся от зарезервированных
слов языка С, начинаются с символа \@ (например \textit{@protocol, @selector, @interface}).
Обычно имена инвариантов классов с ограниченной
областью видимости (\textit{@private, @protected}) начинаются с
символа подчеркивания~\cite{https://habrahabr.ru/post/107126/, developer.apple.com}.


% Swift

В 2014 году на конференции The Apple Worldwide Developers Conference Apple
представила язык программирования Swift --– новый язык программирования
для разработки iOS и OS X приложений, который сочетает в себе все лучшее от C и Objective-C,
но лишен ограничений, накладываемых в угоду совместимости с C.
В Swift используются паттерны безопасного программирования и добавлены
современные функции, превращающие создание приложения в простой,
более гибкий и увлекательный процесс. Swift, созданый с чистого листа, ---
это возможность заново представить себе, как разрабатываются приложения.

На разработку нового языка программирования у Apple ушло около четырёх лет.
Основой нового языка программирования послужили существующие компилятор,
отладчик и фреймворки. Разработчики языка упростили процесс управления памятью
с помощью механизма автоматического подсчета ссылок --- Automatic Reference Counting (ARC).
Фреймворки также подверглись серьезной модернизации. Objective-C начал
поддерживать блоки, литералы и модули --- все это создало благоприятные условия
для внедрения современных технологий. Именно эта подготовительная работа
послужила фундаментом для нового языка программирования,
который будет применяться для разработки будущих программных продуктов для Apple.

Swift вобрал в себя всё лучшее от современных языков и разработан с учетом
обширного опыта компании Apple. Компилятор для Swift --- синоним производительности,
язык оптимизирован для разработки без оглядки на компромиссы.
Он спроектирован таким образом, чтобы вы смогли легко разработать
и ваше первое приложение \textit{«hello, world!»}, и даже целую операционную систему.
Все это делает Swift важным инструментом для разработчиков и для самой компании Apple.

Swift работает в 2,6 раза быстрее, чем Objective-C. Apple включила поддержку
динамических библиотек, которые потребляют меньше ресурсов системы и поставляются,
обновляются отдельно от исполняемых файлов приложений.
В результате готовая программа занимает меньше памяти на устройстве.

В декабре 2015 года, вместе с выходом второй версией языка, компания Apple
объявила от открытии исходного кода Swift.
По словам Федериги, вице-президент Apple по разработке ПО, язык Swift уже
стал самым популярным среди разработчиков на веб-хостинге GitHub.
И создатели Swift сейчас активней всех остальных общаются со сторонними разработчиками.
Внутри самой компании программисты стараются максимально использовать Swift
в своей работе. Это и команда разработчиков iCloud, и разработчики OS X,
которые переводят некоторые аспекты платформы на новый язык (к примеру,
управление окнами). По мнению многих, этот простой язык более практичен в разработке.
Благодаря открытому исходному коду Swift теперь можно внедрять
в школы в качестве базы для обучения программированию. Об этом же говорил
и Тим Кук, посетив лекцию <<Hour of Code>> в нью-йоркском Apple Store.
Федериги даже уверен, что в ближайшие 20 лет Swift станет самым
используемым языком программирования~\cite{https://www.iphones.ru/iNotes/512527,
https://habrahabr.ru/post/225841/}

Язык Swift является совместимым с предыдущим языком Apple Objective-C,
что позволяет разработчикам использовать библиотеки, написанные на Objective-C,
а также вести разработку текущих проектов на новом языке, не ограничиваясь
использованием языка Swift только в новых проектах.

В результате в рамках разрабатываемого проекта в качестве среды разработки
программного продукта будет использована интегрированная среда разработки
от Apple Xcode, языки программирования --- Objective-C и Swift.


\pagebreak

\subsection{Реализация ...}


\pagebreak

\subsection{Реализация ...}


\pagebreak

\subsection{Реализация ...}


\pagebreak

\subsection{Реализация пользовательского интерфейса}


\pagebreak

\subsection{Руководство пользователя}


\pagebreak

\subsection{Перспективы развития}

\pagebreak