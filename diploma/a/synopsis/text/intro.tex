\section*{ВВЕДЕНИЕ}
\addcontentsline{toc}{section}{Введение}

На сегодняшний день сложно переоценить роль мобильных устройств в жизни человека.
Именно они существенно облегчают нашу жизнь, упрощают работу,
принимают участие в поиске, обработке и предоставлении
пользователю релевантной информации.
Важным фактором в развитии мобильных устройств является
существование магазинов приложений, благодаря которым пользователь имеет
возможность подобрать комплекс програм для мобильного устройства по своему усмотрению.
Благодаря достаточно простому процессу размещения программных модулей в магазинах
приложений, сами разработчики способствуют развитию данной сферы услуг,
удовлетворяя спрос на приложения.

Сложившаяся ситуация на экономическом рынке Республики Беларусь
способствует повышению интереса населения к курсам обмена валют.
В связи с этим был произведен сравнительный анализ существующих мобильных
приложений в App Store, дающих возможность возможность просмотра курсов валют.
В результате, было принято решение о разработке программного
модуля для платформы iOS, выполняющего отображение актуальной финансовой
информации. Основные задачи проектируемого приложения: отображение курсов валют,
возможность поиска отделений, формирование списка избранных отделений,
использование конвертера валют.

В разделе~\ref{sec:spec} содержится описание, анализ предметной области,
производится сравнение существующих аналогов проектируемого программного модуля
и постановка задачи проектирования.
В разделе~\ref{sec:design} рассматривается процесс проектирования программного модуля,
описывается его структура, информационное и эргономическое обеспечение,
системные требования.
В разделе~\ref{sec:implementation} изложены детали реализации программного модуля.
В разделе~\ref{sec:teo} приведен расчёт затрат на разработку проекта,
рассмотрен эффект от его использования.
