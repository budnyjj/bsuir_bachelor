\section*{ВВЕДЕНИЕ}
\addcontentsline{toc}{section}{Введение}

На сегодняшний день сложно переоценить роль мобильных устройств в жизни человека.
Именно эти устройства существенно облегчают нашу жизнь, упрощают работу,
повышают темп жизни, принимают участие в поиске, обработке и предоставлении
пользователю релевантной информации.

Важным фактором в развитии мобильных устройств является
существование магазинов приложений, благодаря которым пользователь имеет
возможность подобрать комплекс програм для мобильного устройства по своему усмотрению.

Благодаря достаточно простому процессу размещения программных модулей в магазинах
приложений, сами разработчики способствуют развитию этих магазинов, удовлетворяя
спрос на приложения. Очевидно, что потребности пользователей,
проживающих на разных континентах, различаются.
Проблему удовлетворения таких потребностей решают приложения для локальных рынков,
ограниченные использованием определенного языка, территории и др.

\textit{В результате было принято решение о разработке...}

% Относительно нестабильная ситуация на экономическом рынке Республики Беларусь
% способствует повышению интереса населения к курсам валют. В связи с этим был
% произведен сравнительный анализ существующих мобильных приложений в App Store,
% среди которых предусматривалась возможность просмотра курсов валют.
% Таким образом, было принято решение о разработке программного
% продукта для платформы iOS, выполняющего отображение актуальной финансовой
% информации. Основные задачи проектируемого приложения: отображение курсов валют,
% возможность поиска отделений, использования конвертера валют.



% Проектирование приложения производилось в Interface Builder, встроенном в
% интегрированную среду разработки Xcode, был использован язык программирования Swift,
% в качестве мобильной базы данных использовано кроссплатформенное решение Realm.
% Программный модуль построен с использованием архитектурного паттерна MVC.
% Источником финансовой информации является финансовый портал \url{myfin.by} и
% официальный сайт национального банка Республики Беларусь \url{nbrb.by}.
% Приложение производит сбор данных с указанных источников,
% их обработку, сохранение, таким образом, чтобы можно было
% воспользоваться программным продуктом в режиме оффлайн.
% В качестве аналитического модуля в приложение был интегрирован Google Analytics.

% \section*{Заключение}
% Разработанный программный продукт предоставляет пользователю информацию о курсах
% валют в Республике Беларусь, возможность использования конвертера валют,
% настройки уведомлений. Приложение доступно в App Store для бесплатного скачивания по
% ссылке \url{https://goo.gl/klgy48}.