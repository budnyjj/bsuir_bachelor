\section*{РЕФЕРАТ}
\thispagestyle{empty}

\begin{color}{red}
  Реферат выполняют по ГОСТ 7.9 -- 95. Слово РЕФЕРАТ записывают
  прописными буквами полужирным шрифтом по центру, страницу не нумеруют,
  но включают в общее количество страниц пояснительной записки.
  В реферате выделяют две составные части: собственно реферативную и
  заголовочную.

  Заголовочная часть отражает название темы дипломного проекта (ди-
  пломной работы), фамилию студента с инициалами и выходные данные.

  Пример:
  СИСТЕМА ПОЗИЦИОННОГО УПРАВЛЕНИЯ ПОВОРОТНОГО СТОЛА :
  дипломный проект / В. А. Сергеев. – Минск : БГУИР, 2012, – п.з. – 79 с., черте-
  жей (плакатов) – 6 л. формата А1.

  В реферативной части кратко излагается содержание дипломного проекта
  (дипломной работы). Основными аспектами в содержании должны быть: пред-
  мет проектирования (исследования); цель работы; данные, относящиеся к мето-
  дам проектирования; результаты и выводы.

  Объем реферата ограничен текстом, который можно разместить на одной
  странице пояснительной записки. Рекомендуемый объем реферата 850–1200
  печатных знаков.
\end{color}

\pagebreak