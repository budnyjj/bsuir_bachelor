\addcontentsline{toc}{section}{Заключение}
\section*{ЗАКЛЮЧЕНИЕ}

\begin{color}{red}
  Заключение пишут на отдельной странице. Слово ЗАКЛЮЧЕНИЕ
  записывают прописными буквами полужирным шрифтом по центру строки.
  В заключении необходимо перечислить основные результаты, характеризующие
  степень достижения цели проекта и подытоживающие его содержание.

  Результаты следует излагать в форме констатации фактов, используя слова:
  <<изучены>>, <<исследованы>>, <<сформулированы>>, <<показано>>,
  <<разработана>>, <<предложена>>, <<подготовлены>>, <<изготовлена>>,
  <<испытана>> и т. п.

  Текст перечислений должен быть кратким, ясным и содержать конкрет-
  ные данные.

  Объем заключения не должен занимать более полутора-двух страниц по-
  яснительной записки.
\end{color}