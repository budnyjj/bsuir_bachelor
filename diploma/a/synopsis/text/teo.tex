\section[Технико-экономическое обоснование]{ТЕХНИКО-ЭКОНОМИЧЕСКОЕ ОБОСНОВАНИЕ}

\label{sec:teo}

% Экономическая характеристика проекта

\subsection{Экономическая характеристика программного модуля}

Разрабатываемый программный модуль предназначен для отображения курса валют,
установленного Национальным банком Республики Беларусь, а также курсов
валют в отделениях коммерческих банков нашей страны.

Работы по проектированию, реализации и сопровождению приложения
производятся одним лицом --- программистом.
В разработке программного модуля предполагается участие менеджера проекта,
основной задачей которого является консультирование программиста
по возникающим вопросам.

Финансовые данные для приложения предоставляются бесплатно двумя источниками:
официальным сайтом НБ РБ nbrb.by и финансовым интернет-порталом myfin.by.
Приложение распространяется с использованием магазина приложений App Store
на бесплатной основе.

Предполагается, что потенциальными пользователями разрабатываемого приложения
являются как физические лица, так и юридические лица. Физическим лицам может
быть интересен курс валют в ближайших отделениях с целью покупки или
продажи какой-либо валюты.
Юридическим лицам может быть интересен как курс, установленный Национальным
банком Республики Беларусь, так и курс валют в коммерческом банке,
в котором обслуживается данная организация.

Поскольку извлечение коммерческой прибыли от реализации или использования приложения
не предполагается, экономическое обоснование разработки приложения
сводится к приблизительному расчёту затрат на разработку и
описанию эффекта от его использования, носящего неэкономический характер.


% Расчет затрат на разработку программного модуля

\subsection{Расчет затрат на разработку программного модуля}

Выполним упрощенный расчет затрат на разработку мобильного приложения
по методике, приведенной в~\cite{diploma_teo}, в соответствии с которой,
основная доля затрат на разработку программного обеспечения приходится
на следующие статьи:
\begin{itemize}
  \item затраты на основную заработную плату разработчиков;
  \item затраты на дополнительную заработную плату разработчиков;
  \item отчисления на социальные нужды;
  \item прочие затраты.
\end{itemize}

Итоговое значение затрат получается путем суммирования значений затрат
по приведенным статьям.

Расчет величины основной заработной платы разработчиков программного
продукта производится по следующей формуле:
\begin{equation}
  \text{З}_{\text{о}} =
  \sum^n_i \text{Т}_{\text{ч}_{i}} \cdot t_{i},
\end{equation}

\noindent где
\( n \)
--- количество исполнителей, занятых разработкой приложения; \par
\noindent \hspace{6.5mm} \( \text{Т}_{\text{ч}_{i}} \)
--- часовая заработная плата \( i \)-го исполнителя (руб.); \par
\noindent \hspace{6.5mm} \( t_i \)
--- трудоемкость работ, выполняемых \( i \)-тым исполнителем (ч). \\

Часовая заработная плата определяется по формуле:
\begin{equation}
  \text{Т}_{\text{ч}_{i}} = \dfrac{\text{Т}_{\text{м}_{i}}}{m},
\end{equation}
\noindent где
\( \text{Т}_{\text{м}_{i}} \) --- месячная заработная плата \( i \)-го
исполнителя (руб.); \par
\noindent \hspace{6.2mm} \( m \) --- число рабочих часов в месяце,
принимаемое равным 168 часам. \\

Учитывая данные об установившейся средней заработной
плате различных IT-специалистов~\cite{dev_by_salaries},
выполним расчет величины основной заработной платы:
\begin{equation}
  \label{eq:teo_salary}
  \begin{aligned}
    \text{Т}_{\text{ч}_{\text{р.п.}}}& = \dfrac{30 \: 000}{168} = 178{,}57 \: (\text{тыс. р.}), \\[1mm]
    \text{Т}_{\text{ч}_{\text{м.п.}}}& = \dfrac{13 \: 000}{168}  = 77{,}38 \: (\text{тыс. р.}), \\[1mm]
    \text{З}_{\text{о}_{\text{р.п.}}}& = 178{,}57 \cdot 16 = 2 \: 857{,}14 \: (\text{тыс. р.}), \\
    \text{З}_{\text{о}_{\text{м.п.}}}& = 77{,}38 \cdot 168 = 13 \: 000{,}00 \: (\text{тыс. р.}), \\
    \text{З}_{\text{о}}& = 2 \: 857{,}14 + 13 \: 000{,}00 = 15 \: 857{,}14 \: (\text{тыс. р.}),
  \end{aligned}
\end{equation}
\noindent где
\( \text{Т}_{\text{ч}_{\text{р.п.}}} \), \( \text{З}_{\text{о}_{\text{р.п.}}} \)
--- величины месячной и основной заработной платы руководителя проекта; \par
\noindent \hspace{6.2mm} \( \text{Т}_{\text{ч}_{\text{м.п.}}} \), \( \text{З}_{\text{о}_{\text{м.п.}}} \) --- величины месячной и основной заработной платы программиста; \par
\noindent \hspace{6.2mm} \( \text{З}_{\text{о}} \) --- величина основной
заработной платы программиста и руководителя проекта.

\pagebreak

Результаты расчётов заработной платы работников приведены в таблице~\ref{tbl:teo_salary}.
\begin{table} [h!]
  \caption{
    Расчет затрат на основную заработную плату команды \\
    \hspace{29.5mm} разработчиков
  }\label{tbl:teo_salary}
  \small{
    \begin{tabular}{| c | m{4.3cm} | c | c | c | c | c |}
      \hline
      \textnumero
      &
        \parbox{4.3cm}{
        \smallskip
        \centering Участник \\ команды
      \smallskip
      }
      &
        \parbox{2cm}{
        \smallskip
        \centering Месячная \\ заработная \\ плата, тыс.~р.
      \smallskip
      }
      &
        \parbox{2cm}{
        \smallskip
        \centering Часовая \\ заработная \\ плата, тыс.~р.
      \smallskip
      }
      &
        \parbox{3cm}{
        \smallskip
        \centering Трудоемкость работ, ч
        \smallskip
        }
      &
        \parbox{2cm}{
        \smallskip
        \centering Основная \\ заработная \\ плата, тыс.~р.
      \smallskip
      } \\

      \hline
      1 & Руководитель проекта & \(30 \: 000{,}00 \) & \(178{,}57\) & \( 16 \) & \( 2857{,}14 \) \\
      \hline

      2 & Программист & \(13 \: 000{,}00 \) & \(77{,}38\) & \( 168 \) & \( 13 \: 000{,}00 \) \\
      \hline

      \multicolumn{5}{|r|}{Премия 50\%} & \( 7 \: 928{,}57 \) \\
      \hline

      \multicolumn{5}{|r|}{Итого затраты на основную заработную плату} & \( 23 \: 785{,}71 \) \\
      \hline
    \end{tabular}
  }
\end{table}

Затраты на дополнительную заработную плату
работников включает выплаты, предусмотренные законодательством о
труде (оплата отпусков, льготных часов, времени выполнения государственных
обязанностей и других выплат, не связанных с основной деятельностью
исполнителей), и определяется по формуле:
\begin{equation}
  \text{З}_{\text{д}} =
  \dfrac{\text{З}_{\text{о}} \cdot \text{Н}_{\text{д}}}{100},
\end{equation}

\noindent где
\( \text{З}_{\text{о}} \)
--- затраты на основную заработную плату с учетом премии (руб.); \par
\noindent \hspace{6.5mm} \( \text{Н}_{\text{д}} \)
--- норматив дополнительной заработной платы. \\

Принимая значения норматива дополнительной заработной платы в размере 15\%,
получим значение дополнительной заработной платы:
\begin{equation}
  \text{З}_{\text{д}} =
  \dfrac{23 \: 785{,}71 \cdot 15}{100} = 3 \: 567{,}86 \: (\text{тыс.~р.}).
\end{equation}

Отчисления в фонд социальной защиты населения и на обязательное страхование
определяются по формуле:
\begin{equation}
  \text{З}_{\text{соц}} =
  \dfrac{(\text{З}_{\text{о}} + \text{З}_{\text{д}}) \cdot \text{Н}_{\text{соц}}}{100},
\end{equation}

\noindent где
\( \text{Н}_{\text{соц}} \) --- норматив отчислений на социальные нужды. \\

Согласно действующему законодательству~\cite{law_social_royalties},
норматив отчислений в фонд социальной защиты населения составляет 34\%,
на обязательное страхование --- 0,6\%. Принимая это во внимание,
выполним расчет значения отчислений на социальные нужды:
\begin{equation}
  \text{З}_{\text{соц}} =
  \dfrac{(23 \: 785{,}71 + 3 \: 567{,}86) \cdot 34{,}6}{100} = 9 \: 464{,}34 \: (\text{тыс.~р.}).
\end{equation}

Расчет прочих затрат осуществляется в процентах от затрат на основную
заработную плату команды разработчиков с учетом премии по формуле:
\begin{equation}
  \text{З}_{\text{пз}} =
  \dfrac{\text{З}_{\text{о}} \cdot \text{Н}_{\text{пз}}}{100},
\end{equation}
\noindent где
\( \text{Н}_{\text{пз}} \)
--- норматив прочих затрат, принимаемый в размере 40\%. \\

В нашем случае данная статья расходов включает в себя затраты
на приобретение научно-технической литературы, рассматривающей вопросы
разработки мобильных приложений для операционной системы iOS.
Определим значение данной статьи расходов:
\begin{equation}
  \text{З}_{\text{пз}} =
  \dfrac{23 \: 785{,}71 \cdot 40}{100} = 9 \: 514{,}29 \: (\text{тыс.~р.}).
\end{equation}

Полная сумма затрат на разработку программного обеспечения представлена
в таблице~\ref{tbl:teo_sum_cost}.

\begin{table} [h!]
  \caption{
    Расчет суммы затрат на разработку программного \\
    \hspace{29.5mm} обеспечения
  }\label{tbl:teo_sum_cost}
  \begin{tabular}{| m{13.5cm} | c |}
    \hline

    \parbox{13.5cm}{
    \smallskip
    \centering Статья затрат
    \smallskip
    }
    &
      \parbox{2cm}{
      \smallskip
      \centering Сумма, тыс.~р.
    \smallskip
    } \\
    \hline

    Основная заработная плата команды
    & \( 23 \: 785{,}71 \)\\
    \hline

    Дополнительная заработная плата команды
    & \( 3 \: 567{,}86 \)\\
    \hline

    Отчисления на социальные нужды
    & \( 9 \: 464{,}34 \)\\
    \hline

    Прочие затраты
    & \( 9 \: 514{,}29 \)\\
    \hline

    Общая сумма затрат на разработку
    & \( 46 \: 332{,}19 \) \\
    \hline
  \end{tabular}
\end{table}

Таким образом, сумма затрат на разработку мобильного приложения
составляет приблизительно сорок шесть миллионов белорусских рублей.

\subsection{Оценка эффекта от использования}

Поскольку получение коммерческой прибыли от реализации или использования
разрабатываемого приложения изначально не предполагается,
имеет смысл вести речь об эффекте исключительно социального характера.

Территория Республики Беларусь окружена пятью странами, каждая из которых
имеет собственную официальную валюту. Исключением является Литва и Латвия ---
страны еврозоны, в которых официальной валютой является евро. Таким
образом, любые экономические отношения между физическими и юридическими лицами,
а также между странами, затрагивают вопрос обмена валют.
Так, своевременная реакция на изменение курса валют может помочь
в принятии правильного решения, то есть использование приложения
косвенно влияет на экономические показатели деятельности пользователя.
