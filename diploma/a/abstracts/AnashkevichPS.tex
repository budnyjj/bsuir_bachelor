\documentclass[a4paper,11pt,twoside]{article}
\usepackage{conf_template}
\setcounter{section}{0}
\setcounter{figure}{0}
\setcounter{table}{0}
\setcounter{equation}{0}
\setcounter{secnumdepth}{1}
\setcounter{secnumdepth}{1}
\begin{document}

\authors {П.~С.~Анашкевич}

\topic{Программный модуль отображения актуальной финансовой информации}

\annotation{В работе рассмотрен процесс проектирования, разработки и последующего
размещения в магазине приложений App Store, программного продукта для мобильной платформы iOS.}

\begin{multicols}{2}

\section*{Введение}

На сегодняшний день сложно переоценить роль мобильных устройств в жизни человека.
Именно эти устройства существенно облегчают нашу жизнь, упрощают работу,
повышают темп жизни, принимают участие в поиске, обработке и предоставлении
пользователю релевантной информации [1].

Стоит отметить, что важным фактором в развитии мобильных устройств является
существование магазинов приложений, благодаря которым пользователь имеет
возможность подобрать комплекс програм для мобильного устройства по своему усмотрению.

Благодаря достаточно простому процессу размещения программных модулей в магазинах
приложений, сами разработчики способствуют развитию этих магазинов, удовлетворяя
спрос на приложения. Очевидно, что потребности пользователей,
проживающих на разных континентах, различаются.
Проблему удовлетворения таких потребностей решают приложения для локальных рынков,
ограниченные использованием определенного языка, территории и др. [2].

Относительно нестабильная ситуация на экономическом рынке Республики Беларусь
способствует повышению интереса населения к курсам валют. В связи с этим был
произведен сравнительный анализ существующих мобильных приложений в App Store,
среди которых предусматривалась возможность просмотра курсов валют.
Таким образом, было принято решение о разработке программного
продукта для платформы iOS, выполняющего отображение актуальной финансовой
информации. Основные задачи проектируемого приложения: отображение курсов валют,
возможность поиска ближайших отделений, использования конвертера валют,
получение уведомлений.

\section{Проектирование и реализация программного модуля}

Проектирование приложения производилось в Interface Builder, встроенном в
интегрированную среду разработки Xcode, был использован язык программирования Swift,
в качестве мобильной базы данных использовано кроссплатформенное решение Realm.
Программный модуль построен с использованием архитектурного паттерна MVC.
Источником финансовой информации является финансовый портал \url{myfin.by} и
официальный сайт национального банка Республики Беларусь \url{nbrb.by}.
Приложение производит сбор данных с указанных источников,
их обработку, сохранение, таким образом, чтобы можно было
воспользоваться программным продуктом в режиме оффлайн.
В качестве аналитического модуля в приложение был интегрирован Google Analytics.

\section*{Заключение}
Разработанный программный продукт предоставляет пользователю информацию о курсах
валют в Республике Беларусь, возможность использования конвертера валют,
настройки уведомлений. Приложение доступно в App Store для бесплатного скачивания по
ссылке \url{https://goo.gl/klgy48}.

\references{
Список литературы
}\ListReferences{%
\item Вроблевски Л. Сначала Мобильные!~-- Изд-во Манн, Иванов и Фербер, 2012.~-- 176 с.
\item Усов В. Swift. Основы разработки приложений под iOS.~-- Издательский дом <<Питер>>, 2016.~-- 304 с.
}

\end{multicols}

\authorFIO{Анашкевич Павел Сергеевич}
\authorAbout{
студент 5 курса факультета информационных технологий и управления
Белорусского государственного университета информатики и радиоэлектроники,
anashkevichp@gmail.com.
}

\authorFIO{Научный руководитель: Павловская Евгения Ришардовна}
\authorAbout{
заведующая лабораториями кафедры информационных технологий автоматизированных систем,
pavlovskaya@bsuir.by.
}

\end{document}