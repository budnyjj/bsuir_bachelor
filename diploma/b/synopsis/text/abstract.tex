\section*{РЕФЕРАТ}
\thispagestyle{empty}

ПРОГРАММНЫЙ МОДУЛЬ УЧЕТА ПЕРСОНАЛЬНОЙ ФИНАНСОВОЙ ИНФОРМАЦИИ :
дипломный проект / Р. И. Будный. --- Минск : БГУИР, 2016, --- п.з. ---
{\color{red} 70 с.}, чертежей (плакатов) --- 6 л. формата A1.

Предметом проектирования является мобильное приложение,
выполняющее учет персональной финансовой информации пользователя.
Исследуется алгоритмы оптического распознавание изображений.
Выполняется проектирование пользовательского интерфейса приложения,
модели данных.

% В данной работе рассматривается класс мобильных приложений, вы-
% полняющих учет персональной финансовой информации. Целью работы
% является проектирование и реализация мобильного приложения, выполня-
% ющего учет личных финансов пользователя. Отличительной особенностью
% данной работы является разработка алгоритма оптического распознавания
% числовых данных для облегчения ввода финансовой информации в мобиль-
% ное приложение.

  % Реферат выполняют по ГОСТ 7.9 -- 95. Слово РЕФЕРАТ записывают
  % прописными буквами полужирным шрифтом по центру, страницу не нумеруют,
  % но включают в общее количество страниц пояснительной записки.
  % В реферате выделяют две составные части: собственно реферативную и
  % заголовочную.

  % Заголовочная часть отражает название темы дипломного проекта (ди-
  % пломной работы), фамилию студента с инициалами и выходные данные.

  % Пример:
  % СИСТЕМА ПОЗИЦИОННОГО УПРАВЛЕНИЯ ПОВОРОТНОГО СТОЛА :
  % дипломный проект / В. А. Сергеев. – Минск : БГУИР, 2012, – п.з. – 79 с., черте-
  % жей (плакатов) – 6 л. формата А1.

  % В реферативной части кратко излагается содержание дипломного проекта
  % (дипломной работы). Основными аспектами в содержании должны быть: пред-
  % мет проектирования (исследования); цель работы; данные, относящиеся к мето-
  % дам проектирования; результаты и выводы.

  % Объем реферата ограничен текстом, который можно разместить на одной
  % странице пояснительной записки. Рекомендуемый объем реферата 850–1200
  % печатных знаков.

\pagebreak
