\section[Заголовок первого раздела]{%
  ЗАГОЛОВОК ПЕРВОГО РАЗДЕЛА \\
  ПРОДОЛЖЕНИЕ ЗАГОЛОВКА РАЗДЕЛА
}

\subsection{%
  Заголовок первого подраздела \\
  второго раздела
}

В основном тексте пояснительной записки анализируют существующие решения,
определяют пути достижения цели проектирования, составляют технические требования,
на основании которых разрабатывают конкретные методики и технические решения задач,
принимают схемотехнические, алгоритмические, программные и
конструктивно-технологические решения.

Общие требования к основному тексту пояснительной записки: четкость
и логическая последовательность изложения материала, убедительность
аргументации, краткость и ясность формулировок, исключающих неоднозначность
толкования, конкретность изложения результатов, доказательств и выводов.

\subsection{Заголовок второго подраздела}

Здесь будет ненумерованное перечисление:
\begin{itemize}
  \item пункт перечисления номер один пункт перечисления номер один
    пункт перечисления номер один пункт перечисления номер один;
  \item пункт перечисления номер два;
  \item пункт перечисления номер три.
\end{itemize}

А вот нумерованное перечисление:
\begin{reflist}
  \item пункт перечисления номер один;
  \item пункт перечисления номер два;
  \item пункт перечисления номер три.
\end{reflist}

В основном тексте пояснительной записки анализируют существующие решения,
определяют пути достижения цели проектирования, составляют технические требования,
на основании которых разрабатывают конкретные методики и технические решения задач,
принимают схемотехнические, алгоритмические, программные и
конструктивно-технологические решения.

Общие требования к основному тексту пояснительной записки: четкость
и логическая последовательность изложения материала, убедительность
аргументации, краткость и ясность формулировок, исключающих неоднозначность
толкования, конкретность изложения результатов, доказательств и выводов.
