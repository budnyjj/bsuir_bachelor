\section*{%
  ПЕРЕЧЕНЬ УСЛОВНЫХ ОБОЗНАЧЕНИЙ, \\
  СИМВОЛОВ  И ТЕРМИНОВ}
\addcontentsline{toc}{section}{Перечень условных обозначений, символов и терминов}

CNN (англ. convolutional neural network) ---
архитектура искусственных нейронных сетей,
нацеленная на эффективное распознавание изображений.

BRIEF (англ. binary robust independent elementary features),
BRISK (англ. binary robust invariant scalable keypoints),
HOG (англ. histogram of oriented gradients) ---
дескрипторы ключевых точек
изображения, которые используются в компьютерном зрении и обработке изображений
с целью распознавания объектов.

KNN (англ. k-nearest neighbors algorithm) --- метрический алгоритм для автоматической
классификации объектов.

MVP (англ. model-view-presenter) ---
шаблон проектирования пользовательского интерфейса, разработанный для облегчения
автоматического модульного тестирования и улучшения разделения ответственности
в презентационной логике.

OCR (англ. optical character recognition) ---
перевод изображений рукописного, машинописного или печатного текста в текстовые
данные, использующихся для представления символов в компьютере.

SVM (англ. support vector machine)
--- набор схожих алгоритмов обучения с учителем,
использующихся для задач классификации и регрессионного анализа.

SIFT (англ. scale-invariant feature transform),
SURF (англ. speeded up robust features),
ORB (англ. Oriented FAST and Rotated BRIEF)
--- алгоритмы компьютерного зрения,
предназначенные для обнаружения и описания ключевых точек объектов на изображении.

MNIST (англ. mixed National Institute of Standards and Technology database) ---
крупная база данных изображений рукописных цифр, использумая для обучения
различных систем обработки изображений.

JNI (англ. Java native interface) --- стандартный механизм для запуска кода под
управлением виртуальной машины Java (JVM), который написан на языках С/С++
или Ассемблера.

Бинаризация изображения --- процесс приведения исходного
изображения к черно-белому виду.

Зона интереса --- часть изображения, содержащая данные для оптического распознавания.

Паттерн проектирования ---
общее типовое решение некоторой проблемы,
многократно повторяемое в процессе проектирования архитектуры программного продукта.

Синглтон --- порождающий шаблон проектирования программного обеспечения,
гарантирующий, что в однопоточном приложении имеет место единственный
экземпляр класса с глобальной точкой доступа.
