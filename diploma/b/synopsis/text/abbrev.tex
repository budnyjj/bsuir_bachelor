\section*{%
  ПЕРЕЧЕНЬ УСЛОВНЫХ ОБОЗНАЧЕНИЙ, \\
  СИМВОЛОВ  И ТЕРМИНОВ}
\addcontentsline{toc}{section}{Перечень условных обозначений, символов и терминов}

\textit{CNN} (англ. Convolutional Neural Network) ---
архитектура искусственных нейронных сетей,
нацеленная на эффективное распознавание изображений.

\textit{BRIEF} (англ. Binary Robust Independent Elementary Features),
\textit{BRISK} (англ. Binary Robust Invariant Scalable Keypoints),
\textit{HOG} (англ. Histogram of Oriented Gradients) ---
дескрипторы ключевых точек
изображения, которые используются в компьютерном зрении и обработке изображений
с целью распознавания объектов.

\textit{Google Play} ---
магазин приложений, игр, книг, музыки и фильмов компании Google и других компаний,
позволяющий владельцам устройств с операционной системой Android устанавливать
и приобретать различные приложения.

\textit{JNI} (англ. Java Native Interface) ---
стандартный механизм для выполнения кода, написанного на языках С/С++ или Ассемблера,
под управлением виртуальной машины Java.

\textit{KNN} (англ. K-Nearest Neighbors) ---
метрический алгоритм для автоматической классификации объектов.

\textit{Material Design} ---
свод рекомендаций по оформлению элементов пользовательского интерфейса
мобильных приложений для платформы Android.

\textit{MNIST} (англ. mixed National Institute of Standards and Technology database) ---
база данных изображений рукописных цифр, использумая для обучения
различных систем обработки изображений.

\textit{MVP} (англ. Model-View-Presenter) ---
шаблон проектирования пользовательского интерфейса, разработанный для облегчения
автоматического модульного тестирования и улучшения разделения ответственности
в логике представления.

\textit{OCR} (англ. Optical Character Recognition) ---
перевод изображений рукописного, машинописного или печатного текста в текстовые
данные, использующихся для представления символов в компьютере.

\textit{ORB} (англ. Oriented FAST and Rotated BRIEF),
\textit{SIFT} (англ. scale-invariant feature transform),
\textit{SURF} (англ. speeded up robust features) ---
алгоритмы компьютерного зрения,
предназначенные для обнаружения и описания ключевых точек объектов на изображении.

\textit{SVM} (англ. Support Vector Machine) ---
набор схожих алгоритмов обучения с учителем,
использующихся для задач классификации и регрессионного анализа.

\textit{UML} (англ. Unified Modeling Language) ---
язык графического описания для объектного моделирования
в области разработки программного обеспечения.

\textit{Бинаризация изображения} ---
процесс приведения исходного
изображения к черно-белому виду.

\textit{Дескриптор изображения} ---
алгоритм кодирования ключевых точек изображения;
множество закодированных ключевых точек изображения.

\textit{Зона интереса} ---
часть изображения, содержащая данные для оптического распознавания.

\textit{Открытое программное обеспечение} ---
программное обеспечение, пользователи которого имеют права (<<свободы>>)
на его неограниченную установку, запуск, свободное использование, изучение,
распространение и изменение (совершенствование),
а также распространение копий и результатов изменения.

\textit{Паттерн проектирования} ---
общее типовое решение некоторой проблемы,
многократно повторяемое в процессе проектирования архитектуры программного продукта.

\textit{Синглтон} ---
порождающий шаблон проектирования программного обеспечения,
гарантирующий, что в однопоточном приложении имеет место единственный
экземпляр класса с глобальной точкой доступа.

\textit{Фреймворк} ---
программное обеспечение, облегчающее разработку и объединение разных компонентов
большого программного проекта.