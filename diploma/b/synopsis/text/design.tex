\section[Проектирование структуры программного модуля]{%
  ПРОЕКТИРОВАНИЕ СТРУКТУРЫ \\ ПРОГРАММНОГО МОДУЛЯ
}\label{sec:design}

\subsection{Общая характеристика задачи}

Варианты использования приложения (диаграмма Usecases).
Ввод-вывод данных, необходимость наличия базы данных.
Особенности мобильных устройств (небольшой но сенсорный экран, ???).
Работа с графическим интерфейсом приложения.

\subsection{Структура разрабатываемого модуля}

Высокоуровневая структура приложения. Рисунок, чертеж.
Разделение на UI, BL, CV, P.
Обоснование MVP, отличия от MVC.

\subsection{Информационное обеспечение}

Входные, выходные данные, модель данных. Рисунок, чертеж.

\subsection{Математическое обеспечение}

Этапы процесса распознавания образов.
Краткое математическое описание алгоримов, используемых на каждом этапе.
Достоинства/недостатки/мотивация выбора.

\subsection{Алгоритмическое обеспечение}

% Проектирование алгоритма распознавания изображений.
% Блок-схема окончательного алгоритма распознавания.

\subsection{Системные требования}

Требования к техническому и системному программному обеспечению.
Android >= 4.0.x.

\subsection{Эргономическое обеспечение}

Проектирование пользовательского интерфейса (Material design).
Схема переходов между экранами. Экраны приложения.