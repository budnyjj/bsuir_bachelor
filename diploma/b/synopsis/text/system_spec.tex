\section[Системное описание предметной области и постановка задачи]{%
  СИСТЕМНОЕ ОПИСАНИЕ ПРЕДМЕТНОЙ \\
  ОБЛАСТИ И ПОСТАНОВКА ЗАДАЧИ
}\label{sec:system_spec}

\subsection{Выбор целевой программной платформы}

Для того, чтобы правильно выбрать целевую платформу для разработки
мобильного приложения, следует предварительно оценить состояние и
перспективы развития как рынка мобильных устройств связи в целом,
так и конкурирующих мобильных платформ в частности.
Под состоянием мобильной платформы может пониматься понимается
доля рынка, занимаемого данной платформой,
а под перспективами развития --- намерения компании-разработчика и
сообщества сторонних разработчиков эту платформу поддерживать.

В целом, современный рынок мобильных устройств можно охарактеризовать
как <<крупный>> и <<быстро развивающийся>>.
По данным ITU~\cite{itu_stat_phone}, представленным на рисунке~\ref{fig:stat_phone},
общее число мобильных устройств в мире возросло
более чем в три раза за последние десять лет
и составило более семи миллиардов.

\begin{figure}[h!]
  \centering
  \includegraphics[width=150mm]{fig/stat_phone.eps}
  \caption{Динамика использования персональных устройств связи}
  \label{fig:stat_phone}
\end{figure}

Таким образом, по состоянию на 2015 год, на каждые 100 человек
населения земного шара приходится около 97 устройств мобильной связи,
при этом число станционарных устройств связи за данный период,
наоборот, сократилось.

По данным компании Ericsson~\cite{ericsson_mobility_report},
в 2015 году было зарегистрировано использование около 3{,}4
миллиарда смартфонов для доступа в Интернет,
что составляет около 47\% от общего числа мобильных устройств.

Популярность различных мобильных платформ сравнивают, как правило,
по объемам продаж устройств, на которых они предустановлены.
По данным агенства Gartner~\cite{gartner_smartphone_stat}, представленным
в таблице~\ref{tbl:gartner_platform_stat}, можно утверждать, что наиболее популярной
мобильной платформой является Android --- мобильная платформа от
компании Google, занимающая на данный момент около 70\% рынка.

\begin{table} [h!]
  \caption{
    Динамика количества проданных смартфонов под
    управлением различных операционных систем
  }\label{tbl:gartner_platform_stat}
    \begin{tabular}{| m{6.6cm} | c | c | c | c |}
      \hline

      \multirow{2}{*}{
      \parbox{6.6cm}{
      \smallskip
      \centering Операционная система
      \smallskip
      }
      }
      & \multicolumn{2}{c|}{
          \parbox{4.5cm}{
            \smallskip
            \centering Четвертый квартал 2014 года
            \smallskip
          }
        }
      & \multicolumn{2}{c|}{
          \parbox{4.5cm}{
            \smallskip
            \centering Четвертый квартал 2015 года
            \smallskip
          }
        } \\
      \cline{2-5}

      & млн. штук & \% & млн. штук & \% \\
      \hline

      Android &  279 & 76 & 325{,}4 & 80{,}7 \\
      \hline

      iOS &  75 & 20{,}4 & 71{,}5 & 17{,}7 \\
      \hline

      Windows & 10{,}5 & 2{,}8 & 4{,}5 & 1{,}1 \\
      \hline

      Blackberry & 1{,}7 & 0{,}4 & 0{,}9 & 0{,}3 \\
      \hline

      Другие & 1{,}3 & 0{,}4 & 0{,}9 & 0{,}2 \\
      \hline

      Всего & 367{,}5 & 100{,}0 & 400{,}6 & 100{,}0 \\
      \hline
    \end{tabular}
\end{table}

Кроме этого, из приведенных данных видно, что Android является единственной
платформой, доля продаж мобильных устройств под управлением которой
за наблюдаемый период не сократилась.
Android позиционируется как бесплатная платформа для широкого круга мобильных
устройств любых производителей, что позволяет снизить конечную стоимость устройства,
что, в свою очередь, обуславливает высокую популярность платформы.
Android имеет открытый исходный код; авторы платформы
охотно принимают патчи от сторонних разработчиков,
поэтому вокруг неё сформировалось активное сообщество.
Приведенные факты позволяют сделать вывод, что платформа Android является наиболее
перспективной для разработки мобильных приложений.

Весьма актуальным является вопрос выбора минимальной версии Android,
которая должна поддерживаться разрабатываемым мобильным приложением,
поскольку различные версии этой платформы не обладают прямой совместимостью.
Действительно, поддержка только новых версий Android сужает объем потенциальной
целевой аудитории, а поддержка максимального числа версий значительно усложняет
процесс разработки и тестирования.

На рисунке~\ref{fig:stat_android} представлены официальные
данные об относительной популярности различных версий
ОС Android~\cite{google_stat_android}.

\begin{figure}[h!]
  \centering
  \includegraphics[width=150mm]{fig/stat_android.eps}
  \caption{Относительная популярность различных \\ версий платформы Android}
  \label{fig:stat_android}
\end{figure}

Исходя из этих данных, можно сделать вывод, что наиболее целесообразной
является разработка мобильного приложения для Android версии не ниже 4.0.x.
В этом случае оно будет доступно для 97\% пользователей данной мобильной платформы,
что составляет около двух миллиардов человек.


\subsection{Сравнительный анализ существующих приложений-аналогов}

Для уточнения требований к разрабатываемому приложению
выполним сравнительный анализ существующих решений-аналогов.
Сравнительный анализ по опредению требует предварительного
описания множества объектов, подвергающихся анализу,
и множества критериев сравнения.

Выполним отбор приложений для анализа следующим образом:
введем типичную поисковую фразу для данной категории приложений,
например, ``money manager'' и выберем несколько приложений,
расположившихся на верхних позициях поисковой выдачи.
Данный подход призван сымитировать поведение типичного пользователя
Play Market, желающего установить себе подобное приложение,
при этом не обладающего какой-либо дополнительной информацией о предмете,
а посему полагающегося на качество поисковых алгоритмов Google.
Подобным образом были выбраны следующие приложения, расположенные
в порядке выдачи:
\begin{itemize}
\item Monefy (автор: MonefyApp);
\item Money Manager Expense \& Budget (Realbyte Inc.);
\item Money Lover --- Money Manager (ZooStudio);
\item Money Manager Ex for Android (Android Money Manager Ex Prj)
\item AndroMoney (AndroMoney);
\item Expense Manager (Bishinews);
\item Money Manager Master (SMH17);
\item Money Manager (Praveen Thogarla).
\end{itemize}

Отметим, что как минимум пять приложений из этого списка поддерживаются
компаниями-разработчиками, что косвенно свидетельствует о сложности данных проектов.
Последний пункт списка на самом деле представляет собой два похожих друг
на друга одноименных приложения, разработанных индийскими программистами.

Определим основные категории параметров мобильных приложений, подлежащих сравнению:
\begin{itemize}
\item базовые характеристики;
\item системные характеристики;
\item функциональные возможности;
\item пользовательский интерфейс.
\end{itemize}

Под базовыми характеристиками будем понимать набор параметров приложения,
доступных для оценки без установки приложения. Как правило, эти параметры
создают первое впечатление о приложении у конечного пользователя.
Системные характеристики представляют собой характеристики функционирующего
приложения безотносительно выполняемых им функций.
Набор функциональных возможностей определяет способность приложения
удовлетворять различные запросы пользователя в контексте предметной области.
Пользовательский интерфейс влияет на удобство использования приложения в целом.

Рассмотрим категорию базовых характеристик приложения.
К числу таких характеристик относятся:
количество загрузок приложения, рейтинг приложения,
стоимость платной версии (если таковая имеется), ограничения бесплатной версии.
Результаты сравнения приложений по данным характеристикам приведены
в таблице~\ref{tbl:cmp_basic}.

\begin{table} [h!]
  \caption{
    Сравнение приложений-аналогов по базовым критериям
  }\label{tbl:cmp_basic}
    \begin{tabular}{| m{4cm} | c | c | c | c | c | c | c | c |}
      \hline
      \parbox{4cm}{
        \smallskip
        \centering Критерий \\ сравнения
        \smallskip
      }
      & \rotatebox[origin=c]{90}{
          \parbox{5cm}{
            Monefy
          }
        }
      & \rotatebox[origin=c]{90}{
          \parbox{5cm}{
            Money Manager \\ Expense \& Budget
          }
        }
      & \rotatebox[origin=c]{90}{
          \parbox{5cm}{
            Money Lover --- \\ Money Manager
          }
        }
      & \rotatebox[origin=c]{90}{
          \parbox{5cm}{
            Andro Money
          }
        }
      & \rotatebox[origin=c]{90}{
          \parbox{5cm}{
            Expense Manager
          }
        }
      & \rotatebox[origin=c]{90}{
          \parbox{5cm}{
            Money Manager Ex \\ for Android
          }
        }
      & \rotatebox[origin=c]{90}{
          \parbox{5cm}{
            Money Manager Master
          }
        }
      & \rotatebox[origin=c]{90}{
          \parbox{5cm}{
            Money Manager
          }
        } \\
      \hline

      Приблизительное \par количество \par загрузок
      & \( 10^6 \)
      & \( 10^6 \)
      & \( 10^6 \)
      & \( 10^6 \)
      & \( 10^6 \)
      & \( 10^5 \)
      & \( 5 \cdot 10^3 \)
      & \( 10^3 \) \\
      \hline

      Рейтинг \par приложения
      & \( 4{,}5 \)
      & \( 4{,}5 \)
      & \( 4{,}5 \)
      & \( 4{,}7 \)
      & \( 4{,}3 \)
      & \( 4{,}4 \)
      & \( 4{,}6 \)
      & \( 4{,}5 \) \\
      \hline

      Ограничения \par бесплатной \par версии
      & Р\tablefootnote{принудительный показ рекламы{\color{red}.}},
        Ф\tablefootnote{ограничения функциональности.}
      & Р, Ф
      & Р, Ф
      & Р
      & Р
      & \( - \)
      & Р, Ф
      & \( - \) \\
      \hline

      Стоимость \par
      платной версии, \$
      & \( 1{,}8 \)
      & \( 3{,}3 \)
      & \( 2{,}2 - 5 \)
      & \( 5 \)
      & \( 5 \)
      & \( - \)
      & \( 2{,}3 - 5 \)
      & \( - \) \\
      \hline
    \end{tabular}
\end{table}

По приведенным данным можно сделать следующие выводы:
\begin{itemize}
  \item данный класс приложений пользуются высоким спросом;
  \item наиболее популярные приложения разрабатываются
    компаниями, а не частными лицами;
  \item монетизация приложений осуществляется в основном за
    счет принудительного показа рекламы в бесплатной версии приложения.
\end{itemize}

К системным характеристикам будем относить
размер установленного приложения и набор прав доступа,
требуемых для его функционирования.
Результат сравнения по этим критериям приведен в таблице~\ref{tbl:cmp_system}.

\begin{table} [h!]
  \caption{
    Сравнение приложений-аналогов по системным критериям
  }\label{tbl:cmp_system}
  \small{
    \begin{tabular}{| m{6.7cm} | c | c | c | c | c | c | c | c |}
      \hline
      \parbox{6.7cm}{
        \smallskip
        \centering Критерий \\ сравнения
        \smallskip
      }
      & \rotatebox[origin=c]{90}{
          \parbox{4.5cm}{
            Monefy
          }
        }
      & \rotatebox[origin=c]{90}{
          \parbox{4.5cm}{
            Money Manager \\ Expense \& Budget
          }
        }
      & \rotatebox[origin=c]{90}{
          \parbox{4.5cm}{
            Money Lover --- \\ Money Manager
          }
        }
      & \rotatebox[origin=c]{90}{
          \parbox{4.5cm}{
            Andro Money
          }
        }
      & \rotatebox[origin=c]{90}{
          \parbox{4.5cm}{
            Expense Manager
          }
        }
      & \rotatebox[origin=c]{90}{
          \parbox{4.5cm}{
            Money Manager Ex \\ for Android
          }
        }
      & \rotatebox[origin=c]{90}{
          \parbox{4.5cm}{
            Money Manager Master
          }
        }
      & \rotatebox[origin=c]{90}{
          \parbox{4.5cm}{
            Money Manager
          }
        } \\
      \hline

      Размер установленного \par приложения, Мб
      & \( 15{,}2 \)
      & \( 12{,}1 \)
      & \( 23{,}4 \)
      & \( 19{,}5 \)
      & \( 7{,}3 \)
      & \( 15{,}8 \)
      & \( 15{,}1 \)
      & \( 1{,}7 \) \\
      \hline

      Запрашиваемые права:
      & & & & & & & & \\

      -- доступ к карте памяти
      & +
      & +
      & +
      & +
      & +
      & +
      & +
      & \\

      -- доступ к Интернет
      & +
      & +
      & +
      & +
      & +
      & +
      & +
      & \\

      -- доступ к Google-аккаунту \par пользователя
      &
      &
      & +
      & +
      & +
      &
      & +
      & \\

      -- возможность управления \par вибратором
      &
      & +
      & +
      & +
      &
      & +
      &
      & \\

      -- возможность автоматического \par запуска приложения при \par старте системы
      &
      & +
      & +
      & +
      &
      & +
      &
      & \\

      -- возможность предотвращения \par
      автоматического перехода \par
      устройства в энергосберегающий \par
      режим
      &
      & +
      & +
      & +
      &
      &
      &
      & \\

      -- доступ к SMS пользователя
      &
      & +
      &
      & +
      & +
      &
      &
      & \\

      -- доступ к данным GPS
      &
      &
      & +
      &
      &
      &
      & +
      & \\

      -- доступ к телефонному модулю
      &
      & +
      &
      &
      &
      &
      & +
      & \\

      -- доступ к списку активных \par приложений
      &
      & +
      &
      &
      &
      &
      &
      & \\

      -- управление иконками \par рабочего стола
      &
      &
      & +
      &
      &
      &
      &
      & \\

      -- доступ к персональной \par информации\tablefootnote{%
      доступ к событиям календаря, электронной почте и
      прочим персональным данным пользователя.}
      &
      &
      &
      &
      &
      &
      & +
      & \\

      -- доступ к фотокамере
      &
      &
      &
      & +
      &
      &
      &
      & \\
      \hline
    \end{tabular}
  }
\end{table}


Здравый смысл подсказывает, что установленное приложение должно
занимать как можно меньше места и требовать для работы как
можно меньше прав доступа.
С другой стороны, поскольку стоимость ПЗУ достаточно низка,
его объем для современных смартфонов не является критическим ресурсом,
поэтому размер установленного приложения не играет существенной роли.
Иным образом обстоит дело с правами доступа:
с одной стороны, они могут быть необходимы для обеспечения работоспособности
базовых функций приложения
(например, приложение, предоставляющее прогноз погоды, должно иметь доступ в Интернет),
использоваться для предоставления дополнительных возможностей
(приложение, предназначенное для обмена сообщениями,
может потребовать доступ к карте памяти, где хранятся фотографии, для того, чтобы
прикреплять их к основному тексту).
С другой стороны, существует риск, что приложение, написанное недобросовестным
разработчиком, может использовать предоставленные предоставленные ему права
для во вред владельцу устройства,
(например, рассылать спам или передавать персональные данные третьим лицам).

По данным таблицы~\ref{tbl:cmp_system} можно сделать следующие выводы:
\begin{itemize}
\item для работы большинства приложений необходимы доступ к карте памяти и
  Интернет;
\item приложения Monefy и Money Manager выгодно отличаются от конкурентов
  минимальным набором требуемых прав доступа;
\item некоторые приложения требуют использования прав,
  нетипичных для своего класса.
\end{itemize}

В свете приведенных выше соображений требования приложений
Money Manager Expense \& Budget, Money Lover --- Money Manager,
Andro Money, Expense Manager и Money Manager Master доступа
к пользовательским SMS, данным навигации,
телефонному модулю выглядят весьма неоднозначено.
Особенно подозрительным представляется требование приложения
Money Manager Master доступа к персональным данным пользователя.

Для сравнения функциональности конкурирующих приложений будем использовать
набор типичных возможностей, присущих приложениям данного класса.
Этот набор был составлен во многом субъективно на базе личного пользовательского опыта.
Поясним некоторые из этих возможностей.
Возможность создания повторяющихся транзакций бывает удобна для
периодических статей прибыли и расхода денежных средств, например,
заработная плата, затраты на транспорт, оплата коммунальных услуг, и~т.~д.
Представление данных в виде круговых диаграмм используется,
как правило, для отображения относительной доли транзакций
каждой категории в общей сумме транзакций.
Представление данных в виде столбчатых диаграмм
используется для отображения изменения значений прибыли и затрат во времени.
Функция ограничения величины расходов предупреждает пользователя
о превышении установленного лимита расходов определенной категории.
Результаты сравнения представлены в таблице~\ref{tbl:cmp_func}.

\begin{table} [h!]
  \caption{
    Функциональные возможности приложений-аналогов
  }\label{tbl:cmp_func}
  \small{
    \begin{tabular}{| m{7.8cm} | c | c | c | c | c | c | c | c |}
      \hline
      \parbox{7.8cm}{
        \smallskip
        Функциональная возможность
        \smallskip
      }
      & \rotatebox[origin=c]{90}{
          \parbox{4.5cm}{
            Monefy
          }
        }
      & \rotatebox[origin=c]{90}{
          \parbox{4.5cm}{
            Money Manager \\ Expense \& Budget
          }
        }
      & \rotatebox[origin=c]{90}{
          \parbox{4.5cm}{
            Money Lover --- \\ Money Manager
          }
        }
      & \rotatebox[origin=c]{90}{
          \parbox{4.5cm}{
            Andro Money
          }
        }
      & \rotatebox[origin=c]{90}{
          \parbox{4.5cm}{
            Expense Manager
          }
        }
      & \rotatebox[origin=c]{90}{
          \parbox{4.5cm}{
            Money Manager Ex \\ for Android
          }
        }
      & \rotatebox[origin=c]{90}{
          \parbox{4.5cm}{
            Money Manager Master
          }
        }
      & \rotatebox[origin=c]{90}{
          \parbox{4.5cm}{
            Money Manager
          }
        } \\
      \hline

      Управление данными:
      & & & & & & & & \\

      -- ведение нескольких счетов
      & +
      & +
      & +
      & +
      & +
      & +
      & +
      & \\

      -- редактирование категорий учета
      & *\tablefootnote{в платной версии.}
      & +
      & +
      & +
      & +
      & +
      & +
      & + \\

      -- создание повторяющихся транзакций
      &
      & +
      & +
      & +
      & +
      & +
      & +
      & \\

      -- поддержка нескольких валют
      &
      & +
      & +
      & +
      & +
      & +
      & +
      & \\
      \hline

      Представление данных:
      & & & & & & & & \\

      -- группировка данных по времени
      & +
      & +
      & +
      & +
      & +
      & +
      & +
      & + \\

      -- группировка данных по категориям
      & +
      & +
      & +
      & +
      & +
      & +
      & +
      & + \\

      -- группировка данных по типу транзакций
      &
      & +
      & +
      &
      & +
      & +
      & +
      & + \\

      -- представление данных в виде \par
      круговых диаграмм
      & +
      & +
      &
      & +
      &
      & +
      & *
      & \\

      -- представление данных в виде \par
      столбчатых диаграмм
      &
      & +
      & +
      & +
      & +
      & +
      & *
      & + \\

      \hline

      Дополнительные возможности:
      & & & & & & & & \\

      -- резервное копирование
      & +
      & +
      & +
      & +
      & +
      & +
      &
      & \\

      -- ограничение величины расходов
      & +
      & +
      & +
      & +
      & +
      & +
      &
      & \\

      -- установка напоминаний
      \par о необходимости ввода данных
      &
      &
      & +
      & +
      &
      &
      &
      & \\

      -- поиск транзакций
      &
      & +
      & +
      & +
      &
      & +
      &
      & \\

      -- парольная защита доступа к данным
      & +
      & +
      &
      & +
      & +
      & +
      &
      & \\

      -- экспорт/импорт данных из файла
      & +
      & *
      & +
      & +
      & +
      & +
      &
      & \\

      -- синхронизация данных с \par
      удаленным сервером
      & +
      & *
      & +
      & +
      & +
      & +
      &
      & \\
      \hline
    \end{tabular}
  }
\end{table}

По результатам анализа функциональных возможностей, рассматриваемые приложения
можно разделить на три группы в зависимости от соотношения количества поддерживаемых
возможностей и удобства пользовательского интерфейса.
Первая группа приложений (куда входят
Money Manager Expense \& Budget, Money Lover --- Money Manager,
Andro Money, Expense Manager, Money Manager Ex for Android) имеют
широкий набор функций и большое количество настроек.
Вторая группа приложений, состоящая из Monefy и Money Manager Master,
имеют более скромный набор возможностей, но зачастую более удобный интерфейс.
Третья группа представлена предельно простым в использовании приложением Money Manager,
имеющим лишь минимальный набор функций.

Поскольку сравнение пользовательских интерфейсов программных продуктов
является делом заведомо субъективным, ограничимся следующими критериями:
соответствие дизайна приложения рекомендациям Google, наличие возможности управления жестами,
наличие различных локализаций приложения.
Material design ---
свод рекомендаций Google о пользовательском интерфейсе мобильных приложений ---
описывает ключевые особенности дизайна мобильных приложений на платформе Android:
способы взаимодействия пользователя с приложением,
рекомендации по использованию стандартных виджетов,
иконки, цветовые палитры, начертания и размеры шрифтов, отступов, и~т.~д.
Возможность управления жестами играет ключевую роль в современных мобильных приложениях,
существенно ускоряя процессы навигации и ввода данных.
Наличие различных локализаций приложения значительно расширяет потенциальную
аудиторию пользователей приложения, позволяя пользоваться им людям, не владеющим
английским языком. В таблице~\ref{tbl:cmp_interface} представлен результат сравнения
пользовательских интерфейсов рассмартиваемых приложений.

\begin{table} [h!]
  \caption{
    Пользовательский интерфейс приложений-аналогов
  }\label{tbl:cmp_interface}
  \small{
    \begin{tabular}{| m{8cm} | c | c | c | c | c | c | c | c |}
      \hline
      \parbox{8cm}{
        \smallskip
        Критерий
        \smallskip
      }
      & \rotatebox[origin=c]{90}{
          \parbox{4.5cm}{
            Monefy
          }
        }
      & \rotatebox[origin=c]{90}{
          \parbox{4.5cm}{
            Money Manager \\ Expense \& Budget
          }
        }
      & \rotatebox[origin=c]{90}{
          \parbox{4.5cm}{
            Money Lover --- \\ Money Manager
          }
        }
      & \rotatebox[origin=c]{90}{
          \parbox{4.5cm}{
            Andro Money
          }
        }
      & \rotatebox[origin=c]{90}{
          \parbox{4.5cm}{
            Expense Manager
          }
        }
      & \rotatebox[origin=c]{90}{
          \parbox{4.5cm}{
            Money Manager Ex \\ for Android
          }
        }
      & \rotatebox[origin=c]{90}{
          \parbox{4.5cm}{
            Money Manager Master
          }
        }
      & \rotatebox[origin=c]{90}{
          \parbox{4.5cm}{
            Money Manager
          }
        } \\
      \hline

      Соответствие Material Design
      &
      & +
      & +
      &
      &
      & +
      &
      & + \\
      \hline

      Возможность управления жестами
      & +
      &
      & +
      & +
      & +
      & +
      &
      & \\
      \hline

      Наличие локализаций
      & +
      &
      & +
      &
      & +
      & +
      &
      & \\
      \hline
    \end{tabular}
  }
\end{table}

Результаты сравнения свидетельствуют о том, что наиболее проработанным
пользовательским интерфейсом обладают приложения
Money Lover --- Money Manager и Money Manager Ex for Android.

Подведем итог. Результаты анализа показывают, что рассмотренные приложения
можно разделить на три группы.
Приложения первой группы имеют очень большую аудиторию пользователей,
развитую функциональность и большое количество настроек.
Все эти приложения являются коммерческими,
бесплатные версии которых показывают рекламные баннеры.
Сюда относятся Monefy, Money Manager Expense \& Budget,
Money Lover --- Money Manager, Andro Money, Expense Manager.

Во вторую группу входят такие приложения, как
Money Manager Ex for Android и Manager Master.
Они имеют также равитую функциональность, но меньшую аудиторию пользователей.

Третья группа представлена приложением Money Manager,
обладающим небольшой аудиторией и имеющим весьма скромный набор функций.
С другой стороны, данное приложение является бесплатным, не содержит рекламы
и имеет весьма простой пользовательский интерфейс.

Мобильное приложение, разрабатываемое в рамках данного проекта,
следует причислить к третьей группе приложений,
поскольку в рамках проекта предполагается реализация весьма
ограниченного числа наиболее используемых функций.
Кроме этого, автор приложения не преследует цели получения коммерческой выгоды
от его использования.

\subsection{Постановка задачи проектирования}

В рамках данного проекта требуется выполнить проектирование и
программную реализацию мобильного приложения, выполняющего учет и представление
финансовой информации пользователя.
Данное приложение должно работать под управлением мобильной ОС Android версии
не ниже 4.0.x. Ввод финансовых данных должен осуществляться как вручную,
так и в автоматическом режиме с использованием фотокамеры мобильного устройства.
Приложение должно предусматривать вывод данных в наглядной форме и иметь
удобный пользовательский интерфейс.
