\section*{ВВЕДЕНИЕ}
\addcontentsline{toc}{section}{Введение}

\begin{color}{red}
  Введение (предисловие) помещают на отдельной странице. Слово
  ВВЕДЕНИЕ (ПРЕДИСЛОВИЕ) записывают прописными буквами по центру.
  Введение (предисловие) должно быть кратким и четким, не должно быть общих
  мест и отступлений, непосредственно не связанных с разрабатываемой темой.
  Объем введения не должен превышать двух страниц~\cite{stp2013}.

  Рекомендуется следующее содержание введения (предисловия):
  \begin{itemize}
  \item краткий анализ достижений в той области, которой посвящена тема ди-
    пломного проекта (работы);
  \item цель дипломного проектирования;
  \item принципы, положенные в основу проектирования, научного исследова-
    ния, поиска технического решения;
  \item краткое изложение содержания разделов пояснительной записки с обя-
    зательным указанием задач, решению которых они посвящены.
  \end{itemize}
\end{color}