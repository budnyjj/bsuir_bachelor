\section*{ВВЕДЕНИЕ}
\addcontentsline{toc}{section}{Введение}

В последнее время наблюдается значительный рост
популярности использования мобильных устройств.
Наряду со значительным увеличением числа мобильных устройств
на планете имеет место увеличение их вычислительной мощности:
процессор современного смартфона обладает производительностью,
сравнимой с производительностью процессора персонального
компьютера десятилетней давности.
Высокая производительность мобильных устройств вместе с наличием
встроенных средств беспроводного доступа к сети Интернет открывают
возможности разработки и использования нестандартных мобильных
приложений, призванных удовлетворить потребности пользователей,
не учтенные компанией-разрабочиком мобильной платформы.
Рост популярности мобильных устройств, возможность разработки собственных
приложений для них, а также наличие развитой инфраструктуры для разработки,
сопровождения и обновления приложений, обуславливают актуальность темы
разработки программного обеспечения для мобильных платформ.

В данной работе рассматривается класс мобильных приложений,
выполняющих учет персональной финансовой информации.
Целью работы является проектирование и разработка мобильного приложения,
выполняющего учет личных финансов пользователя.
Отличительной особенностью данной работы является разработка и использование
альтернативного механизма ввода финансовой информации.

Структура работы определяется основными задачами, решаемыми в ней.
Так, раздел~\ref{sec:system_spec} содержит описание предметной области:
выбор наиболее перспективной прогрммной платформы
для разработки мобильного приложения,
сравнительный анализ существующих приложений-аналогов,
постановку задачи проектирования.
Раздел~\ref{sec:design} рассматривает вопросы проектирования:
описание структуры приложения,
его информационного, математического, алгоритмического и эргономического обеспечения,
а также технических и системных программных требований.
Раздел~\ref{sec:implementation} описывает детали реализации,
тестирования и сопровождения приложения.
В разделе~\ref{sec:teo} выполнены расчет
соотношения затрат на разработку программного обеспечения
и ожидаемого эффекта от его использования.
