\section*{ВВЕДЕНИЕ}
\addcontentsline{toc}{section}{Введение}

В последнее время наблюдается значительный рост
популярности мобильных устройств.
Наряду с увеличением общего числа устройств растет их производительность:
можно утверждать, что процессор современного смартфона обладает
вычислительной мощностью, сравнимой с мощностью процессора
персонального компьютера десятилетней давности.
Подобный рост производительности позволяет перенести часть функций
программного обеспечения для персональных компьютеров
на мобильные платформы в виде мобильных приложений.
Популярность мобильных устройств,
сочетание удобства их использования и производительности
обуславливают актуальность темы разработки программного
обеспечения для мобильных платформ.

В данной работе рассматривается класс мобильных приложений,
выполняющих учет персональной финансовой информации.
Целью работы является проектирование и реализация мобильного приложения,
выполняющего учет личных финансов пользователя.
Отличительной особенностью данной работы является разработка алгоритма
оптического распознавания числовых данных для облегчения ввода
финансовой информации в мобильное приложение.

Структура работы определяется основными задачами, решаемыми в ней.
Так, в разделе~\ref{sec:spec} содержится описание предметной области,
рассматриваетя процесс выбора перспективной программной платформы
для разработки мобильного приложения,
выполняется анализ существующих аналогов и постановка задачи проектирования.
В разделе~\ref{sec:design} рассматриваются вопросы проектирования приложения:
описание его структуры, информационного,
алгоритмического и эргономического обеспечения,
а также требований к аппаратной платформе.
В разделе~\ref{sec:implementation} описываются детали реализации,
тестирования и сопровождения приложения.
В разделе~\ref{sec:teo} выполнняется оценка соотношения затрат
на разработку данного мобильного приложения и ожидаемого эффекта
от его использования.
