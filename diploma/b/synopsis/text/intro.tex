\section*{ВВЕДЕНИЕ}
\addcontentsline{toc}{section}{Введение}

В течение последних двадцати лет наблюдается значительный рост
популярности использования мобильных устройств.
Вместе с увеличением числа мобильных устройств на планете имеет место
значительное увеличение их вычислительной мощности:
процессор современного смартфона обладает производительностью,
сравнимой с производительностью процессора персонального
компьютера десятилетней давности.
Высокая производительность мобильных устройств вкупе с наличием
встроенных средств беспроводного доступа к {\color{red} сети Интернет} открывают
возможности эффективной разработки и удобного использования
нестандартных мобильных приложений, призванных удовлетворить
потребности пользователей, не учтенные компанией-разрабочиком
мобильной платформы.

Рост популярности мобильных устройств, возможность разработки собственных
приложений для них, а также наличие развитой инфраструктуры для разработки,
сопровождения и обновления приложений, обуславливают актуальность темы
разработки программного обеспечения для мобильных платформ.

В данной работе рассматривается класс мобильных приложений,
выполняющих учет персональной финансовой информации.
Целью работы является проектирование и разработка мобильного приложения,
выполняющего учет личных финансов пользователя.
Отличительной особенностью данной работы можно назвать попытку разработки
альтернативного метода ввода финансовой информации.

Структура работы определяется основными задачами, решаемыми в ней.
Так, раздел~\ref{sec:spec} содержит описание различных сторон предметной области:
анализ динамики рынка мобильных платформ, сравнительный анализ существующих
приложений-аналогов, беглый обзор основных понятий в контексте тем распознавния
образов и свободного программного обеспечения, а также постановку задачи проектирования.
Раздел~\ref{sec:design} рассматривает вопросы проектирования мобильного приложения:
разработку модели данных, механизмов ввода/вывода финансовой информации,
проектирование пользовательского интерфейса.
Разделы~\ref{sec:implementation} и~\ref{sec:usage} посвящены рассмотрению деталей
реализации и сопровождения. В разделе~\ref{sec:teo} выполнены расчет
величины затрат на разработку программного обеспечения и оценка ожидаемого эффекта
от его использования.
