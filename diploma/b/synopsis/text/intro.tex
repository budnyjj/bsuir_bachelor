\section*{ВВЕДЕНИЕ}
\addcontentsline{toc}{section}{Введение}

В последнее время наблюдается значительный рост
популярности использования мобильных устройств.
Наряду с увеличением общего числа мобильных
устройств растет и их производительность.
Можно утверждать, что процессор современного смартфона обладает
вычислительной мощностью, сравнимой с мощностью процессора
персонального компьютера десятилетней давности.
Подобный рост производительности позволяет перенести часть функций
программного обеспечения для персональных компьютеров
на мобильные платформы в виде мобильных приложений.
Высокая популярность мобильных устройств, а также
их производительность, достаточная для решения большинства
несложных задач, возникающих в реальной жизни, обуславливает
актуальность темы разработки программного обеспечения
для мобильных платформ.

В данной работе рассматривается класс мобильных приложений,
выполняющих учет персональной финансовой информации.
Целью работы является проектирование и разработка мобильного приложения,
выполняющего учет личных финансов пользователя.
% Отличительной особенностью данной работы является разработка и использование
% альтернативного механизма ввода финансовой информации.

Структура работы определяется основными задачами, решаемыми в ней.
Так, раздел~\ref{sec:system_spec} содержит описание предметной области:
выбор перспективной программной платформы для разработки мобильного
приложения, анализ существующих аналогов и постановку задачи проектирования.
Раздел~\ref{sec:design} рассматривает вопросы проектирования:
описание структуры приложения, его информационного, математического,
алгоритмического и эргономического обеспечения,
а также технических и системных программных требований.
Раздел~\ref{sec:implementation} описывает детали реализации,
тестирования и сопровождения приложения.
В разделе~\ref{sec:teo} выполнена оценка соотношения затрат
на разработку данного мобильного приложения и ожидаемого эффекта
от его использования.
