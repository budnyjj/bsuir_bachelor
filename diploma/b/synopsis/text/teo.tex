\section[Технико-экономическое обоснование]{%
  ТЕХНИКО-ЭКОНОМИЧЕСКОЕ ОБОСНОВАНИЕ \\
  РАЗРАБОТКИ ПРОГРАММНОГО МОДУЛЯ
}\label{sec:teo}

\subsection{Экономическая характеристика}

Разрабатываемое мобильное приложение предназначено для учета
персональных финансовых данных.
Данное приложение предоставляет возможности ввода и представления
данных о прибыли и затратах пользователя в наглядной форме.
Разработка данного программного продукта носит преимущественно
индивидуальный характер --- работы по проектированию, реализации и
сопровождению приложения производятся одним лицом.
Основным средством распространения и реализации приложения является сервис
Google Play --- стандартный каталог приложений мобильной платформы Android.

Как было показано в подразделе~\ref{subsec:spec_compare}, целевая аудитория
потенциальных пользователей разрабатываемого приложения весьма велика.
В свете отличительных особенностей разрабатываемого приложения от аналогов
(наличие минимально необходимого набора функций,
простота пользовательского интерфейса,
бесплатность без ограничений функциональности и отсуствие рекламы),
предполагается, что потенциальными пользователями разрабатываемого
приложения являются молодые люди, желающие вести учет личных денежных
средств и не готовые тратить средства на покупку коммерческих приложений.

Поскольку извлечение коммерческой прибыли от реализации или использования приложения
не предполагается, экономическое обоснование разработки приложения
сводится к приблизительному расчету затрат и описанию эффекта от его использования,
носящего неэкономический характер.

\subsection{Расчет затрат на разработку}

Выполним упрощенный расчет затрат на разработку мобильного приложения
по методике, приведенной в~\cite{diploma_teo}. В соответствии с ней,
основная доля затрат на разработку программного обеспечения приходится
на следующие статьи:
\begin{itemize}
  \item затраты на основную заработную плату разработчиков;
  \item затраты на дополнительную заработную плату разработчиков;
  \item отчисления на социальные нужды;
  \item прочие затраты.
\end{itemize}

Итоговое значение затрат получается путем суммирования значений затрат
по приведенным выше статьям.

Расчет величины основной заработной платы разработчиков программного
продукта производится по следующей формуле:
\begin{equation}
  \text{З}_{\text{о}} =
  \sum^n_i \text{Т}_{\text{ч}_{i}} \cdot t_{i},
\end{equation}

\noindent где
\( n \)
--- количество исполнителей, занятых разработкой приложения; \par
\noindent \hspace{6.5mm} \( \text{Т}_{\text{ч}_{i}} \)
--- часовая заработная плата \( i \)-го исполнителя (руб.); \par
\noindent \hspace{6.5mm} \( t_i \)
--- трудоемкость работ, выполняемых \( i \)-тым исполнителем (ч).

Часовая заработная плата определяется путем деления месячной
заработной платы на количество рабочих часов в месяце,
принимаемым равным 168 часам.
Учитывая данные об установившейся средней заработной
плате различных IT-специалистов~\cite{dev_by_salaries},
выполним расчет величины основной заработной платы,
представленный в таблице~\ref{tbl:teo_salary}.

\begin{table} [h!]
  \caption{
    Расчет затрат на основную заработную плату команды разработчиков
  }\label{tbl:teo_salary}
  \small{
    \begin{tabular}{| c | m{4.3cm} | c | c | c | c | c |}
      \hline
      \textnumero
      &
        \parbox{4.3cm}{
        \smallskip
        \centering Участник \\ команды
      \smallskip
      }
      &
        \parbox{2cm}{
        \smallskip
        \centering Месячная \\ заработная \\ плата, тыс.~р.
      \smallskip
      }
      &
        \parbox{2cm}{
        \smallskip
        \centering Часовая \\ заработная \\ плата, тыс.~р.
      \smallskip
      }
      &
        \parbox{3cm}{
        \smallskip
        \centering Трудоемкость работ, ч
        \smallskip
        }
      &
        \parbox{2cm}{
        \smallskip
        \centering Основная \\ заработная \\ плата, тыс.~р.
      \smallskip
      } \\
      \hline
      1
      &
        Руководитель проекта
      &
        \(28 \: 000{,}00 \)
      &
        \(166{,}67\)
      &
        \( 16 \)
      &
        \( 2666{,}67 \) \\
      \hline

      2
      &
        Программист
      &
        \(13 \: 000{,}00 \)
      &
        \(77{,}38\)
      &
        \( 168 \)
      &
        \( 13 \: 000{,}00 \) \\
      \hline

      \multicolumn{5}{|r|}{Премия 50\%}
      & \( 7 \: 833{,}34 \) \\
      \hline

      \multicolumn{5}{|r|}{Итого затраты на основную заработную плату разработчиков}
      & \( 23 \: 500{,}00 \) \\
      \hline
    \end{tabular}
  }
\end{table}

Затраты на дополнительную заработную плату команды
разработчиков включает выплаты, предусмотренные законодательством о
труде (оплата отпусков, льготных часов, времени выполнения государственных
обязанностей и других выплат, не связанных с основной деятельностью
исполнителей), и определяется по формуле:
\begin{equation}
  \text{З}_{\text{д}} =
  \dfrac{\text{З}_{\text{о}} \cdot \text{Н}_{\text{д}}}{100},
\end{equation}

\noindent где
\( \text{З}_{\text{о}} \)
--- затраты на основную заработную плату с учетом премии (руб.); \par
\noindent \hspace{6.5mm} \( \text{Н}_{\text{д}} \)
--- норматив дополнительной заработной платы.

Принимая значения норматива дополнительной заработной платы в размере 15\%,
получим значение дополнительной заработной платы разработчиков
мобильного приложения:
\begin{equation}
  \text{З}_{\text{д}} =
  \dfrac{23 \: 500{,}00 \cdot 15}{100} = 3 \: 525{,}00 \: (\text{тыс.~р.}).
\end{equation}

Отчисления на социальные нужды (в фонд социальной защиты
населения и на обязательное страхование) определяются в соответствии с
действующими законодательными актами по формуле:
\begin{equation}
  \text{З}_{\text{соц}} =
  \dfrac{(\text{З}_{\text{о}} + \text{З}_{\text{д}}) \cdot \text{Н}_{\text{соц}}}{100},
\end{equation}

\noindent где
\( \text{Н}_{\text{соц}} \)
--- норматив отчислений на социальные нужды.

Согласно действующему законодательству~\cite{law_social_royalties},
на данный момент норматив отчислений на социальные нужды для работодателей
составляет 28\%. Принимая это во внимание, выполним расчет значения
отчислений на социальные нужды:
\begin{equation}
  \text{З}_{\text{соц}} =
  \dfrac{(23 \: 500{,}00 + 3 \: 525{,}00) \cdot 28}{100} = 7 \: 567{,}00 \: (\text{тыс.~р.}).
\end{equation}

Расчет прочих затрат осуществляется в процентах от затрат на основную
заработную плату команды разработчиков с учетом премии по формуле:
\begin{equation}
  \text{З}_{\text{пз}} =
  \dfrac{\text{З}_{\text{о}} \cdot \text{Н}_{\text{пз}}}{100},
\end{equation}
\noindent где
\( \text{Н}_{\text{пз}} \)
--- норматив прочих затрат, принимаемый в размере 40\%.

В нашем случае данная статья статья расходов включает в себя затраты на приобретение
научно-технической литературы, рассматривающей вопросы распознавания образов,
а также мобильного устройства, предназначенного
для тестирования разрабатываемого приложения.
Определим значение данной статьи расходов:
\begin{equation}
  \text{З}_{\text{пз}} =
  \dfrac{23 \: 500{,}00 \cdot 40}{100} = 9 \: 400{,}00 \: (\text{тыс.~р.}).
\end{equation}

Полная сумма затрат на разработку программного обеспечения представлена в таблице
~\ref{tbl:teo_sum_cost}.

\begin{table} [h!]
  \caption{
    Расчет затрат на основную заработную плату команды разработчиков
  }\label{tbl:teo_sum_cost}
  \begin{tabular}{| m{13.5cm} | c |}
    \hline

    \parbox{13.5cm}{
    \smallskip
    \centering Статья затрат
    \smallskip
    }
    &
      \parbox{2cm}{
      \smallskip
      \centering Сумма, тыс.~р.
    \smallskip
    } \\
    \hline

    Основная заработная плата команды разработчиков
    & \( 23 \: 500{,}00 \)\\
    \hline

    Дополнительная заработная плата команды разработчиков
    & \( 3 \: 525{,}00 \)\\
    \hline

    Отчисления на социальные нужды
    & \( 7 \: 567{,}00 \)\\
    \hline

    Прочие затраты
    & \( 9 \: 400{,}00 \)\\
    \hline

    Общая сумма затрат на разработку
    & \( 43 \: 992{,}00 \) \\
    \hline
  \end{tabular}
\end{table}

Таким образом, сумма затрат на разработку мобильного приложения
составляет приблизительно сорок четыре миллиона белорусских рублей.

\subsection{Оценка эффекта от использования}

Поскольку получение коммерческой прибыли от реализации или использования
разрабатываемого приложения изначально не предполагается,
имеет смысл вести речь об эффекте исключительно
неэкономического характера.

Основным эффектом от использования приложения является
предоставление даннх пользователя об объемах и характере
его расходов и доходов. Данная информация
позволяет оптимизировать расходы личных денежных средств.
Подобная оптимизация представляет собой косвенное влияние
на экономические показатели деятельности пользователя.