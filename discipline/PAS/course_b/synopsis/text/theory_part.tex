\section[Теоретическая часть]{ТЕОРЕТИЧЕСКАЯ ЧАСТЬ}

\subsection{Информационное обеспечение}

Для функционального моделирования и графической нотации при проектировании
программного обеспечения используются следующие методологии:
\begin{itemize}
\item IDEF0;
\item DFD;
\item IDEF3;
\item IDEF1x.
\end{itemize}

Методология \textbf{IDEF0} используется при для описания процессов,
происходящих в системе. Рассмотрим её концептуальные положения.

\textbf{Модель} --- искусственный объект, представляющий собой отображение
системы и ее компонентов. Модель разрабатывают для понимания, анализа и принятия
решений о реконструкции или замене существующей, либо
проектировании новой системы.

\textbf{Система} представляет собой совокупность
взаимосвязанных и взаимодействующих частей, выполняющих некоторую
полезную работу. Частями системы могут быть любые
комбинации разнообразных сущностей, включающие людей, информацию,
программное обеспечение, оборудование, изделия, сырье или энергию.

Модель описывает, что происходит в системе, как ею
управляют, какие сущности она преобразует, какие средства использует для
выполнения своих функций и что производит.

Основной концептуальный принцип методологии IDEF --- представление любой
изучаемой системы в виде набора взаимодействующих и взаимосвязанных
блоков, отображающих процессы, операции, действия,
происходящие в изучаемой системе.

В IDEF0 всё, что происходит в системе и ее элементах, принято называть
\textbf{функциями}. Каждой функции ставится в соответствие блок.
На диаграмме IDEF0 блок представляет собой прямоугольник.
Интерфейсы, посредством которых блок взаимодействует с другими блоками
или с внешней по отношению к моделируемой системе средой,
представляются стрелками, входящими в блок или выходящими из него.
Входящие стрелки показывают, какие условия должны быть одновременно
выполнены, чтобы функция, описываемая блоком, осуществилась.

Средства IDEF0 облегчают передачу информации от одного
участника разработки модели (отдельного разработчика или рабочей группы)
к другому. К числу таких средств относятся:
\begin{itemize}
\item диаграммы, основанные на простой графике блоков и стрелок, легко
читаемые и понимаемые;
\item метки на естественном языке для описания блоков и стрелок, а также
глоссарий и сопроводительный текст для уточнения смысла
элементов диаграммы;
\item последовательная декомпозиция диаграмм, строящаяся по
иерархическому принципу, при котором на верхнем уровне
отображаются основные функции, а затем происходит их
детализация и уточнение;
\item древовидные схемы иерархии диаграмм и блоков, обеспечивающие
обозримость модели в целом и входящих в нее деталей.
\end{itemize}

Разработка моделей IDEF0 требует соблюдения
ряда строгих формальных правил, обеспечивающих преимущества
методологии в отношении однозначности, точности и целостности сложных
многоуровневых моделей.

Все стадии и этапы разработки и корректировки
модели должны строго, формально документироваться с тем, чтобы при ее
эксплуатации не возникало вопросов, связанных с неполнотой или
некорректностью документации.

\textbf{DFD} --- общепринятое сокращение от англ. Data Flow Diagrams ---
диаграммы потоков данных. Так называется методология графического
структурного анализа, описывающая внешние по отношению к системе
источники и адресаты данных, логические функции, потоки данных и
хранилища данных, к которым осуществляется доступ.

Диаграмма потоков данных является одним из
основных инструментов структурного анализа и проектирования
информационных систем, существовавших до широкого распространения
UML.
Несмотря на имеющее место в современных условиях смещение
акцентов от структурного к объектно-ориентированному подходу в анализе и
проектировании систем, <<старинные>> структурные нотации по-прежнему
широко и эффективно используются как в бизнес-анализе, так и в анализе
информационных систем.

Информационная система принимает извне потоки данных. Для
обозначения элементов среды функционирования системы используется
понятие внешней сущности. Внутри системы существуют процессы
преобразования информации, порождающие новые потоки данных.
Потоки данных могут поступать на вход к другим процессам, помещаться (и
извлекаться) в накопители данных, передаваться к внешним сущностям.
Модель DFD, как и большинство других структурных моделей, является
иерархической. Каждый процесс может быть подвергнут
декомпозиции, то есть разбиению на структурные составляющие, отношения
между которыми в той же нотации могут быть показаны на отдельной
диаграмме. Когда достигнута требуемая глубина декомпозиции, процесс
нижнего уровня сопровождается текстовым описанием.

\textbf{IDEF3} является стандартом документирования технологических
процессов, происходящих на предприятии, и предоставляет инструментарий
для наглядного исследования и моделирования их сценариев.
\textbf{Сценарием} называется описание последовательности изменений свойств
объекта, в рамках рассматриваемого процесса (например, описание
последовательности этапов обработки детали в цеху и изменение её свойств
после прохождения каждого этапа).
Исполнение каждого сценария сопровождается соответствующим документооборотом,
который состоит из двух основных потоков: документов, определяющих структуру и
последовательность процесса (технологических указаний, описаний
стандартов и т.д.), и документов, отображающих ход его выполнения
(результатов тестов и экспертиз, отчетов о браке, и т.~д.).
Для эффективного управления любым процессом необходимо иметь детальное
представление об его сценарии и структуре сопутствующего документооборота.

Средства документирования и моделирования IDEF3 позволяют выполнять следующие
задачи:
\begin{itemize}
\item документировать имеющиеся данные о технологии процесса,
выявленные, скажем, в процессе опроса компетентных сотрудников,
ответственных за организацию рассматриваемого процесса;
\item определять и анализировать точки влияния потоков сопутствующего
документооборота на сценарий технологических процессов;
\item определять ситуации, в которых требуется принятие решения,
влияющего на жизненный цикл процесса, например, изменение
конструктивных, технологических или эксплуатационных свойств конечного
продукта;
\item содействовать принятию оптимальных решений при реорганизации
технологических процессов.
\end{itemize}

Существуют два типа диаграмм в стандарте IDEF3, представляющие
описание одного и того же сценария технологического процесса в разных
ракурсах. Диаграммы, относящиеся к первому типу, называются
диаграммами описания последовательности этапов процесса
(Process Flow Description Diagrams, PFDD), а ко второму ---
диаграммами состояния объекта и его трансформаций
(Object State Transition Network, OSTN).
С помощью диаграмм PFDD документируется последовательность и
описание стадий обработки детали в рамках исследуемого технологического
процесса.
Диаграммы OSTN используются для иллюстрации трансформаций детали,
которые происходят на каждой стадии обработки.

\textbf{IDEF1X} является методом для разработки реляционных баз данных и
использует условный синтаксис, специально разработанный для удобного
построения концептуальной схемы.

Будучи статическим методом разработки, IDEF1X изначально не
предназначен для динамического анализа по принципу "AS IS", тем не менее,
он иногда применяется в этом качестве, как альтернатива методу IDEF1.

Использование метода IDEF1X наиболее целесообразно для построения
логической структуры базы данных после того, как все информационные
ресурсы исследованы (скажем с помощью метода IDEF1) и решение о
внедрении реляционной базы данных, как части корпоративной
информационной системы, было принято.
Однако не стоит забывать, что
средства моделирования IDEF1X специально разработаны для построения
реляционных информационных систем, и если существует необходимость
проектирования другой системы, скажем объектно-ориентированной, то
лучше избрать другие методы моделирования.

Существует несколько очевидных причин, по которым IDEF1X не
следует применять в случае построения нереляционных систем. Во-первых,
IDEF1X требует от проектировщика определить ключевые атрибуты, для того
чтобы отличить одну сущность от другой, в то время как объектно-ориентированные
системы не требуют задания ключевых ключей, в целях идентифицирования объектов.
Во-вторых, в тех случаях, когда более чем один атрибут является однозначно
идентифицирующим сущность, проектировщик должен определить один из
этих атрибутов первичным ключом, а все остальные вторичными.
Таким образом, построенная проектировщиком
IDEF1x-модель и переданная для окончательной реализации программисту
является некорректной для применения методов объектно-ориентированной
реализации, и предназначена для построения реляционной системы.

Отношения многие ко многим обычно используются на начальной стадии
разработки диаграммы, например, в диаграмме зависимости сущностей и
отображаются в IDEF1X в виде сплошной линии с точками на обоих концах. Так
как отношения многие ко многим могут скрыть другие бизнес правила или
ограничения, они должны быть полностью исследованы на одном из этапов
моделирования. Например, иногда отношение многие ко многим на ранних
стадиях моделирования идентифицируется неправильно, на самом деле
представляя два или несколько случаев отношений один-ко-многим между
связанными сущностями. Или, в случае необходимости хранения
дополнительных сведений о связи многие-ко-многим, например, даты или
комментария, такая связь должна быть заменена дополнительной сущностью,
содержащей эти сведения. При моделировании необходимо быть увереным в том,
что все отношения многие ко многим будут подробно обсуждены на более
поздних стадиях моделирования для обеспечения правильного моделирования
отношений.

Выбор первичного ключа для сущности является очень важным шагом,
и требует большого внимания. В качестве первичных ключей могут быть
использованы несколько атрибутов или групп атрибутов. Атрибуты, которые
могут быть выбраны первичными ключами, называются кандидатами в
ключевые атрибуты (потенциальные атрибуты). Кандидаты в ключи должны
уникально идентифицировать каждую запись сущности. В соответствии с
этим, ни одна из частей ключа не может быть NULL, не заполненной или
отсутствующей.

\pagebreak
\subsection{Выбор инструментальной платформы}

В качестве инструментальной платформы проектирования автоматизированной
системы можно использовать как полноценные CASE-средства, такие как ERWin или
IBM Rational Rose, так и обыкновенные векторные графические редакторы с поддержкой
требуемых нотаций проектирования.

В данной работе для проектирования автоматизированной системы будет
использоваться средство для построения диаграмм Dia~\cite{website_dia}. Выбор обусловлен
следующими причинами:
\begin{itemize}
  \item Dia является открытым и бесплатным программным обеспечением;
  \item Dia поддерживает IDEF-нотации, а также UML;
  \item Dia является кросплатформенным ПО;
  \item Dia поддерживает возможность экспорта построенных диаграм в
    издательскую систему LaTeX.
\end{itemize}

Для проектирования визуального дизайна приложения приложения могут использоваться
различные графические редакторы: как растровые, так и векторные.
В даннной работе макеты экранов приложения будут разрабатываться с помощью
свободного и бесплатного программного обеспечения:
InkScape~\cite{website_inkscape} и GIMP~\cite{website_gimp}.