\section{РАЗРАБОТКА МАТЕМАТИЧЕСКОЙ МОДЕЛИ}

Для описания задачи введем следующие обозначения:

\begin{itemize}
  \item \( X_{11} \) --- количество колб, выпускаемых в первую неделю;
  \item \( X_{12} \) --- количество колб, выпускаемых во вторую неделю;
  \item \( X_{13} \) --- количество колб, выпускаемых в третью неделю;
  \item \( X_{14} \) --- количество колб, выпускаемых в четвертую неделю;

  \item \( X_{21} \) --- количество чаш,  выпускаемых в первую неделю;
  \item \( X_{22} \) --- количество чаш,  выпускаемых во вторую неделю;
  \item \( X_{23} \) --- количество чаш,  выпускаемых в третью неделю;
  \item \( X_{24} \) --- количество чаш,  выпускаемых в четвертую неделю.
\end{itemize}

В этом случае ограничения на общий объем производства лабораторных колб и чаш принимают вид:
\begin{align*}
  X_{11} + X_{12} + X_{13} + X_{14} &= 5000, 
  & X_{21} + X_{22} + X_{23} + X_{24} &= 8000.
\end{align*}
Ограничения на выпуск пластин по неделям:
\begin{align*}
  X_{14} &\ge 2000, & X_{23} &\ge 1000.
\end{align*}
Ограничения на время работы оборудования:
\begin{align*}
    20X_{11} + 5X_{21}  &\le 60000,
  & 15X_{11} + 5X_{21}  &\le 60000, 
  & 6X_{11}  + 3X_{21}  &\le 24000, \\  
    20X_{12} + 5X_{22}    &\le 48000,
  & 15X_{12} + 5X_{22}    &\le 60000, 
  & 6X_{12}  + 3X_{22}    &\le 24000, \\
    20X_{13} + 5X_{23}  &\le 72000,
  & 15X_{13} + 5X_{23}  &\le 60000, 
  & 6X_{13}  + 3X_{23}  &\le 72000, \\
    20X_{14} + 5X_{24}    &\le 60000,
  & 15X_{14} + 5X_{24}    &\le 60000, 
  & 6X_{14}  + 3X_{24}    &\le 72000. 
\end{align*}
Здесь, например, первое ограничение для первой недели означает, что 
печь для нагрева может использоваться не более
60000 минут.

Целевая функция для данной задачи принимает следующий вид:
\begin{small}
  \begin{align*}
    E = \max( 
      & 60000 - 20X_{11} - 5X_{21}, \: 60000 - 15X_{11} - 5X_{21}, \: 24000 - 6X_{11} - 3X_{21}, \\
      & 48000 - 20X_{12} - 5X_{22}, \: 60000 - 15X_{12} - 5X_{22}, \: 24000 - 6X_{12} - 3X_{22}, \\
      & 72000 - 20X_{13} - 5X_{23}, \: 60000 - 15X_{13} - 5X_{23}, \: 24000 - 6X_{13} - 3X_{23}, \\  
      & 60000 - 20X_{14} - 5X_{24}, \: 60000 - 15X_{14} - 5X_{24}, \: 72000 - 6X_{14} - 3X_{24} ) \rightarrow \min.
  \end{align*}
\end{small}

\vspace{-7mm}

Следует обратить внимание на то, что общие затраты времени на выпуск
продукции являются фиксированными
(\( 20 \cdot 5000 + 5 \cdot 8000 = 140~000 \) минут на нагрев,
 \( 15 \cdot 5000 + 5 \cdot 8000 = 115~000 \) минут на очистку,
 \( 6  \cdot 5000 + 3 \cdot 8000 = 54~000  \) минут на формование).

Смысл целевой функции заключается в том, чтобы распределение времени работы
оборудования по неделям сделать как можно более равномерным. 
Можно показать, что таким свойством обладает такое распределение, 
у которого наибольшее значение времени простоя некоторого вида
оборудования за неделю минимально.