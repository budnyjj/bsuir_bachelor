\section{РЕЗУЛЬТАТЫ РЕШЕНИЯ}

Для решения задачи воспользуемся табличным процессором.
Результаты решения задачи находятся в приложении А.

Проведем краткий анализ результатов решения.
Переменные модели приняли следующие значения:
\begin{small}
  \begin{align*}
    & X_{11} = 666, \: X_{12} = 111,  \: X_{13} = 2111, \: X_{14} = 2112, \\
    & X_{21} = 22,  \: X_{22} = 1687, \: X_{23} = 1687, \: X_{24} = 3145.
  \end{align*}
\end{small}

\vspace{-7mm}

Таким образом, оптимальный план производства предусматривает:
\begin{itemize}
	\item в течение первой недели следует выпустить 
	  666 лабораторных колб и 22 лабораторные чаши;
	\item в течение второй недели следует выпустить 
	  111 лабораторных колб и 1687 лабораторные чаши;
	\item в течение третьей недели следует выпустить 
	  2111 лабораторных колб и 3146 лабораторные чаши;
	\item в течение четвертой недели следует выпустить 
	  2112 лабораторных колб и 3145 лабораторные чаши.
\end{itemize}

Значение целевой функции для данного решения составляет 49900 минут и
соответствует простою оборудования в течение первой и второй недель.
При этом простои печи для нагрева при выбранном плане производства составят:
\begin{itemize}
	\item на первой неделе --- 46570 минут;
	\item на второй неделе --- 37345 минут;
	\item на третьей неделе --- 14050 минут;
	\item на четвертой неделе --- 2035 минут.
\end{itemize}

Простои оборудования для установки и очистки:
\begin{itemize}
	\item на первой неделе --- 49900 минут;
	\item на второй неделе --- 49900 минут;
	\item на третьей неделе --- 12605 минут;
	\item на четвертой неделе --- 12595 минут;
\end{itemize}

Простои формовочного станка:
\begin{itemize}
	\item на первой неделе --- 19938 минут;
	\item на второй неделе --- 18237 минут;
	\item на третьей неделе --- 49896 минут;
	\item на четвертой неделе --- 49893 минут.
\end{itemize}

\newpage