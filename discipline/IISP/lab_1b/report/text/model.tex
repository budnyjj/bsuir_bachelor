\section{РАЗРАБОТКА МАТЕМАТИЧЕСКОЙ МОДЕЛИ}

Для описания задачи введем следующие обозначения:

\begin{itemize}
\item \( X_{11} \) --- количество пластмассовых пластин для холодильников,
  выпускаемых в первую неделю;
\item \( X_{12} \) --- количество пластмассовых пластин для холодильников,
  выпускаемых во вторую неделю;
\item \( X_{13} \) --- количество пластмассовых пластин для холодильников,
  выпускаемых в третью неделю;
\item \( X_{14} \) --- количество пластмассовых пластин для холодильников,
  выпускаемых в четвертую неделю;
\item \( X_{21} \) --- количество пластмассовых пластин для кухонной мебели,
  выпускаемых в первую неделю;
\item \( X_{22} \) --- количество пластмассовых пластин для кухонной мебели,
  выпускаемых во вторую неделю;
\item \( X_{23} \) --- количество пластмассовых пластин для кухонной мебели,
  выпускаемых в третью неделю;
\item \( X_{24} \) --- количество пластмассовых пластин для кухонной мебели,
  выпускаемых в четвертую неделю.
\end{itemize}

В этом случае ограничения на общий объем производства пластмассовых пластин
принимают вид:
\begin{align*}
  X_{11} + X_{12} + X_{13} + X_{14} &= 3000, 
  & X_{21} + X_{22} + X_{23} + X_{24} &= 2000.
\end{align*}
Ограничения на выпуск пластин по неделям:
\begin{align*}
  X_{22} &\ge 500, 
  & X_{13} &\ge 200, 
  & X_{14} &\ge 200.
\end{align*}
Ограничения на время работы оборудования:
\begin{align*}
  5X_{11} + 2X_{21} &\le 15000,
  & 20X_{11} + 5X_{21} &\le 24000, 
  & 12X_{11} + 10X_{21} &\le 30000, \\  
  5X_{12} + 2X_{22} &\le 15000,
  & 20X_{12} + 5X_{22} &\le 36000, 
  & 12X_{12} + 10X_{22} &\le 30000, \\
  5X_{13} + 2X_{23} &\le 30000,
  & 20X_{13} + 5X_{23} &\le 36000, 
  & 12X_{13} + 10X_{23} &\le 30000, \\
  5X_{14} + 2X_{24} &\le 30000,
  & 20X_{14} + 5X_{24} &\le 36000, 
  & 12X_{14} + 10X_{24} &\le 30000. 
\end{align*}
Здесь, например, первое ограничение для первой недели означает, что 
станки для вырезки плит на первой неделе могут использоваться не более
15000 минут.

Целевая функция для данной задачи принимает следующий вид:
\begin{small}
  \begin{align*}
    E = \max( 
  & 15000 - 5X_{11} - 2X_{21}, \: 24000 - 20X_{11} - 5X_{21}, \: 30000 - 12X_{11} - 10X_{21}, \\
  & 15000 - 5X_{12} - 2X_{22}, \: 36000 - 20X_{12} - 5X_{22}, \: 30000 - 12X_{12} - 10X_{22}, \\
  & 30000 - 5X_{13} - 2X_{23}, \: 36000 - 20X_{13} - 5X_{23}, \: 30000 - 12X_{13} - 10X_{23}, \\  
  & 30000 - 5X_{14} - 2X_{24}, \: 36000 - 20X_{14} - 5X_{24}, \: 30000 - 12X_{14} - 10X_{24} ) \rightarrow \min.
  \end{align*}
\end{small}

\vspace{-7mm}

Следует обратить внимание на то, что общие затраты времени на выпуск
продукции являются фиксированными
(\( 5 \cdot 3000 + 2 \cdot 2000 = 19000 \) минут на вырезку,
\( 20 \cdot 3000 + 5 \cdot 2000 = 70000 \) минут на шлифовку,
\( 12 \cdot 3000 + 10 \cdot 2000 = 46000 \) минут на нанесение покрытия).

Смысл целевой функции заключается в том, чтобы распределение времени работы
оборудования по неделям сделать как можно более равномерным. 
Можно показать, что таким свойством обладает такое распределение, 
у которого наибольшее значение времени простоя некоторого вида
оборудования за неделю минимально.