\section{РЕЗУЛЬТАТЫ РЕШЕНИЯ}

Для решения задачи воспользуемся табличным процессором.
Результаты решения задачи находятся в приложении А.

Проведем краткий анализ результатов решения.
Переменные модели приняли следующие значения:
\begin{small}
  \begin{align*}
    & X_{11} = 550, \: X_{12} = 550, \: X_{13} = 950, \: X_{14} = 950, \\
    & X_{21} = 400, \: X_{22} = 800, \: X_{23} = 400, \: X_{24} = 400.
  \end{align*}
\end{small}

\vspace{-7mm}

Таким образом, оптимальный план производства предусматривает:
\begin{itemize}
\item в течение первой недели следует выпустить 
  550 пластин для холодильников и 400 --- для кухонной мебели;
\item в течение второй недели следует выпустить 
  550 пластин для холодильников и 800 --- для кухонной мебели;
\item в течение третьей недели следует выпустить 
  950 пластин для холодильников и 400 --- для кухонной мебели;
\item в течение четвертой недели следует выпустить 
  950 пластин для холодильников и 400 --- для кухонной мебели.
\end{itemize}

Значение целевой функции для данного решения составляет 24450 минут и
соответствует простою станков для вырезки плит в течение третьей и четвертой недели.
При этом простои станков для вырезки плит при выбранном плане производства составят:
\begin{itemize}
\item на первой неделе --- 24450 минут;
\item на второй неделе --- 24450 минут;
\item на третьей неделе --- 10650 минут;
\item на четвертой неделе --- 11450 минут.
\end{itemize}

Простои станков для шлифовки:
\begin{itemize}
\item на первой неделе --- 11000 минут;
\item на второй неделе --- 21000 минут;
\item на третьей неделе --- 15000 минут;
\item на четвертой неделе --- 15000 минут;
\end{itemize}

Простои установок для нанесения покрытия:
\begin{itemize}
\item на первой неделе --- 19400 минут;
\item на второй неделе --- 15400 минут;
\item на третьей неделе --- 14600 минут;
\item на четвертой неделе --- 14600 минут.
\end{itemize}

\newpage