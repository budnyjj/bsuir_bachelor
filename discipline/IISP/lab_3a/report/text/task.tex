\section{ПОСТАНОВКА ЗАДАЧИ}

Имеются данные о спросе на два некоторых товара за пятнадцать периодов,
приведенные в таблице~\ref{tbl:source_data}.

\begin{table} [h!]
  \caption{
    Исходные данные
  }\label{tbl:source_data}
    \begin{tabular}{| m{2.5cm} | c | c |}
      \hline
      № периода & Фактический спрос (товар 1) & Фактический спрос (товар 2) \\ \hline

      1 & 28 & 32 \\ \hline
      2 & 31 & 33 \\ \hline
      3 & 34 & 29 \\ \hline
      4 & 27 & 26 \\ \hline
      5 & 35 & 24 \\ \hline
      
      6 & 32  & 25 \\ \hline
      7 & 36  & 22 \\ \hline
      8 & 29  & 21 \\ \hline
      9 & 32  & 19 \\ \hline
      10 & 29 & 18 \\ \hline
      
      11 & 35 & 17 \\ \hline
      12 & 34 & 17 \\ \hline
      13 & 29 & 15 \\ \hline
      14 & 25 & 16 \\ \hline
      15 & 31 & 14 \\ \hline
    \end{tabular}
\end{table}

Требуется получить прогноз спроса за 16-й период для каждого из товаров
одним из следующих методов пронозирования:
\begin{itemize}
  \item метод скользящего среднего;
  \item метод экспоненциального сглаживания;
  \item метод линейной регрессии.
\end{itemize}