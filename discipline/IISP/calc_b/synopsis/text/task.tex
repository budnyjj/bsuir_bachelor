\section[Постановка задачи]{ПОСТАНОВКА ЗАДАЧИ}

C помощью системы компоновки данных 1С: Предприятие требуется настроить систему
учета данных в контексте выбранной тематики работы.

Разработанная система должна удовлетворять следующим требованиям:
\begin{itemize}
\item должна состоять не менее чем из
  двух справочников,
  одного документа,
  одного регистра накопления,
  двух отчетов и
  одной диаграммы, созданных с использованием системы компоновки данных (автоматически);
\item должна использовать автоматический ввод данных из справочников;
\item должна выполнять вычисления в документах (не менее двух различных вычислений).
\end{itemize}

В системе должны быть реализованы запросы, отражающие следующие возможности СКД:
\begin{itemize}
\item простой запрос, предусматривающий выборку данных
  из основной и табличной части документа;
\item простой запрос, предусматривающий выборку данных
  из основной и табличной части справочника;
\item запрос, предусматривающий выборку нескольких записей с
  максимальным значением некоторого реквизита (конструкция ПЕРВЫЕ N);
\item запрос с параметром;
\item запрос с агрегатной функцией СУММА;
\item запрос с агрегатной функцией КОЛИЧЕСТВО;
\item запрос с условием отбора (конструкция ГДЕ);
\item запрос с условием отбора (конструкция ИМЕЮЩИЕ);
\item запрос с вычислением итогов (конструкция ИТОГИ);
\item объединение запросов (конструкции ОБЪЕДИНИТЬ, СОЕДИНЕНИЕ,
  ПОЛНОЕ СОЕДИНЕНИЕ, ЛЕВОЕ СОЕДИНЕНИЕ, ПРАВОЕ СОЕДИНЕНИЕ) --- не менее двух запросов;
\item запрос, предусматривающий выборку данных из регистра накопления.
\end{itemize}

Кроме этого, система должна содержать отчет, построенный одним из запросов.