\section{ПОСТАНОВКА ЗАДАЧИ}

Имеются данные о спросе на два некоторых товара за пятнадцать периодов,
приведенные в таблице~\ref{tbl:source_data}.

\begin{table} [h!]
  \caption{
    Исходные данные
  }\label{tbl:source_data}
    \begin{tabular}{| m{5cm} | c | c |}
      \hline
      \multirow{2}{*}{№ периода}
      & \multicolumn{2}{c|}{Фактический спрос} \\ \cline{2-3}

      & \parbox{5.1cm}{
          \centering Товар 1
        }
      & \parbox{5.1cm}{
        \centering Товар 2
        } \\ \hline

      1 & 25 & 24 \\ \hline
      2 & 31 & 22 \\ \hline
      3 & 34 & 25 \\ \hline
      4 & 36 & 22 \\ \hline
      5 & 29 & 27 \\ \hline

      6 & 32  & 23 \\ \hline
      7 & 29  & 24 \\ \hline
      8 & 35  & 27 \\ \hline
      9 & 34  & 29 \\ \hline
      10 & 29 & 31 \\ \hline

      11 & 28 & 29 \\ \hline
      12 & 31 & 36 \\ \hline
      13 & 34 & 35 \\ \hline
      14 & 27 & 32 \\ \hline
      15 & 35 & 31 \\ \hline
    \end{tabular}
\end{table}

Требуется получить прогноз спроса на 16-й период для каждого из товаров
одним из следующих методов прогнозирования:
\begin{itemize}
  \item метод скользящего среднего;
  \item метод экспоненциального сглаживания;
  \item метод линейной регрессии.
\end{itemize}