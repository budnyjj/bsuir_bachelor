\section{ХОД РАБОТЫ}

% choice
\subsection{Выбор метода прогнозирования для каждого из товаров}

Существуют некоторые рекомендации по выбору метода прогнозирования в
зависимости от исходных значений спроса за прошедший период.

Если величина, для которой требуется найти прогноз, имеет достаточно
чёткую тенденцию к росту или, наоборот, к снижению, то для такой величины
обычно оказывается подходящей линейная регрессионная модель. При отсутствии такой
тенденции (то есть если исследуемая величина в некоторые периоды росла,
а в некоторые --- снижалась) более подходящими могут оказаться методы
скользящего среднего или экспоненциального сглаживания.

Построим график зависимости значения фактического спроса за период
для каждого из товаров. Из рисунка~\ref{fig:initial} видно, что
фактический спрос на второй товар имеет чёткую тенденцию к росту.
Таким образом, для прогнозирования спроса цены на второй товар
имеет смысл использовать метод линейной регрессии.

\begin{figure}[h!]
  \centering
  \includegraphics[width=150mm]{pic/initial}
  \caption{Исходные данные}
  \label{fig:initial}
\end{figure}

Значения фактического спроса на первый товар не имеют подобной тенденции.
Следовательно, для прогнозирования фактического спроса на этот товар следует
использовать метод скользящего среднего или экспоненциального сглаживания.


\subsection{Товар 1. Прогноз методом скользящего среднего}

% theory
Краткое описание метода: пусть имеются значения некоторой
величины за $n$ периодов $y_1, y_2, \dots, y_n$.
Прогноз величины $\hat{y}_{n+1}$ на $(n+1)\text{-ый}$ период определяется как
среднее значение за $k$ последних периодов.

Используя этот метод с периодом $k=2$ найдём прогноз на 16-ый период
для первого товара. Для этого воспользуемся инструментом
<<Скользящее среднее>> MS Excel c параметрами:
\begin{itemize}
  \item входной интервал --- ячейки с фактическими данными;
  \item интервал --- 2;
  \item выходной интервал --- ячейка, с которой начнется вывод результата~(со смещением).
\end{itemize}


% k = 2
Рассмотрим, например, как вычислены прогнозируемые значения спроса на 3-й и 4-й период:
\begin{align*}
  \hat{y}_3 &= \dfrac{y_1 + y_2}{2} = \dfrac{25 + 31}{2} = 28{,}0, \\
  \hat{y}_4 &= \dfrac{y_2 + y_3}{2} = \dfrac{31 + 34}{2} = 32{,}5.
\end{align*}

Также найдём прогнозируемое значение спроса на 16-й период:
\begin{equation*}
  \hat{y}_{16} = \dfrac{y_{14} + y_{15}}{2} = \dfrac{27 + 35}{2} = 31{,}0.
\end{equation*}

Для оценки точности этого алгоритма найдём среднее абсолютное отклонение
(\textit{Mean Absolute Deviation, MAD}) фактических данных от прогнозируемых по формуле:
\[
  MAD_{k=2} = \dfrac{\sum_{i=1}^{m} |\hat{y}_i - y_i| }{m} = \dfrac{|\hat{y}_3 - y_3| + |\hat{y}_4 - y_4| + \dots + |\hat{y}_{15} - y_{15}|}{13} = 3{,}85.
\]

Повторно воспользуемся методом скользящего среднего с периодом $k=3$.
Стоит отметить, что прогноз на определённый период в данном случае строится
по трём предыдущим значениям, поэтому первая прогнозируемая величина будет
прогнозом спроса на товар за четвёртый период.

% k = 3
Вычислим прогнозируемые значения спроса на 3-й, 4-й и 16-й период соответственно:
\begin{align*}
  \hat{y}_4 &= \dfrac{y_1 + y_2 + y_3}{3} = \dfrac{25 + 31 + 34}{3} = 30{,}0, \\
  \hat{y}_5 &= \dfrac{y_2 + y_3 + y_4}{3} = \dfrac{31 + 34 + 36}{3} = 33{,}7, \\
  \hat{y}_{16} &= \dfrac{y_{13} + y_{14} + y_{15}}{3} = \dfrac{34 + 27 + 35}{3} = 32{,}0.
\end{align*}

Вычислим \textit{MAD} для данного случая:
\[
  MAD_{k=3} = \dfrac{|\hat{y}_4 - y_4| + |\hat{y}_5 - y_5| + \dots + |\hat{y}_{15} - y_{15}|}{12} = 3{,}7.
\]

Для приведенного примера наиболее точным является метод скользяшего среднего
с периодом $k = 3$, так как значение \textit{MAD} для него меньше.


\subsection{Товар 1. Прогноз методом экспоненциального сглаживания}

% theory
Краткое описание метода: пусть имеются значения некоторой
величины за $n$ периодов $y_1, y_2, \dots, y_n$.
Прогноз величины $\hat{y}_{n+1}$ на $(n+1)\text{-ый}$ период определяется по формуле:
\[
  \hat{y}_{n+1} = (1 - \alpha) y_n + \alpha \hat{y}_n,
\]
где \hspace{2mm} $y_n$ --- фактическое значение за предыдущий ($n$-ый) период, \par
                 $\hat{y}_n$ --- прогноз на предыдущий ($n$-ый) период, \par
                 $\alpha$ --- фактор затухания, обычно принимается равным от $0{,}05$ до $0{,}3$.

Используя этот метод с фактором затухания $\alpha = 0{,}05$,
найдём прогнозируемое значения спроса на 16-ый период
для первого товара. Для этого воспользуемся инструментом
<<Экспоненциальное сглаживание>> MS Excel cо следующими параметрами:
\begin{itemize}
  \item входной интервал --- ячейки с фактическими данными;
  \item фактор затухания $\alpha$ --- $0{,}05$;
  \item выходной интервал --- ячейка, с которой начнется вывод результата.
\end{itemize}

% alpha = 0,05
Рассмотрим, например, как вычислены прогнозируемые значения спроса на 2-й и 3-й период:
\begin{align*}
  \hat{y}_2 &= y_1 = 25{,}0, \\
  \hat{y}_3 &= (1 - \alpha) y_2 + \alpha \hat{y}_2 = 0{,}95 \cdot 34{,}0 + 0{,}05 \cdot 25{,}0 = 30{,}70.
\end{align*}

Также найдём прогнозируемое значение спроса на 16-й период:
\begin{equation*}
  \hat{y}_{16} = (1 - \alpha) y_{15} + \alpha \hat{y}_{15}  = 0{,}95 \cdot 35{,}0 + 0{,}05 \cdot 27{,}34 = 34{,}62.
\end{equation*}

Вычислим значение \textit{MAD} для данного примера:
\[
  MAD_{\alpha = 0{,}05} = \dfrac{|\hat{y}_2 - y_2| + |\hat{y}_3 - y_3| + \dots + |\hat{y}_{15} - y_{15}|}{14} = 4{,}09.
\]

% alpha = 0,10
Повторно воспользуемся методом экспоненциального сглаживания с фактором
затухания $\alpha=0{,}10$. Вычисленим значения прогнозов на 2-й, 3-й и 16-й период соответственно:
\begin{align*}
  \hat{y}_2 &= y_1 = 25{,}0, \\
  \hat{y}_3 &= (1 - \alpha) y_2 + \alpha \hat{y}_2 = 0{,}9 \cdot 34{,}0 + 0{,}1 \cdot 25{,}0 = 30{,}40, \\
  \hat{y}_{16} &= (1 - \alpha) y_{15} + \alpha \hat{y}_{15}  = 0{,}9 \cdot 35{,}0 + 0{,}1 \cdot 27{,}34 = 34{,}27.
\end{align*}

Вычислим значение \textit{MAD} для данного примера:
\[
  MAD_{\alpha = 0{,}10} = \dfrac{|\hat{y}_2 - y_2| + |\hat{y}_3 - y_3| + \dots + |\hat{y}_{15} - y_{15}|}{14} = 4{,}05.
\]

% alpha = 0,15
Повторно воспользуемся методом экспоненциального сглаживания с фактором
затухания $\alpha=0{,}15$. Вычисление прогноза на 2-й, 3-й и 16-й период соответственно:
\begin{align*}
  \hat{y}_2 &= y_1 = 25{,}0, \\
  \hat{y}_3 &= (1 - \alpha) y_2 + \alpha \hat{y}_2 = 0{,}85 \cdot 34{,}0 + 0{,}15 \cdot 25{,}0 = 30{,}1, \\
  \hat{y}_{16} &= (1 - \alpha) y_{15} + \alpha \hat{y}_{15}  = 0{,}85 \cdot 35{,}0 + 0{,}15 \cdot 27{,}34 = 33{,}95.
\end{align*}

Вычислим значение \textit{MAD} для данного примера:
\[
  MAD_{\alpha = 0{,}15} = \dfrac{|\hat{y}_2 - y_2| + |\hat{y}_3 - y_3| + \dots + |\hat{y}_{15} - y_{15}|}{14} = 4{,}00.
\]

Сравнивая значения величин $MAD_{\alpha = 0{,}05}, MAD_{\alpha = 0{,}10}$ и $MAD_{\alpha = 0{,}15}$,
можно сказать, что метод экспоненциального сгаживания с параметром $alpha = 0{,}15$
является наиболее точным, так как значение величины $MAD$ для него является наименьшим.

% total results
Сравнивания общие результаты прогнозирования, можно сказать, что наиболее точным
оказался метод скользящего среднего с периодом $k = 3$, так как величина
$ MAD_{k=3} = 3{,}67$, полученная для этого метода, является минимальной.

Результаты прогнозирования величины спроса для товара 1 приведены в приложении~А.



\subsection{Товар 2. Прогноз методом регрессионного анализа}

% theory
Краткое описание метода прогнозирования на базе линейного регрессионного анализа:
пусть имеются значения некоторой величины за $n$ периодов $y_1, y_2, \dots, y_n$.
Прогноз величины $\hat{y}_{i}$ на $i$-ый период определяется по формуле:
\[
  y_{i} = a_0 + a_1 \cdot i,
\]
где \hspace{2mm} $a_0, a_1$ --- коэффициенты, определяемые методом наименьших квадратов, \par
                 $i$ --- номер периода.

Суть метода наименьших квадратов (МНК) заключается в нахождении
коэффициентов $a_0$ и $a_1$ линейной зависимости, при которых
значение функции $F(a_0, a_1)$ будет минимальным:
\[
  F(a_0, a_1) = \sum_{i=1}^{n} (y_i - (a_0 x_i + a_1))^2 \rightarrow min.
\]

Используя метод регрессионного анализа, найдём прогноз на 16-ый период
для второго товара. Для этого воспользуемся инструментом
<<Регрессия>> MS Excel c параметрами:
\begin{itemize}
  \item входной интервал $Y$ --- ячейки с фактическими данными;
  \item входной интервал $X$ --- ячейки с номерами периодов;
  \item выходной интервал --- ячейка, с которой начнется вывод результата.
\end{itemize}

Прежде чем использовать результаты регрессионного анализа,
следует обратить внимание на величину <<Значимость $F$>>. В случае, если
эта величина принимает значение меньше $0{,}05$, можно полагать, что
изменение значения $y_i$ во времени может быть описано линейной моделью, то
есть при подстановке в построенную модель известных значений $x$ (номеров периодов)
будут получены модельные значения $\hat{y}_i$, достаточно близкие к фактическим
значениям $y_i$. Если же величина <<Значимость $F$>> больше $0{,}05$, то
это означает, что изменение величины $y_i$ во времени невозможно с достаточной
точностью описать линейной моделью.

В нашем случае величина <<Значимость $F$>> гораздо ниже $0,05$, что позволяет
нам использовать коэффициенты, полученные в результате регрессии с достаточной точностью:
\begin{align*}
  a_0 = 20{,}89, \hspace{2cm} a_1 = 0{,}86.
\end{align*}

Зная значения коэффициентов $a_0$ и $a_1$, можно показать, например, как
вычислен прогноз на 1-й и 2-й период:
\begin{align*}
  \hat{y}_1 &= a_0 + a_1 \cdot 1 = 20{,}89 + 0{,}86 \cdot 1 = 21{,}75, \\
  \hat{y}_2 &= a_0 + a_1 \cdot 1 = 20{,}89 + 0{,}86 \cdot 2 = 22{,}61.
\end{align*}

Также найдём прогноз значения спроса на 16-й период:
\begin{equation*}
  \hat{y}_{16} = a_0 + a_1 \cdot 16 = 20{,}89 + 0{,}86 \cdot 16 = 34{,}71.
\end{equation*}

Для определения точности данного метода вычислим величину \textit{MAD}:
\[
  MAD = \dfrac{|\hat{y}_1 - y_1| + |\hat{y}_2 - y_2| + \dots + |\hat{y}_{15} - y_{15}|}{15} = 2{,}00
\]


\subsection{Товар 2. Прогноз методом скользящего среднего}

% k = 2
Для сравнения найдём прогноз спроса методом скользящего среднего
с периодом $k = 2$. Рассмотрим, например, как вычислен прогноз
на 3-й и 4-й период:
\begin{align*}
  \hat{y}_3 &= \dfrac{y_1 + y_2}{2} = \dfrac{24 + 22}{2} = 23{,}0, \\
  \hat{y}_4 &= \dfrac{y_2 + y_3}{2} = \dfrac{22 + 25}{2} = 23{,}5.
\end{align*}

Также найдём прогноз спроса на 16-й период:
\begin{equation*}
  \hat{y}_{16} = \dfrac{y_{14} + y_{15}}{2} = \dfrac{32 + 31}{2} = 31{,}5.
\end{equation*}

Вычисление \textit{MAD} для данного примера:
\[
  MAD_{k = 2} = \dfrac{|\hat{y}_3 - y_3| + |\hat{y}_4 - y_4| + \dots + |\hat{y}_{15} - y_{15}|}{13} = 2{,}69.
\]

Сравнивания величины $MAD_{k = 2} = 2{,}00$ и $MAD = 2{,}69$, можно сделать вывод, что
метод регрессионного анализа для данного примера дал более точный результат, так как
значение величины $MAD$ для него меньше.

Результаты прогнозирования величины спроса на товар 2 приведены в приложении~Б.
