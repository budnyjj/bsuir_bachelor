\section{ПОСТАНОВКА ЗАДАЧИ}

В работе предприятия используются некоторые детали.
Потребность предприятия в деталях представляет собой случайную величину,
распределенную по гауссовскому закону. В среднем потребность в
деталях составляет $80$ шт/день, стандартное отклонение --- $10$ шт/день.
Цена одной детали составляет $2$ ден.ед.
Затраты, связанные с хранением одной детали в течение года,
составляют $0{,}1$ ден.ед. Затраты, связанные с получением
одной партии деталей (не зависящие от размера партии),
составляют $25$ ден.ед. Срок выполнения заказа --- $6$ дней.
Потери от нехватки одной детали в течение года составляют $0{,}4$ ден.ед.

На предприятии предполагается заказывать очередную партию деталей
при снижении запаса до определенного (фиксированного) уровня.

Требуется составить план управления запасом деталей,
при котором общие затраты, связанные с запасом, будут минимальны.