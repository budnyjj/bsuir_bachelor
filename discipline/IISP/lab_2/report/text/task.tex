\section{ПОСТАНОВКА ЗАДАЧИ}

В работе предприятия используются некоторые детали.
Потребность предприятия в деталях представляет собой случайную величину,
распределенную по гауссовскому закону. В среднем потребность в
деталях составляет $80$ шт/день, стандартное отклонение --- $10$ шт/день.
Цена одной детали составляет $2$ ден.ед.
Затраты, связанные с хранением одной детали в течение года,
составляют $0,1$ ден.ед. Затраты, связанные с получением
одной партии деталей (не зависящие от размера партии),
составляют $25$ ден.ед. Срок выполнения заказа --- $6$ дней.
Потери от нехватки одной детали в течение года составляют $0,4$ ден.ед.

На предприятии предполагается заказывать очередную партию деталей
при снижении запаса до определенного (фиксированного) уровня.

Требуется составить план управления запасом деталей,
при котором общие затраты, связанные с запасом, будут минимальны.

% Разрабатывается план производства двух видов изделий из химически
% стойкой пластмассы: лабораторные колбы и чаши.
% Разработана производственная программа: за 4 недели предприятие
% должно выпустить 5000 лабораторных колб и 8000 чаш.

% Изготовление изделия включает три операции:
% нагрев материала, его очистку и формование изделия.
% Затраты времени на выполнение этих операций (в минутах)
% приведены в таблице~\ref{tbl:timers}. 
% Фонды времени работы оборудования приведены в таблице~\ref{tbl:time_available}.

% \begin{table} [h!]
%   \caption{Затраты времени на выполнение технологических операций}
%   \label{tbl:timers}
%   \begin{tabular}{| m{4.9cm} | c | c |}
    
%     \hline
%     \multirow{2}{*}{Операция} & \multicolumn{2}{c|}{Затраты времени на выпуск одного изделия, мин} \\

%     \cline{2-3} & \parbox{5cm}{\centering для колб} & \parbox{5cm}{\centering для чаш} \\ \hline

%     Нагрев материала & 20 & 5 \\ \hline

%     Очистка & 15 & 5 \\ \hline

%     Формование & 6 & 3 \\ \hline
      
%   \end{tabular}
% \end{table}

% \vspace{-4mm}

% \begin{table} [h!]
%   \caption{
%     Фонды времени работы оборудования
%   }\label{tbl:time_available}
%   \begin{tabular}{| m{5.8cm} | c | c | c | c |}
%     \hline
%     \multirow{2}{*}{}
%     & \multicolumn{4}{c|}{Фонд времени, часы (по неделям)} 
%     \\ \cline{2-5}

%     & \parbox{2.1cm}{\centering 1}
%     & \parbox{2.1cm}{\centering 2}
%     & \parbox{2.1cm}{\centering 3}
%     & \parbox{2.1cm}{\centering 4} \\
%     \hline

%     Печь для нагрева        & 1000 & 800 & 1200 & 1000 \\ \hline

%     Установка для очистки   & 1000 & 1000 & 1000 & 1000 \\ \hline

%     Формовочный станок      & 400 & 400 & 1200 & 1200 \\ \hline      

%   \end{tabular}
% \end{table}

% В течение третьей недели необходимо выпустить не менее 1000 чаш,
% а в течение четвертой недели --- не менее 2000 колб.

% Выполнить распределение производственной программы по неделям,
% используя критерий равномерной загрузки оборудования.