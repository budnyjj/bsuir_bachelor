\section{АНАЛИЗ ПОСТАВЛЕННОЙ ЗАДАЧИ}

Решаемая задача является \textit{вероятностной}, так как потребность в
деталях является случайной величиной.
Задачу можно считать \textit{однопродуктовой}, так как в процессе её решения
требуется составить план управления запасом деталей одного типа.

Для решения задачи требуется использовать \textit{уровневую} модель, то есть
заказывать очередную партию деталей при снижении запаса до определенного
(фиксированного) уровня.

Введём обозначения:
\begin{itemize}
  \item $V=80$ --- потребность предприятия в продукте, шт./день;
  \item $C=4$ --- цена продукта, ден. ед.;
  \item $\sigma=5$ --- стандартное отклонение потребности в деталях;
  \item $k=30$ --- затраты предприятия на получение партии продукта, ден.ед.;
  \item $S=0{,}2$ --- затраты предприятия на хранением единицы продукции ден.ед./год;
  \item $d=0{,}5$ --- потери от дефицита, т.е. потери предприятия, связанные с нехваткой
    единицы продукта в течение единицы времени, ден.ед./год;
  \item $\theta=8$ --- срок выполнения заказа, т.е. период времени от момента
    оформления предприятием заказа на очередную партию продукта
    до момента получения предприятием заказанной партии, дней.
\end{itemize}

План управления запасом деталей необходимо составить таким образом,
чтобы минимизировать общие затраты, связанные с запасом деталей.
Эти затраты включают:
\begin{itemize}
  \item затраты на приобретение деталей;
  \item затраты, связанные с партиями деталей;
  \item затраты на хранение деталей;
  \item потери от дефицита деталей.
\end{itemize}

Составление плана управления запасами в данном случае состоит в определении
размера заказа $q$ и точки заказа $r$.
