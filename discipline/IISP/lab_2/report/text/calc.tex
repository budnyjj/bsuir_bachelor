\section{РЕШЕНИЕ ЗАДАЧИ}

\subsection{Определение размера партии}

Размер партии (заказа) определяется по формуле:
\begin{equation}
	q = \sqrt{\dfrac{2KV}{S}}, ~~ q = \sqrt{\dfrac{2 \cdot 30 \cdot (360 \cdot 80))}{0,2}} = 2960 \: (\text{шт.})
\end{equation}

Таким образом, полученная величина означает, что размер партии деталей,
заказываемой предприятием, будет составлять 2960 шт.


\subsection{Анализ потребности в продукте в течение периода \\ выполнения заказа}

Исходя из того, что потребность в деталях в течение одного дня представляет
собой случайную величину, распределенную по Гауссовскому закону со средним значением
(математическим ожиданием) $V=80$ шт., потребность в деталях в течение восьми дней
также можно считать случайной величиной, распределенной по Гауссовскому закону,
со средним значением $V_\theta = V \cdot \theta = 80 \cdot 8 = 640 $ шт.
Это значит, что фактическая потребность в деталях за период $\theta$ может
с вероятностью $50\%$ составлять менее $640$ шт. и с такой же вероятностью превышать $640$ шт.

Стандартное отклонение потребности в деталях за один день составляет $\sigma=5$ шт.
Значит, дисперсия этой случайной величины составляет $D = \sigma^2 = 25 \: \text{шт}^2 $.
Дисперсию потребности в деталях за период $\theta$ можно считать равной
$D_\theta = \theta \cdot D = 8 \cdot 25 = 200 \: \text{шт}^2$. Таким образом,
стандартное отклонение потребности в деталях за период $\theta$ составляет
$\sigma_{\theta} = \sqrt{D_\theta} = 14{,}1 $ шт.

Как показано выше, потребность в деталях за период выполнения заказа ---
случайная величина, распределенная по гауссовскому закону, с математическим
ожиданием (средним значением)  $V_\theta = 640 $ шт. и стандартным отклонением $\sigma_{\theta} = 14{,}1 $ шт.
Рассмотрим возможные значения потребности в деталях, превышающие среднюю потребность, с шагом 10 шт.

Найдем, например, вероятность того, что фактическая потребность в деталях $X$
за период $\theta$ составит от $640$ до $650$ шт.
Из теории вероятностей известно, что эту вероятность можно найти следующим образом:
\[
	P(640<X<500) = P(X<650) - P(X<640).
\]

Закон распределения случайной величины $X$ известен:
это Гауссовская случайная величина с математическим ожиданием 640 и стандартным отклонением 14,1.
Поэтому вероятности $P(X<650)$ и $P(X<640)$ легко определить с помощью таблиц
Гауссовского (нормального) распределения или с помощью средств табличного процессора Excel.

Найдём вероятность $P(X<650)$ используя Excel. Для этого воспользуемся функцией
\texttt{НОРМРАСП} с аргументами \texttt{X}: $650$, \texttt{Среднее}: $640$,
\texttt{Стандартное отклонение}: $14{,}1$, \texttt{Интегральный}: $1$.

Вероятность $P(X<640)$ равна $0{,}5$, так как $640$ --- математическое ожидание
анализируемой Гауссовской случайной величины. 

Таким образом, вероятность того, что фактическая потребность в деталях за
период $\theta$ составит от $640$ до $650$ шт.,
равна $P(640<X<650) = 0{,}7605 - 0{,}5 = 0{,}2603$.

Аналогично найдем вероятности того, что потребность в деталях
будет принимать значения из других диапазонов. Результаты расчётов
сведены в таблицу~\ref{tbl:p}.

\begin{table} [h!]
  \caption{Вероятности попадания в интервал величины $X$}
  \label{tbl:p}
  \begin{tabular}{| m{6cm} | c | c | c | c |}
    \hline

    Диапазон значений, $X$ & $(640;650)$ & $(650;660)$ & $(660;670)$ & $(670;680)$ \\ \hline

   	Вероятность, P & $0{,}2602$ & $0{,}16110$ & $0{,}06170$ & $0{,}0146$ \\ \hline
      
  \end{tabular}
\end{table}



\subsection{Определение точки заказа}

Предположим, что точка заказа $r=645$ (т.е. новая партия деталей заказывается,
когда на предприятии остается $645$ деталей). Заказ будет выполнен за время
восемь дней. Дефицит возникнет, если потребность в деталях за этот период
составит свыше 650 шт. При выполнении расчетов будем предполагать,
что потребность в деталях, превышающая $645$ \: шт., может составлять $X=645, 655, 665 \text{ или } 675$ шт.
Дефицит определяется как $X-r = X-490 \:\text{шт.} $ Таким образом, будем считать,
что дефицит может принимать значения $10$ шт. (с вероятностью $0{,}1611$),
$20$ шт. (с вероятностью $0{,}0617$) и так далее.

Найдём средний дефицит (математическое ожидание дефицита) при каждой поставке партий деталей.
\[
	y_1 = 10 \cdot 0{,}1611 + 20 \cdot 0{,}0617 + 30 \cdot 0{,}0146 + 40 \cdot 0{,}00214 = 3{,}283.
\]

Это означает следующее: если при имеющейся на предприятии потребности в деталях и имеющихся
условиях поставок заказывать новую партию деталей, когда на предприятии остается
$645$ деталей, то дефицит деталей при каждой поставке будет составлять в среднем $3{,}283$ шт.
(конечно, при каждой конкретной поставке дефицит может оказаться как больше,
так и меньше найденной величины).

Так как годовая потребность в деталях составляет в среднем $V=29\:200$ шт.,
а размер партии --- $q=2960$ шт. (см. выше), то в течение года потребуется
в среднем $N = V/q = 29\:200/2960=9{,}87$ поставок.
Значит, средний дефицит за год составит $y=N \cdot y_1=9{,}87 \cdot 3{,}283=32{,}40$ шт.

Теперь найдём состовляющие среднегодовых затрат, связанных с запасом деталей:
\begin{enumerate}
	\item Затраты на приобретение деталей:
		\[Z_{\text{приобр}} = CV = 4 \cdot 29\:200 = 116\:800.\]
	\item Затраты, свзяанные с партиями деталей:
		\[Z_{\text{парт}} = K \dfrac{V}{q} = 30 \cdot \dfrac{29\:200}{2960}. \]
	\item Затраты на хранение деталей:
		\[Z_{\text{хранение}} = S\Big(\dfrac{q}{2} + r - V_\theta\Big) = 0{,}1\Big(\dfrac{2960}{2} + 645 - 640\Big).\]
	\item Потери от дефицита деталей:
		\[P_{\text{реф}} = dy = 0{,}5 \cdot 32{,}40 = 16{,}1962.\]
\end{enumerate}

В итоге среднегодовые затраты:
\begin{align*}
	Z &= Z_{\text{приобр}} + Z_{\text{парт}} + Z_{\text{хранение}} + P_{\text{реф}}, \\
	Z &= 116\:800{,}0 + 295{,}9 + 296{,}9 + 16{,}2.
\end{align*}


Аналогично рассмотрим другие возможные значения точки заказа ($r=655, r=665$), и так далее.
Сравним полученыые значения среднегодовых затрат, и выберем наименьшее значение. Поставим
в соотвествие этому значению точку заказа $r$. В приложении А расположена иллюстрация
произведенных расчётов в табличном процессоре Excel. 

Таким образом, затраты, связанные с запасом, принимают миинимальное значение при
точке заказа $r=665$ шт.

\textbf{Решение задачи управления запасом деталей состоит
в следующем:} требуется заказывать партию деталей в количестве $2\:960$ штук, когда
запас деталей на предприятии снижается до $665$ шт. Средние годовые затраты,
связанные с запасом деталей при этом составят $117\:397{,}88$ ден. ед.

\subsection{Определение вероятности дефицита}

Найдем вероятность возникновения дефицита при таком плане управления запасом деталей.
Дефицит возникнет, если потребность в деталях за период выполнения
заказа ($\theta = 8$ дней) превысит $665$ шт.

Эта потребность представляет собой Гауссовскую случайную величину с математическим
ожиданием $640$ и стандартным отклонением $14{,}1$.
Используя Excel, найдем вероятность того, что эта величина не превысит $640$ шт.
Для этого воспользуемся функцией \texttt{НОРМРАСП} с аргументами 
\texttt{X}: $665$, \texttt{Среднее}: $640$, \texttt{Стандартное отклонение}: $14{,}1$,
\texttt{Интегральный}: $1$. Будет получен результат $0{,}961$.
Таким образом, вероятность дефицита составит $(1 - 0{,}961) = 0{,}0385$.

\newpage