\section{ХОД РАБОТЫ}

\subsection{Постановка задачи}

Рассматривается ЭС, используемая для анализа данных геологической разведки и
принятия решения о бурении скважин. Решение принимается на основе информации о
содержании в пробах трех веществ (\( \text{В}_1, \text{В}_2, \text{В}_3 \)).
Имеются статистические данные о результатах 120 бурений
(из них в 85 случаях было обнаружено полезное ископаемое), приведенные в таблице~\ref{tbl:task}.

\begin{table}[h!]
  \caption{Исходные данные}
  \label{tbl:task}
  \begin{tabular}{| p{25mm} | p{30mm} | c | c |}
    \hline
    Вещество
    & \parbox{30mm}{
      \smallskip
      Содержание в пробах
      \smallskip
    }
    & \parbox{45mm}{
      \smallskip
      Количество случаев \\ обнаружения \\ полезного \\ ископаемого
      \smallskip
    }
    & \parbox{45mm}{
      \smallskip
      Количество случаев \\ неудачного бурения
      \smallskip
    } \\ \hline

    \multirow{3}{*}{\( \text{В}_1 \)}
    & Высокое
    & 62
    & 9  \\
    & Среднее
    & 13
    & 12 \\
    & Низкое
    & 10
    & 14 \\ \hline

    \multirow{2}{*}{\( \text{В}_2 \)}
    & Есть
    & 72
    & 11  \\
    & Нет
    & 13
    & 24 \\ \hline

    \multirow{2}{*}{\( \text{В}_3 \)}
    & Есть
    & 20
    & 22  \\
    & Нет
    & 65
    & 13 \\ \hline
  \end{tabular}
\end{table}

В пробе обнаружено высокое содержание вещества \( \text{В}_1 \), вещество \( \text{В}_2 \)
обнаружено, \( \text{В}_3 \) --- не обнаружено. Требуется оценить вероятность того,
что при бурении будет обнаружено полезное ископаемое.

\newpage

\subsection{Решение задачи}

Введем следующие обозначения:
\begin{itemize}
\item \( H_1 \) --- факт обнаружения полезного ископаемого при бурении;
\item \( H_2 \) --- факт отсутствия полезного ископаемого при бурении;
\item \( E_1 \) --- факт наличия высокого содержания вещества \( \text{В}_1 \)
  в пробе;
\item \( E_2 \) --- факт наличия вещества \( \text{В}_2 \)
  в пробе;
\item \( E_3 \) --- факт отсутствия вещества \( \text{В}_3 \)
  в пробе.
\end{itemize}

В этом случае свидетельство \( E \) представляет собой комбинацию трех
факторов \( E_1, E_2, E_3 \).

Определим значения вероятностей, требуемых для расчета по формуле Байеса.
Априорные вероятности гипотез:
\begin{equation*}
  P(H_1) = \dfrac{85}{120} \approx 0{,}71, \qquad   P(H_2) = \dfrac{120-85}{120} \approx 0{,}29.
\end{equation*}

Условные вероятности факторов \( P(E_i | H_i) \):
\begin{align*}
  P(E_1 | H_1) = \dfrac{62}{62 + 13 + 10} \approx 0{,}73, \qquad &
  P(E_1 | H_2) = \dfrac{9}{9 + 12 + 14} \approx 0{,}26, \\
  P(E_2 | H_1) = \dfrac{72}{72 + 13} \approx 0{,}85, \qquad &
  P(E_2 | H_2) = \dfrac{11}{11 + 24} \approx 0{,}31, \\
  P(E_3 | H_1) = \dfrac{65}{20 + 65} \approx 0{,}76, \qquad &
  P(E_3 | H_2) = \dfrac{13}{22 + 13} \approx 0{,}37. \\
\end{align*}
Считая события \( E_1, E_2, E_3 \) независимыми, определим условные вероятности
свидетельства:
\begin{align*}
  P(E | H_1) &= P(E_1,E_2,E_3 | H_1) = \\
             &= P(E_1| H_1) P(E_2| H_1) P(E_3| H_1) = 0{,}73 \cdot 0{,}85 \cdot 0{,}76 \approx 0{,}47, \\
  P(E | H_2) &= P(E_1,E_2,E_3 | H_2) = \\
             &= P(E_1| H_2) P(E_2| H_2) P(E_3| H_2) = 0{,}26 \cdot 0{,}31 \cdot 0{,}37 \approx 0{,}03. \\
\end{align*}

Здесь, например, величина \( P(E | H_1) \) --- вероятность наличия
высокой концентрации вещества \( \text{В}_1 \),
присутствия вещества \( \text{В}_2 \) и
отсутствия вещества \( \text{В}_3 \)
в исследуемой пробе при условии,
что при бурении было обнаружено полезное ископаемое.

Определим искомую апостериорную вероятность по формуле~\eqref{eq:bayes}:
\begin{align*}
  P(H_1 | E) =
  \dfrac{P(E | H_1) P(H_1)}{P(E | H_1) P(H_1) + P(E | H_2) P(H_2)} &= \\
  = \dfrac{0{,}47 \cdot 0{,}71}{0{,}47 \cdot 0{,}71 + 0{,}29 \cdot 0{,}03} &\approx 0{,}97.
\end{align*}

Таким образом, вероятность обнаружения полезного ископаемого при бурении скважины,
проба которой имеет высокую концентрацию вещества \( \text{В}_1 \),
содержит вещество \( \text{В}_2 \) и
не содержит вещества \( \text{В}_3 \), весьма высока.

Результаты расчетов задачи в табличном процессоре находятся в приложении~А.

\newpage