\section{ТЕОРЕТИЧЕСКИЕ СВЕДЕНИЯ}

Байесовская стратегия оценки выводов --- одна из стратегий,
применяемых для оценки достоверности выводов
(например, заключений продукционных правил) в ЭС.
Основная идея байесовской стратегии заключается в оценке вероятности
некоторого вывода с учетом фактов, подтверждающих или опровергающих этот вывод.

Формулировка теоремы Байеса, известная из теории вероятностей, следующая.

Пусть имеется \( n \) несовместных событий \( H_i, i = \overline{1,n} \).
Несовместность событий означает, что никакие из событий \( H_i, H_j, i,j = \overline{1,n}, i \ne j \)
не могут произойти вместе (другими словами, вероятности их совместного наступления равны нулю).
Известны вероятности этих событий: \( P(H_i) \), причем \( \sum^n_i P(H_i) = 1 \);
это означает, что события \( H_i \) образуют  полную группу событий,
т.~е. одно из них происходит обязательно.
С событиями \( H_i \) связано некоторое событие \( E \).
Известны вероятности события \( E \) при условиях того, что какое-либо из событий
\( H_i \) произошло:
\( P(E | H_i), i = \overline{1,n} \).
Пусть известно, что событие \( E \) произошло.
Тогда вероятность того, что какое-либо из событий \( H_i \)
произошло, можно найти по следующей формуле (формула Байеса):
\begin{equation}
  \label{eq:bayes}
  P(H_i | E) = \dfrac{P(E H_i)}{P(E)} =  \dfrac{P(E | H_i) P(H_i)}{\sum^n_i P(E | H_i) P(H_i)}
\end{equation}

События \( H_i \) называются гипотезами, а событие E --- свидетельством.
Вероятности гипотез \( P(H_i) \) без учета свидетельства (т.~е. без учета того,
произошло событие E или нет) называются доопытными (априорными),
а вероятности \( P(H_i | E) \) --- послеопытными (апостериорными).
Величина \( P(E H_i) \) --- совместная вероятность событий \( E \) и \( H_i \),
т.~е. вероятность того, что произойдут оба события вместе.
Величина \( P(E) \) --- полная (безусловная) вероятность события \( E \).

Формула Байеса позволяет уточнять вероятность гипотез с учетом новой информации,
т.~е. данных о событиях (свидетельствах), подтверждающих или опровергающих гипотезу.