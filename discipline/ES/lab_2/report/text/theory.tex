\section{ТЕОРЕТИЧЕСКИЕ СВЕДЕНИЯ}

Рекурсия --- это вызов одного предиката через самого себя.
В качестве примера рекурсивного вычислительного процесса рассмотрим
программу вычисления факториала, приведенную на рисунке~\ref{lst:factorial}.

\begin{lstlisting}[style=source_code,language=prolog,caption=Программа вычисления факториала в среде Visual Prolog,label=lst:factorial]
 predicates
   nondeterm factor(integer,integer,integer)
 
 clauses
       factor(1,Z,X):- X=Z.
       factor (N,Z,X):-
             Z1=Z*N,
             N1=N-1,
          factor(N1,Z1,X).
      
 goal
  factor(5,1,X), write("Factor=",X), readchar(_).
\end{lstlisting}

Здесь вычисляется факториал 5. Он равен 120 (\(5! = 120\)).
Вычисление факториала \( n! \) реализуется по формуле 
\( factor(n) = n * factor(n-1) \). 
Такая формула называется рекурсивной, поскольку она вычисляется через
саму себя, но с меньшим значением аргумента. 
В программе в формуле \( factor(\dots) \) три аргумента:
целевой аргумент, накопленное значение факториала и переменная
для результирующего значения.

Для организации бесконечного циклического процесса также используется
механизм рекурсии, как показано на рисунке~\ref{lst:repeat}.

\pagebreak
\begin{lstlisting}[style=source_code,language=prolog,caption=Пример организации бесконечного цикла в среде Visual Prolog,label=lst:repeat]
 predicates 
   nondeterm vxod
   nondeterm repeat
 
 clauses
      vxod:-
            repeat,
            nl,
            write("VVedi parol:"),
            readln(P),
            P="morgen",
            nl,
            write("Passed!"),
            readchar(_).
  
          repeat.
          repeat:-repeat.
 
 goal
    vxod.
\end{lstlisting}

Данная программа заставит вводить пароль столь раз, сколько потребуется, чтобы наконец набрать слово <<morgen>>.

\newpage