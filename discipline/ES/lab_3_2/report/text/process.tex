\section{ХОД РАБОТЫ}

\subsection{Постановка задачи}

В ходе данной лабораторной работы требуется выполнить проектирование
и реализацию небольшой семантической сети с использованием языка Пролог.

В качестве темы, определяющей набор объектов и отношений между ними, 
выберем стихотворение Самуила Маршака <<Дом, который построил Джек>>.

\subsection{Проктирование семантической сети}

Приведем размеченный текст стихотворения --- 
основы нашей семантической сети.

\begin{center}
С. Маршак \\
Дом, который построил Джек
\end{center}

\begin{verse}
Вот \textbf{дом}, \\
Который \textit{построил} \textbf{Джек}.

А это \textbf{пшеница}, \\
Которая в \texttt{темном} \textbf{чулане} \textit{хранится} \\
В \textbf{доме}, \\
Который \textit{построил} \textbf{Джек}.

А это \textit{веселая} \textbf{птица-синица}, \\
Которая часто \textit{ворует} \textbf{пшеницу}, \\
Которая в \texttt{темном} \textbf{чулане} \textit{хранится} \\
В \textbf{доме}, \\
Который \textit{построил} \textbf{Джек}.

Вот \textbf{кот}, \\
Который \textit{пугает} и \textit{ловит} \textbf{синицу}, \\
Которая часто \textit{ворует} \textbf{пшеницу}, \\
Которая в \texttt{темном} \textbf{чулане} \textit{хранится} \\
В \textbf{доме}, \\
Который \textit{построил} \textbf{Джек}.

Вот \textbf{пес} \texttt{без хвоста}, \\
Который за шиворот \textit{треплет} \textbf{кота}, \\
Который \textit{пугает} и \textit{ловит} \textbf{синицу}, \\
Которая часто \textit{ворует} \textbf{пшеницу}, \\
Которая в \texttt{темном} \textbf{чулане} \textit{хранится} \\
В \textbf{доме}, \\
Который \textit{построил} \textbf{Джек}.

А это \textbf{корова} \texttt{безрогая}, \\
\textit{Лягнувшая} \texttt{старого} \textbf{пса} \texttt{без хвоста}, \\
Который за шиворот \textit{треплет} \textbf{кота}, \\
Который \textit{пугает} и \textit{ловит} \textbf{синицу}, \\
Которая часто \textit{ворует} \textbf{пшеницу}, \\
Которая в \texttt{темном} \textbf{чулане} \textit{хранится} \\
В \textbf{доме}, \\
Который \textit{построил} \textbf{Джек}.

А это \textbf{старушка}, \texttt{седая} и \texttt{строгая}, \\
Которая \textit{доит} \textbf{корову} \texttt{безрогую}, \\
\textit{Лягнувшую} \texttt{старого} \textbf{пса} \texttt{без хвоста}, \\
Который за шиворот \textit{треплет} \textbf{кота}, \\
Который \textit{пугает} и \textit{ловит} \textbf{синицу}, \\
Которая часто \textit{ворует} \textbf{пшеницу}, \\
Которая в \texttt{темном} \textbf{чулане} \textit{хранится} \\
В \textbf{доме}, \\
Который \textit{построил} \textbf{Джек}.

А это \texttt{ленивый} и \texttt{толстый} \textbf{пастух}, \\
Который \textit{бранится} с \textbf{коровницей} \texttt{строгою}, \\
Которая \textit{доит} \textbf{корову} \texttt{безрогую}, \\
\textit{Лягнувшую} \texttt{старого} \textbf{пса} \texttt{без хвоста}, \\
Который за шиворот \textit{треплет} \textbf{кота}, \\
Который \textit{пугает} и \textit{ловит} \textbf{синицу}, \\
Которая часто \textit{ворует} \textbf{пшеницу}, \\
Которая в \texttt{темном} \textbf{чулане} \textit{хранится} \\
В \textbf{доме}, \\
Который \textit{построил} \textbf{Джек}.

Вот \textbf{два петуха}, \\
Которые \textit{будят} того \textbf{пастуха}, \\
Который \textit{бранится} с \textbf{коровницей} \texttt{строгою}, \\
Которая \textit{доит} \textbf{корову} \texttt{безрогую}, \\
\textit{Лягнувшую} \texttt{старого} \textbf{пса} \texttt{без хвоста}, \\
Который за шиворот \textit{треплет} \textbf{кота}, \\
Который \textit{пугает} и \textit{ловит} \textbf{синицу}, \\
Которая часто \textit{ворует} \textbf{пшеницу}, \\
Которая в \texttt{темном} \textbf{чулане} \textit{хранится} \\
В \textbf{доме}, \\
Который \textit{построил} \textbf{Джек}.
\end{verse}

Здесь объекты будущей сети выделены \textbf{жирным},
отношения между ними --- \textit{курсивом}, 
а свойства --- \texttt{моноширинным шрифтом}. 


\subsection{Реализация семантической сети}

Приведем объекты семантической сети к виду, 
пригодному для исопльзования в языке Пролог.
Для этого введем следующие предикаты:

\begin{itemize}
\item \texttt{who(X)} --- одушевленный объект;
\item \texttt{what(X)} --- неодушевленный объект;
\item \texttt{has\_prop(X)} --- свойство объекта;
\item \texttt{do(X)} --- отношение;
\item \texttt{how(X)} --- свойство отношения;
\item \texttt{is\_a(X)} --- отношение <<является>>.
\end{itemize}

Заметим, что предикаты \texttt{has\_prop(X)} и \texttt{is\_a(X)}
не являются необходимыми для построения сети, 
а введены для удобства и краткости.

Имея в распоряжении вышеперечисленные предикаты, 
а также служебные
\texttt{s(L)} (помещение фрагмента сети \texttt{L} в базу знаний) и
\texttt{m(X, L)} (проверка наличия элемента \texttt{X} в списке \texttt{L})
мы можем построить семантическую сеть стихотворения, 
как показано на рисунке~\ref{lst:jack}.

\pagebreak

\lstinputlisting[style=source_code,numbers=left,numberstyle=\texttt,xleftmargin=2em,
                 caption=Представление семантической сети на языке Пролог,
                 language=prolog,label=lst:jack]{code/jack.pro}
                 
\pagebreak

Приведем тестовые вопросы, которые были заданы разработанной базе знаний:
\begin{itemize}
\item ...
\end{itemize}

\newpage