\section{ТЕОРЕТИЧЕСКИЕ СВЕДЕНИЯ}

Списки --- важный тип данных в языке Пролог.

Для работы со списками нужно использовать специальную символику. 
Обозначение [X|Y] позволяет оторвать от списка голову и занести ее в переменную X.
Оставшаяся часть списка помещается в переменную Y, 
как показано на рисунке~\ref{lst:list_symbolics}.

\begin{lstlisting}[style=source_code,language=prolog,
  caption=Пример работы со списком,label=lst:list_symbolics]
 L = [X|Y] %% L = [1,3,7,8], X == 1, Y == [3,7,8]
\end{lstlisting}

При работе в системе Microsoft Visual Prolog необходимо предварительно объявлять
списки в разделе \texttt{domains}, как показано на рисунке~\ref{lst:list_declaration}.

\begin{lstlisting}[style=source_code,language=prolog,
  caption=Пример декларации списка в среде Microsoft Visual Prolog,label=lst:list_declaration]
 domains
     mylist=integer*
\end{lstlisting}

Рассмотрим пример программы, проверяющей, что заданный элемент содержится в списке. 
Исходный код программы приведен на рисунке~\ref{lst:check_if_exist}.

В этом примере проверку принадлежности к списку реализует предикат \texttt{member},
состоящий из трех правил.
Каждое правило предназначено для выявления одной из возможных ситуаций. 
Первое правило устанавливает, что когда список пустой, 
то элемент не принадлежит пустому списку. 
Команду \texttt{fail} надо вставлять, когда следует
прекратить просмотр и завершить выполнение предиката неудачей.
Второе правило предназначено для выявления ситуации, 
когда элемент совпадает с головой списка.
В этом случае значение хвоста списка не имеет значения.
В Прологе в таком случае используют специальное обозначение для аргументов --- знак подчеркивания.
Наконец, третье правило является рекурсивным. 
Это правило применяется, когда не подошли первые два правила, т.~е.,
когда список не пустой и его голова не совпадает с элементом.
В этом случае предикат member вызывается рекурсивно, но с другими аргументами.

\begin{lstlisting}[style=source_code,language=prolog,
  caption=Программа проверки факта нахождения заданного элемента в списке,label=lst:check_if_exist]
 domains
     li=integer*
 
 predicates
     nondeterm work
     nondeterm member(integer,li)
 clauses
     work:-
         write("Input list:"), nl,
         readterm(li,L),
         nl,
         write("Input element:"),
         readint(E),
         member(E,L),
         nl,
         write("Element prinadlezit spisku"),
         readchar(_).

     %% 1
     member(_,[]):-
         write("Element ne prinadlezit spisku"),
         readchar(_),
         fail.

     %% 2     
     member(X,[X |_]).

     %% 3
     member(X,[_ | Y]):-
         member(X,Y).
 goal
     work.
\end{lstlisting}

\newpage