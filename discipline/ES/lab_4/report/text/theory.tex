\section{ТЕОРЕТИЧЕСКИЕ СВЕДЕНИЯ}

Фреймовая модель основана на концепции Марвина Мински ---
профессора Массачусетского технологического института,
основателя лаборатории искусственного интеллекта,
автора ряда фундаментальных работ.
Фреймовая модель представляет собой систематизированную
психологическую модель памяти человека и его сознания.

Фрейм (англ. frame --- рамка, каркас) --- структура данных
для представления некоторого концептуального объекта.
Информация, относящаяся к фрейму, содержится в составляющих его слотах.

Слот (англ. slot --- щель, прорезь) может быть терминальным (листом иерархии)
или представлять собой фрейм нижнего уровня.

Формально фрейм можно представить как тип данных вида:
\begin{equation*}
  F = < N, S_1, S_2, S_3 >,
\end{equation*}
где \hspace{2mm} $N$ --- имя объекта, \par
$S_1$ --- множество слотов, содержащих факты, определяющие декларативную семантику фрейма, \par
$S_2$ --- множество слотов, обеспечивающих связи с другими фреймами, \par
$S_3$ --- множество слотов, обеспечивающих преобразования, определяющие процедурную семантику фрейма.

Фреймы подразделяются на:
\begin{itemize}
  \item фрейм-экземпляр --- конкретная реализация фрейма, описывающая текущее состояние в предметной области;
  \item фрейм-образец --- шаблон для описания объектов или допустимых ситуаций предметной области;
  \item фрейм-класс --- фрейм верхнего уровня для представления совокупности фреймов образцов.
\end{itemize}

Рассмотрим некоторые вспомогательные предикаты языка Visual Prolog.

\begin{enumerate}
\item Предикат \texttt{assert(<fact>)} добавляет факт или правило \texttt{<fact>} в базу данных Пролога.
\item Предикат \texttt{retractall(<fact>)} удаляет все факты или правила из базы данных Пролога, которые совпадают с аргументом \texttt{<fact>}.
\item Предикат \texttt{makewindow(WinNo, ScrAttr, FrAttr, Title, X, Y, Height, Width)}
  используется для создания нового окна. Имеет следующие параметры: \par
    -- \texttt{WinNo} --- номер окна (уникален по отношению к окну-родителю); \par
    -- \texttt{ScrAttr} --- цвет окна; \par
    -- \texttt{FrAttr} --- цвет рамки окна; \par 
    -- \texttt{Title} --- название окна; \par
    -- \texttt{X} и \texttt{Y} --- определяет положение окна по отношению к окну-родителю; \par 
    -- \texttt{Height} и \texttt{Width} --- высота и ширина окна, включая рамку.
\item Предикат \texttt{removewindow} удаляет активное окно.
\item Предикат \texttt{shiftwindow(WinNo)} делает окно с номером \texttt{WinNo} активным.
\end{enumerate}

\newpage