\section{ПОСТАНОВКА ЗАДАЧИ}

Даны некоторые экспериментальные данные, приведенные в таблице~\ref{tbl:source_data}.
Эти данные носят случайный характер.

\begin{table}[h!]
  \caption{Исходные данные}
  \label{tbl:source_data}
  \small{
    \centering
    \begin{tabular}{| p{0.14\textwidth} | p{0.14\textwidth} | p{0.14\textwidth} | p{0.14\textwidth} | p{0.14\textwidth} | p{0.14\textwidth} |}
      \hline

      $x$ & 0 & 110 & 195 & 410 & 540 \\ \hline
      $y$ & 33 & 24 & 20 & 13 & 12 \\

      \hline
    \end{tabular}
  }
\end{table}

Пользуясь предположением, что величина $ y $ связана с велиной $ x $ соотношением~\ref{eq:appr_f},
необходимо подобрать наиболее подходящие значения параметров $ a $ и $\alpha $ данного соотношения:
\begin{equation}
\label{eq:appr_f}
  y = f(x) = ae^{-\alpha x}.
\end{equation}

Для поиска оптимального значения параметров использовать метод наименьших квадратов.

\newpage
