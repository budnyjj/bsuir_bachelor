\documentclass[a4paper,12pt]{report}
\usepackage{geometry} % Для последующего задания полей
\geometry{a4paper,top=15mm,bottom=20mm,left=10mm,right=10mm}
\usepackage[T2A]{fontenc}	     % Поддержка русских букв
\usepackage[utf8]{inputenc}	     % Кодировка utf8
\usepackage[english, russian]{babel} % Языки: русский, английский
\usepackage{graphicx}                % Подключаем пакет работы с графикой

\renewcommand{\arraystretch}{1.3}

\begin{document}


\section*{Исходные данные}

\begin{table}[h!]
  \renewcommand{\tabcolsep}{0.9em}
  \centering
  \begin{tabular}{cccccccccc}
    -4.86	& -3.7	& -2.41	& -2.24	& -2.12	& -2.07	& -1.87	& -1.57	& -1.05	& -0.95	\\ 
-0.86	& -0.82	& -0.69	& -0.56	& -0.42	& -0.38	& -0.14	& -0.13	& -0.01	& 0.1	\\ 
0.13	& 0.41	& 0.46	& 0.53	& 0.7	& 0.84	& 0.99	& 1.06	& 1.19	& 1.21	\\ 
1.21	& 1.21	& 1.23	& 1.26	& 1.33	& 1.47	& 1.76	& 1.91	& 1.94	& 2.02	\\ 
2.09	& 2.12	& 2.2	& 2.22	& 2.24	& 2.37	& 2.38	& 2.45	& 2.51	& 2.6	\\ 
2.6	& 2.65	& 2.67	& 2.69	& 2.88	& 3.12	& 3.15	& 3.23	& 3.24	& 3.24	\\ 
3.26	& 3.44	& 4.09	& 4.09	& 4.47	& 4.79	& 4.95	& 5.01	& 5.03	& 5.18	\\ 
5.2	& 5.21	& 5.36	& 5.44	& 5.44	& 5.47	& 5.48	& 5.64	& 5.78	& 5.79	\\ 
5.81	& 5.94	& 5.98	& 6.11	& 6.49	& 6.54	& 6.63	& 6.75	& 7.05	& 7.13	\\ 
7.17	& 7.34	& 7.51	& 7.85	& 7.93	& 8.7	& 9.26	& 9.5	& 10.95	& 11.15	\\ 
	\\ 

  \end{tabular}
  \caption{Исходная выборка}
\end{table}

\begin{figure}[h!] 
  \centering
  \includegraphics[width=0.8\linewidth]{../pic/sample.png}
  \caption{График гипотетической функции распределения вероятностей}
\end{figure}

\newpage

\section*{Равноинтервальная гистограмма}

\begin{figure}[h!] 
  \centering
  \includegraphics[width=0.8\linewidth]{../pic/stat_series_eq_size.png}
  \caption{Равноинтервальная гистограмма распределения случайной величины}
\end{figure}

\begin{table}[h!]
  \renewcommand{\tabcolsep}{1.6em}
  \centering
  \begin{tabular}{|c|c|c|c|c|c|c|}
    \hline %%% strange bug...
    $ j $	& $ A_j $	& $ B_j $	& $ h_j $	& $ v_j $	& $ p^{*}_j $	& $ f^{*}_j $ \\ \hline
1	& -4.860	& -3.259	& 1.601	& 2	& 0.0200	& 0.0125 \\ \hline
2	& -3.259	& -1.658	& 1.601	& 5	& 0.0500	& 0.0312 \\ \hline
3	& -1.658	& -0.057	& 1.601	& 11	& 0.1100	& 0.0687 \\ \hline
4	& -0.057	& 1.544	& 1.601	& 18	& 0.1800	& 0.1124 \\ \hline
5	& 1.544	& 3.145	& 1.601	& 20	& 0.2000	& 0.1249 \\ \hline
6	& 3.145	& 4.746	& 1.601	& 9	& 0.0900	& 0.0562 \\ \hline
7	& 4.746	& 6.347	& 1.601	& 19	& 0.1900	& 0.1187 \\ \hline
8	& 6.347	& 7.948	& 1.601	& 11	& 0.1100	& 0.0687 \\ \hline
9	& 7.948	& 9.549	& 1.601	& 3	& 0.0300	& 0.0187 \\ \hline
10	& 9.549	& 11.150	& 1.601	& 2	& 0.0200	& 0.0125 \\ \hline
Всего:	&	&	&16.010	&100	&1.0000	& \\ \hline

  \end{tabular}
  \caption{Данные для построения равноинтервальной гистограммы}
\end{table}

\newpage

\section*{Равновероятностная гистограмма}

\begin{figure}[h!] 
  \centering
  \includegraphics[width=0.8\linewidth]{../pic/stat_series_eq_probability.png}
  \caption{Равновероятностая гистограмма распределения случайной величины}
\end{figure}

\begin{table}[h!]
  \renewcommand{\tabcolsep}{1.6em}
  \centering
  \begin{tabular}{|c|c|c|c|c|c|c|}
    \hline %%% strange bug...
    $ j $	& $ A_j $	& $ B_j $	& $ h_j $	& $ v_j $	& $ p^{*}_j $	& $ f^{*}_j $ \\ \hline
1	& -4.860	& -0.905	& 3.955	& 10	& 0.1000	& 0.0253 \\ \hline
2	& -0.905	& 0.115	& 1.020	& 10	& 0.1000	& 0.0980 \\ \hline
3	& 0.115	& 1.210	& 1.095	& 10	& 0.1000	& 0.0913 \\ \hline
4	& 1.210	& 2.055	& 0.845	& 10	& 0.1000	& 0.1183 \\ \hline
5	& 2.055	& 2.600	& 0.545	& 10	& 0.1000	& 0.1835 \\ \hline
6	& 2.600	& 3.250	& 0.650	& 10	& 0.1000	& 0.1538 \\ \hline
7	& 3.250	& 5.190	& 1.940	& 10	& 0.1000	& 0.0515 \\ \hline
8	& 5.190	& 5.800	& 0.610	& 10	& 0.1000	& 0.1639 \\ \hline
9	& 5.800	& 7.150	& 1.350	& 10	& 0.1000	& 0.0741 \\ \hline
10	& 7.150	& 11.150	& 4.000	& 10	& 0.1000	& 0.0250 \\ \hline
Всего:	&	&	&16.010	&100	&1.0000	& \\ \hline

  \end{tabular}
  \caption{Данные для построения равновероятностной гистограммы}
\end{table}

\newpage

\section*{Гипотеза о равномерном законе \\
  распределения случайной величины}

\begin{figure}[h!] 
  \centering
  \includegraphics[width=0.8\linewidth]{../pic/sample_uniform.png}
  \caption{Иллюстрация гипотезы о равномерном законе распределения случайной величины}
\end{figure}

\begin{table}[h!]
  \renewcommand{\tabcolsep}{1em}
  \centering
  \begin{tabular}{|c|c|c|c|c|c|c|c|}
    \hline %%% strange bug...
    \input{../tbl/pirson_uniform}
  \end{tabular}
  \caption{Промежуточные вычисления критерия согласия Пирсона}
\end{table}

\newpage

\section*{Гипотеза об экспоненциальном законе \\
  распределения случайной величины}

\begin{figure}[h!] 
  \centering
  \includegraphics[width=0.8\linewidth]{../pic/sample_exponential.png}
  \caption{Иллюстрация гипотезы об экспоненциальном законе распределения случайной величины}
\end{figure}

\begin{table}[h!]
  \renewcommand{\tabcolsep}{1em}
  \centering
  \begin{tabular}{|c|c|c|c|c|c|c|c|}
    \hline %%% strange bug...
    $ j $	& $ A_j $	& $ B_j $	& $ F_0(A_j) $	& $ F_0(B_j) $	& $ p_j $	& $ p_j^{*} $	& $ \frac{(p^{*}_j - p_j)^2}{p_j} $ \\ \hline
1	& $ -\infty $	& -3.259	& 0.0000	& 0.0000	& 0.0000	& 0.0200	& $ +\infty $ \\ \hline
	&	&	&	&Всего:	&0.0200	&0.0000	&$ +\infty $ \\ \hline

  \end{tabular}
  \caption{Промежуточные вычисления критерия согласия Пирсона для экспоненциального распределения}
\end{table}

\newpage


\section*{Гипотеза о нормальном законе \\
  распределения случайной величины}

\begin{figure}[h!] 
  \centering
  \includegraphics[width=0.8\linewidth]{../pic/sample_normal.png}
  \caption{Иллюстрация гипотезы о нормальном законе распределения случайной величины}
\end{figure}

\begin{table}[h!]
  \renewcommand{\tabcolsep}{1em}
  \centering
  \begin{tabular}{|c|c|c|c|c|c|c|c|}
    \hline %%% strange bug...
    $ j $	& $ A_j $	& $ B_j $	& $ F_0(A_j) $	& $ F_0(B_j) $	& $ p_j $	& $ p_j^{*} $	& $ \frac{(p^{*}_j - p_j)^2}{p_j} $ \\ \hline
1	& $ -\infty $	& -3.259	& 0.0000	& 0.0269	& 0.0269	& 0.0200	& 0.0018 \\ \hline
2	& -3.259	& -1.658	& 0.0269	& 0.0757	& 0.0487	& 0.0500	& 0.0000 \\ \hline
3	& -1.658	& -0.057	& 0.0757	& 0.1732	& 0.0975	& 0.1100	& 0.0016 \\ \hline
4	& -0.057	& 1.544	& 0.1732	& 0.3268	& 0.1536	& 0.1800	& 0.0045 \\ \hline
5	& 1.544	& 3.145	& 0.3268	& 0.5188	& 0.1921	& 0.2000	& 0.0003 \\ \hline
6	& 3.145	& 4.746	& 0.5188	& 0.7055	& 0.1866	& 0.0900	& 0.0500 \\ \hline
7	& 4.746	& 6.347	& 0.7055	& 0.8492	& 0.1438	& 0.1900	& 0.0149 \\ \hline
8	& 6.347	& 7.948	& 0.8492	& 0.9365	& 0.0873	& 0.1100	& 0.0059 \\ \hline
9	& 7.948	& 9.549	& 0.9365	& 0.9783	& 0.0418	& 0.0300	& 0.0033 \\ \hline
10	& 9.549	& $ +\infty $	& 0.9783	& 1.0000	& 0.0217	& 0.0200	& 0.0001 \\ \hline
	&	&	&	&Всего:	&1.0000	&1.0000	&0.0826 \\ \hline

  \end{tabular}
  \caption{Промежуточные вычисления критерия согласия Пирсона}
\end{table}

\newpage


\end{document}