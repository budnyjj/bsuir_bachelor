\documentclass[14pt,hidelinks]{extarticle}

%%% Поля и разметка страницы %%%
\usepackage{pdflscape}   % Для включения альбомных страниц
\usepackage{geometry} % Для последующего задания полей
\usepackage{setspace} % Для интерлиньяжа
\usepackage[14pt]{extsizes}
\usepackage{titlesec}
\usepackage{tocloft}
\usepackage{enumitem}
\usepackage{fancyhdr}

%%% Кодировки и шрифты %%%
\usepackage{cmap}                    % Улучшенный поиск русских слов в полученном pdf-файле
\usepackage[T2A]{fontenc}	     % Поддержка русских букв
\usepackage[utf8]{inputenc}	     % Кодировка utf8
\usepackage[english=nohyphenation,russian=nohyphenation]{hyphsubst} % Запрет переносов
\usepackage[english, russian]{babel} % Языки: русский, английский
% \usepackage{pscyr}						% Красивые русские шрифты

%%% Математические пакеты %%%
\usepackage{amsthm,amsfonts,amsmath,amssymb,amscd} % Математические дополнения от AMS

%%% Оформление абзацев %%%
\usepackage{indentfirst} % Красная строка

%%% Цвета %%%
\usepackage[usenames]{color}
\usepackage{color}
\usepackage{colortbl}

%%% Таблицы %%%
\usepackage{longtable}		     % Длинные таблицы
\usepackage{multirow,makecell,array} % Улучшенное форматирование таблиц

%%% Общее форматирование
\usepackage{caption}
\captionsetup[figure]{labelsep=space,justification=centering,singlelinecheck=false}

\usepackage{soul}                    % Поддержка переносоустойчивых подчёркиваний и зачёркиваний
\usepackage{multicol}

%%% Библиография %%%
\usepackage{cite} % Красивые ссылки на литературу

%%% Гиперссылки %%%
\usepackage[unicode,plainpages=false,pdfpagelabels=false]{hyperref}

%%% Изображения %%%
\usepackage{graphicx} % Подключаем пакет работы с графикой     % Подключаемые пакеты
../../../../template/lab/sys/styles.tex	     % Пользовательские стили

\begin{document}

../../../../../template/calc/sys/names.tex	     % Переопределение именований

%%%%%%%%%%%%%%%%%%%%%%%%%%%%%%%%%%%%%%%%%%%%%%%%%%%%%%%%%%%%%%%%

\begin{center}
	\textbf{Типовой расчет} \\ 
	выполнил ст. гр. ****** Петров Ю.А. \\
        Задача №11\\
	Вариант XX 
\end{center}

\section{Условие}

По выборке двумерной случайной величины:

\begin{itemize}
	\item вычислить оценку коэффициента корреляции;
        \item прлверить гипотезу об отсутствии \\
          корреляционной зависимости ($ \alpha = 0.05 $);
	\item вычислить параметры линии регрессии $ \alpha_0 $ и $ \alpha_1 $; 
	\item построить диаграмму рассеивания и линию регрессии;
\end{itemize}

Исходные данные для варианта XX приведены в таблице~\ref{tbl:second_sample}.

\begin{table}[h!]
  \renewcommand{\tabcolsep}{0.6em} 
	\centering
	\caption{Двумерная выборка\label{tbl:second_sample}}
	\begin{tabular}{llllllllll}
          -4.86	& -3.7	& -2.41	& -2.24	& -2.12	& -2.07	& -1.87	& -1.57	& -1.05	& -0.95	\\ 
-0.86	& -0.82	& -0.69	& -0.56	& -0.42	& -0.38	& -0.14	& -0.13	& -0.01	& 0.1	\\ 
0.13	& 0.41	& 0.46	& 0.53	& 0.7	& 0.84	& 0.99	& 1.06	& 1.19	& 1.21	\\ 
1.21	& 1.21	& 1.23	& 1.26	& 1.33	& 1.47	& 1.76	& 1.91	& 1.94	& 2.02	\\ 
2.09	& 2.12	& 2.2	& 2.22	& 2.24	& 2.37	& 2.38	& 2.45	& 2.51	& 2.6	\\ 
2.6	& 2.65	& 2.67	& 2.69	& 2.88	& 3.12	& 3.15	& 3.23	& 3.24	& 3.24	\\ 
3.26	& 3.44	& 4.09	& 4.09	& 4.47	& 4.79	& 4.95	& 5.01	& 5.03	& 5.18	\\ 
5.2	& 5.21	& 5.36	& 5.44	& 5.44	& 5.47	& 5.48	& 5.64	& 5.78	& 5.79	\\ 
5.81	& 5.94	& 5.98	& 6.11	& 6.49	& 6.54	& 6.63	& 6.75	& 7.05	& 7.13	\\ 
7.17	& 7.34	& 7.51	& 7.85	& 7.93	& 8.7	& 9.26	& 9.5	& 10.95	& 11.15	\\ 
	\\ 

	\end{tabular}
\end{table}

\newpage

\section{Решение}

\subsection{Вычисление точечных оценок параметров двумерной выборки}

Вычислим оценки математических ожиданий по каждой переменной.
\begin{align}
  m^*_X &= \overline{x} = \frac{1}{n} \sum_{i=1}^{n} x_i, &
  m^*_Y &= \overline{y} = \frac{1}{n} \sum_{i=1}^{n} y_i, \\ \nonumber
  m^*_X &= \input{val/x_expectation_value}, &
  m^*_Y &= \input{val/y_expectation_value}.
\end{align}

Вычислим оценки дисперсий по каждой переменной.
\begin{equation}
  \begin{aligned}
    D^*_X &= S^2_0(x) = \frac{1}{n-1} \sum_{i=1}^{n} (x_i - \overline{x})^2, \\
    D^*_Y &= S^2_0(y) = \frac{1}{n-1} \sum_{i=1}^{n} (y_i - \overline{y})^2,
  \end{aligned}
\end{equation} 
\begin{equation*}
  \begin{aligned}
    D^*_X &= \input{val/x_dispersion_value}, &
    D^*_Y &= \input{val/y_dispersion_value}.
  \end{aligned}
\end{equation*}

Вычислим оценку корреляционного момента:
\begin{equation}
  K^*_{XY} = \frac{1}{n-1} \sum_{i=1}^{n}{(x_i - \overline{x})(y_i - \overline{y})},
\end{equation} 
\begin{equation*}
  K^*_{XY} = \input{val/correlation_moment}.
\end{equation*}

Найдем оценку коэффициента корреляции:
\begin{equation}
  R^*_{XY} = \frac{K^*_{XY}}{\sqrt{S^2_0(x) \cdot S^2_0(y)}},
\end{equation} 
\begin{equation*}
  R^*_{XY} = \dfrac{\input{val/correlation_moment}}
  {\sqrt{\input{val/x_dispersion_value} \cdot \input{val/y_dispersion_value}}} = 
  \input{val/correlation_coefficient_value}.
\end{equation*}

\subsection{Вычисление интервальной оценки коэффициента корреляции}

Вычислим интервальную оценку коэффициента корреляции с надёжностью $\gamma = 0.95$ по следующей формуле:
\begin{align}
  I_{\gamma} (R_{XY}) &= \left[ \frac{e^{2a}-1}{e^{2a}+1}; \frac{e^{2b}-1}{e^{2b}+1} \right]. 
\end{align}

Для этого в таблице функции Лапласа найдем значение, равное $\frac{\gamma}{2} = 0.475$ и определим значение аргумента, ему соответствующее:
\begin{equation*}
  z_{0.95} = arg \Phi (0.475) = 1.96. 
\end{equation*}

Для вычисления интервальной оценки коэффициента корреляции найдем вспомогательные значения $a, b$ по следующим формулам:
\begin{equation}
  \begin{aligned}
    a &= 0.5 \cdot ln \left( \frac{1+R^*_{XY}}{1-R^*_{XY}} \right) - \frac{z_{\gamma}}{\sqrt{n-3}}, \\
    b &= 0.5 \cdot ln \left( \frac{1+R^*_{XY}}{1-R^*_{XY}} \right) + \frac{z_{\gamma}}{\sqrt{n-3}}, \\
  \end{aligned}
\end{equation}
\begin{equation*}
  \begin{aligned}
    a &= \input{val/correlation_coefficient_a}, &
    b &= \input{val/correlation_coefficient_b}.
  \end{aligned}
\end{equation*}

Таким образом, доверительный интервал для коэффициента корреляции имеет вид:
\begin{align}
  I_{\gamma} (R_{XY}) &= \left[ \frac{e^{2a}-1}{e^{2a}+1}; \frac{e^{2b}-1}{e^{2b}+1} \right], \\ \nonumber
  I_{\gamma} (R_{XY}) &= \left[ \input{val/correlation_coefficient_min}; 
    \input{val/correlation_coefficient_max} \right]
\end{align}

\newpage 

\subsection{Проверка гипотезы об отсутствии корреляционной зависимости}
Выдвинем двухальтернативную гипотезу об отсутствии корреляционной зависимости 
между величинами $ X $ и $ Y $:

\begin{itemize}
\item $H_0$ --- $ R_{XY} = 0 $: между величинами $ X $ и $ Y $ корреляционная зависимость отсутствует;
\item $H_1$ --- $ R_{XY} \neq 0 $: между величинами $ X $ и $ Y $ существует корреляционная зависимость.
\end{itemize}

Так как объем выборки велик ($ n \ge 50 $), то вычислим значение критерия по формуле:
\begin{equation}
  Z = \dfrac{|R^*_{XY}| \cdot \sqrt{n}}{1-(R^*_{XY})^2},
\end{equation}
\begin{equation*}
  Z = \input{val/hypothesis_no_correlation}.
\end{equation*}

Определим значение $ Z_\alpha $ из таблицы функции Лапласа ($ \alpha = 0.05 $):
\begin{equation*}
  Z_{0.05} = 1.96.
\end{equation*}

\textbf{Вывод:} так как $ Z = \input{val/hypothesis_no_correlation} {<}notation{>} Z_{0.05} $,
то гипотеза $H_0$ об отсутствии корреляционной зависимости между величинами $ X $ и $ Y $ принимается (отклоняется).

\newpage

\subsection{Построение линии регрессии}

Уравнение линии регрессии имеет следующий вид:
\begin{equation}
  \overline{y}(x) = a^*_0 + a^*_1 x,
\end{equation} 
где $a^*_1 = \dfrac{K^*_{XY}}{S^2_0(x)},\; a^*_0 = \overline{y} - a^*_1 \cdot \overline{x} $
--- коэффициенты линии регрессии.

\vspace{1em}
Найдем значения $ a^*_1, a^*_0 $:
\begin{equation*}
	\begin{aligned}
		a^*_1 &= \input{val/regression_param_1}, &
		a^*_0 &= \input{val/regression_param_0}
	\end{aligned}
\end{equation*}

Таким образом, линия регрессии примет вид:
\begin{equation}
  \overline{y}(x) = \input{val/regression_param_0}
  + ( \input{val/regression_param_1} ) \cdot x
\end{equation}

График линии регрессии изображен на рисунке~\ref{fig:sample_regression}.
\begin{figure}[h!] 
  \centering
  \includegraphics[width=1\linewidth]{pic/sample_regression}
  \caption{График линии регрессии для двумерной случайной величины\label{fig:sample_regression}}
\end{figure}


\end{document}