\section{ХОД РАБОТЫ}

\subsection{Постановка задачи}

Требуется сформировать 4-мерные матрицы $A$ и $B$ четвёртого порядка,
получить матрицу $A^T$, транспонированную относительно $A$ соответственно
подстановке $T = (2,3,4,1)$, а также матрицу $D$, равную $(0, 3)$-свёрнутому
произведению матриц $A$ и $B$.

Сформировать $(0, 3)$-единичную матрицу $E(0,3)$ четвёртого порядка
и найти произведение
\begin{equation*}
  F = {}^{(0,3)}(E(0, 3)B).
\end{equation*}


\subsection{Теоретические сведения}
\label{sub:theory}

Многомерной ($p$-мерной) матрицей $A$ называется система чисел или
переменных $a_{i_1, i_2, \dots, i_p}, i_a = \overline{1, n_a}, a = \overline{1,p}$,
расположенных в точках $p$-мерного пространства, определяемого координатами $i_1, i_2, \dots, i_p$.
Совокупность индексов $(i_1, i_2, \dots, i_p)$ многомерной матрицы $A$
называется мультииндексом и обозначается одним символом $i = (i_1, i_2, \dots, i_p)$.
Таким образом, многомерную матрицу $A$ можно обозначить следующим образом:
\begin{equation*}
  A = (a_i).
\end{equation*}

Многомерная матрица $n$-ого порядка называется гиперквадратной, если выполняется условие:
$n_1 = n_2 = \dots = n_p = n$.

Многомерная матрица $A$ называется симметричной относительно двух своих
индексов $i_\alpha, i_\beta$, если каждые два её элемента, получившиеся
один из другого перестановкой этих индексов, одинаковы, то есть если
выполняется условие:
\begin{equation*}
  a_{i_1 \dots i_\alpha, \dots, i_\beta, \dots, i_p} = a_{i_1 \dots i_\beta, \dots, i_\alpha, \dots, i_p}.
\end{equation*} 

Многоменая матрица называется симметричной,
если она симметрична относительно всех своих индексов.

Пусть $A$ --- $p$-мерная матрица. Тогда матрица $A^{T_1}$ называется транспонированной
относительно матрицы $A$ соответственно подстановке $T_1$:
\begin{equation}
\label{eq:tr}
  A^{T_1} = (a^T_{i_1, i_2, \dots, i_p}),
\end{equation}
где $a^{T^1}_{i_1, i_2, \dots, i_p} = a_{i_{\alpha_1}, i_{\alpha_2}, \dots, i_{\alpha_p}}$,
\[
  \begin{array}{lc}
    T_1 = 
    \left(
      \begin{matrix}
        i_1, i_2, \dots, i_p \\
        i_{\alpha_1} i_{\alpha_2} \dots, i_{\alpha_p}
      \end{matrix}
    \right).
  \end{array}
\]

Перестановкой (подстановкой) называется взаимно однозначное отображение
множества индексов $i_1, i_2, \dots, i_p$ на себя (биекция). Это отображение ставит
в соответствие множеству индексов $(i_1, i_2, \dots, i_p)$ так, что выполняется
равенство:
\[
  (i_{\alpha_1}, i_{\alpha_2}, \dots, i_{\alpha_p}) = T(i_1, i_2, \dots, i_p).
\]

Рассмотрим операцию умножения многомерных матриц. Пусть $A$ --- $p$-мерная
матрица $n$-ого порядка, $B$ --- $q$-мерная матрица $n$-ого порядка,
причём матрицы $A$ и $B$ определяются следующим образом:
\begin{align*}
  A = (a_{i_1, i_2, \dots i_p}), \hspace{3mm} i_{\alpha} = \overline{1,n}, \alpha = \overline{1,p}, \\
  B = (b_{j_1, j_2, \dots i_q}), \hspace{3mm} j_{\alpha} = \overline{1,n}, \alpha = \overline{1,q}.
\end{align*}

Разобъём мультииндексы $i = (i_1, i_2, \dots, i_p), \hspace{2mm} j = (j_1, j_2, \dots, j_q)$ этих матриц
на составляющие мультииндексы следующим образом:
\[
  \begin{array}{lc}
    \begin{matrix}
      i = (i_1, i_2, \dots, i_p) = (l, s, c), \\
      j = (j_1, j_2, \dots, j_q) = (c, s, m), \\
      l = (l_1, l_2, \dots, l_k), \hspace{5mm} s = (s_1, s_2, \dots, s_{\lambda}), \\
      c = (c_1, c_2, \dots, c_\mu), \hspace{5mm} m = (m_1, m_2, \dots, m_{\nu}).
    \end{matrix}
  \end{array}
\]

При этом должны выполняться условия:
\[
  k + \lambda + \mu = p, \hspace{5mm} \lambda + \mu + \nu = q.
\]

Тогда матрицы $A$ и $B$ можно переписать в виде:
\[
  A = (a_{l,s,c}), \hspace{5mm} B = (b_{c,s,m}),
\]
где каждый из мультииндексов $l,s,c,m$ пробегает значения от $1$ до $n$.

Таким образом матрица $D$, определяемая выражением~\eqref{eq:d}, называется $(\lambda, \mu)$-свёрнутым
произведением матриц $A$ и $B$.
\begin{equation}
\label{eq:d}
  D = {}^{\lambda, \mu}(A, B) = \Big( \sum_c a_{l, s, c} \: b_{c, s, m} \Big) = (d_{l,s,m}).
\end{equation}

\subsection{Формирование четырёхмерных матриц $A$ и $B$}

Воспользовавшись языком программирования Python, сформируем матрицы
$A$ и $B$ 4-ого порядка. Нетрудно показать, что число элементов этих
матриц равняется $256$~$(4^4 = 256)$. Заполним матрицы элементами
из диапазона $[0; 255]$, упорядоченными по возрастанию.
На рисунке~\ref{lst:a_and_b} изображены элементы матриц $A$ и $B$.
\lstinputlisting[
  basicstyle=\scriptsize\ttfamily,
  label=lst:a_and_b,
  caption=Элементы матриц $A$ и $B$
]{code/a.txt}


\subsection{Транспонирование матрицы $A$, подстановка $T = (2,3,4,1)$}

Рассмотрим механизм перестановки индексов на примере формирования шестнадцати
элементов матрицы $A^T$ --- $a^T_{i,j,1,1}, \hspace{2mm} i = \overline{1,4}, j = \overline{1,4}$.
\[
\small{
  \begin{array}{lc}
    \begin{matrix}
      a^T_{1,1,1,1} = a_{1,1,1,1} = 0, \hspace{2mm}
      a^T_{1,2,1,1} = a_{2,1,1,1} = 64, \hspace{2mm} 
      a^T_{1,3,1,1} = a_{3,1,1,1} = 128, \hspace{2mm}
      a^T_{1,4,1,1} = a_{4,1,1,1} = 192,
      \\
      a^T_{2,1,1,1} = a_{1,1,1,2} = 1, \hspace{2mm}
      a^T_{2,2,1,1} = a_{2,1,1,2} = 65, \hspace{2mm} 
      a^T_{2,3,1,1} = a_{3,1,1,2} = 129, \hspace{2mm}
      a^T_{2,4,1,1} = a_{4,1,1,2} = 193,
      \\
      a^T_{3,1,1,1} = a_{1,1,1,3} = 2, \hspace{2mm}
      a^T_{3,2,1,1} = a_{2,1,1,3} = 66, \hspace{2mm} 
      a^T_{3,3,1,1} = a_{3,1,1,3} = 130, \hspace{2mm}
      a^T_{3,4,1,1} = a_{4,1,1,3} = 194,
      \\
      a^T_{4,1,1,1} = a_{1,1,1,4} = 3, \hspace{2mm}
      a^T_{4,2,1,1} = a_{2,1,1,4} = 67, \hspace{2mm} 
      a^T_{4,3,1,1} = a_{3,1,1,4} = 131, \hspace{2mm}
      a^T_{4,4,1,1} = a_{4,1,1,4} = 195.
    \end{matrix}
  \end{array}
}
\]

На рисунке~\ref{lst:a_t} изображены элементы матрицы $A^T$.
\lstinputlisting[
  basicstyle=\scriptsize\ttfamily,
  label=lst:a_t,
  caption=Элементы матрицы $A^T$
]{code/a_t.txt}


\subsection{Вычисление $(0,3)$-свёрнутого произведения ${}^{0, 3}(A \: B)$}

Исходные данные: $A = (a_i), \: B = (b_j), \: \lambda = 0, \: \mu = 3$.

Разобъём мультииндексы $i, j$ матриц $A$ и $B$ на мультииндексы:
\begin{equation*}
  i = (l, s, c), \hspace{5mm} j = (c, s, m).
\end{equation*}

При этом должны выполняться условия:
\[
  \begin{array}{lc}
    \begin{matrix}
      l = (l_1, l_2, \dots, l_k), \hspace{5mm} s = (s_1, s_2, \dots, s_{\lambda}), \\
      c = (c_1, c_2, \dots, c_\mu), \hspace{5mm} m = (m_1, m_2, \dots, m_{\nu}), \\
      k + \lambda + \mu = p, \hspace{5mm} \lambda + \mu + \nu = q.
    \end{matrix}
  \end{array}
\]

Из условия имеем $\lambda = 0, \hspace{2mm} \mu = 3, \hspace{2mm} p = q = 4$. Таким образом находим
индексы: $k = 1, \hspace{2mm} \nu = 1$.

В результате матрица $D$: $D = d(l, s, m)$, $l = (l_1)$, $m = (m_1)$, $s = ()$,
элементы которой определяются по формуле:
\begin{equation*}
  d_{l, m} = \sum_{c} a_{lc} \: b_{cm}, \: l = (l_1), \: m = (m_1), \: c = (c_1, c_2, c_3).
\end{equation*}

На рисунке~\ref{lst:dot} изображены элементы матрицы $D$.
\lstinputlisting[
  basicstyle=\scriptsize\ttfamily,
  label=lst:dot,
  caption=Элементы матрицы $D$
]{code/dot.txt}


\subsection{Вычисление $(0,3)$-свёрнутого произведения ${}^{0, 3}(E(0,3) \: B)$}

Сформируем матрицу $E(0,3)$ четвёртого порядка:
\[
  \begin{array}{lc}
    E(0, 3) = (e_{c,s,m}) = 
    \left(
      \left\{
        \begin{matrix}
          1, \text{если } c = m \\
          0, \text{если } c \ne m
        \end{matrix}
      \right.
    \right),
  \end{array}
\]
где $m = (m_{\mu}) = (m_1, m_2, m_3), \hspace{2mm} c = (c_{\mu}) = (c_1, c_2, c_3), \hspace{2mm} s = (s_{\lambda}) = ().$

\pagebreak

На рисунке~\ref{lst:e} изображены некоторые элементы матрицы $E(0,3)$.
\lstinputlisting[
  basicstyle=\scriptsize\ttfamily,
  label=lst:e,
  caption=Некоторые элементы матрицы $E$
]{code/e.txt}

В качестве исходных данных для умножения ${}^{0, 3}(E(0,3) \: B)$ имеем:
\begin{equation*}
  B = (b_i), \hspace{2mm} E = (e_j), \hspace{5mm} \lambda = 0, \hspace{2mm} \mu = 3.
\end{equation*}

Разобъём мультииндексы $i, j$ матриц $B$ и $E$ на мультииндексы:
\begin{equation*}
  i = (l, s, c), \hspace{5mm} j = (c, s, m).
\end{equation*}

При этом должны выполняться условия:
\[
  \begin{array}{lc}
    \begin{matrix}
      l = (l_1, l_2, \dots, l_k), \hspace{5mm} s = (s_1, s_2, \dots, s_{\lambda}), \\
      c = (c_1, c_2, \dots, c_\mu), \hspace{5mm} m = (m_1, m_2, \dots, m_{\nu}), \\
      k + \lambda + \mu = p, \hspace{5mm} \lambda + \mu + \nu = q.
    \end{matrix}
  \end{array}
\]

Из условия имеем $\lambda = 0, \hspace{2mm} \mu = 3, \hspace{2mm} p = 6, \hspace{2mm} q = 4$.
Таким образом находим индексы: $k = 3, \hspace{2mm} \nu = 1$.

В результате матрица $D$: $D = d(l, s, m)$, $l = (l_1, l_2, l_3)$, $m = (m_1)$, $s = ()$,
элементы которой определяются по формуле:
\begin{equation*}
  d_{l, m} = \sum_{c} b_{lc} \: e_{cm}, \: l = (l_1, l_2, l_3), \: m = (m_1), \: c = (c_1, c_2, c_3).
\end{equation*}

В результате вычисления $(0,3)$-свёрнутого произведения ${}^{0, 3}(E(0,3) \: B)$
получаем исходную матрицу $B$ ($D = B$).

Исходный код разработанной программы расположен в приложении~А.