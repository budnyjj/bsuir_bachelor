\section[Построение базовой имитационной модели]
{ПОСТРОЕНИЕ БАЗОВОЙ ИМИТАЦИОННОЙ \\ МОДЕЛИ}

Рассмотрим функции, описанные в программе.

Функция \texttt{TIP\_NEISP} генерирует случайное число в диапазоне $ 0 \dots 1 $.
Если результат меньше либо равен $ 0{,}3 $, функция возвращает единицу,
если результат находится в диапазоне от $ 0{,}3 $ до $ 0{,}9 $ включительно,
то возвращается число два, если же результат больше $ 0{,}9 $, то функция возвращает
число три. Таким образом имитируется определение типа неисправности поступившего
на ремонт двигателя.

Функция \texttt{TIP\_POVT} аналогично функции \texttt{TIP\_NEISP} служит для
определения типа неисправности поступающего двигателя на повторный ремонт. Существует
следующая зависимость попадания в диапазон случайного числа от результата функции:
\begin{itemize}
  \item случайное число оказалось меньше, чем $ 0{,}2 $, результат --- $ 1 $;
  \item случайное число попало в интервал $ 0{,}2 \dots 0{,}9 $, результат --- $ 2 $;
  \item случайное число попало в интервал $ 0{,}9 \dots 1 $, результат --- $ 3 $.
\end{itemize}

Функция \texttt{VREMYA} выполняет подстановку распределения времени проверки
детали в зависимости от её типа. Необходимые распределения предварительно
записаны с помощью операторов \texttt{VARIABLE}.

Функция \texttt{N\_RAB} выполняет подстановку количества рабочих, занятых
для ремонта конкретной детали в зависимости от типа её неисправности.

С помощью оператора \texttt{LINK OZHIDANIE,P2,REMONT} реализуется требование:
узлы, на ремонт которых затрачивается меньше времени, должны обслуживаться в первую
очередь. При этом оператор \texttt{P2} хранит время, затрачиваемое
на обработку детали.

Команда \texttt{GENERATE 14400} имитирует процесс работы мастерской в течение тридцати
рабочих дней~(четырнадцати тысяч четырёхсот минут).

Имитационная модель рассматриваемого объекта моделирования представляет
собой GPSS-модель замкнутого типа, исходный текст которой приведен в приложении~А.

\pagebreak
