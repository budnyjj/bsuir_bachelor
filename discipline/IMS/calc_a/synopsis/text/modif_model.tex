\section[Построение модифицированной имитационной модели]
{ПОСТРОЕНИЕ МОДИФИЦИРОВАННОЙ \\ ИМИТАЦИОННОЙ МОДЕЛИ}

В процессе разработки модифицированной модели рассматриваемого объекта
будем отталкиваться от требования повысить коэффициент загрузки стенда
проверки узлов.

Если бы стенд проверки узлов представлял собой многоканальное
устройство, можно было бы рассмотреть вариант уменьшения количества каналов
для такого устройства, однако в данной задаче стенд проверки узлов является
одноканальным устройством. Таким образом повышение эффективности работы
такого устройства (повышение коэффициента загрузки стенда проверки) можно
произвести путём увеличения входного потока узлов на ремонт.

Увеличение входного потока узлов может быть достигнуто путём заключения
дополнительных договоров в поставщиками узлов для ремонта.

Предположим, что в условиях данной задачи узлы на ремонт поступают в среднем
не раз в двадцать, а раз в двенадцать минут. Следует учесть, что увеличение
потока деталей на обработку приведёт к перегрузке многоканального устройства
\texttt{RABOCHIE}. Для того, чтобы избежать такой перегрузки предлагается
нанять на работку двух рабочих, которые буду ремонтировать детали.

Таким образом, базовая модель поставленной задачи подверглась следующим изменениям:
\begin{itemize}
  \item многоканальное устройство \texttt{RABOСHIE} теперь имеет пять каналов;
  \item узлы поступают на обработку в среднем раз в двенадцать минут~(распределение
    имеет вид \texttt{(EXPONENTIAL(6,0,12))}).
\end{itemize}

Исходный код модифицированной имитационной модели расположен в приложении~Б.

\pagebreak
