\section[Анализ результатов базовой имитационной модели]
{АНАЛИЗ РЕЗУЛЬТАТОВ БАЗОВОЙ \\ ИМИТАЦИОННОЙ МОДЕЛИ}

В качестве результатов работы имитационной модели можно принять следующие
величины:
\begin{itemize}
  \item коэффициент загрузки рабочих;
  \item коэффициент загрузки стенда проверки узлов;
  \item среднее время пребывания узлов в цепи пользователя;
  \item средняя длина цепи пользователя;
  \item число отремонтированных узлов с неисправностями каждого типа;
  \item число узлов, не подлежащих ремонту.
\end{itemize}

Каждую из приведенных величин можно получить из отчёта моделирования,
предоставляемого системой GPSS World, который приведен в приложении~А.
Например, коэффициент загрузки рабочих --- это значение параметра \texttt{UTIL}
в описании многоканального устройства \texttt{RABOCHIE}, в нашем случае
оно равняется $ 0{,}833 $. Исходя из того, что оптимальный коэффициент
загрузки устройств лежит в диапазоне $ 0{,}7 \dots 0{,}9 $, можно сделать
вывод что многоканальное устройство \texttt{RABOCHIE} работает оптимально.

Одноканальное устройство \texttt{PROVER} (стенд проверки узлов) имеет коэффициент
загрузки $ 0{,}445 $, что говорит о недостаточной эффективности его работы.

Среднее время простоя узлов в цепи пользователя \texttt{OZHIDANIE} составляет
$ 45{,}947 $ минут, средняя длина этой цепи --- $ 1{,}477 $ узлов.

С целью понижения времени простоя узлов в цепи пользователя, а также повышения
эффективности работы многоканальных и одноканальных устройств будет создана
модифицированная имитационная модель приведенной задачи.

Файл статистики модифицированной имитационной модели приведен в приложении~А.

\newpage

Результаты базовой имитационной модели сведены в таблицу~\ref{tbl:base_result}.
\begin{table}[h!]
  \hfill
  \caption{Результаты базовой имитационной модели}
  \label{tbl:base_result}
    \begin{tabular}{| l | c |}

      \hline
      \multicolumn{1}{| c |}{Величина} & Значение \\ \hline

      Коэффициент загрузки рабочих & $ 0{,}833 $ \\ \hline

      Коэффициент загрузки стенда проверки узлов & $ 0{,}445 $ \\ \hline

      Среднее время пребывания узлов в цепи пользователя, мин & $ 45{,}947 $ \\ \hline

      Средня длина цепи пользователя, шт & $ 1{,}477 $ \\ \hline

      Число отремонтированных узлов с неисправностью типа $ 1 $, шт & $ 208 $ \\ \hline

      Число отремонтированных узлов с неисправностью типа $ 2 $, шт & $ 431 $ \\ \hline

      Число отремонтированных узлов с неисправностью типа $ 3 $, шт & $ 68 $ \\ \hline

      Число узлов, не подлежащих ремонту, шт & $ 7 $ \\ \hline

    \end{tabular}
\end{table}

Приведенные результаты удобно использовать для сравнения с результатами
имитации модифицированной модели задачи.

\pagebreak
