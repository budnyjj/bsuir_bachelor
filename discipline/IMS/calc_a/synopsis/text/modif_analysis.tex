\section[Анализ результатов модифицированной имитационной модели]
{АНАЛИЗ РЕЗУЛЬТАТОВ МОДИФИЦИРОВАННОЙ \\ ИМИТАЦИОННОЙ МОДЕЛИ}

Увеличение входного потока двигателей позволило добиться требуемого
результата: коэффициент загрузки стенда для проверки отремонтированных узлов
составил $ 0{,}747 $ против прежних $ 0{,}445 $.


Результаты модифицированной имитационной модели представлены в
таблице~\ref{tbl:modified_result}.
\begin{table}[h!]
  \hfill
  % \begin{minipage}{150mm}
  \caption{Результаты модифицированной имитационной модели}
  \label{tbl:modified_result}
    \centering
    \begin{tabular}{| p{7cm} | c | c |}

      \hline
      \multicolumn{1}{| c |}{Величина} &
      \multicolumn{2}{ c |}{Значение} \\
      \cline{2-3}

      & \multicolumn{1}{ c |}{Базовая модель} &
      \multicolumn{1}{>{\centering\arraybackslash}m{4.7cm}|}{Модифицированная модель} \\
      \hline

      Коэффициент загрузки рабочих & $ 0{,}833 $ & $ 0{,}802 $ \\ \hline

      Коэффициент загрузки стенда \newline проверки узлов & $ 0{,}445 $ & $ 0{,}747 $ \\ \hline

      Среднее время пребывания узлов в цепи пользователя, мин & $ 45{,}947 $ & $ 24{,}834 $ \\ \hline

      Средня длина цепи \newline пользователя, шт & $ 1{,}477 $ & $ 4{,}011 $ \\ \hline

      Число отремонтированных \newline узлов с неисправностью типа $ 1 $, шт & $ 208 $ & $ 307 $ \\ \hline

      Число отремонтированных \newline узлов с неисправностью типа $ 2 $, шт & $ 431 $ & $ 729 $ \\ \hline

      Число отремонтированных \newline узлов с неисправностью типа $ 3 $, шт & $ 68 $  & $ 114 $ \\ \hline

      Число узлов, не подлежащих \newline ремонту, шт & $ 7 $ & $ 25 $ \\ \hline

    \end{tabular}
  % \end{minipage}
\end{table}

\newpage

Коэффициент загрузки рабочих удалось сохранить на высоком уровне~($ 0{,}802 $).

Средняя длина цепи пользователя в модифицированной модели составила $ 4{,}011 $,
при этом среднее время пребывания узлов в ней сильно сократилось ---
с $ 45{,}947 $ до $ 24{,}834 $ минут.
Общее число отремонтированных деталей увеличилось примерно в $ 1{,}6 $ раза:
\begin{itemize}
  \item число отремонтированных двигателей с неисправностью первого типа изменилось
    с $ 208 $ до $ 307 $ штук;
  \item число отремонтированных двигателей с неисправностью второго типа изменилось
    с $ 431 $ до $ 729 $ штук;
  \item число отремонтированных двигателей с неисправностью второго типа изменилось
    с $ 68 $ до $ 114 $ штук;
  \item число бракованных двигателей изменилось с $ 7 $ до $ 25 $ штук.
\end{itemize}

Файл статистики модифицированной имитационной модели приведен в приложении~Б.


\pagebreak
