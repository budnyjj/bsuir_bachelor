\section*{ЗАДАНИЕ}
\addcontentsline{toc}{section}{Задание}

Мастерская по ремонту машин состоит из цеха ремонта и цеха контроля. Интервалы
времени между моментами поступления узлов на ремонт --- экспоненциальные случайные
величины; в среднем на ремонт поступает три узла в час.

В ремонтном цехе работают трое рабочих. Из всех узлов, поступающих на ремонт,
$ 30\% $ узлов имеют неисправность типа $1$, для устранения которой требуется
$ 60 \pm 20$ минут; $ 60\% $~--- неисправность типа~$2$~($ 30 \pm 10 $ минут);
$ 10\% $ --- неисправности типа $3$ (время устранения~---~экспоненциальная случайная
величина, в среднем требуется $40$ минут). Устранение неисправностей типа $1$ и $2$
выполняется одним рабочим, неисправности типа $3$~--- двумя рабочими. В первую
очередь ремонтируются узлы, на которые затрачивается меньше времени.

В цехе контроля имеется один стенд для проверки отремонтированных двигателей.
Проверка занимает в среднем восемь минут (время проверки --- экспоненциальная
случайная величина). По результатам проверки $85\%$ узлов признаются исправными,
а $15\%$~--- снова направляются на ремонт. Из узлов, направляемых на повторный
ремонт,~$20\%$ имеют неисправность типа $1$, $70\%$~--- неисправность
типа~$2$,~$10\%$~---~неисправность типа 3. После повторного ремонта снова выполняется
проверка узлов; из них $90\%$ признаются исправными, а $10\%$~--- не подлежащими
ремонту.

Требуется разработать имитационную программу для анализа процесса работы мастерской
в течение тридцати рабочих дней.

\pagebreak
