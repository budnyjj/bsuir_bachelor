\section[Анализ результатов базовой имитационной модели]{
  АНАЛИЗ РЕЗУЛЬТАТОВ БАЗОВОЙ \\
  ИМИТАЦИОННОЙ МОДЕЛИ}
\label{sec:base_analysis}

В качестве результатов работы имитационной модели можно принять
следующие величины: 

\begin{itemize}
  \item коэффициенты загрузки устройств технологической линии;
  \item количество выпущенных изделий.
\end{itemize}

Следует заметить, что вследствие особенностей функционирования модели
коэффициенты загрузки устройств нельзя считать показательными характеристиками:
в каждый момент времени производится обработка только одного изделия на одном из
устройств, при этом остальные устройства простаивают.
По этой причине можно утверждать, что коэффициенты загрузки устройств
всегда будут предсказуемо низкими.

Другая величина --- количество выпущенных изделий --- 
позволяет охарактеризовать общую эффективность производства в достаточной мере. 

Результаты базовой имитационной модели приведены в таблице~\ref{tbl:base_result}.

\begin{table}[h!]
  \hfill
  \begin{minipage}{145mm}
  \caption{Результаты базовой имитационной модели}
  \label{tbl:base_result}
    \begin{tabular}{| l | c |}
      \hline
      \multicolumn{1}{| c |}{Величина} & 
      Значение \\
      \hline

      Коэффициент загрузки первого нагревателя &
      0{,}253 \\
      \hline

      Коэффициент загрузки второго нагревателя &
      0{,}260 \\
      \hline

      Коэффициент загрузки третьего нагревателя &
      0{,}257 \\
      \hline

      Коэффициент загрузки формы &
      0{,}210 \\
      \hline

      Общее число выпущенных изделий, шт &

      153 \\
      \hline

    \end{tabular}
  \end{minipage}
\end{table}

Данные величины удобно использовать для сравнения с результатами имитации
модифицированной модели.
