\section[Описание деталей реализации]{ОПИСАНИЕ ДЕТАЛЕЙ РЕАЛИЗАЦИИ}
\label{sec:realization}

\subsection{Установка окружения разработчика}
\label{ssec:dev_installation}

В этом подразделе приведено описание процесса развертывания рабочего окружения 
для разработки сайта на Drupal: установка Apache, PHP, MySQL.
Процесс настройки FastCGI не рассматривается, так как он является
весьма запутанным, к тому же он достаточно подробно изложен в
интернете, например~\cite{fast_cgi_conf}.
В качестве базовой операционной системы выступает Debian Linux 7.0. 

\paragraph{}
Для создания символьного адреса сайта на локальном компьютере
необходимо изменить файл \textit{/etc/hosts}, как показано на рисунке~\ref{lst:hosts_configuration}.

\begin{lstlisting}[language=bash,
  caption=Содержимое файла \textit{/etc/hosts},
  label=lst:hosts_configuration]
127.0.0.1       localhost.localdomain   localhost
::1             localhost.localdomain   localhost

127.0.0.1       nagrady
\end{lstlisting}

Эти изменения необходимы для того, чтобы по запрос вида http://nagrady/
в адресной строке браузера направлялся на локальный сервер, а не на удаленный.

\paragraph{}
Для инсталляции Drupal необходимы следующие компоненты:

\begin{itemize}
\item
  веб-сервер (для приема и отправки данных);
\item
  база данных (для хранения данных);
\item
  интерпретатор PHP (для обработки данных и генерации контента).
\end{itemize}

В разделе~\ref{sec:choice} было установлено, что в качестве веб-сервера будет
использоваться Apache 2.2, взаимодействующий с PHP через FastCGI,
в качестве базы данных будет выступать MySQL 5.5.
На данный момент последней стабильной версией PHP является PHP 5.4. 

На рисунке~\ref{lst:web_stack_installation} приведены команды установки этих
компонентов в ОС семейства Debian Linux.

\pagebreak

\begin{lstlisting}[language=bash,
  caption=Команды установки веб-стека для работы Drupal,
  label=lst:web_stack_installation]
apt-get install Apache2 libApache2-mod-fcgid
apt-get install php5 php5-curl php5-intl php5-mcrypt \ 
        php5-mysql php5-sqlite php5-xmlrpc php5-gd php5-cgi
apt-get install mysql-server mysql-client
\end{lstlisting}

В процессе установки MySQL потребуется ввести логин и пароль от учетной записи
администратора.

\paragraph{}
Далее необходимо создать виртуальный хост, на котором будет размещаться сайт.
Для этого в директории \textit{/etc/аpache2/sites-available} необходимо создать
файл \textit{nagrady} и наполнить его содержанием, представленным на
рисунке~\ref{lst:Apache_virtual_host}.

\begin{lstlisting}[language=bash,xleftmargin=2em,
  caption=Содержимое конфигурационного файла Apache,
  label=lst:Apache_virtual_host]
NameVirtualHost localhost
<VirtualHost *:80>
    ServerAdmin admin@localhost
    ServerName nagrady
    DocumentRoot /var/www/nagrady
    <Directory /var/www/>
            Options Indexes FollowSymLinks MultiViews
            AllowOverride None
            Order allow,deny
            allow from all
    </Directory>
    ErrorLog ${APACHE_LOG_DIR}/nagrady_error.log
    LogLevel warn
    CustomLog ${APACHE_LOG_DIR}/nagrady_access.log combined
</VirtualHost>
\end{lstlisting}

В первой строке рисунка указывается сетевое имя сервера (\textit{FQDN}). 
Так как разработка будет производиться на том же компьютере, 
где установлен сервер, поэтому в качестве имени выступает \textit{localhost}.
Директива \textit{ServerAdmin} назначает почтовый адрес администратора домена, 
\textit{ServerName} --- адрес, по которому сайт будет доступен.
Директивы, расположенные внутри тега \textit{<Directory>}, применяются только
к содержимому указанной директории.
Они запрещают локальные изменения конфигурации (\textit{AllowOverride None}),
дают доступ к файлам сайта (\textit{allow from all}),
разрешают просматривать содержимое директорий (\textit{Options Indexes}),
а также переходить по символическим ссылкам (\textit{Options FollowSymLinks}).
\textit{DocumentRoot} содержит путь к файлам сайта,
\textit{ErrorLog} и \textit{CustomLog} описывают порядок журналирования
процесса работы сервера.

Кроме этого, необходимо включить модули Apache, которые требуются Drupal для 
нормального функционирования. Команды включения этих модулей и активации 
виртуального домена приведены на рисунке~\ref{lst:Apache_configuration}.

\begin{lstlisting}[language=bash,
  caption=Команды включения дополнительных модулей
  Apache и активации виртуального домена,
  label=lst:Apache_configuration]
a2enmod rewrite
a2enmod fcgid
a2dissite default
a2ensite nagrady
\end{lstlisting}

\paragraph{}
После этого необходимо произвести изменения конфигурации PHP.
Для этого необходимо отредактировать файл \textit{/etc/php5/cgi/php.ini},
внеся в него изменения, приведенные на рисунке~\ref{lst:php_configuration}.

\begin{lstlisting}[language=bash,
  caption=Измененные параметры файла php.ini,
  label=lst:php_configuration]
max_execution_time = 900
max_input_time = 900
memory_limit = 512M
post_max_size = 64M
upload_max_filesize = 64M
max_file_uploads = 32M
default_socket_timeout = 900
\end{lstlisting}

Параметр \textit{max\_execution\_time} устанавливает максимальное 
время работы скрипта,
\textit{default\_socket\_timeout} и \textit{max\_input\_time} ограничивают
максимальное время, в течение которого может происходить ввод данных,  
\textit{memory\_limit} устанавливает лимит на объем оперативной памяти,
используемой интерпретатором,
\textit{post\_max\_size} и \textit{upload\_max\_filesize} ограничивают
максимальный объем загружаемых файлов.

Для применения внесенных изменений необходимо перезагрузить веб-сервер. 
Для этого нужно выполнить команды, приведенные на рисунке~\ref{lst:stack_restart}. 

\begin{lstlisting}[language=bash,
  caption=Команды перезагрузки веб-сервера,
  label=lst:stack_restart]
service php-fastcgi start
service apache2 restart
\end{lstlisting}

\paragraph{}
Для создания базы данных в СУБД MySQL достаточно выполнить команды,
приведенные на рисунке~\ref{lst:mysql_create_db}.

\begin{lstlisting}[language=SQL,alsolanguage=bash,
  caption=Создание базы данных в СУБД MySQL,
  label=lst:mysql_create_db]
mysql -uroot -p<password>
CREATE DATABASE nagrady;
exit;
\end{lstlisting}

Значение \textit{<password>} соответствует паролю администратора СУБД.

\paragraph{}
После установки и настройки частей веб-стека можно
приступить к установке Drush.

\textit{Drush} --- консольная утилита, служащая для автоматизации действий,
выполняемых с сайтом на Drupal~\cite{drush_about}.
На рисунке~\ref{lst:drush_installation} приведена команда установки Drush.

\begin{lstlisting}[language=bash,
  caption=Команда установки Drush,
  label=lst:drush_installation]
apt-get install drush
\end{lstlisting}

После того, как были установлены и настроены Apache, MySQL, PHP и Drush,
можно приступать к установке Drupal.

\subsection{Установка Drupal}
\label{ssec:drupal_installation}

\paragraph{}
После установки Drush установка Drupal является простой задачей.
Достаточно перейти в директорию, где размещаются домены, обслуживаемые сервером,
и набрать команды, приведенные на рисунке~\ref{lst:install_drupal}.

\begin{lstlisting}[language=bash,
  caption=Установки Drupal с помощью Drush,
  label=lst:install_drupal]
drush dl drupal
mv drupal-7.28 nagrady
\end{lstlisting}

Таким образом мы загрузили последнюю стабильную версию Drupal в директорию,
обслуживаемую сервером, следовательно, теперь сайт должен быть доступен через
веб-интерфейс, как показано на рисунке~\ref{fig:drupal_install}.

\begin{figure}[h]
  \centering
  {
    \setlength{\fboxsep}{0pt}%
    \setlength{\fboxrule}{1pt}%
    \fbox{\includegraphics[width=150mm]{pic/drupal_install.png}}
  }
  \caption{Установка Drupal}
  \label{fig:drupal_install}
\end{figure}

В ходе установки нужно будет выбрать профиль установки, 
ввести настройки подключения к базе данных, а также логин и пароль администратора.

После успешной установки главная страница сайта на Drupal имеет вид, 
приведенный на рисунке~\ref{fig:drupal_after_install}.

\begin{figure}[h]
  \centering
  {
    \setlength{\fboxsep}{0pt}%
    \setlength{\fboxrule}{1pt}%
    \fbox{\includegraphics[width=150mm]{pic/drupal_after_install.png}}
  }
  \caption{Drupal после установки}
  \label{fig:drupal_after_install}
\end{figure}

\paragraph{}
Мощность Drupal познается в его модулях. 

\textit{Модуль Drupal} --- самостоятельная сущность, позволяющая заместить либо
расширить стандартную функциональность ядра. 
Модули представляют собой небольшие проекты, которые размещаются на сайте
drupal.org и развиваются силами open-source сообщества~\cite{drupal_modules}.

Скачать модуль можно, скачав с сайта drupal.org, и разместив в директории
sites/all/modules сайта, либо через Drush, воспользовавшись командами,
представленными на рисунке~\ref{lst:drush_modules}.

\begin{lstlisting}[language=bash,
  caption=Установка модуля Drupal с использованием Drush,
  label=lst:drush_modules]
drush dl <module_name>
drush en <module_name>
\end{lstlisting}

В данном случае \textit{<module\_name>} соответсвует системному названию модуля,
узнать которое можно на официальном сайте.
Аналогичным образом устанавливаются темы оформления.

\subsection{Обзор использованных модулей}
\label{ssec:modules_overlook}

В этом разделе перечисляются основные модули Drupal,
которые были установлены и настроены с целью получения требуемой функциональности.
Mодули сгруппированы по категориям; каждый модуль содержит краткое описание функциональности.

\paragraph{}
\textit{Core} --- стандартные модули Drupal. Они входят в комплект стандартной поставки Drupal.

\textit{Block} предоставляет возможность создавать блоки. 
Блок --- это контейнер, в котором размещается контент сайта.
Например, список самых популярных наград на главной странице сайта является блоком.

\textit{Contextual Links} добавляет быстрые ссылки для операций с контентом,
видимые только администраторам.

\textit{Image} используется для обработки изображений: изменение размера, кадрирование.
Позволяет применять шаблоны трансформаций изображений к изображениям,
находящимся в определенном месте сайта. Оригиналы изображений также хранятся на сайте.
Например, оригиналы фотографии медалей имеют различные размеры. 
Данный модуль используется для того, чтобы показывать их в заданном размере
на главной странице, и в другом размере на странице выбранной награды.

\textit{List} предоставляет возможность создания списков через графический интерфейс.

\textit{Locale} предоставляет возможность локализации содержимого сайта.
Drupal изначально имеет интерфейс на английском языке. 
Нам требуется, чтобы наш сайт имел пользовательский интерфейс на беларуском языке.

\textit{Menu} позволяет создавать разнообразные меню сайта из администраторского интерфейса.

\textit{Path} позволяет изменять URL страниц сайта.

\textit{System} обеспечивает поддержку системы конфигурирования
для администраторов.

\textit{Taxonomy} предоставляет возможность разделять контент по казтегориям 
путем назначения терминов таксономии (тегов).

\textit{Text} предоставляет возможность добавления и отображения текстовых полей.

\textit{Update Manager} следит за обновлениями ядра и модулей,
автоматизирует процесс обновления сайта.

\textit{User} --- управляет системой регистрации пользователей, а также входом на сайт.

\paragraph{}
Группа модулей \textit{Administration Menu} добавляет удобное меню для
пользователей с правами администратора.
Состав пунктов меню дублирует иерархию страниц администрирования.

\textit{Chaos Tools} предоставляет низкоуровневые функции для работы с хранением и 
представлением контента. Требуется для установки модулей \textit{Views},
\textit{Display Suite}, \textit{Feeds}.

\textit{Context} используется для изменения внешнего вида отдельной страницы или 
группы страниц в зависимости от определенного условия.
Например, на разрабатываемом сайте этот модуль был использован для размещения блоков
на страницах в зависимости от URL.
  
\textit{Custom breadcrumbs} позволяет задавать содержимое breadcrumbs (<<хлебных крошек>>) 
в зависимости от заданных условий.

\textit{Date} предоставляет функции для работы с датой:
ввод, валидация, хранение, представление.

\textit{Devel} --- группа модулей, используемая для отладки.
Предоставляет API для просмотра содержимого структур данных Drupal и его модулей,
генерации контента, профилирования.

\textit{Display Suite} --- группа модулей, расширяющая возможности
\textit{Field} и \textit{Node} в области отображения контента.

\textit{Meta Tags Quick} используется для автоматического заполнения содержимого
SEO-тегов в зависимости от содержимого страницы.

\textit{Administration Language} предоставляет возможность изменения языка интерфейса для
страниц администратора.

\textit{Localization Update} используется для автоматической загрузки обновления
локализации интерфейса.

\textit{Back To Top} предоставляет всплывающий виджет перемещения вверх по странице.

\textit{Entity API} предоставляет API для работы с сущностями Drupal.
Необходим для установки модулей \textit{Administration Views},
\textit{views bulk operations}.

\textit{Job Scheduler} экслпуатируется в качестве триггера другими модулями.
Требуется для установки группы модулей \textit{Feeds}.

\textit{Menu Item Visibility} управляет показом пунктов меню в зависимости от заданных условий.

\textit{Module Filter} используется для фильтрации модулей по названию на соответвующей странице.

\textit{Pathauto} автоматизирует процесс изменения URL страниц сайта.

\textit{Token} предоставляет возможность использования переменных,
определенных в частях Drupal, при работе с контентом.

\textit{File Cache} используется для хранения кэша в файловой системе,
тем самым снимая часть нагрузки с базы данных.

\textit{Taxonomy Manager} предоставляет расширенный интерфейс для работы с таксономией.

\textit{CKEditor} --- WYSIWYG редактор контента.

\paragraph{}
\textit{Feeds} --- группа модулей, предоставляющих возможность импорта данных 
из заданного источника.

\textit{Feeds Admin UI} --- предоставляет графический интерфейс \textit{Feeds} для администратора.

\textit{Feeds Tamper} --- управляет отображением полей источника на поля контента Drupal. 

\textit{Feeds Tamper Admin UI} --- предоставляет графический интерфейс
\textit{feeds tamper} для администратора.

\textit{Feeds Admin PHP} --- предоставляет возможность использования кода PHP для 
для отображения данных.

Модули этой группы используются для отображения содержимого файлов awards.csv и awarded.csv 
на содержимое типа контента <<награжденный>> и <<награда>> соответственно. 

\paragraph{}
\textit{Views} --- группа модулей, ответственная за представление контента
в разнообразной форме.

\textit{Views Autocomplete Filters} добавляет возможность автодополнения
к полям фильтров Views.

\textit{Views Data Export} --- используется для экспорта данных в машиночитаемом формате.
Например, данный модуль используется для экспорта данных о награжденных в форматах csv и xml.

\textit{Views Dataviz} --- используется для визулизации численных данных.

\textit{Views Date Format SQL} --- предоставляет возможность задания формата даты средствами
языка SQL. Необходим для корректного функционирования модуля \textit{Views Dataviz}.

\textit{Views UI} --- графический интерфейс для администрирования \textit{Views}.

\subsection{Настройка типов контента}
\label{ssec:content_types_setup}

В этом подразделе рассматривается процесс настройки типов контента Drupal для хранения
данных о награжденных и наградах.

В соответствии с подразделом~\ref{ssub:db_info_stage}, информация о различных 
объектах предметной области должна храниться в отдельных сущностях.
В Drupal таким сущностям соответсвуют типы контента либо термины таксономии.

\textit{Тип контента} представляет собой шаблон хранимых и представляемых данных и
характеризуется определенным набором полей, задаваемых при создании.

Всякая страница, содержащая контент, который имеент тип,
назывется \textit{нодой} (англ. \textit{node}).

\textit{Словарь таксономии} определяет шаблон \textit{терминов таксономии}, которые
могут быть связаны с нодами.

Объекту предметной области <<факт награждения>> соответствует тип контента <<узнагароджаны>>,
представленный на рисунке~\ref{fig:awarded_content_type}.

\begin{figure}[h]
  \centering
  \includegraphics[width=160mm]{pic/awarded_content_type.png}
  \caption{Тип контента <<узнагароджаны>>}
  \label{fig:awarded_content_type}
\end{figure}

Объекту предметной области <<награда>> соответствует словарь таксономии <<узнагароды>>,
представленный на рисунке~\ref{fig:awards_taxonomy}.

\begin{figure}[h]
  \centering
  \includegraphics[width=150mm]{pic/awards_taxonomy.png}
  \caption{Словарь таксономии <<узнагароды>>}
  \label{fig:awards_taxonomy}
\end{figure}

Таким образом, термин таксономии, принадлежащий словарю <<узнагароды>>, может
быть связан с любой нодой типа <<узнагароджаны>>.

\subsection{Настройка импорта данных}
\label{ssec:import_setup}

В этом подразделе рассматриваются отображения содержимого входных файлов 
\textit{awarded.csv} и \textit{awards.csv} на тип контента <<узнагароджаны>>
и словарь таксономии <<узнагароды>> соответственно.

Для импорта содержимого входных файлов в базу данных используется модули группы
\textit{Feeds}. 

На рисунке~\ref{fig:awards_map} приведена схема отображения данных файла
\textit{awards.csv} на поля словаря таксономии <<узнагароды>>.

\begin{figure}[h]
  \centering
  \includegraphics[width=150mm]{pic/awards_map.png}
  \caption{Схема импорта данных о наградах}
  \label{fig:awards_map}
\end{figure}

На рисунке~\ref{fig:awarded_map} приведена схема отображения данных файла
\textit{awarded.csv} на поля типа контента <<узнагароджаны>>.
Поле \textit{tip\_nagrady} используется в качестве ключа и состоит из полей
\textit{fio}, \textit{nagrada}, \textit{nomer\_dokument} входного файла.

\begin{figure}[h]
  \centering
  \includegraphics[width=150mm]{pic/awarded_map.png}
  \caption{Схема импорта данных о награжденных}
  \label{fig:awarded_map}
\end{figure}

Так как ноды типа <<узнагароджаны>> ссылаются на <<узнагароды>>, то
необходимо сначала импортировать данные о наградах, а затем --- о награжденных.

\subsection{Настройка экспорта данных}
\label{ssec:export_setup}

В этом подразделе рассматривается настройка экспорта данных о награжденных в 
машиночитаемом формате.

Для экспорта данных в машиночитаемом виде используется модули
\textit{Views} и \textit{Views Data Export}.
Модуль \textit{Views} используется для вывода контента, имеет множество настроек и 
плагинов. Каждый набор настроек вывода называется \textit{видом}.

Настройки страницы экспорта данных о награжденных приведены
на рисунке~\ref{fig:awarded_export}.

\begin{figure}[h]
  \centering
  {
    \setlength{\fboxsep}{0pt}%
    \setlength{\fboxrule}{1pt}%
    \fbox{  \includegraphics[width=150mm]{pic/awarded_export.png}}
  }
  \caption{Настройки экспорта данных о награжденных}
  \label{fig:awarded_export}
\end{figure}

Данный вид позволяет пользователям производить отбор данных о награжденных
по награде, а также по дате награждения, а также загружать данные о награжденных
на локальный носитель.
