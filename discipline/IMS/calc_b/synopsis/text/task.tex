\section*{ЗАДАНИЕ}
\addcontentsline{toc}{section}{Задание}

Цех выпускает пластмассовые изделия.
Сырье (пластмасса) хранится в бункере.
Ёмкость бункера --- 20 кг. На каждое изделие расходуется 1 кг пластмассы.
После выхода из бункера пластмасса последовательно пропускается 
через три нагревателя.
Нагрев пластмассы на каждом нагревателе занимает от 6 до 14 минут.
Нагретая пластмасса подается в форму для изготовления готового изделия.
Изделие извлекается из формы после остывания до установленной температуры;
в среднем это занимает 8 минут (экспоненциальная случайная величина).

Очередная порция пластмассы выпускается из бункера только после
окончания изготовления предыдущего изделия (т.~е. после его извлечения из формы).
Это связано с тем, что в процессе обработки пластмасса нигде не должна
пролеживать (иначе она остывает и бракуется).

При снижении запаса пластмассы в бункере до 3 кг происходит его заполнение.
Оно занимает от 10 до 20 минут.
В это время пластмасса из бункера не извлекается.

Разработать имитационную программу для анализа процесса работы
участка в течение 100 часов.