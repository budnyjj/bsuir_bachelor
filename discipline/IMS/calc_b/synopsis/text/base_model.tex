\section[Построение базовой имитационной модели]{
  ПОСТРОЕНИЕ БАЗОВОЙ ИМИТАЦИОННОЙ \\ 
  МОДЕЛИ}

Имитационная модель рассматриваемого объекта моделирования
представляет собой GPSS-модель замкнутого типа, исходный текст которой приведен
в приложении~А.

Рассмотрим последовательно основные части данной модели.

Переменная \textbf{MIN\_VAL} определяет точку заказа --- 
максимальный уровень заполнения бункера, при котором происходит
его заполнение.

Переменная \textbf{ZAPAS} определяет текущий уровень заполнения бункера.

Команда \textbf{GENERATE ,,,1} создаёт одну заявку.

Команда \textbf{TEST LE X\$ZAPAS,X\$MIN\_VAL,HEAT} производит сравнение
текущего уровня заполнения с пороговым уровнем. 
В случае, если текущий уровень заполнения бункера меньше порогового значения,
производится заполнение бункера 
(команды \textbf{ADVANCE 15,5} и \textbf{SAVEVALUE ZAPAS,20}),
иначе производится переход на метку \textbf{HEAT},
означающую начало процесса производства изделия.

Процесс производства изделия состоит из трех последовательных нагреваний
(устройства \textbf{HEAT\_1}, \textbf{HEAT\_2}, \textbf{HEAT\_3}) и
охлаждения в форме (устройство \textbf{COOL}).

После выхода изделия из формы производится увеличение счетчика готовых изделий
(команда \textbf{SAVEVALUE FINISH+,1}) и безусловный переход на начало обработки
(команда \textbf{TRANSFER ,NEW}), соответствующий началу выпуска очередного
изделия.

Команда \textbf{GENERATE 6000} предназначена для имитации процесса выпуска 
изделий в течение ста часов (шести тысяч минут).
