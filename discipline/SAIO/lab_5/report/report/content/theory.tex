%!TEX root = ../../lab.tex

\section{ТЕОРЕТИЧЕСКИЕ СВЕДЕНИЯ}

Задача о планировании вклада средств в производство с целью получения максимального дохода относится к типичной многоэтапной операции, которую можно планировать с помощью аппарата динамического программирования. Алгоритм динамического программирования включает следующие пункты.

\begin{enumerate}
	\item На выбранном шаге задаем набор (определяется условиями --- ограничениями) значений переменной, характеризующей последний шаг, возможные состояния системы на предпоследнем шаге. Для каждого возможного состояния и каждого значения выбранной переменной вычисляем значения целевой функции. Из них для каждого исхода предпоследнего шага выбираем оптимальные значения целевой функции и соответствующие им значения рассматриваемой переменной. Для каждого исхода предпоследнего шага запоминаем оптимальное(ые) значение(я) переменной и соответствующее значение целевой функции.
	\item Переходим к оптимизации на следующем шаге при любом значении новой переменной и при оптимальных значениях следующих переменных. Оптимальное значение целевой функции на последующих шагах (при оптимальных значениях последующих переменных) считывают из предыдущей таблицы. Если новая переменная характеризует первый шаг, то переходим к п.3. В противном случае повторяем п.2 для следующей переменной.
	\item При заданном в задаче исходном условии для каждого возможного значения первой переменной вычисляем значение целевой функции. Выбираем оптимальное значение целевой функции и соответствующее(ие) оптимальное(ые) значение(ия) первой переменной.
	\item При известном оптимальном значении первой переменной определяем исходные данные для следующего(второго) шага и по последней таблице --- оптимальное(ые) значение(ия) следующей (второй) переменной.
\end{enumerate}

\newpage