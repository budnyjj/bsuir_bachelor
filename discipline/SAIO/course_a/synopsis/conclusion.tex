\section*{ЗАКЛЮЧЕНИЕ}
\addcontentsline{toc}{section}{Заключение}

В процессе разработки оптимального плана развития сельскохозяйственной отрасли было установлено, что для обеспечения максимальной прибыли от продажи урожая хлопка необходимо засеять:
\begin{itemize}

\item  1~млн~гапесчаных почв сортом хлопка ,,Шемаха``;

\item  400~тыс.~га песчаных почв сортом хлопка ,,Зеравшан``;

\item  1,2~млн~га глинистых почв сортом хлопка ,,Эльтон``.

\end{itemize}

Максимальная прибыль составит 15,2~млрд~ден.~ед., однако в данном случае останется не задействован 1~млн человек.

В ходе решения данной проблемы была построена модифицированная модель задачи, в которой предложено увеличить площадь засеиваемых глинистых земель с 1,2~млн~га до 1,4~млн~га. В таком случае проблема с незадействованностью трудовых ресурсов решена, а прибыль от продажи урожая хлопка увеличилась и составила 15,4~млрд~ден.~ед.

Таким образом рекомендуется обратить внимание на предложенный модифицированный план с целью улучшения показателей деятельности отрасли.
\pagebreak