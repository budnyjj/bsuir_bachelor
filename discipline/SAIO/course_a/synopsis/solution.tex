\section[РЕШЕНИЕ ЗАДАЧИ ОПТИМИЗАЦИИ НА ОСНОВЕ \\ СИМПЛЕКС-МЕТОДА]{РЕШЕНИЕ ЗАДАЧИ ОПТИМИЗАЦИИ \\ НА ОСНОВЕ СИМПЛЕКС-МЕТОДА}

Приведём математическую модель задачи к стандартной форме. Для этого в ограничения ,,меньше или равно`` введём остаточные переменные. Целевая функция при этом останется без изменений. Запишем полученную систему:
\begin{equation}
\label{eq:SystemInStandartForm}
	\begin{aligned}
  	&\left\{
    	\begin{aligned}
	      x_1 & + x_2 + x_3 + x_7 = 1 400 000 \\
	      x_4 & + x_5 + x_6 + x_8 = 1 200 000 \\
	      20x_1 & + 30x_2 + 30x_3 + 30x_4 + 30x_5 + 20x_6 + x_9 = 56 000 000 \\
	      2x_1 & + x_2 + 3x_3 + 2x_4 + x_5 + 3x_6 + x_{10} = 7 000 000 \\
	      x_i &\ge 0, i = 1, \ldots ,10  
    	\end{aligned}
  	\right.
  	\\
  	& \hspace{10mm} E = 6x_1 + 8x_2 + 4x_3 + 6x_4 + 5x_5 + 5x_6 \rightarrow \max
	\end{aligned}
\end{equation}

В каждом из ограничений полученной системы присутствует базисная переменная (т.~е. переменная, входящая только в данное ограничение с коэффициентом, равным единице). Рассмотрим их физический смысл:

\begin{itemize}

\item
  $ x_{7} $ --- \textsl{площадь} неизрасходованной \textsl{песчаной почвы},
\item
  $ x_{8} $ --- \textsl{площадь} неизрасходованной \textsl{глинистой почвы},
\item
  $ x_{9} $ --- \textsl{объём} неизрасходованной \textsl{воды},
\item
  $ x_{10} $ --- \textsl{количество} незадействованных \textsl{человек}.

\end{itemize}

Все остальные переменные в системе~(\ref{eq:SystemInStandartForm}) --- небазисные, они равны нулю. Полученное начальное решение задачи отражено в таблице~\ref{tbl:solution_first}.

\renewcommand{\tabcolsep}{0.48em}
\begin{table}[h]
  \caption{Начальное решение задачи\label{tbl:solution_first}}
  \centering
	  \begin{tabular}{{|c}*{10}{|c}}
      \hline
      $x_1$ & $x_2$ & $x_3$ & $x_4$ & $x_5$ & $x_6$ & $x_7$ & $x_8$ & $x_9$ & $x_{10}$ \\ \hline
      0 & 0 & 0 & 0 & 0 & 0 & 1 400 000 & 1 200 000 & 56 000 000 & 7 000 000 \\ \hline
    \end{tabular}
\end{table}

Данное решение является допустимым, так как соответствует системе ограничений. Решение не является оптимальным, так как целевая функция~(\ref{eq:mainFunc}) при данных $ x_i , i = 1,...,6 $ равна нулю. Это говорит о том, что под посев хлопка не будет выделено земель.

Для дальнейшего решения задачи составим первую симплекс-таблицу:

\renewcommand{\tabcolsep}{0.62em}
\begin{table}[h]
  \caption{Первая симплекс-таблица\label{tbl:simplex_first}}
  \centering
    \begin{tabular}{{|c}*{12}{|c}}
      \hline
      Базис & $x_1$ & $x_2$ & $x_3$ & $x_4$ & $x_5$ & $x_6$ & $x_7$ & $x_8$ & $x_9$ & $x_{10}$ & Решение \\ \hline
      
      $ E $ & -6 & -8 & -4 & -6 & -5 & -5 & 0 & 0 & 0 & 0 & 0 \\ \hline
      
      $ x_7 $ & 1 & 1 & 1 & 0 & 0 & 0 & 1 & 0 & 0 & 0 & 1 400 000 \\ \hline

      $ x_8 $ & 0 & 0 & 0 & 1 & 1 & 1 & 0 & 1 & 0 & 0 & 1 200 000 \\ \hline

      $ x_9 $ & 20 & 30 & 30 & 30 & 30 & 20 & 0 & 0 & 1 & 0 & 56 000 000 \\ \hline

      $ x_{10} $ & 2 & 1 & 3 & 2 & 1 & 3 & 0 & 0 & 0 & 1 & 7 000 000 \\ \hline
    \end{tabular}
\end{table}

Неоптимальность начального решения подтвеждается наличием отрицательных элементов в строке целевой функции первой симплекс-таблицы.

Произведем некоторые преобразования в симплекс-таблице. Включаем в базис переменную $ x_2 $, так как ей соответствует максимальный по модулю отрицательный элемент в строке целевой функции первой симплекс-таблицы. \textsl{Столбец $ x_2 $ становится ведущим.} Для определения переменной, исключаемой из базиса, вычисляются симплексные отношения:
\begin{equation}
  \begin{aligned}
    &1 400 000 / 1 = 1 400 000, \\
    &1 200 000 / 0 = +\infty, \\
    &56 000 000 / 30 = 1 866 666 \\
    &7 000 000 / 1 = 7 000 000. \\
  \end{aligned}
\end{equation}

Минимальное симплексное отношение $ 1 400 000 / 1 = 1 400 000 $ соответствует переменной $ x_7 $, значит, эта переменная исключается из базиса. \textsl{Строка $ x_7 $ становится ведущей. Ведущий элемент --- 1.} После оставшихся преобразований по правилам симплекс-метода будет получена новая симплекс-таблица (таблица~\ref{tbl:simplex_second})
\pagebreak
\begin{table}[h]
  \caption{Вторая симплекс-таблица\label{tbl:simplex_second}}
  \centering
    \begin{tabular}{{|c}*{12}{|c}}
      \hline
      Базис & $x_1$ & $x_2$ & $x_3$ & $x_4$ & $x_5$ & $x_6$ & $x_7$ & $x_8$ & $x_9$ & $x_{10}$ & Решение \\ \hline
      
      $ E $ & 2 & 0 & 4 & -6 & -5 & -5 & 8 & 0 & 0 & 0 & 11 200 000 \\ \hline
      
      $ x_2 $ & 1 & 1 & 1 & 0 & 0 & 0 & 1 & 0 & 0 & 0 & 1 400 000 \\ \hline

      $ x_8 $ & 0 & 0 & 0 & 1 & 1 & 1 & 0 & 1 & 0 & 0 & 1 200 000 \\ \hline

      $ x_9 $ & -10 & 0 & 0 & 30 & 30 & 20 & -30 & 0 & 1 & 0 & 14 000 000 \\ \hline

      $ x_{10} $ & 1 & 0 & 2 & 2 & 1 & 3 & -1 & 0 & 0 & 1 & 5 600 000 \\ \hline
    \end{tabular}
\end{table}

Из таблицы~\ref{tbl:simplex_second} видно, что полученное решение является допустимым, но не оптимальном, так как в строке целевой функции всё ещё есть отрицательные элементы.

Далее в базис включается переменная $ x_4 $, так как ей соответствует минимальный по модулю отрицательный элемент в строке целевой функции. \textsl{Столбец $ x_4 $ становится ведущим.}
Затем найдем симплексные отношения: 
\begin{equation}
  \begin{aligned}
    & 1 400 000 / 0 = +\infty, \\
    & 1 200 000 / 1 = 1 200 000, \\
    & 14 000 000 / 30 = 466 666, \\
    & 5 600 000 / 1 = 5 600 000. \\
  \end{aligned}
\end{equation}

Из базиса исключается переменная $ x_9 $, так как ей соответствует минимальное симплексное отношение. \textsl{Строка $ x_9 $ становится ведущей. Ведущий элемент --- 30.} В результате оставшихся преобразований по правилам симплекс-метода будет получена новая симлекс-таблица (таблица~\ref{tbl:simplex_third})

\renewcommand{\tabcolsep}{0.5em}
\begin{table}[h]
  \caption{Третья симплекс-таблица\label{tbl:simplex_third}}
  \centering
    \begin{tabular}{{|c}*{12}{|c}}
      \hline
      Базис & $x_1$ & $x_2$ & $x_3$ & $x_4$ & $x_5$ & $x_6$ & $x_7$ & $x_8$ & $x_9$ & $x_{10}$ & Решение \\ \hline
      
      $ E $ & 0 & 0 & 4 & 0 & 1 & -1 & 2 & 0 & 0,2 & 0 & 14 000 000 \\ \hline
      
      $ x_2 $ & 1 & 1 & 1 & 0 & 0 & 0 & 1 & 0 & 0 & 0 & 1 400 000 \\ \hline

      $ x_8 $ & 0,33 & 0 & 0 & 0 & 0 & 0,33 & 1 & 1 & -0,03 & 0 & 733 333 \\ \hline

      $ x_4 $ & -0,33 & 0 & 0 & 1 & 1 & 0,67 & -1 & 0 & 0,03 & 0 & 466 666 \\ \hline

      $ x_{10} $ & 1,67 & 0 & 2 & 0 & -1 & 1,67 & 1 & 0 & -0,07 & 1 & 4 666 666 \\ \hline
      \end{tabular}
\end{table}

Из таблицы~\ref{tbl:simplex_third} видно, что полученное решение является допустимым, но не оптимальном, так как в строке целевой функции всё ещё есть отрицательные элементы.

Далее в базис включается переменная $ x_6 $, так как ей соответствует минимальный по модулю отрицательный элемент в строке целевой функции. \textsl{Столбец $ x_6 $ становится ведущим.}
Затем найдем симплексные отношения: 
\begin{equation}
  \begin{aligned}
    & 1 400 000 / 0 = +\infty, \\
    & 733 333 / 0,33 = 2 222 222, \\
    & 466 666 / 0,67 = 700 000, \\
    & 4 666 666 / 1,67 = 2 800 000. \\
  \end{aligned}
\end{equation}

Из базиса исключается переменная $ x_4 $, так как ей соответствует минимальное симплексное отношение. \textsl{Строка $ x_4 $ становится ведущей. Ведущий элемент --- 0,67.} В результате оставшихся преобразований по правилам симплекс-метода будет получена новая симлекс-таблица (таблица~\ref{tbl:simplex_fourth})

\begin{table}[h]
  \caption{Четвёртая симплекс-таблица\label{tbl:simplex_fourth}}
  \centering
    \begin{tabular}{{|c}*{12}{|c}}
      \hline
      Базис & $x_1$ & $x_2$ & $x_3$ & $x_4$ & $x_5$ & $x_6$ & $x_7$ & $x_8$ & $x_9$ & $x_{10}$ & Решение \\ \hline
      
      $ E $ & -0,5 & 0 & 4 & 1,5 & 2,5 & 0 & 0,5 & 0 & 0,25 & 0 & 14 700 000 \\ \hline
      
      $ x_2 $ & 1 & 1 & 1 & 0 & 0 & 0 & 1 & 0 & 0 & 0 & 1 400 000 \\ \hline

      $ x_8 $ & 0,5 & 0 & 0 & -0,5 & -0,5 & 0 & 1,5 & 1 & -0,05 & 0 & 500 000 \\ \hline

      $ x_6 $ & -0,5 & 0 & 0 & 1,5 & 1,5 & 1 & -1,5 & 0 & 0,05 & 0 & 700 000 \\ \hline

      $ x_{10} $ & 2,5 & 0 & 2 & -2,5 & -3,5 & 0 & 3,5 & 0 & -0,15 & 1 & 3 500 000 \\ \hline
    \end{tabular}
\end{table}

Из таблицы~\ref{tbl:simplex_fourth} видно, что полученное решение является допустимым, но не оптимальном, так как в строке целевой функции всё ещё есть отрицательные элементы.

Далее в базис включается переменная $ x_1 $, так как ей соответствует минимальный по модулю отрицательный элементы в строке целевой функции. \textsl{Столбец $ x_1 $ становится ведущим.}
Найдем симплексные отношения (вычисляются только для положительных коэффициентов ведущего столбца): 
\begin{equation}
  \begin{aligned}
    & 1 400 000 / 1 = 1 400 000, \\
    & 500 000 / 0,5 = 1 000 000, \\
    & 3 500 000 / 2,5 = 1 400 000. \\
  \end{aligned}
\end{equation}

Из базиса исключается переменная $ x_8 $, так как ей соответствует минимальное симплексное отношение. \textsl{Строка $ x_8 $ становится ведущей. Ведущий элемент --- 0,5.} В результате оставшихся преобразований по правилам симплекс-метода будет получена новая симлекс-таблица (таблица~\ref{tbl:simplex_optimal})

\renewcommand{\tabcolsep}{0.60em}
\begin{table}[h]
  \caption{Пятая симплекс-таблица (оптимальная)
  \label{tbl:simplex_optimal}}
  \centering
    \begin{tabular}{{|c}*{12}{|c}}
      \hline
      Базис & $x_1$ & $x_2$ & $x_3$ & $x_4$ & $x_5$ & $x_6$ & $x_7$ & $x_8$ & $x_9$ & $x_{10}$ & Решение \\ \hline
      
      $ E $ & 0 & 0 & 4 & 1 & 2 & 0 & 2 & 1 & 0,2 & 0 & 15 200 000 \\ \hline
      
      $ x_2 $ & 0 & 1 & 1 & 1 & 1 & 0 & -2 & -2 & 0,1 & 0 & 400 000 \\ \hline

      $ x_1 $ & 1 & 0 & 0 & -1 & -1 & 0 & 3 & 2 & -0,1 & 0 & 1 000 000 \\ \hline

      $ x_6 $ & 0 & 0 & 0 & 1 & 1 & 1 & 0 & 1 & 0 & 0 & 1 200 000 \\ \hline

      $ x_{10} $ & 0 & 0 & 2 & 0 & -1 & 0 & -4 & -5 & 0,1 & 1 & 1 000 000 \\ \hline
    \end{tabular}
\end{table}

Получено оптимальное решение (признак его оптимальности --- отсутствие отрицательных элементов в строке целевой функции). Покажем полученное оптимальное решение в отдельной таблице:

\renewcommand{\tabcolsep}{0.54em}
\begin{table}[h]
  \caption{Оптимальное решение задачи\label{tbl:optimal_solution}}
  \centering
    \begin{tabular}{{|c}*{10}{|c}}
      \hline
      $x_1$ & $x_2$ & $x_3$ & $x_4$ & $x_5$ & $x_6$ & $x_7$ & $x_8$ & $x_9$ & $x_{10}$ \\ \hline
      1 000 000 & 400 000 & 0 & 0 & 0 & 1 200 000 & 0 & 0 & 0 & 1 000 000 \\ \hline
    \end{tabular}
\end{table}

\pagebreak

Таким образом для получения максимальной прибыли от продажи урожая хлопка необходимо:

\begin{itemize}

\item 1~млн гектар \textsl{песчаных почв} засеять хлопком сорта \textsl{,,Шемаха``},

\item 400~тыс. гектар \textsl{песчаных почв} засеять хлопком сорта \textsl{,,Зеравшан``},

\item 1,2~млн гектар \textsl{глинистых почв} засеять хлопком сорта \textsl{,,Эльтон``}.

\end{itemize}

\textbf{Прибыль в таком случае составит 15,2~млрд денежных единиц.}

Отдельно рассмотрим физический смысл остаточных переменных:

\begin{itemize}

\item $ x_{10} = 1 000 000 $ означает, что 1~млн человек не будет задействовано,

\item $ x_7 = 0 $ означает, что при посеве хлопка будет задействована вся песчаная почва,

\item $ x_8 = 0 $ означает, что при посеве хлопка будет задействована вся глинистая почва,

\item $ x_9 = 0 $ означает, что при посеве хлопка будет израсходована вся вода.

\end{itemize}

\pagebreak