\section[ПОСТРОЕНИЕ МОДИФИЦИРОВАННОЙ АНАЛИТИЧЕСКОЙ \\ МОДЕЛИ И АНАЛИЗ РЕЗУЛЬТАТОВ МОДИФИКАЦИИ]{ПОСТРОЕНИЕ МОДИФИЦИРОВАННОЙ \\ АНАЛИТИЧЕСКОЙ МОДЕЛИ И АНАЛИЗ \\ РЕЗУЛЬТАТОВ МОДИФИКАЦИИ}

Проанализировав результаты решения задачи оптимизации, можно выделить следующий недостаток при составлении плана засеивания земель: $ 1 $~млн имеющихся трудовых ресурсов не задействованы. В зависимости от конкретных условий этот недостаток может устраняться по-разному.

\subsection{Обеспечение полного использования запаса трудовых ресурсов}

Для того, чтобы увеличить количество людей, задействованных в обработке земли, следует увеличить количество используемой под посевы хлопка земли.

Предположим, например, что увеличиться количество пригодных для посева \textsl{глинистых земель} с $ 1,2 $~млн~га до $ 1,4 $~млн~га. Внесём соответствующее изменение в правую чать ограничения математической модели~\eqref{eq:base_model}. Получим следующую математическую модель:

\begin{equation}
  \label{eq:base_model_modif}
	\begin{aligned}
  	&\left\{
    	\begin{aligned}
	      x_1 & + x_2 + x_3 \le 1 400 000 \\
	      x_4 & + x_5 + x_6 \le 1 400 000 \\
	      20x_1 & + 30x_2 + 30x_3 + 30x_4 + 30x_5 + 20x_6 \le 56 000 000 \\
	      2x_1 & + x_2 + 3x_3 + 2x_4 + x_5 + 3x_6 \le 7 000 000 \\
	      x_i &\ge 0, i = 1, \ldots ,6  
    	\end{aligned}
  	\right.
  	\\
  	& \hspace{10mm} E = 6x_1 + 8x_2 + 4x_3 + 6x_4 + 5x_5 + 5x_6 \rightarrow \max
	\end{aligned}
\end{equation}

Решив данную задачу получим следующее решение:
\begin{equation}
	\begin{aligned}
		x_1 &= 1 400 000, \\
		x_2 &= 0, \\
		x_3 &= 0, \\
		x_4 &= 0, \\
		x_5 &= 0, \\
		x_6 &= 1 400 000.
	\end{aligned}
\end{equation}

В данном случае будут задействованы все виды ресурсов, включая трудовые.

Сравнительная характеристика начального и модифицированного плана приведена в таблице~\ref{tbl:modif}.

\renewcommand{\tabcolsep}{0.5em}

\begin{table}[h]
  \caption{\label{tbl:modif}}
  \centering
    \begin{tabular}{|c|c|c|c|}
      \hline
      
      \multicolumn{2}{|c|}{Показатели} & \centering Базовый план & Модифицированный план  \\ \hline 
      \multirow{3}{3.5cm}{Необходимо засеять на песчаной почве,~га} & ,,Шемаха`` & 1 000 000 & 1 400 000 \\
      & ,,Зеравшан`` & 0 & 0 \\
      & ,,Эльтон`` & 400 000 & 0 \\  \hline

      \multirow{3}{3.5cm}{Необходимо засеять на глинистой почве,~га} & ,,Шемаха`` & 0 & 0 \\
      & ,,Зеравшан`` & 0 & 0 \\
      & ,,Эльтон`` & 1 200 000 & 1 400 000 \\  \hline

			\multicolumn{2}{|p{6cm}|}{Остаток незадействованных трудовых ресурсов, человек} & 1 000 000 & 0 \\ \hline     
      
      \multicolumn{2}{|p{6cm}|}{Израсходовано воды	на орошение, $\text{м}^3$ } & 56 000 000 & 56 000 000 \\ \hline

    	\multicolumn{2}{|p{6cm}|}{Прибыль, тыс.~ден.~ед.} & 15 200 000 & 15 400 000 \\

      \hline
      \end{tabular}
\end{table}

Данный модифицированный план позволяет обеспечить полное использование ресурсов и, соответсвенно, увеличить прибыль.

\pagebreak