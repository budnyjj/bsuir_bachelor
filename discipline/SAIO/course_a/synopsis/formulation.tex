\section{ПОСТАНОВКА ЗАДАЧИ ОПТИМИЗАЦИИ}

Составляется план развития сельского хозяйства в некоторой стране, где основная сельскохозяйственная культура --- хлопок. В стране выращиваются три сорта  хлопка: ,,Шемаха``, ,,Зеравшан`` и ,,Эльтон``. Для  выращивания  хлопка пригодны 1,4 млн га песчаных почв и 1,2 млн га глинистых почв. Прибыль от  продажи урожая хлопка каждого сорта, выращенного на землях с различными почвами, приведена в таблице~\ref{tbl:formulation_first}.

\renewcommand{\tabcolsep}{1.6em}
\renewcommand{\arraystretch}{1.3}
\begin{table}[h]
  \caption{Прибыль от продажи урожая хлопка\label{tbl:formulation_first}}
  \centering
    \begin{tabular}{|c|c|c|}
      \hline
      \multirow{2}*{Сорт хлопка} & \multicolumn{2}{c|}{Прибыль от продажи урожая с 1 га, тыс.~ден.~ед.} \\ \cline{2-3}
      & \hspace{3mm} песчаные почвы \hspace{3mm} & глинистые почвы \\ \hline
      ,,Шемаха`` & 6 & 6 \\ \hline
      ,,Зеравшан`` & 8 & 5 \\ \hline
      ,,Эльтон`` & 4 & 5 \\ \hline 
    \end{tabular}
\end{table}

Для выращивания хлопка требуется орошение. Имеющаяся ирригационная система обеспечивает не более 56 млн $ \text{м}^3 $ воды в год. Величины расхода воды на орошение одного гектара земли при выращивании хлопка различных сортов приведены в таблице~\ref{tbl:formulation_second}.

\renewcommand{\tabcolsep}{2.1em}
\begin{table}[h]
  \caption{ Расход воды на орошение одного гектара\label{tbl:formulation_second}}
  \begin{center}
    \begin{tabular}{|c|c|c|}
      \hline
      \multirow{2}*{Сорт хлопка} & \multicolumn{2}{c|}{Расход воды на орошение 1 га, $ \text{м}^3 $} \\ \cline{2-3}
      & песчаные почвы & глинистые почвы \\ \hline
      ,,Шемаха`` & 20 & 30 \\ \hline
      ,,Зеравшан`` & 30 & 30 \\ \hline
      ,,Эльтон`` & 30 & 20 \\ \hline 
    \end{tabular}
  \end{center}
\end{table}
\renewcommand{\arraystretch}{1.5}

Население, занятое в сельском хозяйстве, составляет 7 млн человек. Для выращивания хлопка сорта ,,Шемаха`` требуются 2 человека на каждый гектар, для хлопка сорта ,,Зеравшан`` --- 1 человек на гектар, для хлопка сорта ,,Эльтон`` --- 3 человека на гектар.

Составить план использования земель для выращивания хлопка, обеспечивающий получение максимальной прибыли.

\pagebreak