\section{ПОСТРОЕНИЕ БАЗОВОЙ АНАЛИТИЧЕСКОЙ МОДЕЛИ}

Требуется составить план засеивания песчаной и глинистой почвы тремя сортами хлопка: ,,Шемаха``, ,,Зеравшан`` и ,,Эльтон`` , обеспечивающий максимальную прибыль от продажи урожая.

Для построения математической модели задачи введём следующие переменные:

\begin{itemize}

\item
  $ x_{1} $ --- площадь \textsl{песчаной почвы}, используемой под засеивание хлопка сорта \textsl{,,Шемаха``},
\item
  $ x_{2} $ --- площадь \textsl{песчаной почвы}, используемая под засеивание хлопка сорта \textsl{,,Зеравшан``},
\item
  $ x_{3} $ --- площадь \textsl{песчаной почвы}, используемая под засеивание хлопка сорта \textsl{,,Эльтон``},
\item
  $ x_{4} $ --- площадь \textsl{глинистой почвы}, используемой под засеивание хлопка сорта \textsl{,,Шемаха``},
\item
  $ x_{5} $ --- площадь \textsl{глинистой почвы}, используемая под засеивание хлопка сорта \textsl{,,Зеравшан``},
\item
  $ x_{6} $ --- площадь \textsl{глинистой почвы}, используемая под засеивание хлопка сорта \textsl{,,Эльтон``},

\end{itemize}

Для выращивания хлопка пригодны 1,4 млн~га песчаных почв и 1,2 млн~га глинистых почв, значит
\begin{align}
  \label{eq:limit_pesok}
  x_{1} &+ x_{2} + x_{3} \le 1 400 000 \\
  \label{eq:limit_glina}
  x_{4} &+ x_{5} + x_{6} \le 1 200 000 
\end{align}

Для выращивания хлопка требуется орошение. Имеющаяся ирригационная система обеспечивает не более 56 млн $ \text{м}^3 $ воды в год. Величины расхода воды на орошение одного гектара земли при выращивании хлопка различных сортов приведены в таблице~\ref{tbl:formulation_second}. В итоге можем составить уравнение:
\begin{equation}
  \label{eq:limit_voda}
  20x_1 + 30x_2 + 30x_3 + 30x_4 + 30x_5 + 20x_6 \le 56 000 000 
\end{equation}

Население, занятое в сельском хозяйстве, составляет 7 млн человек. Для выращивания хлопка сорта ,,Шемаха`` требуются 2 человека на каждый гектар, для хлопка сорта ,,Зеравшан`` --- 1 человек на гектар, для хлопка сорта ,,Эльтон`` --- 3 человека на гектар, поэтому
\begin{equation}
  \label{eq:pre_limit_naselenie}
  2 \cdot (x_1 + x_4) + (x_2 +x_5) + 3 \cdot (x_3 + x_6) \le 7 000 000 
\end{equation}

Для удобства дальнейших вычислений раскроем скобки:
\begin{equation}
  \label{eq:limit_naselenie}
  2x_1 + x_2 + 3x_3 + 2x_4 + x_5 + 3x_6 \le 7 000 000 
\end{equation}

Кроме того, переменные $ x_1, x_2, \ldots, x_6 $ по своему физическому смылсу не могут принимать отрицательных значений, так как они обозначают объём занятых земель. Поэтому необходимо указать ограничения неотрицательности:

$ x_i \ge 0, i = 1,\ldots,6 $ 

Прибыль от продажи урожая приведена в таблице ~\ref{tbl:formulation_first}. В данной задаче требуется максимизировать прибыль от продажи урожая хлопка, поэтому целевая функция запишется следующим образом:
\begin{equation}
  \label{eq:mainFunc}
  E = 6x_1 + 8x_2 + 4x_3 + 6x_4 + 5x_5 + 5x_6 \rightarrow \max 
\end{equation}

В итоге полная математическая модель примет вид:
\begin{equation}
  \label{eq:base_model}
	\begin{aligned}
  	&\left\{
    	\begin{aligned}
	      x_1 & + x_2 + x_3 \le 1 400 000 \\
	      x_4 & + x_5 + x_6 \le 1 200 000 \\
	      20x_1 & + 30x_2 + 30x_3 + 30x_4 + 30x_5 + 20x_6 \le 56 000 000 \\
	      2x_1 & + x_2 + 3x_3 + 2x_4 + x_5 + 3x_6 \le 7 000 000 \\
	      x_i &\ge 0, i = 1, \ldots ,6  
    	\end{aligned}
  	\right.
  	\\
  	& \hspace{10mm} E = 6x_1 + 8x_2 + 4x_3 + 6x_4 + 5x_5 + 5x_6 \rightarrow \max
	\end{aligned}
\end{equation}

\pagebreak