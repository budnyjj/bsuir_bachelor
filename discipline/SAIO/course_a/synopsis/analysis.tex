\section[АНАЛИЗ БАЗОВОЙ АНАЛИТИЧЕСКОЙ МОДЕЛИ \\ НА ЧУВСТВИТЕЛЬНОСТЬ]{АНАЛИЗ БАЗОВОЙ АНАЛИТИЧЕСКОЙ МОДЕЛИ НА ЧУВСТВИТЕЛЬНОСТЬ}

\subsection{Статус и ценность ресурсов}

В рассматриваемой задаче ресурсами является площадь песчаной, глиняной почвы, количество воды и трудовые ресурсы.

Как видно из значения остаточной перемнной $ x_{10} $ трудовые ресурсы израсходованы не полностью, т.~е. они являются недефицитным ресурсом. Поэтому увеличение запаса трудовых ресурсов нецелесообразно: оно приведет только к увеличению неизрасходованного остатка. Таким образом, количество людей, занятых в сельскохозяйственной отрасли, может быть на $ 500 000 $ человек меньше --- это никак не повлияет на оптимальный план распределения земель. Если запас трудовых ресурсов снизить более чем на 1~млнчеловек (запас составит менее 6~млнчеловек), то потребуется заново определять оптимальный план засеивания почв.

Остаточные переменные $ x_7, x_8, x_9 = 0 $ означают, что запас песчаных, глинистых почв и воды будет использован полностью, т.~е. эти ресурсы являются дефицитными: увеличение их запаса позволит увеличить прибыль, а снижение приведёт к снижению прибыли.

Ценнось ресурсов представляют собой коэффициенты $ E $-строки при остаточных переменных, соответствующих остаткам ресурсов, в симплекс-таблице с оптимальным решением (таблица~\ref{tbl:simplex_optimal}). Ценность песчаных почв равна $2$ тыс. ден. ед., ценность трудовых ресурсов равна нулю, ценность глинистых почв равна $1$ тыс. ден. ед., ценность водных ресурсов --- $0,2$ тыс. ден. ед. Это означает, что увеличение запаса песчаных почв на $1$~га приведет к увеличению прибыли на $2$ тыс.~ден.~ед. Снижение этого запаса приведет к соответствующему снижению прибыли. Нулевое значение ценности трудовых ресурсов означает, что увеличение их запаса или их снижение (не более чем на $1$~млн человек) не приведет к изменению прибыли, так как данный ресурс недефицитен. Увеличение запаса глинистых почв на $1$~га приведет к увеличению прибыли на $1$ тыс.~ден.~ед. Снижение этого запаса приведет к соответствующему снижению прибыли. Увеличение запаса воды на $1$ $ \text{м}^3 $ приведет к увеличению прибыли на $ 0,2 $ тыс.~ден.~ед. Снижение этого запаса приведет к соответствующему снижению прибыли.

%\pagebreak


\subsection[Анализ на чувствительность к изменениям выделенной \\ площади песчаных почв]{Анализ на чувствительность к изменениям выделенной \\ площади песчаных почв}

Проанализируем, как влияют на оптимальный план производства изменение площади выделяемых песчаных почв.

Пусть максимально возможное количество выделяемых под посев почв изменилось на $ d $ единиц, т.~е. составляет $ 1 400 000 + d $ единиц. Для определения нового оптимального решения при изменившемся количестве выделяемых песчаных почв используем коэффициенты оптимальной симплекс-таблицы (таблица~\ref{tbl:simplex_optimal}) из столбца остаточной переменной $ x_7 $, так как эта переменная входит в изменившееся ограничение. Новое оптимальное решение определяется следующим образом:
\begin{equation}
\label{eq:analysis_first}
	\begin{aligned}
  	&\left\{
    	\begin{aligned}
	      x_1 & = 1 000 000 + 3d \\
	      x_2 & = 400 000 - 2d \\
	      x_6 & = 1 200 000 + 0d \\
	      x_{10} & = 1 000 000 - 4d \\ 
    	\end{aligned}
  	\right.
  	\\
  	& \hspace{7mm} E = 15 200 000 + 2d
	\end{aligned}
\end{equation}

Пусть, например, максимально возможная площадь песчаных земель составляет не $ 1 400 000 $, а $ 1 600 000 $~га, т.~е. $ d = 200 000 $. Найдём новое оптимальное решение:
\begin{equation}
\label{eq:analysis_first_result}
	\begin{aligned}
  	&\left\{
    	\begin{aligned}
	      x_1 & = 1 000 000 + 3d = 1 600 000 \\
	      x_2 & = 400 000 - 2d = 0\\
	      x_6 & = 1 200 000 + 0d = 1 200 000 \\
	      x_{10} & = 1 000 000 - 4d = 200 000\\ 
    	\end{aligned}
  	\right.
  	\\
  	& \hspace{7mm} E = 15 200 000 + 2d = 15 600 000
	\end{aligned}
\end{equation}

Из полученной системы видно, что изменение ограничения количества выделяемых земель не приведут к каким-либо изменениям в решении задачи (если эти изменения не выходят за определенный диапазон).

Определим диапазон изменения количества выделяемых земель, при которых состав переменных в оптимальном базисе остается прежним (т.~е. базис оптимального решения будет состоять из переменных $ x_1, x_2, x_6, x_{10} $). Этот диапазон находится из условия неотрицательности всех переменных:
\begin{equation}
	\begin{aligned}
  	\left\{
    	\begin{aligned}
	      x_1 & = 1 000 000 + 3d \ge 0 \\
	      x_2 & = 400 000 - 2d \ge 0\\
	      x_6 & = 1 200 000 + 0d \ge 0 \\
	      x_{10} & = 1 000 000 - 4d \ge 0\\ 
    	\end{aligned}
  	\right.
	\end{aligned}
\end{equation}

Решив данную систему неравенств, получим: $ -333 333 \le d \le 200 000 $.

Таким образом, базис оптимального решения будет состоять из переменных $ x_1, x_2, x_6, x_{10} $, если количество выделенных под посев хлопка песчаных земель будет составлять от $ 1 066 667 $ до $ 1 600 000 $~га. Если земель будет выделено больше (меньше), то для определения оптимального решения потребуется решать задачу заново (с новым ограничением на посев хлопка на песчаных почвах).

%\pagebreak



\subsection[Анализ на чувствительность к изменениям выделенной \\ площади глинистых почв]{Анализ на чувствительность к изменениям выделенной \\ площади глинистых почв}

Проанализируем, как влияют на оптимальный план производства изменение площади выделяемых глинистых почв.

Пусть максимально возможное количество выделяемых под посев почв изменилось на $ d $ единиц, т.~е. составляет $ 1 200 000 + d $ единиц. Для определения нового оптимального решения при изменившемся количестве выделяемых песчаных почв используем коэффициенты оптимальной симплекс-таблицы (таблица~\ref{tbl:simplex_optimal}) из столбца остаточной переменной $ x_8 $, так как эта переменная входит в изменившееся ограничение. Новое оптимальное решение определяется следующим образом:
\begin{equation}
\label{eq:analysis_second}
	\begin{aligned}
  	&\left\{
    	\begin{aligned}
	      x_1 & = 1 000 000 + 2d \\
	      x_2 & = 400 000 - 2d \\
	      x_6 & = 1 200 000 + 1d \\
	      x_{10} & = 1 000 000 - 5d \\ 
    	\end{aligned}
  	\right.
  	\\
  	& \hspace{7mm} E = 15 200 000 + 1d
	\end{aligned}
\end{equation}

Пусть, например, максимально возможная площадь песчаных земель составляет не $ 1 200 000 $, а $ 1 300 000 $~га, т.~е. $ d = 100 000 $. Найдём новое оптимальное решение:
\begin{equation}
\label{eq:analysis_second_result}
	\begin{aligned}
  	&\left\{
    	\begin{aligned}
	      x_1 & = 1 000 000 + 2d = 1 200 000 \\
	      x_2 & = 400 000 - 2d = 200 000\\
	      x_6 & = 1 200 000 + 1d = 1 300 000 \\
	      x_{10} & = 1 000 000 - 5d = 500 000\\ 
    	\end{aligned}
  	\right.
  	\\
  	& \hspace{7mm} E = 15 200 000 + 1d = 15 300 000
	\end{aligned}
\end{equation}

Из полученной системы видно, что изменение ограничения количества выделяемых земель не приведут к каким-либо изменениям в решении задачи (если эти изменения не выходят за определенный диапазон).

Определим диапазон изменения количества выделяемых земель, при которых состав переменных в оптимальном базисе остается прежним (т.~е. базис оптимального решения будет состоять из переменных $ x_1, x_2, x_6, x_{10} $). Этот диапазон находится из условия неотрицательности всех переменных:
\begin{equation}
	\begin{aligned}
  	\left\{
    	\begin{aligned}
	      x_1 & = 1 000 000 + 2d \ge 0 \\
	      x_2 & = 400 000 - 2d \ge 0\\
	      x_6 & = 1 200 000 + 1d \ge 0 \\
	      x_{10} & = 1 000 000 - 5d \ge 0\\ 
    	\end{aligned}
  	\right.
	\end{aligned}
\end{equation}

Решив данную систему неравенств, получим: $ -500 000 \le d \le 200 000 $.

Таким образом, базис оптимального решения будет состоять из переменных $ x_1, x_2, x_6, x_{10} $, если количество выделенных под посев хлопка глиняных земель будет составлять от $ 700 000 $ до $ 1 400 000 $~га. Если земель будет выделено больше (меньше), то для определения оптимального решения потребуется решать задачу заново (с новым ограничением на посев хлопка на глинистых почвах).


%\pagebreak

\subsection[Анализ на чувствительность к изменениям количества \\ потребляемой воды]{Анализ на чувствительность к изменениям количества \\ потребляемой воды}

Проанализируем, как влияют на оптимальный план производства изменение количества потреблямой воды.

Пусть максимально возможное количество потребляемой воды изменилось на $ d $ единиц, т.~е. составляет $ 56 000 000 + d $ единиц. Для определения нового оптимального решения при изменившемся количестве потребляемой воды используем коэффициенты оптимальной симплекс-таблицы (таблица~\ref{tbl:simplex_optimal}) из столбца остаточной переменной $ x_9 $, так как эта переменная входит в изменившееся ограничение. Новое оптимальное решение определяется следующим образом:
\begin{equation}
\label{eq:analysis_third}
	\begin{aligned}
  	&\left\{
    	\begin{aligned}
	      x_1 & = 1 000 000 - 0,1d \\
	      x_2 & = 400 000 + 0,1d \\
	      x_6 & = 1 200 000 + 0d \\
	      x_{10} & = 1 000 000 + 0,1d \\ 
    	\end{aligned}
  	\right.
  	\\
  	& \hspace{7mm} E = 15 200 000 + 0,2d
	\end{aligned}
\end{equation}

Пусть, например, максимально возможное количество потребляемой воды составляет не $ 56 000 000 $, а $ 57 000 000 $~га, т.~е. $ d = 1 000 000 $. Найдём новое оптимальное решение:
\begin{equation}
\label{eq:analysis_third_result}
	\begin{aligned}
  	&\left\{
    	\begin{aligned}
	      x_1 & = 1 000 000 - 0,1d = 900 000 \\
	      x_2 & = 400 000 + 0,1d = 500 000\\
	      x_6 & = 1 200 000 + 0d = 1 200 000 \\
	      x_{10} & = 1 000 000 + 0,1d = 1 100 000\\ 
    	\end{aligned}
  	\right.
  	\\
  	& \hspace{7mm} E = 15 200 000 + 0,2d = 15 400 000
	\end{aligned}
\end{equation}

Из полученной системы видно, что изменение ограничения количества потребляемой воды не приведут к каким-либо изменениям в решении задачи (если эти изменения не выходят за определенный диапазон).

Определим диапазон изменения количества потребляемой воды, при котором состав переменных в оптимальном базисе остается прежним (т.~е. базис оптимального решения будет состоять из переменных $ x_1, x_2, x_6, x_{10} $). Этот диапазон находится из условия неотрицательности всех переменных:
\begin{equation}
	\begin{aligned}
  	\left\{
    	\begin{aligned}
	      x_1 & = 1 000 000 - 0,1d \ge 0 \\
	      x_2 & = 400 000 + 0,1d \ge 0\\
	      x_6 & = 1 200 000 + 0d \ge 0 \\
	      x_{10} & = 1 000 000 + 0,1d \ge 0\\ 
    	\end{aligned}
  	\right.
	\end{aligned}
\end{equation}

Решив данную систему неравенств, получим: \\
$ -4 000 000 \le d \le 10 000 000 $.

Таким образом, базис оптимального решения будет состоять из переменных $ x_1, x_2, x_6, x_{10} $, если количество потребляемой воды будет составлять от $ 52 000 000 $ до $ 66 000 000 $ $ \text{м}^3 $. Если воды будет израсходовано больше (меньше), то для определения оптимального решения потребуется решать задачу заново (с новым ограничением на количество потребляемой воды).



%\pagebreak

\subsection{Анализ на чувствительность к изменениям прибыли}

Проанализируем, как влияют на оптимальный план изменение прибыли от продажи сорта хлопка ,,Шемаха`` на песчаной почве.

Пусть прибыль от продажи хлопка сорта ,,Шемаха`` изменилась на $ d $ тыс.~ден.~ед., т.~е. составляет $ 6 + d $ тыс.~ден.~ед. Для анализа влияния этих изменений на оптимальное решение используем коэффициенты оптимальной симплекс-таблицы (таблица~\ref{tbl:simplex_optimal}) из строки переменной $ x_1 $, так как для этой переменной изменился коэффициент целевой функции. Новые значения коэффициентов $ E $-строки при небазисных переменных для окончательной симплекс-таблицы, а также новое оптимальное значение целевой функции определяются следующим образом:
\begin{equation}
\label{eq:analysis_fourth}
	\begin{aligned}
  	&\left\{
    	\begin{aligned}
	      F_3 & = 4 + 0d \\
	      F_4 & = 1 - 1d \\
	      F_5 & = 2 - 1d \\
	      F_7 & = 2 + 3d \\
	      F_8 & = 1 + 2d \\
	      F_9 & = 0,2 - 0,1d \\
    	\end{aligned}
  	\right.
  	\\
  	& \hspace{7mm} E = 15 200 000 + 1 000 000d
	\end{aligned}
\end{equation}


Пусть, например, прибыль от продажи хлопка сорта ,,Шемаха`` составляет не $ 6 $, а $ 6,5 $ тыс.~ден.~ед., т.~е. $ d = 0,5 $. Найдём новые значения коэффициентов $ E $-строки при небызисных переменных для окончательной симплекс-таблицы и новое оптимальное знычение целевой функции:
\begin{equation}
\label{eq:analysis_third_result}
	\begin{aligned}
  	&\left\{
    	\begin{aligned}
	      F_3 & = 4 + 0d = 4,5\\
	      F_4 & = 1 - 1d = 0,5\\
	      F_5 & = 2 - 1d = 1,5\\
	      F_7 & = 2 + 3d = 3,5\\
	      F_8 & = 1 + 2d = 2\\
	      F_9 & = 0,2 - 0,1d = 0,15 \\ 
    	\end{aligned}
  	\right.
  	\\
  	& \hspace{7mm} E = 15 200 000 + 1 000 000d = 15 700 000
	\end{aligned}
\end{equation}

Из полученной системы видно, что коэффициенты $ E $-строки остались неотрицательными. Это значит, что оптимальное решение задачи не изменится.

Определим диапазон изменения прибыли от продажи хлопка сорта ,,Шемаха``, при котором оптимальное решение, найденное для исходной постановки задачи, не изменится. Условием оптимальности решения является неотрицательность всех коэффициентов $E$-строки. \par
\begin{equation}
	\begin{aligned}
  	&\left\{
    	\begin{aligned}
	      F_3 & = 4 + 0d \ge 0 \\
	      F_4 & = 1 - 1d \ge 0 \\
	      F_5 & = 2 - 1d \ge 0 \\
	      F_7 & = 2 + 3d \ge 0 \\
	      F_8 & = 1 + 2d \ge 0 \\
	      F_9 & = 0,2 - 0,1d \ge 0 \\
    	\end{aligned}
  	\right.
	\end{aligned}
\end{equation}

Решив данную систему неравенств, получим: $ -0,5 \le d \le 1 $.

Таким образом, базис оптимального решения будет состоять из переменных $ x_1, x_2, x_6, x_{10} $, если прибыль от продажи хлопка будет составлять от $6-0,5$ до $6+1$ тыс.~ден.~ед., т.~е. от $5,5$ до $7$ тыс.~ден.~ед. Если эта прибыль будет больше (меньше) указанной, то для определения оптимального решения потребуется решать задачу заново (изменив в математической модели коэффициент целевой функции при $ x_1 $).

\pagebreak