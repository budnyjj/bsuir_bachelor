\section[Анализ базовой аналитической модели на чувствительность]{АНАЛИЗ БАЗОВОЙ АНАЛИТИЧЕСКОЙ МОДЕЛИ \\ НА ЧУВСТВИТЕЛЬНОСТЬ}

\subsection{Статус и ценность ресурсов}

В рассматриваемой задаче ресурсами являются площади, доступные для засева
в западной и восточной части региона.

Как видно из значений остаточных переменных, доступные площади использованы
в полной мере, т.е. все они являются дефицитными ресурсами. Таким образом,
всякое изменение площади, доступной для засева в той или иной части региона, 
приведет к изменению оптимального плана --- увеличение доступной площади
приведет к увеличению прибыли, и наоборот.

Ценности ресурсов представляют собой коэффициенты $ E $-строки при
остаточных переменных, соответствующих остаткам ресурсов, в симплекс-
таблице с оптимальным решением (таблица~\ref{tbl:Simplex2_3}).
Ценность земель западной части региона равна 196 млн ден. ед.,
ценность земель восточной части региона --- 245 млн ден.ед.
Это означает, что увеличение площади, доступной в западной части региона
на 1 млн га приведет к увеличению прибыли в среднем на 196 млн ден. ед.
Увеличение площади, доступной в восточной части региона
на 1 млн га приведет к увеличению прибыли в среднем на 245 млн ден. ед.
Уменьшение площадей, доступных для засева в западной или восточной части региона
приведет к соответствующему снижению прибыли.

Из значений ценностей ресурсов можно также сделать следующий вывод:
если имеется возможность увеличить площадь, доступную для засева
  только в одной части региона, то целесообразно увеличить площадь в восточной части региона.

\subsection{Анализ на чувствительность к изменениям площадей, \\ доступных для засева}

Проанализируем, как влияют на оптимальный план земледелия изменения площади,
доступной для засева в некоторой части региона, например, в \textbf{восточной}.

Пусть площадь, доступная для посева в восточной части региона изменилась на
$ d $ млн га, т.~е. составляет не $ 0{,}6 $, а $ 0{,}6 + d $ млн га. 
Для определения нового оптимального решения при изменившейся доступной площади
используются коэффициенты окончательной симплекс-таблицы (таблица~\ref{tbl:Simplex2_3})
из столбца остаточной переменной $ x_{4} $ так как эта переменная входит в изменившееся ограничение.
Новое оптимальное решение определяется следующим образом:
\begin{equation}
  \label{eq:Analisys_1m}
  \begin{aligned}
    x_{11} &= 0{,}25 - 1{,}25d, &
    x_{12} &= 0{,}6 + d, \\
    x_{21} &= 0{,}55 + 1{,}25d, &
    x_{2} &= 9{,}4 + 35d.
  \end{aligned}
\end{equation}
\begin{equation}
  E = 267{,}8 + 245d.
\end{equation}

Пусть, например, площадь, доступная для засева в восточной части региона равна не 0{,}6, а 0{,}8 млн га.
Найдем новое оптимальное решение: 
\begin{equation}
    \begin{aligned}
      x_{11} &= 0{,}25 - 1{,}25 \cdot 0{,}2 = 0, \\
      x_{12} &= 0{,}6 + 0{,}2 = 0{,}8, \\
      x_{21} &= 0{,}55 + 1{,}25 \cdot 0{,}2 = 0{,}8, \\
      x_{2} &= 9{,}4 + 35 \cdot 0{,}2 = 16{,}4, \\
      E &= 267{,}8 + 245 \cdot 0{,}2 = 316{,}8.
    \end{aligned}
\end{equation}

Таким образом, в новых условиях следует использовать следующий план использования земель:
\begin{itemize}
\item Засеять 0 млн га земель \textbf{западной} части региона \textbf{озимыми};
\item Засеять 0{,}8 млн га земель \textbf{восточной} части региона \textbf{озимыми};
\item Засеять 0{,}8 млн га земель \textbf{западной} части региона \textbf{яровыми};
\item Засеять 0 млн га земель \textbf{восточной} части региона \textbf{яровыми}.
\end{itemize}

В таком случае будут получены следующие результаты:
\begin{itemize}
\item Значение целевой функции $ E = 316{,}8 $ показывает, что прибыль при таком
  плане использования земель составит 316{,}8 млн. ден.ед.
\item Избыточная переменная $ x_{2} = 16{,}4 $ означает, что план урожайности яровых 
  будет перевыполнен на 16{,}4 млн. центнеров (требуется не менее 6 млн. центнеров, а
  вырастет 22{,}4 млн. центнеров).
\item Избыточная переменная $ x_{1} = 0 $ означает, что при рассчитанном плане использования
  земель урожай озимых будет равен минимальному ограничению плана.
\item Остаточные переменные $ x_{3} = x_{4} = 0 $ означают, что для засева будут
  использованы все доступные площади (то есть неиспользуемых земель не останется).
\end{itemize}

\textbf{Вывод:} увеличение площади, доступной для засева в восточной части региона позволит
увеличить прибыль от продажи выращенного урожая.

Пусть теперь площадь, доступная для засева в восточной части региона равна не 0{,}6, а 0{,}4 млн га.
Найдем новое оптимальное решение: 
\begin{equation}
    \begin{aligned}
      x_{11} &= 0{,}25 - 1{,}25 \cdot (-0{,}2) = 0{,}5, \\
      x_{12} &= 0{,}6 + (-0{,}2) = 0{,}4, \\
      x_{21} &= 0{,}55 + 1{,}25 \cdot (-0{,}2) = 0{,}3, \\
      x_{2} &= 9{,}4 + 35 \cdot (-0{,}2) = 2{,}4, \\
      E &= 267{,}8 + 245 \cdot (-0{,}2) = 218{,}8.
    \end{aligned}
\end{equation}

Таким образом, в новых условиях следует использовать следующий план использования земель:
\begin{itemize}
\item Засеять 0{,}5 млн га земель \textbf{западной} части региона \textbf{озимыми};
\item Засеять 0{,}4 млн га земель \textbf{восточной} части региона \textbf{озимыми};
\item Засеять 0{,}3 млн га земель \textbf{западной} части региона \textbf{яровыми};
\item Засеять 0 млн га земель \textbf{восточной} части региона \textbf{яровыми}.
\end{itemize}

В таком случае будут получены следующие результаты:
\begin{itemize}
\item Значение целевой функции $ E = 218{,}8 $ показывает, что прибыль при таком
  плане использования земель составит 218{,}8 млн. ден.ед.
\item Избыточная переменная $ x_{2} = 2{,}4 $ означает, что план урожайности яровых 
  будет перевыполнен на 2{,}4 млн. центнеров (требуется не менее 6 млн. центнеров, а
  вырастет 8{,}4 млн. центнеров).
\item Избыточная переменная $ x_{1} = 0 $ означает, что при рассчитанном плане использования
  земель урожай озимых будет равен минимальному ограничению плана.
\item Остаточные переменные $ x_{3} = x_{4} = 0 $ означают, что для засева будут
  использованы все доступные площади (то есть неиспользуемых земель не останется).
\end{itemize}

\textbf{Вывод:} уменьшение площади, доступной для засева в восточной части региона
уменьшит прибыль от продажи выращенного урожая.


Рассмотрим случай, когда площадь, доступная для засева в восточной части региона равна не 0{,}6, а 0{,}3 млн га.
Попробуем найти новое оптимальное решение, подставив значение $ d = -0{,}3 $ в систему уравнений~\eqref{eq:Analisys_1m}:
\begin{equation}
    \begin{aligned}
      x_{11} &= 0{,}25 - 1{,}25 \cdot (-0{,}3) = 0{,}625, \\
      x_{12} &= 0{,}6 + (-0{,}3) = 0{,}3, \\
      x_{21} &= 0{,}55 + 1{,}25 \cdot (-0{,}3) = 0{,}175, \\
      x_{2} &= 9{,}4 + 35 \cdot (-0{,}3) = -1{,}1, \\
      E &= 267{,}8 + 245 \cdot (-0{,}3) = 194{,}3.
    \end{aligned}
\end{equation}

Таким образом, одна из переменных приняла отрицательное значение, что недопустимо по её физическому смыслу.

\textbf{Вывод:} для поиска оптимального решения в новых условиях требуется решить задачу заново,
изменив ограничение на максимальную площадь используемых земель в восточной части региона следующим образом:
$ x_{12} + x_{22} \le 0{,}3 $.


Определим диапазон изменений площади, доступной для засева в восточной части региона, при которых состав
переменных в оптимальном базисе останется прежним (т.~е. базис оптимального решения будет состоять из переменных
$ x_{11}, x_{12}, x_{21}, x_{2} $). Этот диапазон находится из условия неотрицательности всех переменных:

\begin{equation}
    \begin{aligned}
      x_{11} &= 0{,}25 - 1{,}25d \ge 0, \\
      x_{12} &= 0{,}6 + d \ge 0, \\
      x_{21} &= 0{,}55 + 1{,}25d \ge 0, \\
      x_{2} &= 9{,}4 + 35d \ge 0.
    \end{aligned}
\end{equation}

Решив эту систему неравенств, получим: $ -0{,}268 \le d \le 0{,}2 $.

Это означает, что базис оптимального решения будет состоять из переменных 
$ x_{11}, x_{12}, x_{21}, x_{2}, $ если площадь, доступная для засева в восточной
части региона будет составлять от $ 0{,}6 - 0{,}268 $ до $ 0{,}6 + 0{,}2 $ млн га
(т.е. от 0{,}332 до 0{,}8 млн га).

\textbf{Вывод:}
для любой величины площади, доступной для засева в восточной
части региона, входящей в диапазон от 0{,}332 до 0{,}8 млн га, новое оптимальное решение можно найти
из уравнений~\eqref{eq:Analisys_1m}.
Если площадь, доступная для засева в восточной
части региона составит менее 0{,}332 или более 0{,}8 млн га, то для
определения оптимального решения потребуется решать задачу заново (с
новым ограничением на величину площади, доступной для засева в восточной
части региона).

\subsection{Анализ на чувствительность к изменениям минимально \\
  допустимой величины урожая}

Проанализируем, как влияют на оптимальный план земледелия изменения минимально допустимого
объёма выращенных \textbf{озимых}.

Пусть минимально допустимый объём выращенных озимых изменился на
$ d $ млн центнеров, т.~е. составляет не $ 20 $, а $ 20 + d $ млн центнеров. 
Для определения нового оптимального решения при изменившейся доступной площади
используются коэффициенты окончательной симплекс-таблицы (таблица~\ref{tbl:Simplex2_3})
из столбца избыточной переменной $ x_{1} $ так как эта переменная входит в изменившееся ограничение.
Так как ограничение, для которого выполняется анализ на чувствительность, имеет вид <<больше или равно>>,
коэффициенты из столбца избыточной переменной используются с обратными знаками.
Новое оптимальное решение определяется следующим образом:

\begin{equation}
  \label{eq:Analisys_2m}
  \begin{aligned}
    x_{11} &= 0{,}25 + 0{,}05d, &
    x_{12} &= 0{,}6 - 0d, \\
    x_{21} &= 0{,}55 - 0{,}05d, &
    x_{2} &= 9{,}4 - 1{,}4d, \\
  \end{aligned}
\end{equation}
\begin{equation}
  E = 267{,}8 - 1{,}8d.
\end{equation}

Пусть, например, минимально допустимый объём выращенных озимых равeн не 20, а 22 млн центнеров.
Найдем новое оптимальное решение: 
\begin{equation}
    \begin{aligned}
      x_{11} &= 0{,}25 + 0{,}05 \cdot 2 = 0{,}35, \\
      x_{12} &= 0{,}6 - 0 \cdot 2 = 0{,}6, \\
      x_{21} &= 0{,}55 - 0{,}05 \cdot 2 = 0{,}45, \\
      x_{2} &= 9{,}4 - 1{,}4 \cdot 2 = 6{,}6, \\
      E &= 267{,}8 - 1{,}8 \cdot 2 = 264{,}2.
    \end{aligned}
\end{equation}
Таким образом, в новых условиях следует использовать следующий план использования земель:
\begin{itemize}
\item Засеять 0{,}35 млн га земель \textbf{западной} части региона \textbf{озимыми};
\item Засеять 0{,}6 млн га земель \textbf{восточной} части региона \textbf{озимыми};
\item Засеять 0{,}45 млн га земель \textbf{западной} части региона \textbf{яровыми};
\item Засеять 0 млн га земель \textbf{восточной} части региона \textbf{яровыми}.
\end{itemize}

В таком случае будут получены следующие результаты:
\begin{itemize}

\item Значение целевой функции $ E = 264{,}2 $ показывает, что прибыль при таком
  плане использования земель составит 264{,}2 млн. ден.ед.

\item Избыточная переменная $ x_{2} = 6{,}6 $ означает, что план урожайности яровых 
  будет перевыполнен на 6{,}6 млн. центнеров (требуется не менее 6 млн. центнеров, а
  вырастет 12{,}6 млн. центнеров).

\item Избыточная переменная $ x_{1} = 0 $ означает, что при рассчитанном плане использования
  земель урожай озимых будет равен минимальному ограничению плана.

\item Остаточные переменные $ x_{3} = x_{4} = 0 $ означают, что для засева будут
  использованы все доступные площади (то есть неиспользуемых земель не останется).
\end{itemize}

\textbf{Вывод:} увеличение минимально допустимого объёма выращенных озимых
снизит прибыль от продажи выращенного урожая.


Пусть минимально допустимый объём выращенных озимых равeн не 20, а 18 млн центнеров.
Найдем новое оптимальное решение: 
\begin{equation}
    \begin{aligned}
      x_{11} &= 0{,}25 + 0{,}05 \cdot (-2) = 0{,}15, \\
      x_{12} &= 0{,}6 - 0 \cdot (-2) = 0{,}6, \\
      x_{21} &= 0{,}55 - 0{,}05 \cdot (-2) = 0{,}65, \\
      x_{2} &= 9{,}4 - 1{,}4 \cdot (-2) = 12{,}2, \\
      E &= 267{,}8 - 1{,}8 \cdot (-2) = 271{,}4.
    \end{aligned}
\end{equation}

Таким образом, в новых условиях следует использовать следующий план использования земель:
\begin{itemize}
\item Засеять 0{,}15 млн га земель \textbf{западной} части региона \textbf{озимыми};
\item Засеять 0{,}6 млн га земель \textbf{восточной} части региона \textbf{озимыми};
\item Засеять 0{,}65 млн га земель \textbf{западной} части региона \textbf{яровыми};
\item Засеять 0 млн га земель \textbf{восточной} части региона \textbf{яровыми}.
\end{itemize}

В таком случае будут получены следующие результаты:
\begin{itemize}

\item Значение целевой функции $ E = 271{,}4 $ показывает, что прибыль при таком
  плане использования земель составит 271{,}4 млн. ден.ед.

\item Избыточная переменная $ x_{2} = 12{,}2 $ означает, что план урожайности яровых 
  будет перевыполнен на 12{,}2 млн. центнеров (требуется не менее 6 млн. центнеров, а
  вырастет 12{,}2 млн. центнеров).

\item Избыточная переменная $ x_{1} = 0 $ означает, что при рассчитанном плане использования
  земель урожай озимых будет равен минимальному ограничению плана.

\item Остаточные переменные $ x_{3} = x_{4} = 0 $ означают, что для засева будут
  использованы все доступные площади (то есть неиспользуемых земель не останется).
\end{itemize}

\textbf{Вывод:} уменьшение минимально допустимого объёма выращенных озимых
позволит увеличить прибыль от продажи выращенного урожая.


Определим диапазон изменений минимально допустимого объёма выращенных озимых, при котором состав
переменных в оптимальном базисе останется прежним (т.~е. базис оптимального решения будет состоять из переменных
$ x_{11}, x_{12}, x_{21}, x_{2} $). Этот диапазон находится из условия неотрицательности всех переменных:
\begin{equation}
    \begin{aligned}
      x_{11} &= 0{,}25 + 0{,}05d \ge 0, \\
      x_{12} &= 0{,}6 - 0d \ge 0, \\
      x_{21} &= 0{,}55 - 0{,}05d \ge 0, \\
      x_{2} &= 9{,}4 - 1{,}4d \ge 0.
    \end{aligned}
\end{equation}

Решив эту систему неравенств, получим: $ -5 \le d \le 6{,}71 $.

Это означает, что базис оптимального решения будет состоять из переменных 
$ x_{11}, x_{12}, x_{21}, x_{2}, $ если минимально допустимый объём выращенных озимых
будет составлять от $ 20 - 5 $ до $ 20 + 6{,}71 $ млн центнеров
(т.е. от 15 до 26{,}71 млн центнеров).

\textbf{Вывод:}
Для любой величины минимально допустимого объёма выращенных озимых,
входящей в диапазон от 15 до 26{,}71 млн центнеров, новое оптимальное решение можно найти
из уравнений~\eqref{eq:Analisys_2m}.
Если величина минимально допустимого объёма выращенных озимых
составит менее 15 или более 26{,}71 млн центнеров, то для
определения оптимального решения потребуется решать задачу заново (с
новым ограничением на минимально допустимый объём выращенных озимых).


\subsection{Анализ на чувствительность к изменениям 
  прибыли \\ от продажи урожая}

Проанализируем, как влияют на оптимальный план земледелия изменения прибыли от
продажи одного центнера урожая.

Заметим, что изменение цены за один центнер выращенных озимых или яровых
представляет собой изменение общего множителя, находящегося за скобками,
в целевой функции базовой аналитической модели~\eqref{eq:BaseOpt}.

Например, пусть стоимость центнера озимых изменилась на $ d $, а стоимость центнера 
яровых изменилась на $ t $. Тогда целевая функция в базовой аналитической модели,
примет следующий вид:
\begin{equation}
\label{eq:Analisys_3m}
  E = (8 + d)(20x_{11} + 25x_{12}) + (7 + t)(28x_{21} + 18x_{22}) \rightarrow \max.
\end{equation} 

Как видно из выражения~\eqref{eq:Analisys_3m}, \textbf{изменение цены на
  урожай влияет на значение двух переменных}. Следовательно, \textbf{при
  всяком изменении цены за урожай нам потребуется делать полный
  перерасчет задачи с учетом изменившейся целевой функции}.

С другой стороны, мы можем провести анализ на чувствительность на изменение коэффициента, влияющего
на одну переменную в целевой функции. Для этого нам потребуется записать целевую
функцию~\eqref{eq:BaseOpt} в следующем виде:
\begin{equation}
\label{eq:Analisys_3m_2}
  E = 8 \cdot 20x_{11} + 8 \cdot 25x_{12} + 7 \cdot 28x_{21} + 7 \cdot 18x_{22} \rightarrow \max.
\end{equation} 

Пусть прибыль от продажи одного центнера яровых, выращенных в западной части региона изменилась на $ d $ единиц,
т.~е. составляет не $ 7 $, а $ 7 + d $ ден. единиц.

Для анализа влияния этих изменений на оптимальное решение используются коэффициенты
окончательной симплекс-таблицы (таблица~\ref{tbl:Simplex2_3}) из строки переменной $ x_{21} $, 
так как для этой переменной изменился коэффициент целевой функции.
Новые значения коэффициентов $ Е $-строки при небазисных переменных для окончательной симплекс-таблицы
(т.~е. при переменных $ x_{22}, x_{1}, x_{3}, x_{4} $),
а также новое оптимальное значение целевой функции определяются следующим образом: 
\begin{equation}
  \label{eq:Analisys_3m_3}
  \begin{aligned}
    F_{22} &= 119 + 1{,}25d, &
    F_{1} &= 1{,}8 + 0{,}05d, \\
    F_{3} &= 196 + d, &
    F_{4} &= 245 + 1{,}25d.
  \end{aligned}
\end{equation}
\begin{equation}
      E = 267{,}8 + 0{,}55d.
\end{equation}

Пусть, например, прибыль от продажи одного центнера яровых, выращенных в западной части региона 
снизилась на 1 ден.~ед., т.~е. составляет не 7, а 6 ден.~ед. ($ d = -1 $). Найдем новые значения 
коэффициентов $ E $-строки при небазисных переменных для окончательной симплекс-таблицы 
и новое оптимальное значение целевой функции:
\begin{equation}
    \begin{aligned}
      F_{22} &= 119 + 1{,}25 \cdot (-1) = 117{,}75 \\
      F_{1} &= 1{,}8 + 0{,}05d \cdot (-1) = 1{,}75, \\
      F_{3} &= 196 + d \cdot (-1) = 195, \\
      F_{4} &= 245 + 1{,}25d \cdot (-1) = 243{,}75, \\
      E &= 267{,}8 + 0{,}55d \cdot (-1) = 267{,}25.
    \end{aligned}
\end{equation}

Видно, что коэффициенты $ E $-строки остались неотрицательными. Это значит, что оптимальное решение 
не изменяется: $ x_{11} = 0{,}25, x_{12} = 0{,}6, x_{21} = 0{,}55,
x_{22} = 0, x_{1} = 0, x_{2} = 9{,}4, x_{3} = 0, x_{4} = 0 $. 


Определим диапазон изменений прибыли от продажи одного центнера яровых, выращенных в западной части региона,
при которых остается оптимальным решение, найденное для исходной постановки задачи (система~\eqref{eq:Solution}).
Условием оптимальности решения является неотрицательность всех коэффициентов $ E $-строки:
\begin{equation}
    \begin{aligned}
      F_{22} &= 119 + 1{,}25d \ge 0, \\
      F_{1} &= 1{,}8 + 0{,}05d \ge 0, \\
      F_{3} &= 196 + d \ge 0, \\
      F_{4} &= 245 + 1{,}25d \ge 0. 
    \end{aligned}
\end{equation}

Решив эту систему неравенств, получим: $ -36 \le d \le +\infty $. Это означает, решение,
найденное для исходной постановки задачи, оптимально, если прибыль от продажи одного центнера яровых,
выращенных в западной части региона, будет составлять как минимум $ 7 - 36 = -29 $ д.~е.

\textbf{Вывод:}
Для любой величины прибыли от продажи одного центнера яровых в западной части региона,
большей минус 26 д.~е, оптимальный план использования земель не изменится.
Если же убыток составит меньше 29 д.~е., то для получения оптимального решения потребуется решить задачу
заново, используя симплекс-метод.
Новое оптимальное решение будет отличаться от прежнего не только значениями, но и составом переменных в 
оптимальном базисе.

\newpage