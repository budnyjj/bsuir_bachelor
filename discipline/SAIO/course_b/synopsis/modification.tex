\section[Построение модифицированной аналитической модели и \\ анализ результатов модификации]{ПОСТРОЕНИЕ МОДИФИЦИРОВАННОЙ \\ АНАЛИТИЧЕСКОЙ МОДЕЛИ И АНАЛИЗ \\ РЕЗУЛЬТАТОВ МОДИФИКАЦИИ}
 
Проанализировав результаты решения задачи оптимизации (см. раздел~3), можно выделить
следующие особенности в разработанном плане использования земель:
\begin{itemize}
  \item Все используемые ресурсы являются дефицитными, то есть все земли используются.
    Эта особенность носит позитивный характер.
  \item Земли восточной части региона не будут засеваться яровыми. В некоторых случаях
    эта особенность может принимать негативный характер, например, если восточная часть региона нуждается 
    не только в озимых, но и в яровых. 
\end{itemize}

Рассмотрим различные варианты модификации базовой аналитической модели~\eqref{eq:BaseNSM} с целью исправления
неравномерности использования площадей для засева:
\begin{itemize}
  \item Изменение объема доступных ресурсов. В нашем случае это означает внесение правок в распределение
    площадей земель, доступных для засева, надо полагать, в сторону увеличения. На практике такая возможность
    редко встречается.
  \item Изменение величин, входящих в минимальные требования к объемам выращенного урожая.
    Такая возможность также представляется маловероятной, так как на практике такие требования не зависят
    от субъекта, отвечающего за его выполнения.
  \item Увеличение урожайности культур на используемых землях. Как правило, урожайность зависит от многих факторов,
    и в ряде случаев принципиально не может быть увеличена, к тому же прогнозы изменения урожайности зачастую 
    носят случайных характер.
  \item Изменение прибыли от выращенного урожая. Как правило, субъект имеет возможность влиять на прибыль
    от выращенного урожая (путем изменения стоимости). Рассмотрим подробнее этот вариант.
\end{itemize}

Исходя из условия задачи (таблица~\ref{tbl:Cond})
можно сделать вывод, что для получения желаемого плана
использования земель следует либо увеличивать прибыль от выращивания яровых, либо
уменьшать прибыль от выращивания озимых.
Заметим также, что уменьшение прибыли от выращивания культур нежелательно, так как это негативно сказывается на 
общей прибыли от урожая.
Таким образом, мы пришли к выводу, что \textbf{изменение плана использования земель путем варьирования прибыли
  от продажи урожая сводится к увеличению стоимости центнера яровых, выращенных в восточной части региона}. 

Далее, получим граничное значение цены яровых, выращенных в восточной части региона, при которой происходит 
перераспределение используемых площадей. Для этого введем параметр $ d $ в базовую математическую модель задачи
~\eqref{eq:BaseNSM}, отражающий изменение прибыли от продажи центнера яровых, выращенных в восточной части региона:
\begin{equation}
  E = 160x_{11} + 200x_{12} + 196x_{21} + (7+d) \cdot 18x_{22} \rightarrow \max.
\end{equation}

Последовательно увеличивая параметр $ d \in \pmb{Z} $, будем отслеживать, как будет изменяться решение задачи.
Для расчетов будем использовать табличный процессор Microsoft Office Excel.

При значении $ d = 7 $ происходит искомое перераcпределение, сопровождающееся увеличением целевой функции.
Переменные при таком решении имеют следующие значения:  
$ x_{11} = 0{,}8, x_{12} = 0{,}16, x_{21} = 0, x_{22} = 0{,}44 $.

Таким образом, модифицированный план использования земель будет иметь следующий вид:

\begin{itemize}
\item Засеять 0{,}8 млн га земель \textbf{западной} части региона \textbf{озимыми};
\item Засеять 0{,}16 млн га земель \textbf{восточной} части региона \textbf{озимыми};
\item Засеять 0{,}0 млн га земель \textbf{западной} части региона \textbf{яровыми};
\item Засеять 0{,}44 млн га земель \textbf{восточной} части региона \textbf{яровыми}.
\end{itemize}

Целевая функция $ E $ примет значение, равное $ 270{,}88 $, таким образом, прибыль при таком
плане использования земель составит 270{,}88 млн. ден.ед, что на $ 3{,}08 $ ден.~ед. больше,
чем в исходном плане.

К сожалению, измененному плану присущ ряд недостатков:
\begin{itemize}
\item Для перераспределения площадей импольхуемых земель, используемых под засев, потребовалось изменить 
  стоимость центнера яровых, выращенного в восточной части региона на семь ден.~ед., что представляет
  собой \textbf{увеличение стоимости в два раза}.
\item Модифицированный план предусматривает, что земли западной части региона не будут 
    засеваться яровыми, то есть нам снова \textbf{придется решать аналогичную проблему для западной части региона}.
\end{itemize}

Таким образом, следует признать, что \textbf{изменение величины прибыли от продажи яровых, выращенных в восточной
части региона не является подходящим выходом из сложившейся ситуации}. 

Рассмотрим еще одну возможность влияния на использование доступных земель.
Исходя из задачи модификации плана использования земель, нам требуется, чтобы каждая культура выращивалась
в каждой из доступных частей региона. Исходя из этого, введем дополнительные ограничения в базовую математическую
модель~\eqref{eq:BaseNSM}, отражающие минимальные размеры площадей различных частей региона, используемых для выращивания культур:
\begin{equation}
  \label{eq:ModifiedNSM}
    \begin{aligned}
      E = 160x_{11} + 200x_{12} &+ 196x_{21} + 126x_{22} \rightarrow \max, \\
      x_{11} \ge 0{,}2, x_{12} \ge &0{,}15, x_{21} \ge 0{,}2, x_{22} \ge 0{,}15, \\
      20x_{11} &+ 25x_{12} \ge 20, \\
      28x_{21} &+ 18x_{22} \ge 6, \\
      x_{11} &+ x_{21} \le 0{,}8, \\
      x_{12} &+ x_{22} \le 0{,}6. \\
    \end{aligned}
\end{equation}

Ограничения, накладываемые на значения переменных $ x_{ij}, i{,}j = 1,2 $ выбраны из следующих рассуждений.
На практике площадь используемых земель пропорциональна той части региона, в которой эти земли расположены.
Спрос на сельскохозяйственную продукцию пропорционален численности населения, проживающей в некоторой части 
региона, а численность населения при равномерном заселении территории прямо пропорциональна площади проживания.
Таким образом, нам \textbf{следует взять ограничения на минимльный объем выращенных культур, прямо
  пропорциональный площади используемых земель}. Условимся, что для удовлетворения спроса жителей части
региона достаточно использования 25 процентов земель каждой части региона для каждой из культур.
Отсюда получим левые части соответствующих ограничений:
$ 0{,}8 \cdot 0,25 = 0{,}2, 0{,}6 \cdot 0,25 = 0{,}15 $. 

Получим решение модифицированной математической модели~\eqref{eq:ModifiedNSM}, используя Microsoft Office Excel:
$ x_{11} = 0{,}6, x_{12} = 0{,}45, x_{21} = 0,2, x_{22} = 0{,}15 $. Целевая функция $ E $ примет значение,
равное $ 270{,}88 $, таким образом, прибыль при таком плане использования земель составит $ 245{,}4 $ млн. ден.ед,
что на $ 22{,}4 $ ден.~ед. меньше, чем в исходном плане.

Сравнительная характеристика двух планов использования земель (в базовом и новом варианте) приведена
в таблице~\ref{tbl:Comparation}.

\renewcommand{\tabcolsep}{0.3em}
\begin{table}[h!]
  \centering
    \caption{Сравнительная характеристика планов использования земель\label{tbl:Comparation}}
    \begin{tabular}{|p{0.5\linewidth}|c|c|}
      \hline
      \centeringПоказатели & Базовый вариант & Новый вариант \\ \hline
      Площади используемых земель, млн~га: & & \\ 
      \hspace{2ex}озимые в западной части региона & 0{,}25  &  0{,}6  \\ 
      \hspace{2ex}озимые в восточной части региона &  0{,}6  &  0{,}45  \\ 
      \hspace{2ex}яровые в западной части региона &  0{,}55  &  0{,}2  \\ 
      \hspace{2ex}яровые в восточной части региона &  0  &  0{,}15  \\ \hline
      Объём выращенных культур, млн~центнеров: & & \\ 
      \hspace{2ex}озимые &  20  &  23{,}25  \\ 
      \hspace{2ex}яровые &  15{,}4  &  8{,}3  \\ \hline
      Неиспользованные площади, млн~га: & & \\
      \hspace{2ex}западная часть региона & 0 & 0 \\ 
      \hspace{2ex}восточная часть региона & 0 & 0 \\ \hline
      Прибыль, млн ден.~ед.: &  267{,}8  &  245{,}4  \\ \hline
    \end{tabular}
\end{table}

Видно, что введение дополнительных ограничений на использование земель для засева различных культур
приводит к снижению прибыли.

Рабочий лист с результатами решения модифицированной задачи с использованием
табличного процессора Excel приведен в приложении Б.

\newpage