\section{Задание}
\addcontentsline{toc}{section}{Задание}	% Добавляем его в оглавление

Написать хранимые процедуры для следующих видов запросов:
\begin{enumerate}
\item Три примера на изменение записи – изменить цену, название и количество товара. 
Название товара ввести с экрана.
\item Вывести количество товара данного наименования.
\item Вывести названия и цену товара, причем названия товара упорядочить по алфавиту.
\item Вывести название товара и выручку как вычисляемое поле=цена*количество.
\item Вывести название товара с минимальной ценой, цену также отобразить.
\item Создать выборку из двух таблиц – товары и производители. Вывести
название товара, цену, фирму и ее адрес.
\item Вывести название фирмы, производящей самый дешевый товар.
\item Привести два различных примера работы SELECT ... GROUP BY c группированием.
\item Вывести название фирмы, продающей товар по самой низкой цене.
Название товара ввести с экрана.
\item Продемонстрировать использование курсора.
\item Продемонстрировать работу представления.
\item Вывести таблицу по SELECT с вычисляемым полем НАЛОГ.
Для вычисления налога использовать функцию IIF (налог считать так: если выручка от
продажи товара больше 10 000, то налог равен 20\% от выручки, иначе – 13\% от
выручки.
\end{enumerate}

\clearpage