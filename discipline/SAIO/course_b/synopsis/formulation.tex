\section[Постановка задачи оптимизации]{ПОСТАНОВКА ЗАДАЧИ ОПТИМИЗАЦИИ}

Для посевов зерновых культур может использоваться 0,8 млн га земли в
западной части некоторого региона и 0,6 млн га --- в восточной.
Данные об урожайности зерновых культур с одного гектара приведены в
таблице~\ref{tbl:Cond}.  Прибыль от продажи одного центнера озимых
составляет 8 ден. ед., от продажи одного центнера яровых --– 7
ден. ед.
Необходимо произвести не менее 20 млн центнеров озимых и не менее 6
млн центнеров яровых.

Требуется составить план использования земель, обеспечивающий
максимальную прибыль от выращенного урожая.
\begin{table}[h]
  \centering
    \caption{Урожайность зерновых культур\label{tbl:Cond}}
    \begin{tabular}{|c|c|c|}
      \hline
      \multirow{2}*{Зерновая культура} & \multicolumn{2}{c|}{Урожайность, ц/га} \\ \cline{2-3}
      & запад & восток \\ \hline
      Озимые & 20 & 25 \\ \hline
      Яровые & 28 & 18 \\ \hline
    \end{tabular}
\end{table}

\pagebreak