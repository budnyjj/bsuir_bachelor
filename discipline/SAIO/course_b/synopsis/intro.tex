\section*{ВВЕДЕНИЕ}
\addcontentsline{toc}{section}{Введение}

\textbf{Исследование операций} --- наука, занимающаяся разработкой и
практическим применением методов наиболее оптимального управления
организационными системами.

\textbf{Предмет исследования операций} --- системы организационного управления
или организации, которые состоят из большого числа взаимодействующих
между собой подразделений не всегда согласующихся между собой и могут
быть противоположны.

\textbf{Цель исследования операций} --- количественное обоснование принимаемых
решений по управлению организациями

Решение, которое оказывается наиболее выгодным для всей организации
называется \textbf{оптимальным}, а решение наиболее выгодное одному или
нескольким подразделениям называется \textbf{субоптимальным}.

Основные этапы исследования операций:

\begin{itemize}
\item \textbf{Постановка задачи.}
  Происходит формирование концептуальной модели исследуемой
  системы (задачи), в которой в содержательной форме описывается состав
  системы, ее компоненты и их взаимосвязи, перечень основных показателей
  качества, переменных, контролируемых и неконтролируемых
  внешних факторов, а также их взаимосвязей с показателями качества
  системы, перечень стратегий управления (или решений), которые требуется
  определить в результате решения поставленной задачи.
\item \textbf{Построение аналитической модели рассматриваемого объекта
    (процесса).}
  Происходит формализация цели управления
  объектом, выделение возможных управляющих воздействий, влияющих на
  достижение сформулированной цели, а также описание системы ограничений
  на управляющие воздействия.
\item \textbf{Решение задач, сформулированных на базе построенной математической модели.}
  После достижения удовлетворительного уровня адекватности
  аналитической модели применяется соответствующий метод или алгоритм для нахождения
  оптимального (или субоптимального) решения задачи.
\item \textbf{Реализация полученного решения на практике.}
  Это один из важнейших этапов, завершающий операционное исследование. Внедрение в
  практику найденного на модели решения можно рассмотреть как
  самостоятельную задачу, применив системный подход и анализ. Полученной
  на модели оптимальной стратегии управления необходимо предоставить
  соответствующую содержательную форму в виде инструкций и правил,
  которая была бы понятной для административного персонала
  и легкой для выполнения в производственных условиях.
\end{itemize}

Краеугольным камнем исследования операций является \textbf{математическое
моделирование}. Хотя данные, полученные в процессе исследования
математических моделей, являются основой для принятия решений,
окончательный выбор обычно делается с учетом многих других
<<нематериальных>> факторов, которые невозможно отобразить в
математических моделях \cite{taxa}.

В исследовании операций нет единого общего метода решения всех
математических моделей, которые встречаются на практике. Вместо этого
выбор метода решения диктуют тип и сложность исследуемой
математической модели.
Наиболее известными методами исследования операций являются:

\begin{itemize}
\setlength{\itemsep}{0.2ex}
\item методы линейного программирования;
\item методы целочисленного программирования;
\item методы динамического программирования;
\item методы нелинейного программирования;
\item методы исследования функций классического анализа;
\item методы, основанные на использовании множителей Лагранжа;
\item вариационное исчисление;
\item принцип максимума;
\item метод геометрического программирования.
\end{itemize}

Ниже приведена краткая характеристика указанных методов и областей их
применения, что до некоторой степени может облегчить выбор того или
иного метода для решения конкретной оптимальной задачи.

Задачи оптимального планирования, связанные с отысканием оптимума заданной
целевой функции (линейной формы) при наличии ограничений в виде
линейных уравнений или линейных неравенств относятся к задачам
линейного программирования.

\textbf{Линейное программирование} --- направление математического программирования,
изучающее методы решения экстремальных задач, которые характеризуются
линейной зависимостью между переменными и линейным критерием.

Необходимым условием постановки задачи линейного программирования
являются ограничения на наличие ресурсов, величину спроса,
производственную мощность предприятия и другие производственные
факторы.

Математическая модель любой задачи линейного
программирования включает в себя:

\begin{itemize}
\setlength{\itemsep}{0.2ex}
\item критерий оптимальности;

\item систему ограничений в форме
  линейных уравнений и неравенств.

\end{itemize}

Для решения большого круга задач линейного программирования имеется
практически универсальный алгоритм --- \textbf{симплексный метод},
позволяющий за конечное число итераций находить оптимальное решение
подавляющего большинства задач. Тип используемых ограничений не
сказывается на возможности применения указанного
алгоритма. Дополнительной проверки на оптимальность для получаемых
решений не требуется.

Как правило, практические задачи линейного программирования отличаются 
значительным числом независимых переменных. Поэтому для их решения
обычно используют вычислительные машины, необходимая мощность которых
определяется размерностью решаемой задачи.

\textbf{Задачами нелинейного программирования} называются
задачи математического программирования, в которых нелинейны и/или
целевая функция, и/или ограничения в виде неравенств или равенств.
Задачи нелинейного программирования можно классифицировать в
соответствии с видом функции $ F(x) $, функциями ограничений и
размерностью вектора $ х $ (вектора решений). Общих способов решения,
аналогичных симплекс-методу линейного программирования, для
нелинейного программирования не существует. В каждом конкретном
случае способ выбирается в зависимости от вида функции $ F(x) $. Задачи
нелинейного программирования на практике возникают довольно часто,
когда, например, затраты растут непропорционально количеству
закупленных или произведённых товаров.

Многие задачи нелинейного программирования могут быть приближены к задачам линейного
программирования, и найдено решение, близкое к оптимальному. Встречаются
\textbf{задачи квадратичного программирования}, когда
функция $ F(x) $ является полиномом второй степени относительно переменных, а
ограничения линейны. В ряде случаев может быть применён \textbf{метод штрафных
функций}, сводящий задачу поиска экстремума при наличии ограничений к
аналогичной задаче при отсутствии ограничений, которая обычно решается
проще. Но в целом задачи нелинейного программирования относятся к
трудным вычислительным задачам. При их решении часто приходится
прибегать к приближенным методам оптимизации. Мощным средством для
решения задач нелинейного программирования являются численные
методы. Они позволяют найти решение задачи с заданной степенью
точности. 

\textbf{Динамическое программирование} --- вычислительный метод для решения задач
определенной структуры. Возникло и сформировалось в 1950--1953
гг. благодаря работам Р. Беллмана над динамическими задачами
управления запасами. В упрощенной формулировке динамическое
программирование представляет собой направленный последовательный
перебор вариантов, который обязательно приводит к глобальному
максимуму. 

Основные необходимые свойства задач, к которым возможно
применить этот метод:

\begin{itemize}
\setlength{\itemsep}{0.2ex}
\item задача должна допускать интерпретацию как
  n-шаговый процесс принятия решений;

\item задача должна быть определена для
  любого числа шагов и иметь структуру, не зависящую от их числа;

\item при рассмотрении k-шаговой задачи должно быть задано некоторое множество
  параметров, описывающих состояние системы, от которых зависят
  оптимальные значения переменных, причем это множество не должно
  изменяться при увеличении числа шагов;

\item выбор решения на k-м шаге не должен оказывать влияния на предыдущие решения, кроме
  необходимого пересчета переменных. 

\end{itemize}

 В основе метода ДП лежит \textbf{принцип оптимальности},
впервые сформулированный в 1953 г. американским
математиком Р. Э. Беллманом: каково бы ни было состояние системы в
результате какого-либо числа шагов, на ближайшем шаге нужно выбирать
управление так, чтобы оно в совокупности с оптимальным управлением на
всех последующих шагах приводило к оптимальному выигрышу на всех
оставшихся шагах, включая выигрыш на данном шаге. При решении задачи
на каждом шаге выбирается управление, которое должно привести к
оптимальному выигрышу. Если считать все шаги независимыми, тогда
оптимальным управлением будет то управление, которое обеспечит
максимальный выигрыш именно на данном шаге.

При решении задач методом динамического программирования, как
правило, используют вычислительные машины, обладающие достаточным
объемом памяти для хранения промежуточных результатов решения, которые
обычно получаются в табличной форме.

\textbf{Методы исследования функций классического анализа}
представляют собой наиболее известные методы решения несложных
задач оптимизации, которые известны из курса математического
анализа. Областью использования данных методов являются задачи
с известным аналитическим выражением критерия оптимальности, что
позволяет найти аналитическое выражение для
производных.

\textbf{Методы исследования при наличии ограничений на область изменения
независимых переменных} можно использовать только для отыскания
экстремальных значений внутри указанной области. Это
относится к задачам с большим числом независимых переменных
(больше двух), в которых анализ значений критерия
оптимальности на границе допустимой области изменения переменных
становится весьма сложным.

\textbf{Метод множителей Лагранжа} применяют для решения задач такого
же класса сложности, как и при использовании обычных методов
исследования функций, но при наличии ограничений типа равенств на
независимые переменные. В основном при использовании метода множителей
Лагранжа приходится решать те же задачи, что и без
ограничений. Некоторое усложнение в данном случае возникает лишь от
введения дополнительных неопределенных множителей, вследствие чего
порядок системы уравнений, решаемой для нахождения экстремумов
критерия оптимальности, соответственно повышается на число
ограничений. В остальном, процедура поиска решений и проверки их на
оптимальность отвечает процедуре решения задач без ограничений.

Множители Лагранжа можно применять для решения задач оптимизации
объектов на основе уравнений с частными производными и задач
динамической оптимизации. При этом вместо решения системы конечных
уравнений для отыскания оптимума необходимо интегрировать систему
дифференциальных уравнений.

Следует отметить, что множители Лагранжа используют также в качестве
вспомогательного средства и при решении специальными методами задач
других классов с ограничениями типа равенств, например, в вариационном
исчислении и динамическом программировании. Особенно эффективно
применение множителей Лагранжа в методе динамического
программирования, где с их помощью иногда удается снизить размерность
решаемой задачи.


\textbf{Методы вариационного исчисления} обычно используют для решения задач, в
которых критерии оптимальности представляются в виде функционалов и
решениями которых служат неизвестные функции. Такие задачи возникают
обычно при статической оптимизации процессов с распределенными
параметрами или в задачах динамической оптимизации.

При наличии ограничений типа равенств, имеющих вид функционалов,
применяют множители Лагранжа, что дает возможность перейти от условной
задачи к безусловной. Наиболее значительные трудности при
использовании вариационных методов возникают в случае решения задач с
ограничениями типа неравенств. Заслуживают внимания \textbf{прямые методы
решения задач оптимизации функционалов}, обычно позволяющие свести
исходную вариационную задачу к задаче нелинейного программирования,
решить которую иногда проще, чем краевую задачу для уравнений Эйлера.


\textbf{Принцип максимума} применяют для решения задач оптимизации процессов,
описываемых системами дифференциальных уравнений. Достоинством
математического аппарата принципа максимума является то, что решение
может определяться в виде разрывных функций; это свойственно многим
задачам оптимизации, например задачам оптимального управления
объектами, описываемыми линейными дифференциальными уравнениями.

Термин \textbf{геометрическое программирование}, появившийся в
математической литературе сравнительно недавно, применяется для
обозначения теории решения важного класса оптимизационных задач. Эти
задачи можно сформулировать в терминах функций специального вида — так
называемых «позиномов». Именно, это задачи отыскания наименьших
значений позиномов в областях, описываемых неравенствами
«позиномиального» вида. Оптимизационные задачи такого типа особенно
важны для приложений. Они постоянно возникают при решении
экономических задач, задач, связанных с проектированием разного рода
сооружений и др.

В общем случае задачи геометрического программирования требуют для
своего решения привлечения средств современной высшей математики и
использования ЭВМ. Однако, если несколько упростить задачу, например,
искать минимумы позиномов в области их определения, то можно
разработать общие методы их нахождения, которые базируются на
элементарной основе, т. е. не используют понятий и методов
дифференциального исчисления.


Некоторые математические модели могут быть такими сложными, что их
невозможно решить никакими доступными методами оптимизации. В этом
случае остается только \textbf{эвристический подход}: поиск подходящего
«хорошего» решения вместо оптимального решения. Эвристический подход
предполагает наличие эмпирических правил, в соответствии с которыми
ведется поиск подходящего решения.

\newpage