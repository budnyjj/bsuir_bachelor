\section[Построение базовой аналитической модели]{ПОСТРОЕНИЕ БАЗОВОЙ АНАЛИТИЧЕСКОЙ \\ МОДЕЛИ}

В соответствии с условием задания, требуется составить план использования земель,
обеспечивающий максимальную прибыль от выращенного урожая. 
Принимая это во внимание, введем следующие переменные:

\begin{itemize}

\item
  $ x_{11} \ge 0 $ --- площадь земель \textbf{западной части региона}, используемая для выращивание \textbf{озимых}, млн га;
\item
  $ x_{12} \ge 0 $ --- площадь земель \textbf{восточной части региона}, используемая для выращивание \textbf{озимых}, млн га;
\item
  $ x_{21} \ge 0 $ --- площадь земель \textbf{западной части региона}, используемая для выращивание \textbf{яровых}, млн га;
\item
  $ x_{22} \ge 0 $ --- площадь земель \textbf{восточной части региона}, используемая для выращивание \textbf{яровых}, млн га.

\end{itemize}

Определим ограничения математической модели в соответствии с данными, взятыми из условия задачи, а также из таблицы~\ref{tbl:Cond}:
\begin{equation}
  \label{eq:LimitA_b}
  \begin{aligned}
    20x_{11} + 25x_{12} & \ge 20, \\
    28x_{21} + 18x_{22} & \ge 6.
  \end{aligned}
\end{equation}

Неравенства~\eqref{eq:LimitA_b} соответствуют минимально допустимым объемам выращенных культур, для озимых и яровых соответственно.
\begin{equation}
  \label{eq:LimitS_b}
  \begin{aligned}
    x_{11} + x_{21} & \le 0{,}8, \\
    x_{12} + x_{22} & \le 0{,}6.
  \end{aligned}
\end{equation}

Неравенства~\eqref{eq:LimitS_b} соответствуют ограничениям площади, доступной для засева в западной и восточной части региона соответственно.

Целевая функция определяется из требования максимальности выручки, полученной от продажи урожая:
\begin{equation}
  \label{eq:BaseOpt}
  \begin{aligned}
    E &= 8(20x_{11} + 25x_{12}) + 7(28x_{21} + 18x_{22}) \rightarrow \max = \\
    &= 160x_{11} + 200x_{12} + 196x_{21} + 126x_{22} \rightarrow \max.
  \end{aligned}
\end{equation}

\pagebreak

Приведем полную математическую модель рассматриваемой задачи:

\begin{equation}
  \label{eq:BaseNSM}
    \begin{aligned}
      E = 160x_{11} + 200x_{12} &+ 196x_{21} + 126x_{22} \rightarrow \max, \\
      20x_{11} &+ 25x_{12} \ge 20, \\
      28x_{21} &+ 18x_{22} \ge 6, \\
      x_{11} &+ x_{21} \le 0{,}8, \\
      x_{12} &+ x_{22} \le 0{,}6, \\
      x_{ij} & \ge 0, i, j = 1, 2.
    \end{aligned}
\end{equation}

В данной задаче все переменные по своему физическому смыслу могут
принимать дробные значения, поэтому ограничения целочисленности на них не
накладываются.

\pagebreak