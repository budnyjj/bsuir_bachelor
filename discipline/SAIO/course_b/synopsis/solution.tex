\section[Решение задачи оптимизации на основе симплекс-метода]{РЕШЕНИЕ ЗАДАЧИ ОПТИМИЗАЦИИ \\ НА ОСНОВЕ СИМПЛЕКС-МЕТОДА}

\subsection{Приведение базовой аналитической модели к стандартной форме}

Приведем математическую модель задачи~\eqref{eq:BaseNSM} к стандартной форме.
Для этого в ограничения <<больше или равно>> требуется ввести избыточные
переменные, а в ограничения <<меньше или равно>> --- остаточные:
\begin{align}
  \label{eq:LimitAW_2}
  20x_{11} + 25x_{12} - x_{1} & = 20, \\
  \label{eq:LimitAE_2}
  28x_{21} + 18x_{22} - x_{2} & = 6, \\
  \nonumber
  x_{11} + x_{21} + x_{3} & = 0{,}8, \\
  \nonumber
  x_{12} + x_{22} + x_{4} & = 0{,}6, \\ 
  \nonumber
  x_{ij} \ge 0, i, j & = 1, 2, \\
  \nonumber
  x_{k} \ge 0,  k &= \overline{1,4}. 
\end{align}

В ограничениях~\eqref{eq:LimitAW_2} и~\eqref{eq:LimitAE_2}
отсутствует базисная переменная (переменная, входящая только в это
ограничение с коэффициентом, равным единице), поэтому следует
ввести исскуственные переменные $ x_{5}$ и $ x_{6} $ соответственно:
\begin{equation}
  \begin{aligned}
    20x_{11} + 25x_{12} - x_{1} + x_{5} & = 20, \\
    28x_{21} + 18x_{22} - x_{2} + x_{6} & = 6.
  \end{aligned}
\end{equation}

Таким образом, в каждом ограничении имеется по базисной переменной
($ x_{3}, x_{4}, x_{5}, x_{6}$). Остальные переменные --- небазисные.

Составим искусственную целевую функцию, равную сумме искусственных переменных:
\begin{equation}
  W = x_{5} + x_{6} \rightarrow \min.
\end{equation}

Выразим искусственную целевую функцию через небазисные переменные.
Для этого сначала выразим их через небазисные переменные:
\begin{equation}
  \label{eq:LimitA}
  \begin{aligned}
    x_{5} & = 20 - 20x_{11} - 25x_{12} + x_{1}, \\
    x_{6} & = 6 - 28x_{21} - 18x_{22} + x_{2}.
  \end{aligned}
\end{equation}

Подставим~\eqref{eq:LimitA} в исскуственную целевую функцию:
\begin{equation}
  \label{eq:Synthetic_1}
  W = 26 - 20x_{11} - 25x_{12} - 28x_{21} - 18x_{22} + x_{1} + x_{2} \rightarrow \min.
\end{equation}

Для приведения всей задачи к стандартной форме требуется перейти к искусственной
целевой функции, подлежацей максимизации. Для этого умножим выражение~\eqref{eq:Synthetic_1} на минус один:
\begin{equation}
  -W = -26 + 20x_{11} + 25x_{12} + 28x_{21} + 18x_{22} - x_{1} - x_{2} \rightarrow \max.
\end{equation}

Приведем полную математическую модель задачи с искусственным базисом в стандартной форме:

\begin{subequations}
  \renewcommand{\theequation}{\theparentequation\asbuk{equation}}
  \begin{equation}
    \begin{aligned}
      E = 160x_{11} + 200x_{12} &+ 196x_{21} + 126x_{12} \rightarrow \max, \\
      -W = -26 + 20x_{11} &+ 25x_{12} + 28x_{21} + \\
      &+ 18x_{22} - x_{1} - x_{2} \rightarrow \max. 
    \end{aligned}
  \end{equation}
  \vspace{-14pt}
  \begin{equation}
    \begin{aligned}
      20x_{11} + 25x_{12} - x_{1} + x_{5} & = 20,  \\
      28x_{21} + 18x_{22} - x_{2} + x_{6} & = 6,  \\
      x_{11} + x_{21} + x_{3} & = 0{,}8, \\
      x_{12} + x_{22} + x_{4}  & = 0{,}6,  \\
      x_{ij} \ge 0, i, j &= 1, 2, \\
      x_{k} \ge 0, k &= \overline{1,6}.
    \end{aligned}
  \end{equation}
\end{subequations}

Составим первую симплекс-таблицу (таблица~\ref{tbl:Simplex1_1}).

% redefine column width
\renewcommand{\tabcolsep}{0.55em}
\begin{table}[h]
  \centering
    \caption{\label{tbl:Simplex1_1}}
    \begin{tabular}{|c|c|c|c|c|c|c|c|c|c|c|c|}
      \hline
      Базис & $ x_{11} $ & $ x_{12} $ & $ x_{21} $ & $ x_{22} $ & $ x_{1} $ & $ x_{2} $ & $ x_{3} $ & $ x_{4} $ & $ x_{5} $ & $ x_{6} $ & Решение \\  
      \hline
      $ E $ & -160 & -200 & -196 & -126 & 0 & 0 & 0 & 0 & 0 & 0 & 0 \\  
      \hline
      $ -W $ & -20 & -25 & -28 & -18 & 1 & 1 & 0 & 0 & 0 & 0 & -26 \\  
      \hline
      $ x_{5} $ & 20 & 25 & 0 & 0 & -1 & 0 & 0 & 0 & 1 & 0 & 20 \\  
      \hline
      $ x_{6} $ & 0 & 0 & 28 & 18 & 0 & -1 & 0 & 0 & 0 & 1 & 6 \\  
      \hline
      $ x_{3} $ & 1 & 0 & 1 & 0 & 0 & 0 & 1 & 0  & 0 & 0 & 0{,}8 \\  
      \hline
      $ x_{4} $ & 0 & 1 & 0 & 1 & 0 & 0 & 0 & 1 & 0 & 0 & 0{,}6 \\  
      \hline
    \end{tabular}
\end{table}

Приведенное в таблице~\ref{tbl:Simplex1_1} начальное решение
$ x_{3} = 0{,}8, x_{4} = 0{,}6, x_{5} = 20, x_{6} = 6, x_{k} = x_{ij} = 0, $ при $ i,j,k=1{,}2 $
является недопустимым, так как оно не удовлетворяет начальной системе ограничений~\eqref{eq:BaseNSM} --- не выполняются условия~\eqref{eq:LimitA}.

Для поиска начального допустимого решения реализуется первый этап двухэтапного метода:
максимизация искусственной целевой функции на основе процедур симплекс-метода.

\subsection{Максимизация искусственной целевой функции}

Выберем переменную для включения в базис: это переменная $ x_{21} $, так как
ей соответствует максимальный по модулю отрицательный коэффициент в
строке искусственной целевой функции. Столбец $ x_{21} $ --- ведущий.
Для определения переменной, исключаемой из базиса, найдём
симплексные отношения:
\begin{align}
  \dfrac{20}{0} & = +\infty; &
  \dfrac{6}{28} & \approx 0{,}214; &
  \dfrac{0{,}8}{1} & = 0{,}8; &
  \dfrac{0{,}6}{0} & = +\infty.
\end{align}

Минимальное симплексное отношение соответствует переменной $ x_{6} $, следовательно,
эта переменная исключается из базиса. Строка $ x_{6} $ --- ведущая.
Ведущий элемент находится на пересечении ведущего столбца и
ведущей строки симплекс-таблицы. В нашем случае ведущий элемент принимает значение,
равное 28.

В результате преобразований по правилам симплекс-метода\footnote{
Правила преобразования симплекс-таблицы:
\begin{itemize}
\item в столбце <<Базис>> вместо переменной $ x_{6} $ указывается переменная $ x_{21} $.
\item все элементы ведущей строки делятся на ведущий элемент;
\item все элементы ведущего столбца (кроме ведущего элемента) заменяются нулями;
\item все остальные элементы таблицы (включая $ E $-строку и столбец <<Решение>>)
  пересчитываются по <<правилу прямоугольника>>:
  ведущий и пересчитываемый элемент образуют диагональ прямоугольника;
  находится произведение ведущего и пересчитываемого прямоугольника;
  из этого произведения вычитается произведение элементов,
  образующих противоположную диагональ прямоугольника; результат делится
  на ведущий элемент.
\end{itemize}}
получим 
следующую симплекс-таблицу (таблица~\ref{tbl:Simplex1_2}).

% redefine column width
\renewcommand{\tabcolsep}{0.45em}
\begin{table}[h]
  \centering
    \caption{\label{tbl:Simplex1_2}}
    \begin{tabular}{|c|c|c|c|c|c|c|c|c|c|c|c|}
      \hline
      Базис & $ x_{11} $ & $ x_{12} $ & $ x_{21} $ & $ x_{22} $ & $ x_{1} $ & $ x_{2} $ & $ x_{3} $ & $ x_{4} $ & $ x_{5} $ & $ x_{6} $ & Решение \\  
      \hline
      $ E $ & -160 & -200 & 0 & 0 & 0 & -7 & 0 & 0 & 0 & 7 & 42 \\  
      \hline
      $ -W $ & -20 & -25 & 0 & 0 & 1 & 0 & 0 & 0 & 0 & 1 & -20 \\  
      \hline
      $ x_{5} $ & 20 & 25 & 0 & 0 & -1 & 0 & 0 & 0 & 1 & 0 & 20 \\  
      \hline
      $ x_{21} $ & 0 & 0 & 1 & 0{,}64 & 0 & -0{,}04 & 0 & 0 & 0 & 0{,}04 & 0{,}2143 \\  
      \hline
      $ x_{3} $ & 1 & 0 & 0 & -0{,}64 & 0 & 0{,}04 & 1 & 0  & 0 & -0{,}04 & 0{,}5857 \\  
      \hline
      $ x_{4} $ & 0 & 1 & 0 & 1 & 0 & 0 & 0 & 1 & 0 & 0 & 0{,}6 \\  
      \hline
    \end{tabular}
\end{table}

Как видно из таблицы~\ref{tbl:Simplex1_2}, значение искусственной
целевой функции не равно нулю, следовательно, требуется сделать ещё
как минимум одну итерацию симплекс-метода.

Выберeм переменную для включения в базис: это переменная $ x_{12} $, так как
ей соответствует максимальный по модулю отрицательный коэффициент в
строке искусственной целевой функции.
Для определения переменной, исключаемой из базиса, найдём
симплексные отношения:
\begin{align}
  \dfrac{20}{25} & = 0{,}8; &
  \dfrac{0{,}21429}{0} & = +\infty; &
  \dfrac{0{,}58571}{0} & = +\infty; &
  \dfrac{0{,}6}{1} & = 0{,}6.
\end{align}

Минимальное симплексное отношение соответствует переменной $ x_{4} $, следовательно,
эта переменная исключается из базиса.
В результате преобразований по правилам симплекс-метода получим 
следующую симплекс-таблицу (таблица~\ref{tbl:Simplex1_3}).

\begin{table}[h]
  \centering
    \caption{\label{tbl:Simplex1_3}}
    \begin{tabular}{|c|c|c|c|c|c|c|c|c|c|c|c|}
      \hline
      Базис & $ x_{11} $ & $ x_{12} $ & $ x_{21} $ & $ x_{22} $ & $ x_{1} $ & $ x_{2} $ & $ x_{3} $ & $ x_{4} $ & $ x_{5} $ & $ x_{6} $ & Решение \\  
      \hline
      $ E $ & -160 & 0 & 0 & 200 & 0 & -7 & 0 & 200 & 0 & 7 & 162 \\  
      \hline
      $ -W $ & -20 & 0 & 0 & 25 & 1 & 0 & 0 & 25 & 0 & 1 & -5 \\  
      \hline
      $ x_{5} $ & 20 & 0 & 0 & -25 & -1 & 0 & 0 & -25 & 1 & 0 & 5 \\  
      \hline
      $ x_{21} $ & 0 & 0 & 1 & 0{,}64 & 0 & -0{,}04 & 0 & 0 & 0 & 0{,}04 & 0{,}2143 \\  
      \hline
      $ x_{3} $ & 1 & 0 & 0 & -0{,}64 & 0 & 0{,}04 & 1 & 0  & 0 & -0{,}04 & 0{,}5857 \\  
      \hline
      $ x_{12} $ & 0 & 1 & 0 & 1 & 0 & 0 & 0 & 1 & 0 & 0 & 0{,}6 \\  
      \hline
    \end{tabular}
\end{table}


Как видно из таблицы~\ref{tbl:Simplex1_3}, значение искусственной
целевой функции не равно нулю, следовательно, требуется сделать
еще как минимум одну итерацию симплекс-метода.

Выберeм переменную для включения в базис: это переменная $ x_{11} $, так как
ей соответствует максимальный по модулю отрицательный коэффициент в
строке искусственной целевой функции.
Для определения переменной, исключаемой из базиса, найдем
симплексные отношения:
\begin{align}
  \dfrac{5}{20} & = 0{,}25; &
  \dfrac{0{,}2143}{0} & = +\infty; &
  \dfrac{0{,}5857}{1} & = 0{,}5857; &
  \dfrac{0{,}6}{0} & = +\infty.
\end{align}

Минимальное симплексное отношение соответствует переменной $ x_{5} $, следовательно,
эта переменная исключается из базиса.
В результате преобразований по правилам симплекс-метода получим 
следующую симплекс-таблицу (таблица~\ref{tbl:Simplex1_4}).

% redefine column width
\renewcommand{\tabcolsep}{0.35em}
\begin{table}[h]
  \centering
    \caption{\label{tbl:Simplex1_4}}
    \begin{tabular}{|c|c|c|c|c|c|c|c|c|c|c|c|}
      \hline
      Базис & $ x_{11} $ & $ x_{12} $ & $ x_{21} $ & $ x_{22} $ & $ x_{1} $ & $ x_{2} $ & $ x_{3} $ & $ x_{4} $ & $ x_{5} $ & $ x_{6} $ & Решение \\  
      \hline
      $ E $ & 0 & 0 & 0 & 0 & -8 & -7 & 0 & 0 & 8 & 7 & 202 \\  
      \hline
      $ -W $ & 0 & 0 & 0 & 0 & 0 & 0 & 0 & 0 & 1 & 1 & 0 \\  
      \hline
      $ x_{11} $ & 1 & 0 & 0 & -1{,}25 & -0{,}05 & 0 & 0 & -1{,}25 & 0{,}05 & 0 & 0{,}25 \\  
      \hline
      $ x_{21} $ & 0 & 0 & 1 & 0{,}64 & 0 & -0{,}04 & 0 & 0 & 0 & 0{,}04 & 0{,}2143 \\  
      \hline
      $ x_{3} $ & 0 & 0 & 0 & 0{,}61 & 0{,}05 & 0{,}04 & 1 & 1{,}25 & -0{,}05 & -0{,}04 & 0{,}3357 \\  
      \hline
      $ x_{12} $ & 0 & 1 & 0 & 1 & 0 & 0 & 0 & 1 & 0 & 0 & 0{,}6 \\  
      \hline
    \end{tabular}
\end{table}

Как видно из таблицы~\ref{tbl:Simplex1_4}, искусственная целевая функция равна нулю, и в
базисе нет искусственных переменных. Получено допустимое решение:
\begin{equation}
    \begin{aligned}
      x_{11} &= 0{,}25; &
      x_{12} &= 0{,}6; &
      x_{21} &= 0{,}2143; &
      x_{3} &= 0{,}3357; \\
    \end{aligned}
\end{equation}
\begin{equation}
  x_{k} = x_{22} = 0, k=1,2,4,5,6.
\end{equation}

В том, что оно допустимо, нетрудно убедиться, подставив значения переменных в систему
ограничений~\eqref{eq:BaseNSM}.

Таким образом, первый этап двухэтапного метода завершен. Искусственная целевая функция
и искусственные переменные исключаются из симплекс-таблицы (таблица~\ref{tbl:Simplex2_1})

% redefine column width
\renewcommand{\tabcolsep}{0.65em}
\begin{table}[h]
  \centering
    \caption{\label{tbl:Simplex2_1}}
    \begin{tabular}{|c|c|c|c|c|c|c|c|c|c|c|c|}
      \hline
      Базис & $ x_{11} $ & $ x_{12} $ & $ x_{21} $ & $ x_{22} $ & $ x_{1} $ & $ x_{2} $ & $ x_{3} $ & $ x_{4} $ & Решение \\  
      \hline
      $ E $ & 0 & 0 & 0 & 0 & -8 & -7 & 0 & 0 & 202 \\  
      \hline
      $ x_{11} $ & 1 & 0 & 0 & -1{,}25 & -0{,}05 & 0 & 0 & -1{,}25 & 0{,}25 \\  
      \hline
      $ x_{21} $ & 0 & 0 & 1 & 0{,}64 & 0 & -0{,}04 & 0 & 0 & 0{,}2143 \\  
      \hline
      $ x_{3} $ & 0 & 0 & 0 & 0{,}61 & 0{,}05 & 0{,}04 & 1 & 1{,}25 & 0{,}3357 \\  
      \hline
      $ x_{12} $ & 0 & 1 & 0 & 1 & 0 & 0 & 0 & 1 & 0{,}6 \\  
      \hline
    \end{tabular}
\end{table}

\newpage

\subsection{Максимизация основной целевой функции}

Полученное решение является допустимым, но не оптимальным: признак
неоптимальности решения --- наличие отрицательных коэффициентов в строке
целевой функции $ E $. Поэтому реализуется второй этап двухэтапного метода ---
максимизация основной целевой функции E.

Включим в базис переменную $ x_{1} $, так как ей соответствует
максимальный по модулю отрицательный коэффициент в строке целевой
функции. Для определения переменной, исключаемой из базиса, вычислим
симплексные отношения:
\begin{align}
  \dfrac{0{,}21429}{0} & = +\infty; &
  \dfrac{0{,}3357}{0{,}05} & = 6{,}714; &
  \dfrac{0{,}6}{0} & = +\infty.
\end{align}

Минимальное симплексное отношение соответствует переменной $ x_{3} $,
следовательно, эта переменная исключается из базиса.
После преобразований по правилам симплекс-метода
будет получена новая симплекс-таблица (таблица~\ref{tbl:Simplex2_2}).

% redefine column width
\renewcommand{\tabcolsep}{0.7em}
\begin{table}[h]
  \centering
    \caption{\label{tbl:Simplex2_2}}
    \begin{tabular}{|c|c|c|c|c|c|c|c|c|c|c|c|}
      \hline
      Базис & $ x_{11} $ & $ x_{12} $ & $ x_{21} $ & $ x_{22} $ & $ x_{1} $ & $ x_{2} $ & $ x_{3} $ & $ x_{4} $ & Решение \\  
      \hline
      $ E $ & 0 & 0 & 0 & 97{,}14 & 0 & -1{,}29 & 160 & 200 & 255{,}7143 \\  
      \hline
      $ x_{11} $ & 1 & 0 & 0 & -0{,}64 & 0 & 0{,}04 & 1 & 0 & 0{,}5857 \\  
      \hline
      $ x_{21} $ & 0 & 0 & 1 & 0{,}64 & 0 & -0{,}04 & 0 & 0 & 0{,}2143 \\  
      \hline
      $ x_{1} $ & 0 & 0 & 0 & 12{,}14 & 1 & 0{,}71 & 20 & 25 & 6{,}7143 \\  
      \hline
      $ x_{12} $ & 0 & 1 & 0 & 1 & 0 & 0 & 0 & 1 & 0{,}6 \\  
      \hline
    \end{tabular}
\end{table}

Полученное решение также не является оптимальным, поэтому повторим процесс 
максимизации основной целевой функции E.

Включим в базис переменную $ x_{2} $, так как ей соответствует
максимальный по модулю отрицательный коэффициент в строке целевой
функции. Для определения переменной, исключаемой из базиса, вычислим
симплексные отношения:
\begin{align}
  \dfrac{0{,}5857}{0{,}04} & = 14{,}6425; &
  \dfrac{6{,}7143}{0{,}71} & = 9{,}4568; &
  \dfrac{0{,}6}{0} & = +\infty.
\end{align}
Минимальное симплексное отношение соответствует переменной $ x_{1} $,
следовательно, эта переменная исключается из базиса.
После преобразований по правилам симплекс-метода
будет получена новая симплекс-таблица (таблица~\ref{tbl:Simplex2_3}),
содержащая оптимальное решение (признак его оптимальности --- отсутствие
отрицательных элементов в строке целевой функции).

\begin{table}[h]
  \centering
    \caption{\label{tbl:Simplex2_3}}
    \begin{tabular}{|c|c|c|c|c|c|c|c|c|c|c|c|}
      \hline
      Базис & $ x_{11} $ & $ x_{12} $ & $ x_{21} $ & $ x_{22} $ & $ x_{1} $ & $ x_{2} $ & $ x_{3} $ & $ x_{4} $ & Решение \\  
      \hline
      $ E $ & 0 & 0 & 0 & 119 & 1{,}8 & 0 & 196 & 245 & 267{,}8 \\  
      \hline
      $ x_{11} $ & 1 & 0 & 0 & -1{,}25 & -0.05 & 0 & 0 & -1{,}25 & 0{,}25 \\  
      \hline
      $ x_{21} $ & 0 & 0 & 1 & 1{,}25 & 0{,}05 & 0 & 1 & 1{,}25 & 0{,}55 \\  
      \hline
      $ x_{2} $ & 0 & 0 & 0 & 17 & 1{,}4 & 1 & 28 & 35 & 9{,}4 \\  
      \hline
      $ x_{12} $ & 0 & 1 & 0 & 1 & 0 & 0 & 0 & 1 & 0{,}6 \\  
      \hline
    \end{tabular}
\end{table}

\subsection{Анализ результатов оптимизации}

В соответствии со значениями таблицы ~\ref{tbl:Simplex2_3}, основные переменные
задачи приняли следующие значения:
\begin{equation}
  \label{eq:Solution}
    \begin{aligned}
      x_{11} &= 0{,}25; &
      x_{12} &= 0{,}6; &
      x_{21} &= 0{,}55; &
      x_{22} &= 0; &
      x_{k} &= 0, k=\overline{1,4}.
    \end{aligned}
\end{equation}

Это означает, что для получения максимальной прибыли от продажи урожая следует
составить следующий план использования земель:
\begin{itemize}
\item Засеять 0{,}25 млн га земель \textbf{западной} части региона \textbf{озимыми};
\item Засеять 0{,}6 млн га земель \textbf{восточной} части региона \textbf{озимыми};
\item Засеять 0{,}55 млн га земель \textbf{западной} части региона \textbf{яровыми};
\item Засеять 0 млн га земель \textbf{восточной} части региона \textbf{яровыми}.
\end{itemize}

В таком случае будут получены следующие результаты:
\begin{itemize}
\item Значение целевой функции $ E = 267{,}8 $ показывает, что прибыль при таком
  плане использования земель составит 267{,}8 млн. ден.ед.

\item Избыточная переменная $ x_{1} = 0 $ означает, что при рассчитанном плане использования
  земель урожай озимых будет равен минимальному ограничению плана.

\item Избыточная переменная $ x_{2} = 9{,}4 $ означает, что план урожайности яровых 
  будет перевыполнен на 9{,}4 млн. центнеров (требуется не менее 6 млн. центнеров, а
  вырастет 15{,}4 млн. центнеров).

\item Остаточные переменные $ x_{3} = x_{4} = 0 $ означают, что для засева будут
  использованы все доступные площади (то есть неиспользуемых земель не останется).
\end{itemize}

Рабочий лист с результатами решения задачи с использованием
табличного процессора Microsoft Office Excel приведен в приложении А.

\pagebreak