\section*{ЗАКЛЮЧЕНИЕ}
\addcontentsline{toc}{section}{Заключение}

В результате решения поставленной задачи было получено следующее оптимальное решение:
\begin{itemize}
\item Засеять 0{,}25 млн га земель \textbf{западной} части региона \textbf{озимыми};
\item Засеять 0{,}6 млн га земель \textbf{восточной} части региона \textbf{озимыми};
\item Засеять 0{,}55 млн га земель \textbf{западной} части региона \textbf{яровыми};
\item Засеять 0 млн га земель \textbf{восточной} части региона \textbf{яровыми}.
\end{itemize}

В результате будут использованы все доступные площади земель, а выручка составит 267{,}8 ден.~ед.

Данное решение не изменится при следующих изменениях исходных данных:
\begin{itemize}
  \item изменение величины площади, доступной для засева в восточной
    части региона, входящей в диапазон от 0{,}332 до 0{,}8 млн га;
  \item изменение величины минимально допустимого объёма выращенных озимых,
    входящей в диапазон от 15 до 26{,}71 млн центнеров;
  \item изменение величины прибыли от продажи одного центнера яровых
    в западной части региона, большее, чем минус 26 д.~е.
\end{itemize}

В разработанном плане использования земель был обнаружена особенность:
земли восточной части региона не будут засеваться яровыми.
Были проанализированы различные пути её устранения, рассмотрен наиболее реалистичный.
В рамках его реализации были введены дополнительные ограничения на характер 
использования доступных земель, построена модифицированная математическая модель задачи,
получено её решение, а также сформирован модифицированный план засева земель с учетом 
данной особенности.

\newpage

