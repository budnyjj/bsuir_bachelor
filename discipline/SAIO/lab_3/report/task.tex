\section{Задание}

На участке выпускаются детали двух видов. Интервалы времени между моментами поступления заготовок для выпуска деталей примерно постоянные и составляют 5 минут. Все заготовки обрабатываются на станке A; время обработки на станке составляет от 2 до 4 минут.
10\% деталей, выпущенных на станке A, продаются как готовые изделия (детали типа 1). Остальные проходят дальнейшую обработку (из них выпускаются детали типа 2). Детали типа 1 со станка A поступают на два одинаковых станка (B1 и B2); время обработки одной детали на этих станках распределено по экспоненциальному закону и составляет в среднем 15 минут. Перед станками B1 и B2 установлен общий накопитель, вмещающий пять деталей; при его заполнении все поступающие детали типа 1 направляются на станок C, на котором обработка занимает в среднем 10 минут (экспоненциальная случайная величина).

Затраты (в денежных единицах), связанные с работой и простоями каждого станка (в минуту), приведены в таблице. 

\begin{table}[h]
	\caption{}
	\centering
	\begin{tabular}{|c|c|c|c|}
		\hline
		& A & B & C \\ \hline
		Работа & 0,2 & 0,5 & 0,7 \\ \hline
		Простой & 0,1 & 0,1 & 0,1 \\ 
		\hline
	\end{tabular}
\end{table}

Прочие расходы, связанные с выпуском деталей типа 1 и 2, составляют 3 и 10 д.е. соответственно. Детали типа 1 продаются по цене 8 д.е., типа 2 --- 35 д.е.

\begin{enumerate}

\item Найти характеристики работы станка A (10.4, 10.7).
\item Найти характеристики работы группы станков B1-B2 (10.13, 10.4, 10.9). Поток деталей на эту группу станков считать пуассоновским.
\item Рассчитать характеристики работы станка C (10.13, 10.4, 10.7). Поток деталей на станок C считать пуассоновскими.
\item Найти прибыль от работы участка за 8 часов (10.6, 10.7, 10.9).
\item Найти вероятность того, что деталь, поступившая на станки B1-B2, сразу же начнет обрабатываться (не будет ждать в очереди) (10.5, 10.9, пример из 10.8).
\item Найти характеристики работы всех станков и прибыль от работы участка (за 8 часов) при следующих изменениях: заготовки поступают на обработку чаще (через каждые 4 минуты), а станок A заменен на новый (A1); время обработки одной детали на станке A1 --- от 1 до 3 минут. Для нового станка A1 затраты на одну минуту работы и простоя --- 0,4 и 0,2 д.е. соответственно. Определить, являются ли предлагаемые изменения целесообразными.

\end{enumerate}

\pagebreak