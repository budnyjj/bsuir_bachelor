\documentclass[a4paper,12pt]{article}

\usepackage[T2A]{fontenc}
\usepackage[utf8]{inputenc}
\usepackage[russian]{babel}
\usepackage{cmap}

\usepackage{xcolor}

\usepackage{helvet}
\usepackage{pscyr}


\usepackage{multicol}

% \usepackage{amssymb,amsfonts,amsmath,mathtext}
% \usepackage{cite,enumerate,float}

\graphicspath{{images/}}		% Подключаемые пакеты
%%% Макет страницы %%%
\geometry{a4paper,top=20mm,bottom=27mm,left=30mm,right=15mm}
\setstretch{1.15}

%%% Язык текста %%%
\selectlanguage{russian}

%%% Кодировки и шрифты %%%
\renewcommand{\rmdefault}{ftm} % Включаем Times New Roman

%%% Выравнивание и переносы %%%
\sloppy				% Избавляемся от переполнений
\clubpenalty=10000		% Запрещаем разрыв страницы после первой строки абзаца
\widowpenalty=10000		% Запрещаем разрыв страницы после последней строки абзаца
\interfootnotelinepenalty=10000 % Запрет разрывов сносок

%%% Нумерация страниц %%%
\fancypagestyle{empty}{%
\fancyhf{} % clear all header and footer fields
\renewcommand{\headrulewidth}{0pt}
\renewcommand{\footrulewidth}{0pt}
\setlength{\headheight}{5mm} 
}

\fancypagestyle{plain}{%
\fancyhf{} % clear all header and footer fields
\fancyfoot[R]{\thepage} 
\renewcommand{\headrulewidth}{0pt}
\renewcommand{\footrulewidth}{0pt}
\setlength{\headheight}{5mm}
}

\pagestyle{plain}

%%% Библиография %%%

\makeatletter
\bibliographystyle{ugost2003s} % Оформляем библиографию в соответствии с ГОСТ 7.1 2003

\let\oldthebibliography=\thebibliography
\let\endoldthebibliography=\endthebibliography
\renewenvironment{thebibliography}[1]{
  \begin{oldthebibliography}{#1}
    \setlength{\parskip}{0mm}
    \setlength{\itemsep}{0mm}
}
{
\end{oldthebibliography}
}

%%% Изображения %%%
\graphicspath{{images/}} % Пути к изображениям

%%% Содержание %%%
\renewcommand{\cfttoctitlefont}{\hfil \large\bfseries}

\setlength{\cftparskip}{0mm}
\setlength{\cftbeforesecskip}{0mm}
\setlength{\cftaftertoctitleskip}{14pt}

\renewcommand{\cftsecaftersnumb}{\:}
\renewcommand{\cftsecfont}{}   
\renewcommand{\cftsecpagefont}{\normalsize}
\renewcommand{\cftsecleader}{\cftdotfill{\cftdotsep}}
\setlength{\cftsecindent}{0mm}
\setlength{\cftsecnumwidth}{3mm}

\setlength{\cftsubsecindent}{4mm}
\setlength{\cftsubsecnumwidth}{8mm}

%%% Требования ЕСКД/СТП %%%

%%% Размеры заголовков
\newcommand{\sectionbreak}{\clearpage}

\titleformat{\section}{\large\bfseries}{\thesection}{\wordsep}{}
\titlespacing*{\section}{12mm}{14pt}{14pt}

\titleformat{name=\section,numberless}{\large\bfseries\filcenter}{}{0mm}{}
\titlespacing*{name=\section,numberless}{0mm}{14pt}{14pt}

\titleformat{name=\subsection}{\normalsize\bfseries}{\thesubsection}{\wordsep}{}
\titlespacing*{\subsection}{12mm}{14pt}{14pt}

\titleformat{name=\subsection,numberless}{\normalsize\bfseries}{}{0mm}{}
\titlespacing*{name=\subsection,numberless}{0mm}{14pt}{14pt}

%%% Нумерация параграфов

\counterwithout{paragraph}{subsubsection}
\counterwithin{paragraph}{subsection}
\renewcommand{\theparagraph}{\thesubsection.\arabic{paragraph}}
\setcounter{secnumdepth}{4}

\titleformat{name=\paragraph}[runin]{\normalsize\bfseries}{\theparagraph}{\wordsep}{}
\titlespacing*{\paragraph}{12mm}{14pt}{\wordsep}

%%% Размеры текста формул %%%

\DeclareMathSizes{12}{12}{6}{4}

%%% Расстояние между формулами

\AtBeginDocument{%
  \setlength\abovedisplayskip{14pt}%
  \setlength\belowdisplayskip{14pt}%
  \setlength\abovedisplayshortskip{14pt}%
  \setlength\belowdisplayshortskip{14pt}%
}

%%% Оформление текста

\setlength{\parskip}{0pt}
\setlength{\parindent}{12mm}

%%% Расстояние между плавающими элементами

\setlength{\floatsep}{14pt}     % between top floats
\setlength{\textfloatsep}{14pt} % between top/bottom floats and text
\setlength{\intextsep}{14pt}    % between text and float
\setlength{\dbltextfloatsep}{14pt}
\setlength{\dblfloatsep}{14pt}

 % костыль для того, чтобы убрать расстояние от картинки до текста
\setlength{\abovecaptionskip}{0pt}
\setlength{\belowcaptionskip}{0pt}
           
%%% Оформление списков
\AddEnumerateCounter{\asbuk}{\@asbuk}{\cyrm}

\setlist{nosep,listparindent=\parindent}
\setlist[1]{itemindent=18.5mm,leftmargin=0mm,itemsep=0mm,topsep=0mm,parsep=0mm}             
\setlist[itemize,1]{label=$-$}
\setlist[enumerate,1]{label=\arabic*)}

\setlist[2]{itemindent=20.5mm,leftmargin=0mm,itemsep=0mm,topsep=0mm,parsep=0mm}             

% Определяем новый стиль для списков,
% на которые есть ссылки в тексте
\newlist{reflist}{enumerate*}{1}
\setlist*[reflist,1]{%
  label=\asbuk*),
}

\setlist*[reflist,2]{%
  label=\arabic*),
}

%% Нумерация плавающих элементов

\counterwithin{figure}{section}
\counterwithin{table}{section}

\makeatletter
\AtBeginDocument{%
\renewcommand{\thetable}{\thesection.\arabic{table}}
\renewcommand{\thelstlisting}{\thesection.\arabic{lstlisting}}
\renewcommand{\thefigure}{\thesection.\arabic{figure}}
\let\c@lstlisting\c@figure}
\makeatother 

%% Подписи плавающих элементов

\captionsetup[figure]{
  labelsep=endash,
  justification=centering,
  singlelinecheck=false,
  position=bottom,
  skip=14pt}

\captionsetup[table]{
  labelsep=endash,
  justification=raggedright,
  singlelinecheck=false,
  position=top,
  skip=0mm}

\captionsetup[lstlisting]{
  labelsep=endash
}

\lstset{
basicstyle=\scriptsize\ttfamily,
numberstyle=\scriptsize\ttfamily,
keywordstyle=\bfseries,
commentstyle=\itshape,
numbers=left,
stepnumber=1,
frame=single,
resetmargins=true,
xleftmargin=7mm,
xrightmargin=2mm,
captionpos=b,
keepspaces=true,
breaklines=true,
aboveskip=22pt,
belowskip=10pt,
abovecaptionskip=16pt}

\renewcommand{\arraystretch}{1.5}

%%% Настройка размеров вертикальных отступов

\renewcommand{\smallskip}{\vspace{6pt}}
\renewcommand{\bigskip}{\vspace{14pt}}
			% Пользовательские стили

\begin{document}

\section*{Лабораторная работа №2.  Классы \\  Вариант №4}

\lstset{ %
  language=C++,                 % выбор языка для подсветки
  basicstyle=\small\ttfamily, % размер и начертание шрифта для подсветки кода
  numbers=left,               % где поставить нумерацию строк (слева\справа)
  numberstyle=\tiny,           % размер шрифта для номеров строк
  stepnumber=1,                   % размер шага между двумя номерами строк
  numbersep=5pt,                % как далеко отстоят номера строк от подсвечиваемого кода
  backgroundcolor=\color{white}, % цвет фона подсветки - используем \usepackage{color}
  showspaces=false,            % показывать или нет пробелы специальными отступами
  showstringspaces=false,      % показывать или нет пробелы в строках
  showtabs=false,             % показывать или нет табуляцию в строках
  frame=single,              % рисовать рамку вокруг кода
  tabsize=2,                 % размер табуляции по умолчанию равен 2 пробелам
  captionpos=t,              % позиция заголовка вверху [t] или внизу [b] 
  breaklines=true,           % автоматически переносить строки (да\нет)
  breakatwhitespace=false % переносить строки только если есть пробел
}

\begin{multicols}{2}
  \begin{flushleft}
    \textbf{Проверил:} \\
    ассистент кафедры ИТАС \\
    Прищепчик М. В.

  \end{flushleft}

  \begin{flushright}
    \textbf{Выполнили:} ст. группы 120602 \\
    Анашкевич П. С. \\
    Будный Р. И. 
  \end{flushright}
\end{multicols}

\subsection{Цель работы}
\begin{enumerate}

\item Изучение возможностей языка C++ в определении пользовательских типов данных.
\item Изучение правил определения и переопределения функций доступа к объектам класса.
\item Использование статических элементов классов и дружественных функций.

\end{enumerate}

\subsection{Задание}

Определите узел бинарного дерева следующим образом:

\begin{lstlisting}
  class Node
  {
    char name[10];	
    Node * left;	
    Node * right;	
    ...
  };
\end{lstlisting}

Определите в классе следующие функции:

\small{\ttfamily{Init()}} -- инициализация узла. Функция должна установить указатели на левый и правый узел в ноль;

\small{\ttfamily{AddNode()}} -- добавление узла в левую или правую ветви. Если слева нет узла, то добавить слева, если справа нет узла, то добавить справа, иначе ничего не добавлять;

\small{\sffamily{DelTree()}} -- удаление поддеревьев;

\small{\sffamily{Print()}} -- рекурсивная функция вывода дерева на экран.

\subsection{Листинг программы}

\lstinputlisting[caption=main.cpp]{code/main.cpp}
\lstinputlisting[caption=tree.h]{code/tree.h}
\lstinputlisting[caption=tree.cpp]{code/tree.cpp}

\subsection{Выводы}

В ходе проведения лабораторной работы были изучены возможности языка C++ в определении пользовательских типов данных.

\end{document}