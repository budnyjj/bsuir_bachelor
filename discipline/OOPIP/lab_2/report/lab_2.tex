\documentclass[a4paper,12pt]{article}

%%% Поля и разметка страницы %%%
\usepackage{pdflscape}   % Для включения альбомных страниц
\usepackage{geometry} % Для последующего задания полей
\usepackage{setspace} % Для интерлиньяжа
\usepackage[14pt]{extsizes}
\usepackage{titlesec}
\usepackage{tocloft}
\usepackage{enumitem}
\usepackage{fancyhdr}

%%% Кодировки и шрифты %%%
\usepackage{cmap}                    % Улучшенный поиск русских слов в полученном pdf-файле
\usepackage[T2A]{fontenc}	     % Поддержка русских букв
\usepackage[utf8]{inputenc}	     % Кодировка utf8
\usepackage[english=nohyphenation,russian=nohyphenation]{hyphsubst} % Запрет переносов
\usepackage[english, russian]{babel} % Языки: русский, английский
% \usepackage{pscyr}						% Красивые русские шрифты

%%% Математические пакеты %%%
\usepackage{amsthm,amsfonts,amsmath,amssymb,amscd} % Математические дополнения от AMS

%%% Оформление абзацев %%%
\usepackage{indentfirst} % Красная строка

%%% Цвета %%%
\usepackage[usenames]{color}
\usepackage{color}
\usepackage{colortbl}

%%% Таблицы %%%
\usepackage{longtable}		     % Длинные таблицы
\usepackage{multirow,makecell,array} % Улучшенное форматирование таблиц

%%% Общее форматирование
\usepackage{caption}
\captionsetup[figure]{labelsep=space,justification=centering,singlelinecheck=false}

\usepackage{soul}                    % Поддержка переносоустойчивых подчёркиваний и зачёркиваний
\usepackage{multicol}

%%% Библиография %%%
\usepackage{cite} % Красивые ссылки на литературу

%%% Гиперссылки %%%
\usepackage[unicode,plainpages=false,pdfpagelabels=false]{hyperref}

%%% Изображения %%%
\usepackage{graphicx} % Подключаем пакет работы с графикой		% Подключаемые пакеты
../../../../template/lab/sys/styles.tex			% Пользовательские стили

\begin{document}

\section*{Лабораторная работа №2.  Классы \\  Вариант №4}

\lstset{ %
  language=C++,                 % выбор языка для подсветки
  basicstyle=\small\ttfamily, % размер и начертание шрифта для подсветки кода
  numbers=left,               % где поставить нумерацию строк (слева\справа)
  numberstyle=\tiny,           % размер шрифта для номеров строк
  stepnumber=1,                   % размер шага между двумя номерами строк
  numbersep=5pt,                % как далеко отстоят номера строк от подсвечиваемого кода
  backgroundcolor=\color{white}, % цвет фона подсветки - используем \usepackage{color}
  showspaces=false,            % показывать или нет пробелы специальными отступами
  showstringspaces=false,      % показывать или нет пробелы в строках
  showtabs=false,             % показывать или нет табуляцию в строках
  frame=single,              % рисовать рамку вокруг кода
  tabsize=2,                 % размер табуляции по умолчанию равен 2 пробелам
  captionpos=t,              % позиция заголовка вверху [t] или внизу [b] 
  breaklines=true,           % автоматически переносить строки (да\нет)
  breakatwhitespace=false % переносить строки только если есть пробел
}

\begin{multicols}{2}
  \begin{flushleft}
    \textbf{Проверил:} \\
    ассистент кафедры ИТАС \\
    Прищепчик М. В.

  \end{flushleft}

  \begin{flushright}
    \textbf{Выполнили:} ст. группы 120602 \\
    Анашкевич П. С. \\
    Будный Р. И. 
  \end{flushright}
\end{multicols}

\subsection{Цель работы}
\begin{enumerate}

\item Изучение возможностей языка C++ в определении пользовательских типов данных.
\item Изучение правил определения и переопределения функций доступа к объектам класса.
\item Использование статических элементов классов и дружественных функций.

\end{enumerate}

\subsection{Задание}

Определите узел бинарного дерева следующим образом:

\begin{lstlisting}
  class Node
  {
    char name[10];	
    Node * left;	
    Node * right;	
    ...
  };
\end{lstlisting}

Определите в классе следующие функции:

\small{\ttfamily{Init()}} -- инициализация узла. Функция должна установить указатели на левый и правый узел в ноль;

\small{\ttfamily{AddNode()}} -- добавление узла в левую или правую ветви. Если слева нет узла, то добавить слева, если справа нет узла, то добавить справа, иначе ничего не добавлять;

\small{\sffamily{DelTree()}} -- удаление поддеревьев;

\small{\sffamily{Print()}} -- рекурсивная функция вывода дерева на экран.

\subsection{Листинг программы}

\lstinputlisting[caption=main.cpp]{code/main.cpp}
\lstinputlisting[caption=tree.h]{code/tree.h}
\lstinputlisting[caption=tree.cpp]{code/tree.cpp}

\subsection{Выводы}

В ходе проведения лабораторной работы были изучены возможности языка C++ в определении пользовательских типов данных.

\end{document}