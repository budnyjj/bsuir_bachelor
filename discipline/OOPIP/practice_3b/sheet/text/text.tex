\section*{КОНТРОЛЬНЫЕ ВОПРОСЫ}

\begin{enumerate}[label=\arabic*)]

\item 
  Какие диаграммы использованы в данном ПЗ? Объяснить их назначение.
  
  \textbf{Ответ:} 
  В данном ПЗ было использовано два вида диаграмм: диаграмма классов
  и диаграмма пакетов.
  
  Диаграмма классов описывает типы объектов системы и различного
  рода статические отношения, которые существуют между ними. На
  диаграммах классов отображаются также свойства классов, операции
  классов и ограничения, которые накладываются на связи между объектами.

  Диаграмма пакетов показывает пакеты и зависимости между ними.

  Пакет --– это инструмент группирования, который позволяет
  взять любую конструкцию UML и объединить ее элементы в единицы
  высокого уровня~\cite{fowler04}.

\item
  Почему разработчики ПО (АСУ) игнорируют использование UML и чем это чревато?

  \textbf{Ответ:} 
  Дело в том, что некоторые разработчики пытаются отобразить все возможные детали 
  работы проектируемой системы на своих диаграммах, поэтому они 
  получаются чересчур сложными. Такая избыточная сложность отпугивает разработчиков.

  UML при правильном использовании позволяет наглядно представить ключевые моменты
  работы системы, тем самым снижая сложность проектирования и упрощая поиск
  узких мест в работе системы.

  Под правильным использованием я понимаю изображение на диаграммах только 
  ключевых элементов системы в рассматрваемом контексте.

\item
  За что отвечает модель в шаблоне MVC?

  \textbf{Ответ:} 
  Модель (model) отвечает за хранение данных о предметной области.
  Она изменяем свое состояние в соответствии с запросами контроллера (controller),
  а также предоставляет интрефес доступа к данным для вида
  (view)~\cite{wiki_mvc}.

\item
  Какие существуют рекомендации по количеству зависимостей между модулями?
  
  \textbf{Ответ:}
  Чем меньше зависимостей, тем лучше.
  Следует избегать циклических зависимостей.

\item Объяснить сущность понятия множественности на основе примера из ПЗ.

  \textbf{Ответ:} 
  Множественность показывает нижнюю и верхнюю границы количества объектов,
  которые могут участвовать в связи~\cite{itteach_class_diagrams}.

  Пример: одно окно программы (объект класса MainWindow) может содержать
  ровно одну статусную строку (StatusBar).
  Экземпляр класса StatsReportViewStack может содержать один или несколько
  экземпляров класса, унаследованных от StatsReportView.

\end{enumerate}