%!TEX root = ../lab.tex

\section{ХОД РАБОТЫ}

\subsection{Текст задания}



\subsection{Особенности разработанной программы}

Для написания данной программы была использована представленная на рисунке~\ref{lst:struct} структура данных, которая включает в себя действительную и мнимую части комплексного числа. 

\begin{lstlisting}[caption=Структура данных Complex,label=lst:struct]
  struct Complex {
  double re;
  double im;
  };
\end{lstlisting}

Функции по работе с комплексными числами, продемонстрированные на рисунке~\ref{lst:functions}, принимают на вход два комплексных числа. Результатом является новое комплексное число.

\begin{lstlisting}[caption=Функции работы с комплексными числами,label=lst:functions]
  struct Complex add(struct Complex c1, struct Complex c2);
  struct Complex sub(struct Complex c1, struct Complex c2);
  struct Complex mul(struct Complex c1, struct Complex c2);
  struct Complex div(struct Complex c1, struct Complex c2);
\end{lstlisting}

Изначально пользователю предлагается ввести два комплексных числа, затем выбрать действие между ними. В случае неккорректного ввода информации, пользователь получит сообщение об ошибке. Если введенные данные верны, на экран будет выведет результат вычислений, записанный в алгебраической и показательных формах.

Разработанная программа предоставляет пользователю удобный и простой в использовании интерфейс для работы с комплексными числами, защищённый от попыток некорректного ввода информации.

Исходный текст программы расположен в приложении А.

\newpage