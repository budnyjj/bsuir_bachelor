%!TEX root = ../lab.tex

\section{ХОД РАБОТЫ}

\textbf{Текст задания:} заданы три числа D, M, Y, которые обозначают число, месяц и год. Найти номер N этого дня с начала года (високосные года --- это те, у которых номер делится на 400, и те, у которых номер делится на четыре, но не делится на 100).

На рисунке 1 представлены переменные, участвующие во взаимодействии с пользователем, а также массив из 12 чисел, соответствующих количеству дней в соответствующем месяце.

\begin{lstlisting}[caption=Объявление переменных и объявление и инициализация массива]
  int d, m, y;
  int result;
  int days[12] = {31, 29, 31, 30, 31, 30, 31, 31, 30, 31, 30, 31};  
\end{lstlisting}

Часть исходного кода, обеспечивающего вычисление количества дней, прошедших с начала года, изображена на рисунке 2.

\begin{lstlisting}[caption=Вычисление количества дней]
  int i = 0;
  while (i < m-1) {
    result += days[i];
    i++;
  }
  result += d;
\end{lstlisting}

Разработанная программа предоставляет удобный в использовании интерфейс, защищённый от попыток переполнения и некорректного ввода информации.

Полный исходный текст программы расположен в приложении А.

\newpage