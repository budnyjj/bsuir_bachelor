\section{ТЕОРЕТИЧЕСКИЕ СВЕДЕНИЯ}

Для работы с локальными журанлами событий предназначен командлет \textit{Get-EventLog},
представленный на рисунке~\ref{lst:get-eventlog_description}. 

\begin{lstlisting}[caption=Синтаксис использования командлета Get-Eventlog, label=lst:get-eventlog_description]
 Get-Eventlog [-list] [-asString] [<CommonParameters>]
  
 Get-Eventlog [-logName] <string> [<CommonParameters>]
\end{lstlisting}

При вызове командлета с параметром \textit{-list} выводится список всех журналов событий, 
существующих в системе. Кроме названия журнала, также выводятся такие параметры журналов,
как максимальный размер, количество записей, действие, которое будет выполнять система
при переполнении журнала.

При вызове командлета с параметром \textit{-logName} <название журнала> происходит вывод списка 
системных событий, записанных в указанном журнале. Каждому событию присвоено имя,
уникальный номер, тип, время регистрации и название источника возникновения. 

Для обработки списка событий средствами PowerShell могут использоваться командлеты 
\textit{Where-Object} (для отбора по значению параметров), 
\textit{Select-Object} (для отбора определенного числа событий)
и \textit{Foreach-Object} (для итерации по событиям).

\newpage
