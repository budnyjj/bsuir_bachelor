\section{ТЕОРЕТИЧЕСКИЕ СВЕДЕНИЯ}

WMI --- это одна из базовых технологий для централизованного управления и
слежения за работой различных частей компьютерной инфраструктуры под
управлением платформы Windows.


Поскольку WMI построена по объектно-ориентированному принципу, то все данные
об операционной системе, ее свойствах, управляемых приложениях и
обнаруженном оборудовании представлены в виде объектов. Каждый тип объекта
описан классом, в состав которого входят свойства и методы.
Определения классов описаны в MOF-файлах, а объекты этих классов с заполненными
свойствами и доступными методами при их вызове возвращаются WMI-провайдерами.
Управляет созданием и удалением объектов, а также вызовом их методов
служба CIM Object Manager.

Например, для доступа к WMI объекту в PowerShell достаточно использовать команду:
\textit{Get-WmiObject}.

Пример команды PowerShell получения списка запущенных процессов
представлен на рисунке~\ref{lst:get_processes}.

\begin{lstlisting}[caption=Команда получения списка запущенных процессов,
label=lst:get_processes]
 Get-WmiObject -Class win32_process
\end{lstlisting}

Для получения WMI объекта в VBScript необходимо проделать более сложные операции,
представленные на рисунке~\ref{lst:get_wmi_vbscript}.

\begin{lstlisting}[caption=Команда получения объекта WMI в VBScript,
label=lst:get_wmi_vbscript,language=VBScript]
 Set objWMIService = GetObject("winmgmts:\\" & strComputer & "\root\cimv2")
 Set cProducts = objWMIService.ExecQuery(<<query>>)
\end{lstlisting}

\newpage
