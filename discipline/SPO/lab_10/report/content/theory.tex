\section{ТЕОРЕТИЧЕСКИЕ СВЕДЕНИЯ}

Пример команды PowerShell получения списка запущенных процессов
представлен на рисунке~\ref{lst:get_processes}.

\begin{lstlisting}[caption=Команда получения списка запущенных процессов, label=lst:get_processes]
  Get-WmiObject -Class win32_process
\end{lstlisting}

Стандартный блок кода для добавления справочной информации о программе (help) продемонстрирован
на рисунке~\ref{lst:help_structure}.

\begin{lstlisting}[caption=Стандартная форма добавления справки к программе, label=lst:help_structure]
  <#
  .SYNOPSIS
      <short description>
  .DESCRIPTION
      <full description>
  .PARAMETER
      <parameters>
  .EXAMPLE
      <run example>
  .NOTES
      <additional notes>
  .LINK
      <additional links>
  #>
\end{lstlisting}

После добавления справочной информации о программе,
вызвать данную справку можно будет командой,
изображенной на рисунке~\ref{lst:help_call}.
\begin{lstlisting}[caption=Пример вызова справки к программе, label=lst:help_call]
  Get-Help \.ProgramName
\end{lstlisting}

\newpage
