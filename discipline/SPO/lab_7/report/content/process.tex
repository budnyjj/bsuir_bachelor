\section{ХОД РАБОТЫ}

\subsection{Текст задания}

Данная лабораторная работа выполняется на основе предыдущих лабораторных работ №4-6.

На рисунке~\ref{lst:task_struct} представлена структура, содержащая массив элементов типа Complex.

\begin{lstlisting}[caption=Структура\text{,} содержащая массив элементов типа Complex,label=lst:task_struct]
#define COMPLEXES_SIZE 5

struct Complexes {
  int count;
  int complexes_size;
  struct Complex complexes[COMPLEXES_SIZE];
};
\end{lstlisting}

Главный поток программы (функция main()) создаёт вторичный поток, передав в него указатель на структуру типа Complex. Вторичный поток запоминает значение из поля count, открывает файл, и затем в цикле, если значение count изменилось, записывает элементы массива complexes в файл. После этого поток закрывает файл и завершается.

Главный поток организует цикл ввода комплексных чисел следующим образом:

\begin{itemize}
  \item инициализируется временная переменная типа Complex (ввод числа с клавиатуры);
  \item c помощью функции SuspendThread() приостанавливается поток;
  \item значение переменной заносится в массив типа Complex;
  \item с помощью функции ResumeThread() поток запускается на выполнение.
\end{itemize}

\newpage

\subsection{Исходные данные}


На рисунке~\ref{lst:complex_h_code} представлено содержимое наиболее важных заголовочных файлов, реализованных в предыдущих лабораторных работах.

\begin{lstlisting}[caption=Исходный код заголовочного файла complex.h,label=lst:complex_h_code]
struct Complex {
    double re;
    double im;
};

struct Complex add(struct Complex c1, struct Complex c2);
struct Complex sub(struct Complex c1, struct Complex c2);
struct Complex mul(struct Complex c1, struct Complex c2);
struct Complex division(struct Complex c1,
                        struct Complex c2,
                        int * error);

void writeComplex(char *fname,
                  struct Complex *complex,
                  int count);

int readComplex(char *fname,
                struct Complex *complex,
                int count);

struct Complex input_complex();
void print_algebraic_form(struct Complex complex);
void print_polar_form(struct Complex c);
void print_complex(struct Complex complex);
void print_complex_array(struct Complex *complex, int count);
\end{lstlisting}

Функции, объявленные в на рисунке~\ref{lst:complex_h_code},
были использованы при выполнении лабораторной работы №7.

\subsection{Особенности разработанной программы}

В ходе проведения данной лабораторной работы была разработана программа для записи комплексных чисел в файл во вторичном потоке.

На рисунке~\ref{fig:uml} проиллюстрировано взаимодействие файлов проекта с помощью диаграммы пакетов UML.

\begin{figure}[htbp]
  \centering
  \includegraphics[width=150mm,height=80mm]{img/uml}
  \caption{UML диаграмма пакетов лабораторной работы №7}\label{fig:uml}
\end{figure}

После запуска программы на выполнение главный поток (функция создаёт вторичный поток, передав в него указатель на функцию записи комплексных чисел в файл. В главном потоке пользователь вводит комплексные числа.

Далее организуется следующий цикл: вторичный поток приостанавливается. Пользователь вводит несколько файлов, затем вторичный поток запускается и происходит запись в файл.

Исходный текст разработанной программы расположен в приложении~А.

\newpage
