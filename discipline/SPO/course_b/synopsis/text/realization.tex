\section[Описание деталей реализации]{ОПИСАНИЕ ДЕТАЛЕЙ РЕАЛИЗАЦИИ}
\label{sec:realization}

\subsection{Схема работы программы}

Разрабатываемый архиватор должен быть способен выполнять как минимум две
операции: кодирование (сжатие, упаковку) и декодирование (расжатие, распаковку)
данных.
Так как архиватор имеет консольный интерфейс, то он также должен быть способен
выполнять разбор аргументов командной строки для определения желаемой 
пользователем операции.
На рисунке~\ref{lst:hf} приведен исходный код модуля, содержащего функцию 
\texttt{main} --- стартовую функцию работы программы.

\begin{lstlisting}[basicstyle=\scriptsize\ttfamily,
                   numberstyle=\scriptsize\ttfamily,
                   xleftmargin=7mm,
                   language=C,caption=Исходный код функции main,
                   label=lst:hf]
#include <stdio.h>
#include <stdlib.h>
#include <io_options.h>
#include <compress.h>
#include <decompress.h>

int
main (int argc, char *argv[])
{
  struct io_options options;
  if (!get_options(argc, argv, &options))
    exit(1);

  switch (options.command)
    {
    case CREATE:
      {
        if (!compress(options.src_filename, options.dest_filename, options.verbose))
          {
            fprintf(stderr, "Some errors happened. Aborting.\n");
            exit(1);
          }
	break;
      }
    case EXTRACT:
      {
        if (!decompress(options.src_filename, options.dest_filename, options.verbose))
          {
            fprintf(stderr, "Some errors happened. Aborting.\n");
            exit(1);
          }
        break;
      }
    default:
      puts("This functionality was not implemented yet!");
      break;
    }
  exit (0);
}
\end{lstlisting}

Из исходного текста нетрудно проследить логику работы программы:
в 10-й строке производится объявление структуры \texttt{io\_options},
в которую помещаются разобранные аргументы командной строки
(функция \texttt{get\_options}, строка 11),
далее на основании поля \texttt{command} данной структуры производится
создание архива (функция \texttt{compress}, строка 18)
либо распаковка (функция \texttt{decompress}, строка 27).
Каждая из перечисленных функций возвращает код, 
который позволяет судить об успешности выполнения функции.
В случае, если он равен нулю, выводится сообщения об ошибке и 
производится аварийный выход из программы.

Отметим также тот факт, что на рисунке~\ref{lst:hf} были умышленно опущены
комментарии документации исходного кода, пример которых показан на 
рисунке~\ref{lst:comments}.

\begin{lstlisting}[basicstyle=\scriptsize\ttfamily,
                   numberstyle=\scriptsize\ttfamily,
                   xleftmargin=7mm,
                   language=C,caption=Пример документации исходного кода,
                   label=lst:comments]
/**
   @file hf.c
   @brief Main function
   @author Roman Budny
*/

#include <stdio.h>
...

/**
   @brief Main function
   @param[in] argc Number of console arguments
   @param[in] argv Array of  console arguments
   @return 0 if success, 1 if fail
 */
int
main (int argc, char *argv[])
{
...
\end{lstlisting}

Данные комментарии используются с целью снабжения системы 
автоматического документирвания doxygen информацией об исходном коде:
авторах исходного текста, описаниях предназначений модулей и функций,
входных и выходных параметров и т.~п. 

Несмотря на то, что данные комментарии имеют немалую ценность для программиста,
непосредственно читающего исходный текст, в дальнейшие примеры исходного кода
программы они включаться не будут.

Полная версия исходного текста рассматриваемых частей программы
расположена в приложении~А. 

\subsection{Разбор аргументов командной строки}

Разбор аргументов командной строки производится функцией \texttt{get\_options},
описанной в модуле \texttt{io\_options.c}. Результаты работы этой функции помещаются в 
структуру \texttt{io\_options}, описанную в заголовочном файле \texttt{io\_options.h}.
Данная структура приведена на рисунке~\ref{lst:io_options}.

\begin{lstlisting}[basicstyle=\scriptsize\ttfamily,
                   numberstyle=\scriptsize\ttfamily,
                   xleftmargin=7mm,
                   language=C,caption=Структура io\_options,
                   label=lst:io_options]
struct io_options {
  command_t command;
  char * src_filename;
  char * dest_filename;
  verbosity_t verbose;
};
\end{lstlisting}

На рисунке~\ref{lst:get_options} приведен исходный код функции \texttt{get\_options}.

\begin{lstlisting}[basicstyle=\scriptsize\ttfamily,
                   numberstyle=\scriptsize\ttfamily,
                   xleftmargin=7mm,
                   language=C,caption=Функция get\_options,
                   label=lst:get_options]
int
get_options(int argc, char ** argv,
               struct io_options *const dest_opts)
{
  CHKPTR(argv);
  CHKPTR(dest_opts);

  init_options(dest_opts);
  cli_get_options(argc, argv, dest_opts);
  return check_options(dest_opts);
}
\end{lstlisting}

Не вдаваясь в детали, можно отметить, что данная функция производит проверку
передаваемых указателей на равенство нулю (макросы \texttt{CHKPTR}),
инициализацию структуры типа \texttt{io\_options} безопасными значениями
(функция \texttt{init\_options}),
собственно получение аргументов из командной строки
(функция \texttt{cli\_get\_options}) и
базовую проверку корректности аргументов
(функция \texttt{check\_options}).

Собственно процесс разбора аргументов командной строки производится
с использованием системной библиотечной функции \texttt{getopt\_long}, 
описанной в стандартном заголовочном файле 
\texttt{getopt.h}~\cite{man_getopt_long}.

\subsection{Реализация алгоритма сжатия данных}

Рассмотрим реализацию алгоритма кодирования данных Хаффмана,
выраженную в виде функции \texttt{compress},
приведенной на рисунке~\ref{lst:compress}.

\begin{lstlisting}[basicstyle=\scriptsize\ttfamily,
                   numberstyle=\scriptsize\ttfamily,
                   xleftmargin=7mm,
                   language=C,caption=Функция сжатия данных compress,
                   label=lst:compress]
int
compress(const char *const src_fname,
         const char *const dest_fname,
         verbosity_t verbose)
{
  ppl_t char_ppl[MAX_PPL_SIZE] = {0};
  struct node_t * ppl_tree = NULL; 
  struct hf_code char_code[MAX_CODE_TBL_SIZE] = {{0, CODE_NOT_EXISTS}};
  struct header_t archive_info = {0}; /* contains info about archive */

  FILE* src_file = NULL;

  CHKPTR(src_fname);
  CHKPTR(dest_fname);
  
  src_file = fopen(src_fname, "r");
  archive_info.num_char = calculate_ppl(src_file, char_ppl, verbose);
  fclose(src_file);
  
  if (!archive_info.num_char)
    {
      fprintf(stderr, "Can't read source!\n");
      return 0;
    }

  ppl_tree = build_tree(char_ppl);

  if (!ppl_tree)
    {
      fprintf(stderr, "Error of memory allocation!\n");
      return 0;
    }

  tree_export_code(ppl_tree, char_code, verbose);
  clear_tree(ppl_tree);

  return write_archive(src_fname, dest_fname,
                       char_ppl, char_code,
                       archive_info, verbose);
}
\end{lstlisting}

Разберем поэтапно работу этой функции. 
Данная функция принимает следующие аргументы:
\begin{itemize}
\item строку \texttt{src\_fname}, содержащую название исходного файла;
\item строку \texttt{dst\_fname}, содержащую название выходного архива;
\item аргумент \texttt{verbose}, указывающий на степень подробности вывода
  информации о статусе работы функции.
\end{itemize}

Первый этап работы алгоритма --- подсчет числа повторов одинаковых символов 
в исходном файле. Для этого в шестой строке рассматриваемой функции производится
инициализация массива \texttt{char\_ppl} нулями и последующее занесение 
в данный массив числа повторов одинаковых символов
(вызов функции \texttt{calculate\_ppl}, строка 17).

Второй этап работы алгоритма --- построение дерева Хаффмана, структура которого
описана в подразделе~\ref{ssec:huffman_encode}.
Построение выполняется в функции \texttt{build\_tree}, строка 26.

Третий этап работы алгоритма --- извлечение кодов Хаффмана из дерева и
помещение их в массив. Данный массив можно представить в виде хранилища типа
<<ключ-значение>>, где ключом является код ASCII-символа, 
а значением --- соответствующий код переменной длины.
Данное извлечение кодов выполняется в функции \texttt{tree\_export\_code}, 
строка 34.

Последний этап работы алгоритма --- собственно кодирование --- выполняется
в функции \texttt{write\_archive}, расположенной на строке 37.
Возвращаемое значение описанных вспомогательных функций говорит об статусе
их выполнения.
В случае, если он неудовлетворительный, функция compress возвращает нулевое
значение, которое затем обрабатывается в главной функции.
На мой взгляд, подобный подход подобен механизму
обработки исключений в объектно-ориентированных языках программирования.

\subsection{Реализация алгоритма расжатия данных}

Рассмотрим реализацию алгоритма декодирования данных Хаффмана,
выраженную в виде функции \texttt{decompress},
приведенной на рисунке~\ref{lst:decompress}.

Разберем поэтапно работу этой функции. 
Данная функция принимает следующие аргументы:
\begin{itemize}
\item строку \texttt{src\_fname}, содержащую название исходного архива;
\item строку \texttt{dst\_fname}, содержащую название выходного файла;
\item аргумент \texttt{verbose}, указывающий на степень подробности вывода
  информации о статусе работы функции.
\end{itemize}

Первый этап работы алгоритма --- получение информации о содержимом архива.
Данная информация помещается в структуру типа \texttt{header\_t} и 
представляет собой набор следующих значений:
\begin{itemize}
  \item \texttt{num\_char} --- число закодированных символов;
  \item \texttt{num\_code} --- число используемых кодов Хаффмана;
  \item \texttt{buffer\_size} --- размер буффера, используемого при кодировании;
  \item \texttt{last\_offset} --- длина значимой части последнего байта архива.
\end{itemize}

\begin{lstlisting}[basicstyle=\scriptsize\ttfamily,
                   numberstyle=\scriptsize\ttfamily,
                   xleftmargin=7mm,
                   language=C,caption=Функция расжатия данных decompress,
                   label=lst:decompress]
int
decompress(const char *const src_fname,
         const char *const dest_fname,
         verbosity_t verbose)
{
  ppl_t char_ppl[MAX_PPL_SIZE] = {0};
  struct header_t archive_info = {0, 0, 0}; /* contains info about archive */
  long int data_offset = 0;
  FILE* src_file = NULL;

  CHKPTR(src_fname);
  CHKPTR(dest_fname);

  src_file = fopen(src_fname, "r");
  fread(&archive_info, sizeof(archive_info), 1, src_file);
  data_offset = read_ppl(src_file, char_ppl, archive_info.num_code, verbose);
  fclose(src_file);
  src_file = NULL;

  if (verbose == INFO || verbose == DEBUG)
    {
      printf("=== ARCHIVE INFO ===\n");
      printf("NUMBER OF ENCODED CHARACTERS: %ld\n", archive_info.num_char);
      printf("NUMBER OF CODES: %d\n", archive_info.num_code);
      printf("CODE BUFFER SIZE: %d\n", archive_info.buffer_size);
      printf("OFFSET IN LAST BYTE: %d\n", archive_info.last_offset);
    }

  if (archive_info.num_code == 1) /* src file contatins data of single character */
    {
      read_single_char(char_ppl, dest_fname, archive_info.num_char);
    }
  else
    {
      struct node_t * ppl_tree = NULL; 
      ppl_tree = build_tree(char_ppl);
      if (!ppl_tree)
        {
          fprintf(stderr, "Error of memory allocation!\n");
          return 0;
        }

      read_archive(src_fname, dest_fname, ppl_tree,
                   archive_info, data_offset, verbose);

      clear_tree(ppl_tree);
    }

  return 1;
}
\end{lstlisting}

Данные значения наряду с набором соответствий кодов символов числам повторений
этих символов в файле до сжатия извлекаются из исходного архива с
помощью функций \texttt{fread} и \texttt{read\_ppl}, расположенных
в строках 15 и 16 рассматриваемой функции.

В случае, если параметр \texttt{verbose} имеет значения
\texttt{INFO} или \texttt{DEBUG}, выполняется вывод информации 
о содержимом архива.

Второй этап работы алгоритма --- используя данные о содержимом архива,
с помощью функции \texttt{read\_single\_char} (строка 31) или
\texttt{build\_tree} (строка 36) выполняется построение дерева Хаффмана.

Третий этап работы алгоритма --- чтение архива и декодирование данных
с использованием построенного дерева выполняется в функции \texttt{read\_archive}
(строка 43). 

\subsection{Сборка проекта}

Как уже упоминалось в подразделе~\ref{ssec:choice_build_system},
в качестве системы сборки приложения из исходных кодов была выбрана утилита
GNU Make. Данная утилита использует специальный файл конфигурации Makefile,
в котором задается последовательность сборки проекта.
Данный файл представляет собой набор связанных целей сборки и 
использует декларативный синтаксис. 
Это означает, что в нем нет жесткого описания последовательности 
выполнений операций, приводящих к сборке проекта, 
вместо этого GNU Make сама определяет требуемую последовательность сборки, 
исходя из зависимостей целей.

Пример части Makefile, используемого для сборки проекта арзиватора,
приведен на рисунке~\ref{lst:part_makefile}

\begin{lstlisting}[basicstyle=\scriptsize\ttfamily,
                   numberstyle=\scriptsize\ttfamily,
                   xleftmargin=7mm,
                   language=make,caption=Пример Makefile сборки проекта,
                   label=lst:part_makefile]
clean:
	$(RM) -r $(OBJDIR)
	$(RM) $(shell find . -name '*~')
	$(RM) $(shell find . -name '.\#*')

distclean: clean
	$(RM) $(APP)

doc: 
	doxygen Doxyfile
\end{lstlisting}

Из этой части Makefile видно, что для выполнения цели \texttt{distclean}
необходимо, чтобы сначала была выполнена цель \texttt{clean} (строка 6).
Цель \texttt{clean} выполняет рекурсивное удаление всех файлов, 
которые начинаются на <<.\#>> или оканчиваются на <<\textasciitilde>> 
(строки 1--5). 
Кроме этого, нетрудно заметить, что для генерации документации используется 
цель \texttt{doc}, которая приводит к вызову программы doxygen (строки 9, 10).

Полный текст Makefile, используемого для сборки программного продукта, 
расположен в приложении~Б.
