\section{ХОД РАБОТЫ}

\subsection{Текст задания}

Заданы три числа D, M, Y, которые обозначают
число, месяц и год. Найти номер N этого дня с начала года (високосные
года --- это те, у которых номер делится на 400, и те, у которых номер
делится на четыре, но не делится на 100).

\subsection{Теоретические сведения}

В С принят способ передачи параметров, который называется передачей по
значению. Выглядит он так: 

\begin{itemize}
\item формальные параметры являются собственными
  переменными функции;
\item при вызове функции происходит присваивание
  значений фактических параметров формальным (копирование первых во
  вторые);
\item при изменении формальных параметров значения соответствующих
  им фактических параметров не меняются.
\end{itemize}

Единственным исключением из этого правила является передача имени
массива в качестве параметра.
В этом случае формальный параметр также
является собственной переменной, но не массивом, а указателем на
него. Поэтому размерность такого массива в функции несущественна и
может отсутствовать, а изменение элементов массива --- формального
параметра приводит к изменению значений массива --- фактического
параметра функции.

\subsection{Особенности разработанной программы}
Разработанная 
программа имеет следующие функциональные особенности:

\begin{itemize}
\item интерактивный консольный интерфейс, устойчивый к ошибкам переполнения;
\item осуществление проверки корректности вводимых данных с возможностью 
  замены некорректных данных;
\item разделение различных функциональных частей исходного кода программы 
  по различным файлам с исходным кодом;
\item реализована поддержка инкрементальной компиляции.
\end{itemize}

Для представления данных, введенных пользователем, была разработана 
соответствующая структура, представленная на рисунке~\ref{lst:struct}.

\begin{lstlisting}[caption=Структура Date,label=lst:struct]
struct Date {
  int d;
  int m;
  int y;
};
\end{lstlisting}

Для вычисления количества дней,
прошедших с начала года, была разработана функция, изображенная на рисунке 2. 

\begin{lstlisting}[caption={Функция вычисления результата}]
int get_day_number(struct Date * date) {
  int months[] = {-1, 31, 28, 31, 30, 31, 30, 31, 31, 30, 31, 30, 31}; /* months[0] is bogus */
  int num_days = 0;
  int i;

  for (i = 1; i < date->m; i++)
    num_days += months[i];

  if (is_leap_year(date->y) && (date->m > 2))
    num_days++;

  num_days += date->d;
  return num_days;
}
\end{lstlisting}

Полный исходный текст разработанной программы расположен в приложении А.
