\section{ТЕОРЕТИЧЕСКИЕ СВЕДЕНИЯ}

\subsection{Описание цепочечных команд}

Цепочечные команды позволяют производить действия над блоками памяти,
представляющими собой последовательности элементов следующего размера:

\begin{itemize}
\item 8 битов (byte);
\item 16 битов (word);
\item 32 бита (double word).
\end{itemize}

Цепочечные команды оперируют адресами,
формирующимися из значений пар DS:SI (адрес источника), и
ES:DI (адрес приемника). 
Таким образом, перед использованием этих команд необходимо проинициализировать
регистры ES, SI, DI.

Важно также отметить, что направление работы цепочечных команд определяется 
значеним флага DF: при DF=0 обработка ведется в сторону возрастания адресов,
при DF=1 --- в сторону убывания.

Рассмотрим операции-примитивы обработки цепочек и реализующие их команды
ассемблера.

\subsection{Пересылка цепочки}

Синтаксис команды:
\begin{itemize}
\item MOVS destination, source;
\item MOVSB;
\item MOVSW;
\item MOVSD.
\end{itemize}

Команда MOVS копирует байт, слово или двойное слово из цепочки, адресуемой операндом
source, в цепочку, адресуемую операндом destination. При трансляции в зависимости от
типа операндов транслятор преобразует ее в одну из трех машинных команд:
MOVSB, MOVSW, MOVSD.

\subsection{Сравнение цепочек}

Синтаксис команды:
\begin{itemize}
\item CMPS destination, source;
\item CMPSB;
\item CMPSW;
\item CMPSD.
\end{itemize}

Данная команда выполняет последовательное сравнение байтов, слов или двойных слов 
(в зависимости от команды или операндов) и выставляет флаги ZF, SF, OF,
подобно команде CMP.

Удобно использовать с префиксами повторения для поиска одинаковых или различающихся 
элементов цепочек без модификации этих цепочек.

\subsection{Сканирование цепочки}

Синтаксис команды:
\begin{itemize}
\item SCAS source;
\item SCASB;
\item SCASW;
\item SCASD.
\end{itemize}

Команда SCAS выполняет последовательное сравнение значения, находящегося
в регистрах AL, AX, EAX (в зависимости от размера операнда или типа команды), 
с элементом цепочки, адремуемым парой ES:DI.
Результат сравнения можно отследить по значениям флагов ZF, SF, OF.

Вместе с префиксами повторения удобно использовать для поиска вхождений
элементов в цепочке с заданным значением.

\subsection{Загрузка элемента из цепочки}

Синтаксис команды:
\begin{itemize}
\item LODS source;
\item LODSB;
\item LODSW;
\item LODSD.
\end{itemize}

Данная группа команд позволяет загрузить элемент цепочки, адресуемый парой DS:SI,
в регистры AL, AX, EAX (в зависимости от размера операнда или типа команды).

\newpage 

\subsection{Сохранение элемента в цепочке}

Синтаксис команды:
\begin{itemize}
\item STOS destination;
\item STOSB;
\item STOSW;
\item STOSD.
\end{itemize}

Данная группа команд позволяет загрузить элемент, находящийся в регистрах AL, AX, EAX
(в зависимости от размера операнда или типа команды) в место цепочки, адресуемой парой DS:SI.

\subsection{Получение элементов цепочки из порта ввода-вывода}

Синтаксис команды:
\begin{itemize}
\item INS destination, source\_port;
\item INSB;
\item INSW;
\item INSD.
\end{itemize}

Эта команда вводит элемент из порта, номер которого находится в регистре DX,
в элемент цепочки, адрес памяти которого определяется операндом destination.
Несмотря на то, что цепочка, в которую вводится элемент, адресуется указанием
этого операнда, ее адрес должен быть явно сформирован в паре регистров
ES:DI. Размер элементов цепочки должен быть согласован с размером порта ---
он определяется директивой резервирования памяти, с помощью которой выделяется
память для размещения элементов цепочки.

\subsection{Вывод элементов цепочки в порт ввода-вывода}

Синтаксис команды:
\begin{itemize}
\item OUTS destination\_port, source;
\item OUTSB;
\item OUTSW;
\item OUTSD.
\end{itemize}

Эта команда выводит элемент цепочки в порт, номер которого находится в регистре DX.
Адрес элемента цепочки определяется операндом source.

Несмотря на то, что цепочка, из которой выводится элемент, адресуется указанием
того операнда, значение адреса должно быть явно сформировано в паре регистров
DS:SI. Размер структурных элементов цепочки должен быть согласован
с размером порта --- он определяется директивой резервирования памяти, 
с помощью которой выделяется память для размещения элементов цепочки.

\subsection{Префиксы повторения}

Префиксы повторения предназначены для использования цепочечными командами.
Они имеют свои мнемонические обозначения:
\begin{itemize}
\item REP;
\item REPE или REPZ;
\item REPNE или REPNZ.
\end{itemize}

Префикс REP заставляет команду выполняться столько раз, сколько находится 
в регистре CX. После каждого повторения команды значение CX уменьшается на единицу, таким
образом, по окончании работы команды с данным префиксом значение регистра CX равно нулю.

Префиксы REPE и REPZ заставляют команду выполняться такое количество раз, сколько находится 
в регистре CX, либо пока флаг ZF равен единице. Таким образом, этот префикс удобен для поиска 
различных частей анализируемых цепочек.

Префиксы REPNE и REPNZ заставляют команду выполняться такое количество раз, сколько находится 
в регистре CX, либо пока флаг ZF равен нулю. Таким образом, этот префикс удобен для поиска 
одинаковых частей анализируемых цепочек.