\section{ХОД РАБОТЫ}

\subsection{Постановка задачи}

Необходимо написать программу, выполняющую наложение пары изображений друг на друга.
Процесс обработки пикселей изображения должен выполняться в функции,
написанной на языке Ассемблер с использованием MMX-команд.

\subsection{Особенности разработанной программы}

Основной модуль программы написан на языке программирования высокого уровня --- С++ с
использованием специализированной библиотеки обработки изображений Magick++. 
Данная библиотека позволяет считывать и записывать изображения практически 
во всех популярных форматах: BMP, GIF, PNG, JPG, TIFF.

Основной модуль программы выполняет разбор аргументов командной строки,
производит считываение изображений в оперативную память, после этого
предоставляет низкоуровневый доступ к пикселям изображения в виде 
массива байтов, как показано на рисунке~\ref{lst:low_access}.

\begin{lstlisting}[caption=Предоставление низкоуровневого доступа к изображению,
label=lst:low_access,language={C++},basicstyle=\scriptsize\ttfamily]

 // Write raw pixels of a
 unsigned char * raw_pixels_a = (unsigned char *) calloc(columns*rows, SIZE_RGBA);
   
 image_a.write(0, 0, columns, rows, "RGBA", CharPixel, raw_pixels_a);

 // Write raw pixels of b
 unsigned char * raw_pixels_b = (unsigned char *) calloc(columns*rows, SIZE_RGBA);

 image_b.write(0, 0, columns, rows, "RGBA", CharPixel, raw_pixels_b);
   
 // Allocate raw pixels of res
 unsigned char * raw_pixels_res = (unsigned char *) calloc(columns*rows, SIZE_RGBA);

 // call ASM
 merge_raw_pixels(raw_pixels_a, raw_pixels_b, raw_pixels_res, rows*columns);
 end_MMX();
   
 // Write chages to result image
 image_res.read(columns, rows, "RGBA", CharPixel, raw_pixels_res);
   
 // Write result image to file 
 image_res.write(path_dst);
\end{lstlisting}

Функция merge\_raw\_pixels, представленная на рисунке~\ref{lst:merge}
выполняет наложение изображений друг на друга.

\begin{lstlisting}[caption=Функция наложения изображений,
label=lst:merge,language={[x86masm]Assembler},basicstyle=\scriptsize\ttfamily]
 global  merge_raw_pixels:function
 global  end_MMX:function
     
 segment .text
 merge_raw_pixels:
     
 .process:    
     movq        mm0, [rdi]
     movq        mm1, [rsi]
     pmaxub      mm0, mm1
     movq        [rdx], mm0
 
     add         rdi, 8
     add         rsi, 8
     add         rdx, 8
     
     sub         rcx, 2
     jrcxz       .end
     jmp         .process
     
 .end:    
     ret
 
 end_MMX:
     emms
     ret
\end{lstlisting}

Данная фунцкия принимает пару указателей на массивы пикселей исходных изображений,
указатель на массив пикселей результирующего изображения, а также число пикселей
подлежащих обработке. Далее она в цикле выполняет следующие операции:
\begin{enumerate}
\item загрузка двух пикселей (восьми байт) первого изображения в регистр mm0;
\item загрузка двух пикселей (восьми байт) второго изображения в регистр mm1;
\item помещение максимума из сравниваемых пар пикселей в регистр mm0;
\item помещение содержимого регистра mm0 в память по адресу результирующего изображения.
\end{enumerate}

Данный цикл выполняется (rows*columns)/2 раз, 
так каз за одну итерацию производится обработка двух пикселей.

Исходный текст разработанной программы находится в приложении~А.
