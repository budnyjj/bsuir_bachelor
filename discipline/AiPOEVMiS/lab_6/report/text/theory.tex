\section{ТЕОРЕТИЧЕСКИЕ СВЕДЕНИЯ}

\subsection{Описание MMX-команд}

Команды технологии MMX работают с 64-разрядными целочисленными данными,
а также с данными, упакованными в группы (векторы) общей длиной 64 бита.
Такие данные могут находиться в памяти или в восьми MMX-регистрах. 
Эти регистры называются MM0, MM1, \dots , MM7.

Команды технологии MMX работают со следующими типами данных:
\begin{itemize}
\item упакованные байты (восемь байтов в одном регистре);
\item упакованные слова (четыре слова слова в регистре);
\item упакованные двойные слова (два двойных слова в регистре);
\item 64-разрядные слова.
\end{itemize}

MMX-команды исполняются в том же режиме процессора, 
что и команды с плавающей запятой.
Поэтому при исполнении всех MMX-команд (кроме EMMS) <<портится>>
слово состояния регистров с плавающей запятой.

Большинство команд имеют суффикс, который определяет тип данных 
и используемую арифметику:
\begin{itemize}
\item US --- арифметика с насыщением, 
  данные без знака;
\item S или SS --- арифметика с насыщением, данные со знаком.
  Если в суффиксе нет ни S, ни SS, используется циклическая арифметика;
\item B, W, D, Q указывают тип данных.
  Если в суффиксе есть две из этих букв, 
  первая соответствует входному операнду, а вторая --- выходному.
\end{itemize}

Перечислим MMX-команды
(обозначения: mm --- MMX-регистр;
m64 --- память объема 64 бит; 
imm --- непосредственный операнд), 
которые были использованы при выполнении лабораторной работы:
\begin{itemize}
\item EMMS --- 
  обеспечивает переход процессора от исполнения MMX-команд к
  исполнению обычных команд с плавающей запятой;

\item MOVQ mm, mm/m64 ---
  команда пересылки данных в MMX-регистр;

\item MOVQ mm/m64, mm ---
  команда пересылки данных из MMX-регистра;

\item MAXPS mm0, mm1 --
  попарно сравнивает элементы данных и записывает большее значение из
  каждой пары в соответствующий элемент выходного операнда.
\end{itemize}
