\section{ТЕОРЕТИЧЕСКИЕ СВЕДЕНИЯ}

Если часто пользоваться Интернетом и не придавать особого значения вопросам
безопасности при работе в Сети, весьма вероятно, что на жестком диске скопилось
множество нежелательных файлов и программ. Некоторые неопасны и просто занимают место,
но есть и вредные разновидности. Далее будут представлены некоторые из
них в порядке возрастания риска.

\textit{Копии посещаемых ресурсов.} В обозревателе сохраняются не только
адреса веб-страниц, которые посетил пользователь, но и их копии.
Такие файлы могут гостить на жестком диске от нескольких дней до нескольких недель
(в зависимости от настройки обозревателя) и в целом никому и ничему
не угрожают --- за исключением случаев, когда вы не хотите, чтобы другие
пользователи компьютера (например, члены семьи) знали, какие именно узлы вы посещаете.

\textit{«Cookies»} --- это небольшие информационные текстовые файлы,
сами по себе не представляющие особой опасности. Однако они могут
рассказать другим о ваших путешествиях по Интернету --- так же,
как «Журнал» и копии веб-страниц. Кроме того, с помощью этих файлов
определенные коммерческие узлы собирают информацию о привычках
и интересах пользователей буквально по крупицам.

Пользователям Интернета хорошо известны \textit{вирусы} и \textit{черви}.
Однако теперь они создаются не только для того, чтобы просто размножаться
и время от времени уничтожать данные. Многие из них стали разносчиками
шпионских программ, которые внедряются на ПК с целью передать управление
разработчику вируса. За исключением тех случаев, когда на компьютере
нет никакого антивирусного ПО или межсетевого экрана, вероятность заражения
вирусами в наши дни сравнительно невелика. Дело в том,
что специальные современные программные продукты весьма эффективно
от них защищают, а сами вирусы сравнительно просто устроены.

\textit{Рекламные программы (их принято называть adware)} применяются для показа
рекламных объявлений в то время, когда пользователь путешествует по Сети.
Некоторые из их числа безвредны и не скрываются: они предназначены для сбора
денег на бесплатные сервисы, о чем и сообщается в лицензионном соглашении,
которое все же следует читать перед установкой ПО. Другие программы действуют
тайком: самостоятельно устанавливаются на компьютер, используя для этого «дыры»
в веб-обозревателях. Опасность рекламных модулей не стоит недооценивать:
недавно проведенное исследование показало, что на 80\% персональных компьютеров,
подключенных к Интернету, живет по крайней мере один такой «паразит» --- работающий,
разумеется, без ведома владельца.
\textit{Шпионские программы (spyware)} очень опасны --- возможно,
не меньше, чем вирусы.
Программы-шпионы иногда подразделяют на два вида:
\begin{itemize}
\item программы-шпионы для получения конфиденциальной
информации (программы-шпионы в узком смысле слова, это программы-шпионы по цели использования);
\item программы, перехватывающие управление компьютером (программы-шпионы по способу проникновения).
\end{itemize}

Первый вид программ может осуществлять: несанкционированный перехват информации,
политический и экономический шпионаж,
несанкционированный доступ к счетам Интернет-банков, к данным кредитных карточек,
к криптографическим данным и т.д. В технологическом плане эти программы-шпионы
могут регистрировать нажатия клавиш, клики мыши, копировать экран и активные окна,
перехватывать звук с микрофона и видео-изображения и др. Программы-шпионы могут
встраиваться в коммерческие, бесплатные и условно-бесплатные программы,
троянские программы, вирусы и черви.
Программы второго вида осуществляют перехват управления компьютером в целях
рассылки спама или атаки веб-узлов (отказ в обслуживании). «Отказ в обслуживании» или
так называемая DoS-атака, выполняется удаленно с высокой степенью анонимности
и приводит либо к захвату ресурсов системы, либо к ее аварийной остановке.

Для борьбы с программами-шпионами используются программы антишпионы.
Эти программы относятся к средствам обеспечения безопасности и осуществляют
выявление и удаления активных компонент программ-шпионов.

Антишпионы защищают ваш компьютер не только от шпионских программ,
но и от прочего «не полезного» софта, например, назойливой рекламы,
непонятных браузерных дополнений, мошеннических программ, сайтов, распространяющих вирусы и прочего.

Далее будут представлены некоторые из антишпионских программ.

\textit{Spybot} --- программа, предназначенная для предотвращения и устранения
заражения компьютера пользователя шпионским программным обеспечением (spyware),
троянами, malware, scumware, вредоносными дополнениями к браузеру,
компонентами, отслеживающими предпочтения пользователя без его согласия.
Программа не является антивирусным программным обеспечением в привычном
понимании этого термина и представляет собой скорее дополнительный инструментарий
для повышения уровня безопасности работы в интернете.

Основные возможности:
\begin{itemize}
\item сканирование компьютера;
\item восстановление удалённых компонентов в случае возникновения проблем;
\item иммунизация системы путём внесения изменений в настройки браузеров
(поддерживаются Internet Explorer, Firefox,Flock, K-Meleon и Opera) и файл hosts;
\item автоматические обновления через Интернет;
\item надёжное удаление файлов;
\item резидентные компоненты: SDHelper --- для защиты от вредоносных
ActiveX-компонентов и TeaTimer --- для защиты критических настроек операционной системы;
\item просмотр информации об установленных в системе ActiveX-компонентах,
BHO, программах в автозагрузке, сокетах LSP с указанием степени их опасности;
\item обозреватель стартовых страниц браузеров и установленных в них поисковых машинах;
\item защита файла hosts и настроек Internet Explorer;
\item просмотр файла hosts;
\item функция отказа от рассылок;
\item перечень запущенных в системе процессов с возможностью их завершения;
\item исправление ошибок в реестре;
\item деинсталляция установленных программ.
\end{itemize}

\textit{Spy Sweeper} --- программа для поиска и уничтожения шпионского
программного обеспечения (spyware). Spy Sweeper способен успешно бороться также
и со всеми типами рекламного ПО, keylogger-ами, троянами и т.д.
Spy Sweeper спроектированный специально для защиты от данных типов угроз,
сочетает технологию безопасного удаления вредоносных программ
Comprehensive Removal Technology (CRT), с защитными механизмами Smart Shields.
Технология CRT позволяет достойно противостоять даже наиболее коварным и совершенным
вредоносным программам, включая Commonname, LOP, SurfSidekick,
VX2/Abetterinternet/Aurora, Elite Keylogger и другие.
Программа функционирует в фоновом режиме, защищая ПК от дальнейшего заражения,
<<безболезненно>> для операционной системы удаляет угрозы и вирусы,
а встроенный Smart Shield постоянно защищает веб-браузер и ОС во время работы в Сети.
\textit{Ad-Aware} --- программа, предназначенная для удаления шпионского
программного обеспечения (англ. spyware) с компьютера пользователя.
Также обнаруживает троянские программы, malware, scumware, вредоносные
дополнения к браузеру, руткиты, а также программы,
отслеживающие предпочтения пользователя без его согласия.

Программа предлагает широкий набор функций:
\begin{itemize}
\item полное сканирование компьютера, а также сканирование выбранных дисков и папок;
\item сканирование реестра, файлов в архивах, избранного Internet Explorer и файла hosts;
\item исключение из сканирования файлов больше заданного размера;
\item сканирование процессов в оперативной памяти компьютера;
\item автоматическое обновление базы вредоносных программ через Интернет;
\item поддержка скинов;
\item поддержка плагинов.
\end{itemize}
