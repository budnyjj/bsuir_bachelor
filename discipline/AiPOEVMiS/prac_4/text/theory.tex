\section{ТЕОРЕТИЧЕСКИЕ СВЕДЕНИЯ}

Интернет страдает от серьезных проблем с безопасностью. Организации,
которые игнорируют эти проблемы, подвергают себя значительному риску того,
что они будут атакованы злоумышленниками, и что они могут стать стартовой
площадкой при атаках на другие сети. Даже те организации,
которые заботятся о безопасности, имеют те же самые проблемы из-за появления
новых уязвимых мест в сетевом программном обеспечении (ПО) и отсутствия мер
защиты от некоторых злоумышленников.

Некоторые из проблем безопасности в Интернете --- результат наличия уязвимых
мест из-за ошибок при проектировании в службах (и в протоколах, их реализующих),
в то время как другие --- результат ошибок при конфигурировании хоста или
средств управления доступом, которые или плохо установлены или настолько сложны,
что с трудом поддаются администрированию. Кроме того: роль и важность администрирования
системы часто упускается при описании должностных обязанностей сотрудников,
что при приводит к тому, что большинство администраторов в лучшем случае
нанимаются на неполный рабочий день и плохо подготовлены.
Это усугубляется быстрым ростом Интернета и характера использования Интернета;
государственные и коммерческие организации теперь зависят от Интернета (иногда даже больше:
чем они думают) при взаимодействии с другими организациями и исследованиях
и поэтому понесут большие потери при атаках на их хосты.

Следующие главы описывают проблемы в Интернете и причины, приводящие к их возникновению.
NIST считает, что Интернет, хотя и является очень полезной и важной сетью,
в то же самое время очень уязвим к атакам. Сети, которые соединены с Интернетом,
подвергаются некоторому риску того, что их системы будут атакованы или
подвергнуты некоторому воздействию со стороны злоумышленников,
и что риск этого значителен. Следующие факторы могут повлиять на уровень риска:
\begin{itemize}
\item число систем в сети;
\item какие службы используются в сети;
\item каким образом сеть соединена с Интернетом;
\item профиль сети, или насколько известно о ее существовании;
\item насколько готова организация к улаживанию инцидентов с компьютерной безопасностью.
\end{itemize}

Высокая динамика развития информационных технологий, наблюдаемая в настоящее время,
не только дает новые преимущества для развития бизнеса,
но и влечет за собой многократный рост числа уязвимостей информационных систем.
Это, в свою очередь, обуславливает развитие существующих и появление новых методов
и средств их защиты. В результате возрастает потребность в квалифицированных
и компетентных специалистах по информационной безопасности.

Эффективность работы сотрудников, которые должны обеспечить безопасность информационных систем,
зависит от уровня их подготовки и наличия у них опыта работы. Лишь в последние годы вузы
в массовом порядке стали готовить специалистов в области защиты информации,
поэтому в организациях различного масштаба и сфер деятельности за информационную безопасность
отвечают специалисты, которые для этого были вынуждены перепрофилироваться.

Новые специальности, о которых идет речь, возникли на стыке двух направлений ---
информационных технологий и технологий безопасности. В связи с этим
информационной безопасностью занялись те, кто пришел из <<других видов>>
безопасности (выходцы из различных силовых структур) либо специалисты в области
программирования, вычислительной техники и связи. Третья, пока немногочисленная
категория, присутствующая на рынке труда, --- выпускники вузов не имеющие опыта работы,
но прошедшие базовую подготовку. В любом случае, всем трем указанным категориям
приходится осваивать <<недостающую часть>> специальности.

Компьютерная безопасность требует комплексного решения широкого спектра
правовых и организационно-технических вопросов, поэтому руководителям
и аналитикам подразделений обеспечения информационной безопасности в автоматизированных
системах необходимы знания и навыки по:


\begin{itemize}
\item теоретическим и правовым вопросам защиты информации в АС и безопасности информационных технологий;
\item задачам, функциям и основным направлениям деятельности подразделений обеспечения информационной безопасности в АС;
\item принципам построения комплексных систем защиты АС, технологиям информационной безопасности,
включая умение рационально распределить функции и организовать эффективное
взаимодействие по вопросам защиты информации сотрудников всех подразделений,
использующих АС и поддерживающих ее функционирование;
\item основным защитным механизмам, возможностям средств защиты информации от несанкционированного доступа,
средств криптографической защиты информации, межсетевых экранов, средств анализа защищенности,
систем обнаружения атак (вторжений);
\item порядку применения вышеуказанных средств для обнаружения и устранения уязвимостей
в информационных системах и для защиты компьютерных сетей.
\end{itemize}

Аналитикам в области компьютерной безопасности, кроме того, необходимы знания и навыки по:
\begin{itemize}
\item основам проведения информационных обследований, выявления значимых угроз,
анализа и оценки рисков, способам управления рисками,
применению различных видов защитных мер в подсистемах АС;
\item вопросам разработки нормативно-методических и организационно-распорядительных
документов, необходимых для реализации той или иной технологии;
\item подходам к оценке характеристик и выбору необходимых программно-аппаратных
средств защиты ресурсов компьютерных сетей, основам поиска и использования
оперативной информации о новых средствах защиты и другой
актуальной информации по безопасности АС;
\item контролю и анализу эффективности применения и достаточности конкретных мер и средств защиты.
\end{itemize}

Администраторам средств защиты необходимы знания и навыки по:
\begin{itemize}
\item проблемам информационной безопасности в сетях Интернет/интранет;
\item уязвимостям распространенных ОС;
\item СУБД и приложений;
\item сетевых протоколов и служб;
\item атакам в IP-сетях;
\item типовым приемам проникновения в корпоративные сети;
\item реализуемым за счет использования определенных уязвимостей;
\item а также инструментальным средствам взлома систем защиты, популярным в среде нарушителей;
\item основам поиска и использования оперативной информации о новых уязвимостях в системном и прикладном программном обеспечении;
\item эффективному применению конкретных средств защиты информации от несанкционированного доступа, средств криптографической защиты информации, межсетевых экранов, средств анализа защищенности и средств обнаружения атак.
\end{itemize}
