\section{ТЕОРЕТИЧЕСКИЕ СВЕДЕНИЯ}

Наиболее значимые протоколы стека TCP/IP:
\begin{itemize}
\item описание стека протоколов TCP/IP (RFC 1112, RFC 1123, RFC 1823);
\item протокол PPP (RFC-1661,  RFC-1548);
\item протокол Ipv4 (RFC 791);
\item организация подсетей  (RFC 950);
\item протокол IPv6 (RFC-1752, RFC-1826, RFC-1827, RFC-1883, RFC-1885, RFC-1887, RFC-1924, RFC-1971, RFC-2073,RFC 3919);
\item протокол ARP (RFC 826);
\item протокол UDPRFC 768);
\item протокол TCP (RFC 793);
\item DNS (RFC 1034, 1035, 1348, 1535-1537, 1876, 1982, 1995, 1996, 2065, 2136, 2137, 2181, 2308, 2533, 2845, 3425, 3658);
\item ТELNET (RFC-854, RFC-855);
\item FTP (RFC 114, RFC 765, RFC 959);
\item FTP (RFC 1350);
\item SMTP (RFC 821, RFC 822 RFC 974, RFC 1870, RFC 2821);
\item MIME (RFC 2045 - RFC 2049);
\item POP 3 (RFC 1939);
\item IMAP 4   (rfc 2060);
\item HTTP/1.1 (RFC 2068).
\end{itemize}

Рабочее предложение (англ. RequestforComments, RFC) --- документ из серии пронумерованных
информационных документов Интернета, содержащих технические спецификации и стандарты,
широко применяемые во всемирной сети. Название «RequestforComments» ещё можно перевести
как «заявка (запрос) на отзывы» или«тема для обсуждения». В настоящее время первичной
публикацией документов RFC занимается IETF под эгидой открытой организации Общество
Интернета(англ. InternetSociety, ISOC). Правами на RFC обладает именно Общество Интернета.
Несмотря на название, запросы на отзывы RFC сейчас рассматриваются как стандарты
Интернета (а рабочие версии стандартов обычно называют драфтами,
отангл. draft здесь --- проект). Согласно RFC 2026, жизненный цикл
стандарта выглядит следующим образом:

Выносится на всеобщее рассмотрение интернет-проект (InternetDraft).
Проекты не имеют официального статуса и удаляются из базы через шесть
месяцев после последнего изменения.
Если проект стандарта оказывается достаточно удачным и непротиворечивым,
он получает статус предложенного стандарта (ProposedStandard), и свой номер RFC.
Наличие программной реализации стандарта желательно, но не обязательно.
Следующая стадия --- проект стандарта (DraftStandard) --- означает, что предложенный
стандарт принят сообществом, в частности, существуют две независимые по коду
совместимые реализации разных команд разработчиков. В проекты стандартов
ещё могут вноситься мелкие правки, но они считаются достаточно стабильными и рекомендуются для реализации.
Высший уровень --- стандарт Интернета (InternetStandard). Это спецификации
с большим успешным опытом применения и зрелой формулировкой. Параллельно с нумерацией RFC
они имеют свою собственную нумерацию STD. Список стандартов имеется в
документе STD 1 (сейчас это RFC 5000, но нумерация может измениться).

Из более чем трёх тысяч RFC этого уровня достигли только несколько десятков.
Многие старые RFC замещены более новыми версиями под новыми
номерами или вышли из употребления. Такие документы получают статус исторических(Historic)
Практически все стандарты Глобальной сети существуют в виде опубликованных заявок RFC.
Но в виде документов RFC выходят не только стандарты, но также концепции,
введения в новые направления в исследованиях, исторические справки, результаты экспериментов,
руководства по внедрению технологий, предложения и рекомендации по развитию
существующих Стандартов и другие новые идеи в информационных технологиях:
Экспериментальные (Experimental) спецификации содержат информацию об
экспериментальных исследованиях, интересных для интернет-сообщества.
Это могут быть, например, прототипы, реализующие новые концепции.
Информационные (Informational) RFC предназначены для ознакомления общественности,
не являются стандартами и не являются результатом консенсуса или рекомендациями.
Некоторые проекты, не получившие статуса Предложенного стандарта, но представляющие интерес,
могут быть опубликованы как Информационные RFC.
Лучший современный опыт (BestCurrentPractice). Эта серия RFC содержит рекомендации
по реализации стандартов, в том числе от сторонних организаций,
а также внутренние документы о структуре и процедурах стандартизации.
Почти все стандарты разрабатываются под эгидой каких-либо научных или
интернет-организаций (например W3C, IETF, консорциум Юникода, Интернет2).
Запросы на отзывы официально существуют только на английском языке.
Строгих требований к оформлению нет. Встречаются RFC, написанные
в строгом академическом стиле, иные --- в дружеской неформальной манере.
Существует традиция выпуска первоапрельских шуточных RFC, например, RFC 1149
рассказывает о передаче пакетов IP с помощью почтовых голубей.

