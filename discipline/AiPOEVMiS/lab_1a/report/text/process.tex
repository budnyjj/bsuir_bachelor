\section{ХОД РАБОТЫ}

\subsection{Текст задания}

Разработать программу, которая будет осуществлять диалог с пользователем в форме
<<вопрос-ответ>>. Необходимо использоватать конструкции ветвления, обработать
некорректный ввод символов с использованием командной строки.

\subsection{Разработка программы}

Каждая программа на ассемблере включает в себя три сегмента: сегмент кода,
данных и стека. Пример описания этих сегментов представлен
на рисунке~\ref{lst:segments}.
\begin{lstlisting}[caption={Пример объявления сегмента кода, данных и стека в
программе на языке Ассемблер},label=lst:segments]
 ...
 .STACK
 ...
 .DATA
 ...
 .CODE
 ...
\end{lstlisting}

В сегменте данных разместим объявление сообщений, которые будут учавствовать в
диалоге с пользователем. Пример объявления сообщений приведен на
рисунке~\ref{lst:messages}.
\begin{lstlisting}[caption={Пример объявления сообщений, используемых
 для диалога с пользователем},label=lst:messages]
 ...
 .DATA
 ...
 BreakfastMsg    db CR, LF, 'The breakfast is ready!', CR, LF, EOS
 ...
\end{lstlisting}

Сегмент стека в нашем случае останется пустым. Сегмент сода заполним командами
Ассемблера, используемыми для диалога с пользователем. Пример использования
команд Ассемблера приведен на рисунке~\ref{lst:code}.
\begin{lstlisting}[caption={Пример команд Ассемблера, используемых
 для диалога с пользователем},label=lst:code]
 ...
 .CODE
 ...
 mov dx, offset BreakfastMsg
 call WriteMessagep
 call ExitProgramp
 ...
 WriteMessagep proc
     mov ah, 09h
     int 21h
     ret
 WriteMessagep endp

 ExitProgramp proc
     mov ah, 4Ch
     int 21h
     ret
 ExitProgramp endp
\end{lstlisting}

Исходный текст разработанной программы находится в приложении~А.
