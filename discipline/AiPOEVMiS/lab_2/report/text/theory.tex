\section{ТЕОРЕТИЧЕСКИЕ СВЕДЕНИЯ}

\subsection{Понятие и виды BCD-чисел}

BCD-числа --- специальный вид представления числовой информации, в основу которого
положен принцип кодирования каждой десятичной цифры числа группой из четырех битов.
При этом каждый байт числа содержит одну или две десятичные цифры в так называемом
\textit{двоично-десятичном} виде (Binary-Coded Decimal, BCD).

Для описания двоично-десятичных чисел в программе могут быть использованы только
две директивы описания и инициализации данных --- \textit{db} и \textit{dt}.
Это обусловлено тем, что к таким числам также применяется принцип
\textit{<<младший байт по младшему адресу>>}, что очень удобно для их обработки.

Каждое BCD-число может быть представлено в упакованном или неупакованном формате.
Формат BCD-чисел определяет специфику выполнения арифметических операций над ними,
которые будут подробнее рассмотрены в подразделах~\ref{sub:zoned}~и~\ref{sub:packed}.


\subsection{Неупакованные BCD-числа}
\label{sub:zoned}

В неупакованном формате BCD-числа каждый байт содержит одну десятичную цифру в
четырех младших битах. Старшие четыре бита имеют нулевое значение. Это так называемая
\textit{зона}. Следовательно, диапазон представления десятичного неупакованного
числа в одном байте составляет от \textit{нуля} до \textit{девяти}. Пример объявления
неупакованного BCD-числа представлен на рисунке~\ref{lst:zoned_example}.

\begin{lstlisting}[caption=Пример объявления неупакованного BCD-числа,label=lst:zoned_example,
language={[x86masm]Assembler}]
 number  db  01h, 02h, 03h  ; first example
 number  dw  0102h          ; second example
\end{lstlisting}

Рассмотрим арифметические операции, производимые над двумя неупакованными BCD-числами.

\pagebreak

Пример \textbf{сложения двух неупакованных BCD-чисел} приведен на рисунке~\ref{lst:zoned_addition}.

\begin{lstlisting}[caption=Пример сложения двух неупакованных BCD-чисел,label=lst:zoned_addition]
 07 = 0000 0111
 +
 08 = 0000 1000
 =
 15 = 0000 1111
\end{lstlisting}

Из рисунка видно, что если результат сложения получается больше
девяти, то нарушается формат записи BCD-числа.
Правильный результат в неупакованном
BCD-формате в двоичном представлении должен быть таким: данном случае:
$ 0000~0001~0000~0101 $.

Для получения правильного результата при выполнении арифметических операций над BCD-числами
в языке ассемблер используются \textit{команды корректировки}. Для коррекции операции сложения
двух неупакованных BCD-чисел и представления результата в символьном виде используется
команда \textit{AAA (ASCII Adjust for Addition)}. Команда \textit{AAA} работает неявно
с регистром \textit{al} и анализирует значение его младшей тетрады.
Если это значение меньше девяти, то флаг \textit{CF} сбрасывается в ноль и
осуществляется переход к следующей команде. Если это значение больше девяти, то
выполняются следующие действия:
\begin{itemize}
  \item к содержимому младшей тетрады \textit{al} прибавляется шесть, тем самым значение
  десятичного результата корректируется в нужную сторону;
  \item флаг \textit{CF} устанавливается в единицу, тем самым фиксируется её
  перенос в старший разряд.
\end{itemize}

Далее необходимо использовать команду \textit{ADC}, которая при сложении учтёт перенос
из предыдущего разряда.

Для выполнения операции \textbf{вычитания двух неупакованных BCD-чисел} используется
корректирующая команда \textit{AAS (ASCII Adjust for Substraction)}. Команда \textit{AAS}
анализирует младшую тетраду регистра \textit{al} следующим образом: если её значение
меньше девяти, то флаг \textit{CF} сбрасывается в ноль и управление передаётся
следующей команде. Если значение тетрады больше девяти, то команда \textit{AAS}
выполняет следующие действия:
\begin{itemize}
  \item из содержимого младшей тетрады регистра \textit{al} вычитается шесть;
  \item старшая тетрада регистра \textit{al} обнуляется;
  \item флаг \textit{CF} устанавливается в единицу, тем самым фиксируется воображаемый
  заём из старшего разряда.
\end{itemize}

Команда \textit{AAS} применяется вместе с основными командами вычитания \textit{SUB} и \textit{SBB}.
При этом команду \textit{SUB} есть смысл использовать только один раз при вычитании самых
младших цифр операндов, далее должна применяться команда \textit{SBB}, которая будет учитывать
возможный заём из старшего разряда.

Для выполнения операции \textbf{умножения одноразрядных неупакованных BCD-чисел} необходимо
выполнить следующие операции:
\begin{itemize}
  \item поместить один из сомножителей в регистр \textit{al};
  \item поместить второй сомножитель в регистр или память;
  \item перемножить сомножители командой \textit{MUL};
  \item скорректировать результат.
\end{itemize}

Для коррекции результата после умножения в целях представления его в символьном виде
применяется специальная команда \textit{AAM (ADCII Adjust for Multiplication)}. Она
работает с регистром \textit{ax} следующим образом: делит \textit{al} на десять, частное
от деления помещает в \textit{al}, остаток --- в \textit{ah}.

Для того, чтобы перейти от умножения одноразрядных чисел к умножению чисел произвольной
длинны, необходимо учитывать смещение одного из операндов, которое возникает при
умножении в <<столбик>>.

Для выполнения операции \textbf{деления неупакованных BCD-чисел} необходимо предварительно
выполнить действия по коррекции следующим образом:
\begin{itemize}
  \item предварительно в регистре \textit{ax} нужно получить две неупакованные
    BCD-цифры делимого;
  \item выполнить команду \textit{AAD (ASCII Adjust for Division)}.
\end{itemize}

Команда \textit{AAD} преобразует двузначное неупакованное BCD-число в регистре \textit{ax}
в двоичное число, которое впоследствии будет играть роль делимого в операции деления.
Кроме преобразования, команда \textit{AAD} помещает полученное десятичное число в регистр \textit{al}.

Алгоритм, по которому команда \textit{AAD} выполняет данное преобразования выглядит следующим образом:
\begin{itemize}
  \item умножить на 10 содержимое \textit{ah} исходного BCD-числа;
  \item выполнить сложение \textit{ah} + \textit{al}, результат которого занести в \textit{al};
  \item обнулить содержимое \textit{ah}.
\end{itemize}

Затем необходимо выполнить команду деления \textit{DIV} для деления содержимого
\textit{ax} на одну BCD-цифру, находящуюся в байтовом регистре или байтовой ячейке памяти.
Для деления чисел большой разрядности нужно реализовать алгоритм деления в <<столбик>>.

\subsection{Упакованные BCD-числа}
\label{sub:packed}

В упакованном формате BCD-числа каждый байт содержит две десятичные цифры.
Пример объявления упакованного BCD-числа представлен
на рисунке~\ref{lst:packed_example}.

\begin{lstlisting}[caption=Пример объявления упакованного BCD-числа,label=lst:packed_example,
language={[x86masm]Assembler}]
 number  db  12h   ; first example
 number  dw  1234h ; second example
\end{lstlisting}


Десятичная цифра представляет собой двоичное значение в диапазоне от нуля до девяти
размером четыре бита. При этом код старшей цифры числа занимает старшие четыре бита.
Следовательно, диапазон представления десятичного упакованного числа в одном байте
составляет от $ 00 $ до $ 99 $. 

% \textbf{Сложение упакованных BCD-чисел}

% \textbf{***}

% \textbf{Вычитание упакованных BCD-чисел}

% \textbf{***}

\subsection{Текст задания}

Требуется разработать программу для конвертации температуры, выраженной 
в градусах Цельсия в градусы Фаренгейта и обратно.

Программа должна корректно обрабатывать как целочисленные,
так и рациональные входные данные.

Программа должна иметь интерактивный пользовательский интерфейс,
протоколирование ввода и вывода в файл.

\newpage
