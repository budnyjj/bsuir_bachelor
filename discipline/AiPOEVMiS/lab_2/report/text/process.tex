\section{ХОД РАБОТЫ}

\subsection{Разработка пользовательского интерфейса}

При разработке пользовательского интерфейса будем
руководствоваться следующими соображениями:
\begin{itemize}
    \item пользователь должен иметь возможность выбирать направление
    конвертации (из градусов Цельсия в градусы Фаренгейта, или наоборот);

    \item пользователь должен иметь возможность вводить только числа --- как целые,
      так и дробные, при этом разделитель (<<точка>>) при вводе может встретиться
      только один раз;

    \item пользователь может вводить отрицательные числа,
      при этом знак <<минус>> при вводе может встретиться только один раз в начале ввода;

    \item пользователь может завершить программу с помощью горячей клавиши;
\end{itemize}

На основе вышеперечисленных требований реализуем пользовательский интерфейс.
Создадим меню, в котором пользователю будет дана возможность перейти к конвертации
в выбранном направлении, либо выйти. Принцип построения меню на языке Ассемблер
представлен на рисунке~\ref{lst:menu}.

\begin{lstlisting}[caption=Принцип построения пользователького меню,
label=lst:menu,language={[x86masm]Assembler},basicstyle=\scriptsize\ttfamily]
 OutMenu macro
     OutStr  slct_prmpt
     OutStr  var_cf
     OutStr  var_fc
     OutStr  var_quit
     OutStr  newline
     endm
 ...
 code segment
 _main_menu_:
     OutMenu

 _main_menu_choose_variant_:
     InChar                      ; Choose variant
     cmp al, 31h                 ; '1'
     je _cf_
     cmp al, 32h                 ; '2'
     je _fc_
     cmp al, 113                 ; q
     je _end_
     cmp al, 81                  ; Q
     je _end_
     jmp _main_menu_choose_variant_
 ...
\end{lstlisting}

\subsection{Пользовательский ввод данных}

Для ввода BCD-числа будем использовать процедуру, значительная часть которой
приведена на рисунке~\ref{lst:input_proc}.

\begin{lstlisting}[caption={Процедура обработки пользователького ввода данных},
label=lst:input_proc,language={[x86masm]Assembler},basicstyle=\scriptsize\ttfamily]
 InBCDp  proc
   cld
     mov     dx, 0
     mov     bx, 0
     mov     si, 0
 _in_:
     InChar
     cmp     al, 0                   ; extended keystroke
     je      _extended_
     cmp     al, 13                  ; enter
     je      _success_
     cmp     al, 113                 ; q
     je      _error_
     cmp     al, 81                  ; Q
     je      _error_
     cmp     al, 45                  ; got -
     je      _minus_
     cmp     al,46                   ; got .
     je      _dot_
     cmp     al, 30h                 ; got char with code <= 30h
     jl      _not_digit_
     cmp     al, 3Ah                 ; got char with code >= 40h
     jge     _not_digit_
     jmp     _digit_
 ...
 endp
 ...
\end{lstlisting}

Макрос, помещающий нужные аргументы в регистры и вызывающий данную процедуру,
представлен на рисунке~\ref{lst:input_macros}.

\begin{lstlisting}[caption={Макрос обработки пользовательского ввода},
label=lst:input_macros,language={[x86masm]Assembler},basicstyle=\scriptsize\ttfamily]
 InBCD macro mem,len_input,max_size,len_read,pos_dot,sign
   mov     di,offset mem
   mov     cx,len_input
   call  InBCDp
   mov     len_read, ax
   mov     pos_dot, bx
   mov     sign, dl

   AlignBCD mem,len_read,max_size
 endm
\end{lstlisting}

\subsection{Вычисление результата}

Приведём последовательность получения результата на примере перевода числа из
градусов Цельсия в градусы Фаренгейта. Последовательность конвертации на языке 
Ассемблер представлена на рисунке~\ref{lst:convertion_main}.

\begin{lstlisting}[caption={Последовательность конвертации с
    использованием макроподстановок в языке Ассемблер},
  label=lst:convertion_main,language={[x86masm]Assembler},basicstyle=\scriptsize\ttfamily]
 ...
 DivBCD  a,sign_a, k1,sign_k1, len_mem,len_a   ; /5
 MulBCD  a,sign_a, k2,sign_k2, len_mem,len_a   ; *9
 AddBCD  a,sign_a, b,sign_b, len_mem,len_a     ; +32
 ...
\end{lstlisting}

Рассмотрим команды, представленные на рисунке~\ref{lst:convertion_main} подробнее.
\textit{DivBCD}, \textit{MulBCD}, \textit{AddBCD} --- макросы, имеющие следующие параметры:
\begin{itemize}
  \item \textit{a} --- число, вводимое пользователем (первый операнд);
  \item \textit{sign\_a} --- знак первого операнда;
  \item \textit{k1 (k2, b)} --- второй операнд;
  \item \textit{sign\_k1 (sign\_k1, sign\_b)} ---  знак второго операнда;
  \item \textit{len\_mem} --- максимальный размер операндов;
  \item \textit{len\_a} --- длина исходного числа (первого операнда).
\end{itemize}

Таким образом, результат конвертации получается путём умножения
исходного числа на $\frac{9}{5}$ и прибавления к полученному значению 32.

\subsection{Вывод результата на экран}

Для вывода BCD-числа на экран пользователя используется 
макрос \textit{OutBCD}, сигнатура которого представлена на рисунке~\ref{lst:out_bcd}.
Этот макрос производит вывод BCD-числа, расположенного
в области памяти \textit{mem} размера \textit{max\_size},
длиной \textit{len}, знаком \textit{sign} и позицией точки \textit{pos\_dot}.

\begin{lstlisting}[caption={Сигнатура макроса вывода BCD-числа на экран пользователя},
label=lst:out_bcd,language={[x86masm]Assembler}]
 OutBCD	macro	mem,max_size,len,pos_dot,sign
\end{lstlisting}

Исходный текст разработанной программы находится в приложении~А.
