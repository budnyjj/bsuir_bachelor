 %%% Макет страницы %%%
\geometry{a4paper,top=20mm,bottom=27mm,left=30mm,right=15mm}

%%% Язык текста %%%
\selectlanguage{russian}

%%% Кодировки и шрифты %%%
\renewcommand{\rmdefault}{ftm} % Включаем Times New Roman

%%% Выравнивание и переносы %%%
\sloppy				% Избавляемся от переполнений
\clubpenalty=10000		% Запрещаем разрыв страницы после первой строки абзаца
\widowpenalty=10000		% Запрещаем разрыв страницы после последней строки абзаца
\interfootnotelinepenalty=10000 % Запрет разрывы сносок

%%% Требования СТП %%%
%%% Интерлиньяж
\setstretch{1.15}

%%% Нумерация страниц %%%
\fancypagestyle{empty}{%
\fancyhf{} % clear all header and footer fields
\renewcommand{\headrulewidth}{0pt}
\renewcommand{\footrulewidth}{0pt}
\setlength{\headheight}{5mm} 
}

\fancypagestyle{plain}{%
\fancyhf{} % clear all header and footer fields
\fancyfoot[R]{\thepage} 
\renewcommand{\headrulewidth}{0pt}
\renewcommand{\footrulewidth}{0pt}
\setlength{\headheight}{5mm}
}

\pagestyle{plain}

%%% Библиография %%%
\makeatletter
\bibliographystyle{ugost2003s}	% Оформляем библиографию в соответствии с ГОСТ 7.1 2003

%%% Изображения %%%
\graphicspath{{images/}} % Пути к изображениям

%%% Содержание %%%
\renewcommand{\cfttoctitlefont}{\hfil \large\bfseries}

\renewcommand{\cftsecaftersnum}{}
\renewcommand{\cftsecfont}{}   
\renewcommand{\cftsecpagefont}{\normalsize}
\renewcommand{\cftsecleader}{\cftdotfill{\cftdotsep}}

\setlength{\cftparskip}{0em}
\setlength{\cftbeforesecskip}{0em}
\setlength{\cftaftertoctitleskip}{\baselineskip}
\setlength{\cftsecindent}{0em}
\setlength{\cftsecnumwidth}{1ex}

\setlength{\cftsubsecindent}{1em}
\setlength{\cftsubsecnumwidth}{2ex}


%%% Размеры заголовков
\newcommand{\sectionbreak}{\clearpage}

\titleformat{\section}{\large\bfseries}{\thesection}{0.7em}{}
\titlespacing*{\section}{5ex}{\baselineskip}{\baselineskip}

\titleformat{name=\section,numberless}{\large\bfseries\filcenter}{}{0em}{}
\titlespacing*{name=\section,numberless}{0ex}{\baselineskip}{\baselineskip}

\titleformat{name=\subsection}{\normalsize\bfseries}{\thesubsection}{0.7em}{}
\titlespacing*{\subsection}{5ex}{\baselineskip}{\baselineskip}

\titleformat{name=\subsection,numberless}{\normalsize\bfseries}{}{0em}{}

%%% Нумерация параграфов

\counterwithout{paragraph}{subsubsection}
\counterwithin{paragraph}{subsection}
\renewcommand{\theparagraph}{\thesubsection.\arabic{paragraph}}
\setcounter{secnumdepth}{4}

\titleformat{name=\paragraph}[runin]{\normalsize\bfseries}{\theparagraph}{\wordsep}{}
\titlespacing*{\paragraph}{5ex}{\baselineskip}{\wordsep}

%%% Размеры текста формул %%%

\DeclareMathSizes{12}{12}{6}{4}

%%% Оформление текста

\setlength{\parskip}{0pt}
\setlength{\parindent}{1.25cm}

%%% Расстояние между плавающими элементами

\setlength{\floatsep}{\baselineskip} % between top floats
\setlength{\textfloatsep}{\baselineskip} % between top/bottom floats and text
\setlength{\intextsep}{\baselineskip} % between text and float
\setlength{\dbltextfloatsep}{\baselineskip}
\setlength{\dblfloatsep}{\baselineskip}

 % костыль для того, чтобы убрать расстояние от картинки до текста
\setlength{\abovecaptionskip}{0pt}
\setlength{\belowcaptionskip}{0pt}

%%% Расстояние между формулами

\AtBeginDocument{%
  \setlength\abovedisplayskip{\baselineskip}%
  \setlength\belowdisplayskip{\baselineskip}%
  \setlength\abovedisplayshortskip{\baselineskip}%
  \setlength\belowdisplayshortskip{\baselineskip}%
}

%%% Оформление списков

\AddEnumerateCounter{\asbuk}{\@asbuk}{\cyrm}

\setlist{nosep,listparindent=\parindent}
\setlist[1]{itemindent=17.5mm,leftmargin=0em,itemsep=0em,topsep=0em,parsep=0em}             
\setlist[itemize,1]{label=$-$}
\setlist[enumerate,1]{label=\arabic*)}

\setlist[2]{itemindent=6.5em,leftmargin=0em,itemsep=0em,topsep=0em,parsep=0em}             

% Определяем новый стиль для списков,
% на которые есть ссылки в тексте
\newlist{reflist}{enumerate*}{1}
\setlist*[reflist,1]{%
  label=\asbuk*),
}

\setlist*[reflist,2]{%
  label=\arabic*),
}

%%% Оформление подписей к плавающим элементам

\captionsetup[figure]{labelsep=endash,
  justification=centering,
  singlelinecheck=false,
  position=bottom,
  skip=1\baselineskip
}

\captionsetup[subfigure]{margin=0pt,
  position=bottom,
  font=normalsize,
  labelformat=simple,
  labelsep=endash,
  skip=\baselineskip}

\captionsetup[table]{
  labelsep=endash,
  justification=raggedright,
  singlelinecheck=false,
  skip=0pt}

\captionsetup[lstlisting]{labelsep=endash,
  justification=centering,
  singlelinecheck=false,
  skip=\baselineskip}

\lstset{
  basicstyle=\small\ttfamily,
  keywordstyle=\bfseries,
  commentstyle=\itshape,
  numbers=none,
  stepnumber=1,
  frame=single,
  framesep=0ex,
  resetmargins=true,
  captionpos=b,
  keepspaces=true,
  breaklines=true,
  floatplacement=!h,
  aboveskip=1.2\baselineskip,
  belowskip=\baselineskip,
  abovecaptionskip=\baselineskip,
  belowcaptionskip=0pt}

\renewcommand{\arraystretch}{1.5}

%%% Нумерация плавающих элементов %%%

\renewcommand{\thesubfigure}{\asbuk{subfigure}}

%%% Совместная нумерация для рисунков и листингов

\makeatletter
\AtBeginDocument{%
\renewcommand{\thelstlisting}{\arabic{lstlisting}}
\renewcommand{\thefigure}{\arabic{figure}}
\let\c@lstlisting\c@figure}
\makeatother 

%%% Настройка размеров вертикальных отступов

\renewcommand{\smallskip}{\vspace{0.5ex}}
\renewcommand{\bigskip}{\vspace{\baselineskip}}
