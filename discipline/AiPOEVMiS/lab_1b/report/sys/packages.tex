%%% Поля и разметка страницы %%%
\usepackage{pdflscape}   % Для включения альбомных страниц
\usepackage{geometry} % Для последующего задания полей
\usepackage{setspace} % Для интерлиньяжа
\usepackage{titlesec}
\usepackage{tocloft}
\usepackage{enumitem}
\usepackage{fancyhdr}
\usepackage{chngcntr} % Для нумерации параграфов

%%% Кодировки и шрифты %%%
\usepackage{cmap}                    % Улучшенный поиск русских слов в полученном pdf-файле
\usepackage[T1,T2A]{fontenc}	     % Поддержка русских букв
\usepackage[utf8x]{inputenc}	     % Кодировка utf8
% \usepackage[english=nohyphenation,russian=nohyphenation]{hyphsubst} % запрет перносов
\usepackage[english, russian]{babel} % Языки: русский, английский

\usepackage{courier}
\usepackage{pscyr}  % Красивые русские шрифты

%%% Математические пакеты %%%
\usepackage{amsthm,amsfonts,amsmath,amssymb,amscd} % Математические дополнения от AMS

%%% оформление абзацев %%%
\usepackage{indentfirst} % Красная строка

%%% Цвета %%%
\usepackage{color}

%%% Таблицы %%%
\usepackage{longtable}		     % Длинные таблицы
\usepackage{tabularx}    
\usepackage{tabulary}                                     
\usepackage{multirow,makecell,array} % Улучшенное форматирование таблиц

%%% Общее форматирование
\usepackage[figureposition=bottom,tableposition=top]{caption}
\usepackage{subcaption}

\usepackage{soul}     % Поддержка переносоустойчивых подчёркиваний и зачёркиваний
\usepackage{multicol}

%%% Библиография %%%
\usepackage{cite} % Красивые ссылки на литературу

%%% Гиперссылки %%%
\usepackage[unicode,plainpages=false,pdfpagelabels=false]{hyperref}

%%% Изображения %%%
\usepackage{graphicx} % Подключаем пакет работы с графикой

%%% Листинги %%%
\usepackage{listings} %% собственно, это и есть пакет listings

%%% Рисунки TEX %%%
\usepackage{tikz}

