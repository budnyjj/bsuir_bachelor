\section{ХОД РАБОТЫ}
\subsection{Индивидуальное задание}

Требуется написать скрипт на языке Power Shell, демонстрирующий работу
командлетов~\texttt{ConvertTo-SecureString}~и~\texttt{Get-Credential}.


\subsection{Теоретические сведения}

Истоки создания PowerShell (первоначальное название Monad) связаны с
доработкой WMIC для обеспечения доступа из командной строки к любым классам
платформы .NET Framework. Windows PS является и оболочкой командной строки
и средой выполнения сценариев.
Язык сценариев PS поддерживает управление объектами .NET,
обладает совместимостью с языками, используемыми при программировании
для .NET и имеет синтаксическое сходство с C\# (язык PS по существу
представляет собой упрощённый C\#); это обусловлено тем, что  оболочка PS
основана на .NET Framework.

Windows  PS принципиально  отличается от других оболочек тем,
что PS обрабатывает не текст, а объекты платформы .NET.
Windows  PowerShell содержит встроенные команды (командлеты; cmdlets),
которые имеют унифицированный интерфейс и обрабатываются одним
синтаксическим анализатором.

Командлет представляет собой команду, выполняющую одну единственную функцию.
Версия 1.0 Windows PowerShell была выпущена в 2006 г.
Она доступна для Windows XP SP2/SP3, Windows Server 2003, Windows Vista,
и встроена в Windows Server 2008. Версия 2.0 Windows PowerShell является
компонентом Windows 7 и Windows Server 2008 R2.
Версия 2.0 PS доступна для Windows XP SP3, Windows Server 2003 SP2,
Windows Vista SP1 и Windows Server 2008.

\pagebreak

Синтаксис командлета \texttt{ConvertTo-SecureString} представлен на
рисунке~\ref{lst:convertto_secure_string}.
\begin{lstlisting}[caption={Синтаксис командлета \texttt{ConvertTo-SecureString}},
label=lst:convertto_secure_string,basicstyle=\scriptsize\ttfamily]
 Parameter Set: Secure
 ConvertTo-SecureString [-String] <String> [[-SecureKey] <SecureString> ] [ <CommonParameters>]

 Parameter Set: Open
 ConvertTo-SecureString [-String] <String> [-Key <Byte[]> ] [ <CommonParameters>]

 Parameter Set: PlainText
 ConvertTo-SecureString [-String] <String> [[-AsPlainText]] [[-Force]] [ <CommonParameters>]
\end{lstlisting}

Синтаксис командлета \texttt{Get-Credential} представлен на
рисунке~\ref{lst:get_credential}.
\begin{lstlisting}[caption={Синтаксис командлета \texttt{Get-Credential}},
label=lst:get_credential,basicstyle=\scriptsize\ttfamily]
 Parameter Set: CredentialSet
 Get-Credential [-Credential] <PSCredential> [ <CommonParameters>]

 Parameter Set: MessageSet
 Get-Credential [[-UserName] <String> ] -Message <String> [ <CommonParameters>]
\end{lstlisting}

\subsection{Ход работы}

Создадим объект учётной записи (Credential object), при этом используем
стандартное шифрование для поля password. Используемые команды
продемонстрированы на рисунке~\ref{lst:results}.
\begin{lstlisting}[caption={Пример совместного использования командлетов \texttt{Get-Credential}
 и \texttt{ConvertTo-SecureString}},
label=lst:results]
 PS C:\> $User = "Domain01\User01"
 PS C:\> $PWord = ConvertTo-SecureString -String "P@sSwOrd" -AsPlainText -Force
 PS C:\> $Credential = New-Object -TypeName System.Management.Automation.PSCredential -ArgumentList $User, $PWord
\end{lstlisting}

