\section{ТЕОРЕТИЧЕСКИЕ СВЕДЕНИЯ}

\subsection{Разработка оконных приложений в
  операционных системах на базе ядра GNU/Linux}

В операционных системах на базе ядра GNU/Linux, как и на всех Unix-подобных
системах, графическая подсистема находится в пространстве пользователя, 
а не интегрирована в ядро ОС, подобно тому, как это сделано в Windows.
Благодаря этому пользователь может выбирать, какой набор 
программного обеспечения ему использовать для отрисовки графики.
На данный момент наиболее популярной реализацией графической подсистемы является
X.Org, реализующая протокол передачи графической информации X11.

Среди наиболее популярных свободных реализаций библиотек, 
определяющих вид и интерфейс взаимодействия, можно выделить следующие:
GTK, KDE, XFCE, LXDE. 

В ходе данной лабораторной работы будет разработан графический интерфейс 
приложения на базе GTK\# --- библиотеки, предоставляющей интерфейс 
взаимодействия приложений, написанных С\#, и GTK.

В качестве компилятора и исполнителя программного кода
будет использоваться mono --- свободная реализация .NET стека.
Для проектирования графического интерфейса будет использоваться приложение
Glade.