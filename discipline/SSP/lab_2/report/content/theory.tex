\section{ТЕОРЕТИЧЕСКИЕ СВЕДЕНИЯ}

Пример команд языка С\#, используемых для чтения строковых данных из файла
представлен на рисунке~\ref{lst:read_from_file}.

\begin{lstlisting}[
                  caption=Команды чтения строковых данных из файла на языке C\#,
                  label=lst:read_from_file,
                  basicstyle=\scriptsize\ttfamily,
                  language=C++
                  ]
  OpenFileDialog openFileDialog = new OpenFileDialog();
  openFileDialog.Filter = "Text files|*.txt";
  openFileDialog.Title = "Select text file with cars";
  if (openFileDialog.ShowDialog() == DialogResult.OK)
  {
      string openedFileName = openFileDialog.FileName;
      string[] lines = {};
      if (openedFileName != string.Empty)
      {
          lines = System.IO.File.ReadAllLines(openedFileName);
      }
  }
\end{lstlisting}

Для записи строковых данных в файл может быть использована аналогичная конструкция,
пример которой представлен на рисунке~\ref{lst:save_to_file}.

\begin{lstlisting}[
                  caption=Команды записи строковых данных в файл на языке C\#,
                  label=lst:save_to_file,
                  basicstyle=\scriptsize\ttfamily,
                  language=C++
                  ]
  SaveFileDialog saveFileDialog = new SaveFileDialog();
  saveFileDialog.FileName = "filteredCars";
  saveFileDialog.DefaultExt = "txt";
  saveFileDialog.Filter = "Text files|*.txt";

  if (saveFileDialog.ShowDialog() == DialogResult.OK)
  {
      List<string> cars = new List<string>();
      foreach (Car car in carList)
      {
          cars.Add(car.ToString());
      }
      File.WriteAllLines(saveFileDialog.FileName, cars);
  }
\end{lstlisting}

\newpage
