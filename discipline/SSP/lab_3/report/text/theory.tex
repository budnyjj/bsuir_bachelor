\section{ТЕОРЕТИЧЕСКИЕ СВЕДЕНИЯ}

\subsection{Работа с базами данных в среде Mono}

Свободная реализация Microsoft .Net Framework Mono поддерживает 
работу с различными СУБД посредством интерфейса ADO.NET:
\begin{itemize}
  \item PostrgreSQL;
  \item MySQL;
  \item sqlite;
  \item Oracle;
  \item Microsoft SQL Server;
\end{itemize}

В данной лабораторной работе для реализации задания будет использоваться sqlite.
Sqlite --- система управления базами данных, предназначенная для работы
со сравнительно небольшими объемами инофрмации. 
Отличительная особенность этой системы --- хранение данных и метаданных,
в одном файле. Это значительно упрощает процесс переноса данных, а также 
избавляет от необходимости использования сложного в установке серверного ПО,
обслуживающего запросы.

Интерфейс для взаимодействия с СУБД sqlite является унифицированным,
следовательно, в случае роста нагрузки можно будет заменить хранилище 
данных на более масштабируемое, например, MySQL или Oracle.
