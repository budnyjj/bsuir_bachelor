\section{ХОД РАБОТЫ}

\subsection{Текст задания}

Требуется создать собственный невизуальный компонент, который
имеет два метода: один используется для получения головной части строки,
второй --- хвостовой. Например, первый метод можно объявить так, как показано
на рисунке~\ref{lst:first_method_declaration}.
\begin{lstlisting}[caption= Пример объявления первого метода,
                   label=lst:first_method_declaration]
 String Head(String str, String mask)
\end{lstlisting}

Приведенный метод получает исходную строку, например <<Hello, friends>> и
строку--разделитель <<,>>. Данный метод отыскивает в строке <<Hello, friends>>
первое вхождение строки--разделителя и возвращает часть исходной строки,
стоящую до разделителя. Если же разделитель не встретится, то возвращается
вся исходная строка. Метод, возвращающий хвостовую часть, отличается от описанного
только тем, что возвращает часть строки, стоящую после разделителя.

\subsection{Детали реализации программы}

Создадим собственный компонент с использованием среды Microsoft Visual Studio.
Объявим в классе два метода работы со строками так, как это показано
на рисунке~\ref{lst:methods_declaration}.
\begin{lstlisting}[caption= Объявление методов для работы со строками,
                   label=lst:methods_declaration]
 public String HeaderOfString(String str, String delimiter);
 public String FooterOfString(String str, String delimiter);
\end{lstlisting}

Скомпилируем проект. В результате получим библиотеку, которая будет содержать
описанные выше методы.

Создадим новый проект, подключим созданный ранее компонент, который позволит
использовать методы \texttt{HeaderOfString()} и \texttt{FooterOfString()}.
С помощью формы покажем пользователю пример использования методом обработки строк.


Исходный текст разработанного приложения расположен в приложении~А.

\newpage
