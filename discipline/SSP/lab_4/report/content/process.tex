\section{ХОД РАБОТЫ}

\subsection{Текст задания}

Требуется построить приложение с клиент-серверной архитектурой на основе протокола TCP.
На стороне сервера определить интерфейс для работы с базой данных.
Клиент, используя интерфейс, получает необходимые ему данные из базы данных,
расположенной на стороне сервера. Предусмотреть передачу по протоколу TCP
картинки и открытие её на стороне клиента.

\subsection{Детали реализации программы}

С помощью реляционной СУБД Microsoft Office Access создадим базу данных, которая
будет содержать таблицу \texttt{students}. В Visual Studio на языке C\# создадим
класс Student, который в дальнейшем будем использовать для связи с базой данных и
сериализации, десериализации объектов в формат XML. Схема отношения \texttt{<<students>>}
приведена в таблице~\ref{tbl:students_scheme}.

\begin{table}[h!]
  \caption{Схема отношения \texttt{<<students>>}}
  \label{tbl:students_scheme}
  \small{
    \centering
    \begin{tabular}{| p{0.23\textwidth} | p{0.22\textwidth} | p{0.22\textwidth} | p{0.22\textwidth} |}
      \hline
      Название атрибута в \newline отношении \texttt{students} &
      Название сеттера в \newline классе \texttt{Student} &
      Тип хранимых данных  &
      Описание \\

      \hline
      id & Id & BIGINT & ID студента \\

      \hline
      firstName & FirstName & VARCHAR(255) & Имя \\

      \hline
      lastName & LastName & VARCHAR(255) & Фамилия  \\

      \hline
      groupNumber & GroupNumber & INT & Номер группы \\

      \hline
      course & Course & BYTE & Курс обучения \\

      \hline
      averageScore & AverageScore & FLOAT & Средний балл \\

      \hline
      photoPath & PhotoPath & VARCHAR(255) & Путь к фотографии \\

      \hline
    \end{tabular}
  }
\end{table}

Создадим приложение-сервер, которое будет постоянно ожидать подключения клиента.
В случае подключения клиента, сервер будет распознавать запрос от клиента (<<request>>),
и формировать ответ на него (<<response>>). Все сообщения между клиентом и сервером
будут производиться с использованием языка разметки XML.

Приложение-клиент, определив требования пользователя будет определенным образом
формировать запрос, который посредством TCP соединения будет отправляться серверу.

Для сериализации на стороне сервера и десериализации на стороне клиента объекта
класса Student будем использовать класс \texttt{XmlSerializer} языка C\#.

Исходный текст разработанного приложения-сервера, приложения-клиента и общей
для них библиотеки \texttt{StudentLibrary} расположен в приложении~А,
Б~и~В~соответственно.

\newpage
