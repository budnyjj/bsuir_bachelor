\section{ХОД РАБОТЫ}

\subsection{Формулировка задачи}

Требуется рассчитать предельные вероятности для цепи Маркова, воспользовавшись
исходными данными:
\begin{equation*}
P = \begin{pmatrix}
      0{,}6 & 0 & 0{,}4 \\
      0{,}4 & 0{,}6 & 0 \\
      0{,}4 & 0{,}2 & 0{,}4 \\
    \end{pmatrix}.
\end{equation*}


\subsection{Теоретические сведения}
\label{subs:theory}

\textit{Теорема (Маркова)}. Если существует такое $s>0$, что все
$p_{i,j}(s)>0$, то существуют такие числа $p_j^*, j = 1,2,...,k$, что независимо
от индекса $i$ выполняется следующие соотношения:
\begin{equation*}
  \lim_{n\to\infty} p_{i,j}(n) = p_j^*, j=1,2, ...,k, \hspace{7mm}
  \sum\limits_{j=1}^{k}p_j^* = 1.
\end{equation*}

Физический смысл этой теоремы состоит в том, что вероятности $p_{i,j}(n)$
перехода из состояния $E_i$ в состояние $E_j$ за $n$ шагов при $n\to\infty$
не зависят от состояния $E_i$ из которого был начат переход. Система как бы
забывает о своём состоянии в далёком прошлом.

\textit{Теорема}. Если для цепи Маркова существует вектор-столбец предельных
вероятностей $p^*$, то он удовлетворяет следующей системе линейных алгебраических
уравнений:
\begin{equation*}
  A^* = P^*.
\end{equation*}

\textit{Теорема}. Если существует вектор предельных вероятностей $P^*$, то он
является и вектором безусловных предельных вероятностей:
\begin{equation*}
  P^T p^* = p^*, \hspace{7mm} \sum\limits_{j=1}^{k} p_j^* = 1.
\end{equation*}


\subsection{Ход работы}

На языке программирования Python создадим программу, выполняющую расчет
предельных вероятностей цепи Маркова. В результате расчётов получим:
\begin{equation*}
p_1^* = 0.5, \hspace{7mm} p_2^* = 0.1(6), \hspace{7mm} p_1^* = 0.(3).
\end{equation*}

Приведенные значения для предельных вероятностей практически полностью
совпадают с рассчитанным в лабораторной работе №~4 вектором безусловных вероятностей
цепи Маркова на десятом шаге, изображенных на рисунке~\ref{lst:state_probs}.

\begin{lstlisting}[caption=Вектор безусловных вероятностей перехода цепи Маркова \\
  на десятом шаге,label=lst:state_probs]
 STATE PROBABILITIES #10:
 [ 0.5         0.16669742  0.33330258]
\end{lstlisting}

Исходный код разработанной программы представлен в приложении~А.