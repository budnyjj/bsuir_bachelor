\section{ХОД РАБОТЫ}

\subsection{Формулировка задачи}

Требуется определить функцию для моделирования полной группы случайных событий.
Входными параметрами этой функции должен быть вектор вероятностей случайных
событий $ p = (p_1, p_2, \dots, p_k) $ и количество независимых испытаний n,
а выходным --- вектор смоделированной последовательности
событий $ e = (e_1, e_2, \dots, e_n) $. С помощью такой функции смоделировать
последовательность независимых испытаний для случайных событий,
при $ p = (0~0{,}3~0{,}7) $. По достаточно большой последовательности
испытаний рассчитать оценки вероятностей этих событий.

Определить функцию для моделирования выборки объёмом n из дискретного
распределения с возможными значениями $ x = (x_1, x_2, \dots, x_3) $ и
их вероятностями $ p = (p_1, p_2, \dots, p_k) $. Входными параметрами функции
должны быть векторы $ x $, $ p $ и объём выборки $ n $, а выходными --- выборка
возможных значений $ y = (y_1, y_2, \dots, y_n) $, при $ x = (-1~0~1) $.
По выборке достаточно большого объёма рассчитать оценки среднего значения и
дисперсии дискретного распределения.


\subsection{Теоретические сведения}

Случайные события $ A_i \subseteq F $, заданные на вероятностном пространстве $ (\Omega, F, P) $, образуют полную группу событий, если выполняются условия:
$ \cup_{i = 1}^{k} A_i = \Omega, A_i \cap A_j = O, i \neq j$ ,
$ P(A_i) \ge 0, i = \overline{1,k}, \sum\limits_{i = 1}^{k} P(A_i) = 1 $.

Математическое ожидание и дисперсия дискретной случайной величины определяется
по следующим формулам:
\begin{align*}
  M(x) &= \dfrac{x_1 + x_2 + \dots + x_n}{n}. \\
  D(x) &= \dfrac{\sum\limits_{i = 1}^{n} (x_i - M(x))^2}{n}.
\end{align*}

\pagebreak


\subsection{Ход работы}

Воспользоваашись языком программирования Python 3, определим функцию для
моделирования полной группы случайных событий, входными параметрами которой
будут вектор вероятностей случайных событий $ p $ и количество независимых
испытаний n. Пример такой функции изображен на рисунке~\ref{lst:gen_events}.
\begin{lstlisting}[caption=Функция для моделирования полной группы случайных событий,
label=lst:gen_events, language=python, basicstyle=\scriptsize\ttfamily]
  def gen_events(event_probs, n):
       sum_ps = event_probs[:]

       for i in range(1, len(sum_ps)):
           sum_ps[i] = sum_ps[i-1] + sum_ps[i]

       if sum_ps[-1] != 1:
           raise ValueError("Sum of event probabilities != 1")

       for _ in range(n):
           v = random.uniform(0, 1)
           yield event_idx(sum_ps, v)
\end{lstlisting}

Функция для моделирования выборки объёмом n из дискретного распределения c
возможными значениями $ x $ и их вероятностями $ p $ представлена на
рисунке~\ref{lst:gen_values}.
\begin{lstlisting}[caption=Функция для моделирования выборки объёмом $n$ из
дискретного распределения с возможными значениями $ x $ и их вероятностями $ p $,
label=lst:gen_values, language=python, basicstyle=\scriptsize\ttfamily]
   def gen_values(values, event_probs, n):
     for event in gen_events(event_probs, n):
         yield values[event]
\end{lstlisting}

Предположим, что объём выборки равен 1000. Тогда значения математического
ожидания и дисперсии последовательности значений, полученных
в ходе независимых испытаний:
\begin{align*}
  M(x) = 0.711, \hspace{15mm} D(x) = 0.205.
\end{align*}

Исходный код разработанной программы представлен в приложении~А.

\newpage
