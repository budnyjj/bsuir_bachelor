\section{Результаты экспериментальных исследований}
\addcontentsline{toc}{section}{Результаты экспериментальных исследований}	% Добавляем его в оглавление

\begin{enumerate}

\item{Полевой транзистор с управляющим p-n-переходом с ОИ.}

\begin{enumerate}

\item 
Характеристика $ I_{с} = f( U_{\text{ЗИ}} ) $:

\begin{table} [h!]
  \caption{ Измерения при $ U_{\text{СИ}} = 5 $ В }
  \begin{tabular}{| m{1.5cm} | m{1.5cm} | m{1.5cm} | m{1.5cm} | m{1.5cm} | m{1.5cm} | m{1.5cm}l |}
    \hline
    \centering $ U_{\text{ЗИ}} $, В & & & & & & &\\
    \hline
    \centering $ I_c $, мА & & & & & & &\\
    \hline
  \end{tabular}
\end{table}

$ U_{\text{ЗИ}_{\text{отс}}} = $

\item 
Характеристики $ I_{с} = f( U_{\text{CИ}} ) $ в зависимости от разных значений напряжения $ U_{\text{ЗИ}} $:

\begin{table} [h!]
  \caption{ Измерения при $ U_{\text{ЗИ}} = 0 $ В }
  \begin{tabular}{| m{1.5cm} | m{1.5cm} | m{1.5cm} | m{1.5cm} | m{1.5cm} | m{1.5cm} | m{1.5cm}l |}
    \hline
    \centering $ U_{\text{CИ}} $, В & & & & & & &\\
    \hline
    \centering $ I_c $, мА & & & & & & &\\
    \hline
  \end{tabular}
\end{table}

\begin{table} [h!]
  \caption{ Измерения при $ U_{\text{ЗИ}} =  0,3 * U_{\text{ЗИ}_{\text{отс}}} = \hspace{8mm} $ В }
  \begin{tabular}{| m{1.5cm} | m{1.5cm} | m{1.5cm} | m{1.5cm} | m{1.5cm} | m{1.5cm} | m{1.5cm}l |}
    \hline
    \centering $ U_{\text{CИ}} $, В & & & & & & &\\
    \hline
    \centering $ I_c $, мА & & & & & & &\\
    \hline
  \end{tabular}
\end{table}

\begin{table} [h!]
  \caption{ Измерения при $ U_{\text{ЗИ}} =  0,6 * U_{\text{ЗИ}_{\text{отс}}} = \hspace{8mm} $ В }
  \begin{tabular}{| m{1.5cm} | m{1.5cm} | m{1.5cm} | m{1.5cm} | m{1.5cm} | m{1.5cm} | m{1.5cm}l |}
    \hline
    \centering $ U_{\text{CИ}} $, В & & & & & & &\\
    \hline
    \centering $ I_c $, мА & & & & & & &\\
    \hline
  \end{tabular}
\end{table}

\vspace{50mm}

\begin{figure}[h!]
  \begin{minipage}[h]{0.5\linewidth}
  		\caption{График зависимости $ I(U_{\text{ЗИ}}) $ }
  \end{minipage}
  \hfill   
  \begin{minipage}[h]{0.5\linewidth}
  		\caption{График зависимости $ I(U_{\text{CИ}}) $ }
  \end{minipage}   
\end{figure}

\end{enumerate}

\item{МДП-транзистор с индуцированным каналом с ОИ.}

\begin{enumerate}

\item 
Характеристика $ I_{с} = f( U_{\text{ЗИ}} ) $:

\begin{table} [h!]
  \caption{  Измерения при $ U_{\text{СИ}} = 5 $ В }
  \begin{tabular}{| m{1.5cm} | m{1.5cm} | m{1.5cm} | m{1.5cm} | m{1.5cm} | m{1.5cm} | m{1.5cm}l |}
    \hline
    \centering $ U_{\text{ЗИ}} $, В & & & & & & &\\
    \hline
    \centering $ I_c $, мА & & & & & & &\\
    \hline
  \end{tabular}
\end{table}

$ U_{\text{ЗИ}_{\text{пор}}} = $

\item 
Характеристики $ I_{с} = f( U_{\text{CИ}} ) $ в зависимости от разных значений напряжения $ U_{\text{ЗИ}} $:

\begin{table} [h!]
  \caption{ Измерения при $ U_{\text{ЗИ}} =  1,5 * U_{\text{ЗИ}_{\text{пор}}} = \hspace{8mm} $ В }
  \begin{tabular}{| m{1.5cm} | m{1.5cm} | m{1.5cm} | m{1.5cm} | m{1.5cm} | m{1.5cm} | m{1.5cm}l |}
    \hline
    \centering $ U_{\text{СИ}} $, В & & & & & & &\\
    \hline
    \centering $ I_c $, мА & & & & & & &\\
    \hline
  \end{tabular}
\end{table}

\begin{table} [h!]
  \caption{ Измерения при $ U_{\text{ЗИ}} =  2,5 * U_{\text{ЗИ}_{\text{пор}}} = \hspace{8mm} $ В }
  \begin{tabular}{| m{1.5cm} | m{1.5cm} | m{1.5cm} | m{1.5cm} | m{1.5cm} | m{1.5cm} | m{1.5cm}l |}
    \hline
    \centering $ U_{\text{СИ}} $, В & & & & & & &\\
    \hline
    \centering $ I_c $, мА & & & & & & &\\
    \hline
  \end{tabular}
\end{table}

\begin{table} [h!]
  \caption{ Измерения при $ U_{\text{ЗИ}} =  3,5 * U_{\text{ЗИ}_{\text{пор}}} = \hspace{8mm} $ В }
  \begin{tabular}{| m{1.5cm} | m{1.5cm} | m{1.5cm} | m{1.5cm} | m{1.5cm} | m{1.5cm} | m{1.5cm}l |}
    \hline
    \centering $ U_{\text{СИ}} $, В & & & & & & &\\
    \hline
    \centering $ I_c $, мА & & & & & & &\\
    \hline
  \end{tabular}
\end{table}

\vspace{50mm}

\begin{figure}[h!]
  \begin{minipage}[h]{0.5\linewidth}
    \caption{График зависимости $ I(U_{\text{ЗИ}}) $ }
  \end{minipage}
  \hfill   
  \begin{minipage}[h]{0.5\linewidth}
    \caption{График зависимости $ I(U_{\text{CИ}}) $ }
  \end{minipage}   
\end{figure}


\end{enumerate}

\newpage

\item{Расчет дифференциальных параметров.}

\begin{enumerate}

\item
Используемые формулы:
\begin{equation*}
  \begin{aligned}
    S &= \dfrac{dI_c}{dU_{\text{ЗИ}}} \Bigg|_{U_{\text{СИ} = const}} - \text{крутизна}
    \\
    \\
    R_i &= \dfrac{dU_{\text{СИ}}}{dI_c} \Bigg|_{U_{\text{ЗИ} = const}} - \text{внутреннее (дифференциальное) сопротивление}
    \\
    \\
    \mu &= \dfrac{dU_{\text{СИ}}}{dU_{\text{ЗИ}}} \Bigg|_{I_c = const} - \text{статистический коэффициент усиления}
  \end{aligned}
\end{equation*}


\item
Расчет дифференциальных параметров ПТ с управляющим p-n-переходом в рабочей точке $ U_{\text{СИ}} = 5 $ В и $ U_{\text{ЗИ}} = 0,3 * U_{\text{ЗИ}_{\text{отс}}} = \hspace{8mm} $ В.

\vspace{80mm}

\item
Расчет дифференциальных параметров МДП-транзистора в рабочей точке $ U_{\text{СИ}} = 5 $ В и $ U_{\text{ЗИ}} = 2,5 * U_{\text{ЗИ}_{\text{пор}}} = \hspace{8mm} $ В.

\end{enumerate}

\end{enumerate}

\newpage
