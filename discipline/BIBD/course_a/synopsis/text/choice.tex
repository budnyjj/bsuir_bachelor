\section[Выбор и обоснование средств реализации]{ВЫБОР И ОБОСНОВАНИЕ \\ СРЕДСТВ РЕАЛИЗАЦИИ}

В данном разделе рассмотрены различные аспекты реализации разрабатываемого интернет-магазина.

\subsection{Основные определения}
\label{sub:choice_theory}

Приведём основные определения терминов, встречающихся в этом разделе.

\textit{Язык программирования} --- формальная знаковая система, предназначенная для записи команд.

\textit{Веб-программирование} --- раздел программирования, ориентиро-ванный на разработку веб-приложений.

Языки веб-программирования можно разделить на две группы: \textit{клиентские}
и \textit{серверные}. К серверным языкам программирования можно
отнести следующие: PHP, C\#, Java, Groovy, Scala, Ruby, Python.

\textit{Клиент-серверная архитектура} --- способ взаимодействия программных компонентов,
при котором они образуют единую систему. Существует некий \textit{клиентский процесс},
требующий определенных ресурсов, а также \textit{серверный процесс}, который
эти ресурсы предоставляет. На практике принято размещать сервер на одном узле локальной сети,
а клиенты --- на других узлах~\cite{konnolli03}.

\textit{Веб-сервер} --- сервер, принимающий HTTP-запросы от клиентов, и выдающий им
HTTP-ответы, как правило, вместе с HTML-страницей, файлами, изображениями или другими данными.

\subsection{Обзор серверных языков программирования}
\label{sub:choice_server_language}

Рассмотрим наиболее популярные на сегодняшний день серверные языки программирования.

\paragraph{}
\textit{PHP (PHP: Hypertext Preprocessor)} --- скриптовый язык про-
граммирования общего назначения, часто применяемый для разработки веб-приложений.
В настоящее время PHP поддерживается большинством хостинг-провайдеров
и является одним из лидеров среди языков программирования, применяющихся для создания
динамических веб-сайтов.

В отличие от подавляющего большинства остальных языков программирования,
PHP не был изначально задуман как полноценный язык программирования,
а постепенно вырос из шаблонизатора для Perl. Это привело как к появлению в
первоначальном варианте языка некоторых спорных решений, так и
к отсутствию контроля со стороны создателя языка, который бы следил за
стройностью его архитектуры.

Стоит отметить, что язык разрабатывается группой энтузиастов в рамках проекта
с открытым исходным кодом. Проект распространяется под собственной лицензией,
несовместимой с GNU GPL.

К недостаткам языка можно отнести перегруженность языка различными функциями,
а также специфический синтаксис.

\paragraph{}
\textit{C\#} --- объектно-ориентированный язык с C-подобным синтаксисом, разработанный
группой инженеров в компании Microsoft Получил распространение в качестве
языка веб-программирования благодаря технологии .NET Framework.
В последнее время активно развивается технология
\textit{ASP.NET (Active Server Pages .NET)}.

Технология ASP.NET имеет два существенных недостатка: отсутствие кроссплатформенности и
платная лицензия.

\paragraph{}
\textit{Java} --- объектно-ориентированный язык программирования, разработанный
компанией Sun Microsystems, особенностью которого является то, что все программы,
написанные на Java транслируются в байт-код, выполняемый виртуальной машиной Java (JVM).
Таким образом достигается практически полная независимость от платформы. То есть
программа, написанная на Java, может выполнятся везде где возможен запуск JVM.

Стоит отметить, что в рамках концепции Java существует несколько семейств технологий:
\begin{itemize}
  \item \textit{Java ME (Java Micro Edition)} --- создана для использования в устройствах,
    имеющих ограничения по вычислительной мощности, например в КПК, встроенных системах;
  \item \textit{Java SE (Java Standard Edition)} содержит в себе компиляторы, API, Java Runtime
    Environment, подходит для создания настольных приложений;
  \item \textit{Java EE (Java Enterprise Edition)} представляет собой набор спецификаций для
    создания программного обеспечения уровня предприятия;
  \item \textit{JavaFX} --- технология, предназначенная для создания графических интерфейсов
    корпоративных приложения и бизнеса;
  \item \textit{Java Card} --- технология предоставляет безопасную среду для приложений, работающих
    на смарт-картах и других устройствах с ограниченным набором памяти.
\end{itemize}

К недостаткам языка можно отнести невысокую скорость работы, а также
сравнительно большой объём исходного кода.

\paragraph{}
\textit{Groovy} --- объектно-ориентированный язык программирования, разработанный для
платформы Java как дополнение к данному языку с возможностями таких языков, как Python,
Ruby и Smalltalk. Groovy использует Java-подобный синтаксис и работает напрямую с другим
Java-кодом и библиотеками.

\paragraph{}
\textit{Scala} --- мультипарадигмальный язык программирования, спроектированный кратким
и типобезопасным для простого и быстрого создания компонентного программного обеспечения,
сочетающий в себе возможности функционального и объектно-ориентированного программирования.
Scala, также как и Groovy, свободно взаимодействует с Java-кодом.

Языки Scala и Groovy являются относительно молодыми среди рассматриваемых языков.
Недостаточная <<зрелость>> языков программирования как правило сказывается на количестве
документации.

\paragraph{}
\textit{Ruby} --- высокоуровневый язык программирования для быстрого и удобного
объектно-ориентированного программирования. Основное назначение Ruby --- создание
простых и в то же время понятных программ, где важна не скорость работы программы,
а малое время разработки, понятность и простота синтаксиса.

Язык следует принципу <<наименьшей неожиданности>>: программа должна вести себя так,
как ожидает программист. Однако следует отметить, что в контексте Ruby
это означает наименьшее удивление не при знакомстве с языком, а при его основательном изучении.
Для разработки веб-приложений с помощью языка Ruby используется
фреймворк \textit{Ruby on Rails}.

К недостаткам языка можно отнести медленную скорость работы.

\paragraph{}
\textit{Python} --- высокоуровневый язык программирования общего назначения, ориентированный
на повышение производительности разработчика и читаемости кода. Python поддерживает
несколько парадигм программирования, в том числе структурное, объектно-ориентированное,
функциональное, императивное и аспектно-ориентированное.

Эталонной реализацией Python является интерпретатор CPython, поддерживающий большинство
активно используемых платформ. Он распространяется под свободной
лицензией Python Software Foundation License, позволяющей использовать его без ограничений
в любых приложениях, включая проприетарные.

Для разработки веб-приложений с использованием языка Python как правило используется
фреймворк \textit{Django}.

\subsection{Обзор клиентских языков программирования}
\label{sub:choice_client_language}

Рассмотрим наиболее популярные языки программирования, используемые на стороне клиента.

\paragraph{}
\textit{JavaScript} --- сценарийный язык программирования, обычно используется как
встраиваемый язык для программного доступа к объектам приложений. Наиболее широкое
распространение получил в браузерах как язык сценариев для придания интерактивности
веб-страницам.

На JavaScript оказали влияние многие языки, при разработке была цель сделать язык
похожим на Java, но при этом лёгким для использования непрограммистами.
Языком JavaScript не владеет какая-либо компания или организация, однако
название <<JavaScript>> является зарегистрированным товарным знаком компании Oracle Corporation.

\paragraph{}
\textit{CoffeScript} --- транслируемый в JavaScript язык программиро-вания. CoffeScript
добавляет синтаксический сахар в духе языков Python, Ruby, Haskell и Erlang для того,
чтобы улучшить читаемость кода и уменьшить его размер.

\paragraph{}
\textit{Dart} --- язык программирования, созданный корпорацией Google. Dart позиционируется в качестве
альтернативы JavaScript, страдающего от <<фундаментальных>> изъянов, которые невозможно
исправить путём эволюционного развития. В настоящее время существует два способа исполнения
Dart-программ: с использованием виртуальной машины или с промежуточной трансляцией в JavaScript.

\paragraph{}
\textit{VBScript} --- объектно-ориентированный скриптовый язык программирования, широко
используемый при создании скриптов в операционных системах семейства Microsoft Windows.

Сценарий не компилируется, а интерпретируется. То есть для обработки скрипта в системе
должен присутствовать интерпретатор языка VBS, и таких интерпретаторов в Windows два:
оконный WScript и консольный CScript.

\subsection{Обзор СУБД}
\label{sub:choice_DBMS}

Исходя из выбранной реляционной модели данных для проекта, следует обратить
внимание на следующие СУБД: MySQL, MS SQL Server, PosgreSQL, Microsoft Access.

\paragraph{}
\textit{MySQL} --- свободная реляционная система управления базами данных, являющаяся
решением для малых и средних приложений.
Разработку и поддержку MySQL осуществляет корпорация Oracle, получившие права на
торговую марку вместе с поглощённой Sun Microsystems.
Сообществом разработчиков MySQL созданы различные ветвления кода, такие как Drizzle,
OurDelta и MariaDB. MySQL имеет API для языков C, C++, Delphi, Java, Perl, Lisp,
PHP, Python, Ruby, Smalltalk, имеет библиотеки для языков платформы .NET и некоторых других.

\paragraph{}
\textit{MS SQL Server} --- система управления реляционными базами данных, разработанная
компанией Microsoft. В качестве языка запросов используется язык Transact-SQL.
Microsoft SQL Server используется для работы с базами данных размером от персональных до
крупных баз данных масштаба предприятия.
MS SQL Server поддерживает следующие языки программирования: языки на платформе .NET, а также
Microsoft Visual C++.

\paragraph{}
\textit{PostgreSQL} --- свободная объектно-реляционная система управления базами
данных. Сильными сторонами PosgreSQL считается поддержка БД практически неограниченного размера,
мощные механизмы транзакций, расширяемая система встроенных языков программирования и
лёгкая расширяемость.

\paragraph{}
\textit{Microsoft Access} --- реляционная система управления базами данных компании Microsoft.
Особенностью MS Access является встроенный язык программирования VBA, позволяющий создавать
приложения, работающие с базой данных.

Основными недостатками Microsoft Access является платная лицензия и работа исключительно в
операционной системе от Microsoft.

\subsection{Обзор веб-серверов}
\label{sub:choice_server}

\paragraph{}
\textit{Apache} --- кроссплатформенный, свободный веб-сервер. Основными достоинствами
Apache считаются надёжность и гибкость конфигурации. Он позволяет подключать
внешние модули для предоставления данных, использовать СУБД для аутентификации пользователей,
модифицировать сообщения об ошибках и~т.~д. Поддерживает версию IP протокола IPv6.

Поддержка языков программирования реализована путём установки дополнительных модулей.
В настоящее время существуют модули для языков PHP, Python, Ruby, Perl, ASP и некоторых других.

\paragraph{}
\textit{Tomcat (Apache Tomcat)} --- контейнер сервлетов с открытым исходным кодом,
разрабатываемый Apache Software Foundation. Реализует спецификацию JSP и JSF. Написан
на языке Java.

Tomcat используется в качестве самостоятельного веб-сервера, в качестве сервера контента
в сочетании с веб-сервером Apache HTTP Server, а также в качестве контейнера сервлетов
в серверах приложений JBoss и GlassFish.

\paragraph{}
\textit{Nginx} --- веб-сервер и почтовый прокси-сервер, работающий на Unix-подобных системах.
Основными характеристиками веб-сервера nginx являются простота, скорость работы и надёжность.
Применение nginx целесообразно прежде всего для статических веб-сайтов и как прокси-сервера перед
динамическими сайтами.

\paragraph{}
\textit{Jetty} --- свободный контейнер сервлетов, написанный на Java. Может использоваться
как HTTP-сервер или в паре со специализированным HTTP-сервером (например, Apache HTTP Server).

\paragraph{}
\textit{GlassFish} --- кроссплатформенный сервер приложений с открытым исходным кодом,
реализующий спецификации Java EE. В качестве сервлет-контейнера в GlassFish используется
модифицированный Apache Tomcat, дополненный компонентами Grizzly.


\newpage
\subsection{Выбор средств реализации}
\label{sub:choice_results}

\pagebreak
