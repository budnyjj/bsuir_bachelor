\section[Анализ предметной области]{АНАЛИЗ ПРЕДМЕТНОЙ ОБЛАСТИ}

\subsection{Определение понятия <<открытые данные>>}

\textit{Открытые данные} --- концепция, отражающая идею о том, что определённые данные
должны быть свободно доступны для машиночитаемого использования и дальнейшей републикации без
ограничений авторского права, патентов и других механизмов контроля~\cite{wiki_opendata}.

Несмотря на то, что идея открытых данных не является новой, строгое определение понятия
<<открытые данные>> было сформулировано сравнительно недавно.

\textit{Открытые данные} --- это информация, которую кто угодно может свободно использовать и распространять.
Допустимы лишь требования указывать источник данных и распространять их на тех же условиях,
что и исходные.

Открытые данные характеризуется следующими признаками:
\begin{itemize}

\item
Доступность и читаемость: данные должны быть доступны целиком не дороже
разумной стоимости их воспроизведения; желательно через интернет.
Формат данных должен быть удобным для чтения и изменения.

\item
Повторное использование и распространение: данные должны предоставляться на условиях,
которые разрешают их повторное использование и распространение,
в том числе --- в комбинации с другими наборами данных.

\item
Всеобщее участие: каждый должен иметь возможность использовать и распространять данные.
Не должно быть дискриминации областей применения, людей или групп.
Например, ограничение «только для некоммерческого использования»,
которое запрещает «коммерческое» применение, или ограничение возможных областей применения
(к примеру, только в образовании), недопустимы.

\end{itemize}

Благодарю чёткому определению <<открытости>> есть возможность комбинировать данные
из различных источников~\cite{opendatahandbook_open_data}.

Доступ к данным, как и последующее их использование, контролируется
организациями --- государственными и частными. 
Контроль может производиться через ограничения, лицензии, копирайт, патенты и требования оплаты
для доступа или повторного использования.

\textit{Авторское право} --- это юридический термин, используемый для описания прав,
которыми обладают авторы на свои литературные и художественные произведения~\cite{wipo_copyright}. 

Освободить данные от ограничений авторского права можно с помощью свободных
лицензий~\cite{wiki_opendata}.


\subsection{Типы cвободных лицензий}

\textit{Свободная лицензия} --- такой лицензионный договор,
 условия которого содержат разрешения пользователю от правообладателя
на конкретный перечень способов использования его произведения, которые дают ему четыре важнейшие свободы:

\begin{itemize}
\item  
  свобода использовать произведение в любых целях, изучать его (в случае ПО требуется доступность исходного кода);
\item
  свобода создавать и распространять копии произведения;
\item
  свобода вносить в произведение изменения;
\item
  свобода публиковать и распространять такие изменённые производные произведения (в случае ПО требуется доступность
  исходного кода и возможность внесения в него изменений)~\cite{wiki_free_license}.
\end{itemize}

\textit{Копилефт} --- это обобщенный метод сделать продукт
(программа, текст, фото- или видеоматериал) свободным и потребовать,
чтобы все последующие измененные и дополненные версии этого продукта также оставались свободными~\cite{gnu_copyleft}.

Различают пермиссивные и копилефтные свободные лицензии, а также договоры, связанные с общественным достоянием.

\textit{Пермиссивные лицензии на свободное ПО} --- лицензии на программное обеспечение,
которые практически не ограничивают свободу действий пользователей ПО и разработчиков,
работающих с исходным кодом.
В частности, пермиссивные лицензии сами по себе не ограничивают выбор лицензии для работ,
производных от работы с пермиссивной лицензией. Следовательно, пермиссивные лицензии не являются копилефтом.

Наиболее распространенные пермиссивные лицензии:
\begin{itemize}
\item лицензия BSD;
\item лицензия MIT;
\item Mozilla Public License;
\item Creative Commons Attribution 2.0.
\end{itemize}

\textit{Копилефтной лицензией} называется лицензия, использующая копилефт.
Наиболее распространенные копилефтные лицензии:
\begin{itemize}
\item универсальная общественная лицензия GNU;
\item лицензия свободной документации GNU;
\item Creative Commons Attribution-Sharealike 2.0.
\end{itemize}

\subsection{Мировое движение <<за открытые данные>>}

Рассказать про открытые данные, преставляемые государственными в мире и
примеры их использования.

\subsection{Открытые данные в Беларуси}

Рассказать про положение дел в Беларуси, про источники данных, про отсутствие поддержки
со стороны государства.

\subsection{Сервис представления информации о государственных
  наградах как реализация концепции открытых данных}


