\section[Анализ предметной области]{АНАЛИЗ ПРЕДМЕТНОЙ ОБЛАСТИ}

\subsection{Этапы проектирования баз данных}

В базе данных (БД) отражается информация об определенной предметной
области. Под предметной областью будем понимать часть реального мира, дан-
ные о которой должны храниться в проектируемой БД.

В процессе проектирования баз данных выделяют три основных этапа.

Концептуальное (инфологическое) проектирование – анализ пред-
метной области и ее описание. Этот этап осуществляется без ориентации на ка-
кие-либо конкретные программные или технические средства. Результатом
данного этапа является формализованная модель предметной области, постро-
енная с использованием специальных языковых средств (обычно – графиче-
ских). Инфологическая модель, как правило, включает следующие основные
элементы:
− описание объектов предметной области;
− описание атрибутов (свойств) объектов;
− описание связей между объектами.
Пусть, например, проектируется база данных о промышленном предпри-
ятии. Объектами предметной области в данном случае могут являться работни-
ки предприятия, его подразделения, другие предприятия – поставщики сырья
или потребители продукции, заключенные договоры, виды выпускаемой про-
дукции, конкретные выпущенные изделия и т.д. Эти объекты имеют свойства
(атрибуты) – фамилия, адрес и профессия работника, стоимость изделия, объем
поставок по договору и т.д. Наконец, объекты находятся в различных связях
друг с другом: работник работает в некотором подразделении, подразделения
выпускают продукцию, продукция поставляется по договорам другим предпри-
ятиям и т.д.
4Для описания объектов предметной области, их атрибутов и связей между
ними обычно применяются стандартизированные системы графических обо-
значений. Чаще всего применяются ER-модели (ER-диаграммы), подробно рас-
сматриваемые в разделе 2, и семантические объектные модели (COM-модели),
рассматриваемые в [2].
Кроме того, инфологическая модель может включать:
− описание основных запросов к проектируемой БД;
− описание документооборота, т.е. документов, используемых в качестве
источников данных для БД или составляемых на основе БД;
− описание алгоритмических связей между данными (например, алгорит-
мы и формулы для вычисления каких-либо величин, хранящихся в БД или оп-
ределяемых на основе БД);
− описание ограничений целостности, т.е. правил, обеспечивающих акту-
альность и непротиворечивость данных. Ограничения целостности представ-
ляют собой требования к допустимым значениям данных и связям между ними.

Даталогическое (логическое) проектирование – описание логиче-
ской структуры данных средствами системы управления базами данных
(СУБД), для которой проектируется БД. Такое описание (даталогическая мо-
дель) строится на основе инфологической модели по определенным правилам.
Для реляционных БД (т.е. БД, где данные представлены в виде таблиц) датало-
гическая модель включает:
− описание таблиц;
− описание связей между таблицами;
− описание атрибутов.

Физическое проектирование – описание физической структуры БД,
т.е. ее размещения на запоминающем устройстве. Такое описание называется
физической моделью. Физическая модель включает:
− тип носителя;
− способы организации данных;
− способы управления свободной памятью;
− способы сжатия данных и т.д.
Этот этап, как правило, в основным скрыт от проектировщика БД, так как
реализуется средствами СУБД.

\subsection{Нормальные формы}

БД, полученной по результатам даталогического проектиро-
вания, должна выполняться проверка на соблюдение свойств нормализации.

Нормализация обеспечивает:
− минимальную избыточность данных (все данные хранятся один раз);
− минимальный физический объем данных;
− ускорение доступа к данным;
− сокращение риска потери данных или их искажения при внесении изме-
нений в БД.
Нормализация выполняется для каждой таблицы, входящей в БД. Выделя-
ется несколько уровней нормализации (нормальных форм): первая, вторая, тре-
тья и т.д. Нормализация каждой таблицы выполняется поэтапно: сначала таб-
лица приводится к первой нормальной форме, затем – ко второй, и. т.д. При пе-
реходе к каждой последующей нормальной форме устраняются недостатки
(аномалии), присущие предыдущей нормальной форме.

Таблица находится в первой нормальной форме (1НФ), если все содер-
жащиеся в ней атрибуты имеют атомарные (одиночные) значения.
Для приведения к 1НФ строки и/или столбцы, содержащие неатомарные
атрибуты, разбиваются на несколько строк и/или столбцов.

Таблица находится во второй нормальной форме (2НФ), если она нахо-
дится в 1НФ, и все ее неключевые атрибуты находятся в полной функциональ-
ной зависимости от ключа.

Атрибут B находится в полной функциональной зависимости от группы
атрибутов A, если он находится в функциональной зависимости от этой группы
атрибутов, но не находится в функциональной зависимости ни от какой из час-
тей этой группы (т.е. ни от каких из атрибутов, входящих в группу атрибу-
тов A).

Таблица приводится к 2НФ в следующем порядке:
− выделяются все атрибуты, находящиеся в полной функциональной зави-
симости от ключа. Эти атрибуты, вместе с ключом, выделяются в отдельную
таблицу;
− выделяются атрибуты, находящиеся в функциональной зависимости от
какой-либо из частей ключа. Эти атрибуты вместе с той частью ключа, от кото-
рой они зависят, выделяются в отдельную таблицу. Такое действие выполняет-
ся для всех неполных функциональных зависимостей, имеющихся в таблице.

Таблица находится в третьей нормальной форме (3НФ), если она нахо-
дится в 2НФ (а значит, и в 1НФ) и не содержит транзитивных зависимостей.

Пусть атрибут B находится в функциональной зависимости от атрибута
или группы атрибутов A (A → B), а атрибут С – в функциональной зависимости
от атрибута B (B → C), причем A - ключ, B и C – неключевые атрибуты. Такая
зависимость (A → B → C) называется транзитивной.

Чтобы привести к 3НФ таблицу, содержащую транзитивную зависимость
A → B → C, требуется выделить атрибуты B и C в отдельную таблицу, где B
становится ключевым атрибутом. Другими словами, для приведения к 3НФ
таблица разбивается на две: в одну из них помещаются все атрибуты, кроме ат-
рибута C, в другую - атрибуты B и C.

Таблица находится в нормальной форме Бойса-Кодда (НФБК), если она
находится в 3НФ, и при этом каждый детерминант в таблице является возмож-
ным ключом.

Атрибут (или группа атрибутов) называется детерминантом, если от него
функционально зависит какой-либо другой атрибут. Другими словами, если
имеется функциональная зависимость A → B, то A – детерминант.
Атрибут (или группа атрибутов) называется возможным ключом, если он
может использоваться в качестве ключа.

Чтобы привести таблицу к НФБК, требуется выделить детерминант, не яв-
ляющийся возможным ключом, и атрибут, находящийся в функциональной за-
висимости от него, в отдельную таблицу. Другими словами, если в таблице
имеется функциональная зависимость A → B, где A не является возможным
ключом, то для приведения к НФБК таблица разбивается на две: в одну из них
помещаются все атрибуты, кроме атрибута B, в другую - атрибуты A и B (атри-
бут A в ней становится ключом).

\subsection{Физическое проектирование – описание физической структуры БД,
т.е. ее размещения на запоминающем устройстве. Такое описание называется
физической моделью. Физическая модель включает:
− тип носителя;
− способы организации данных;
− способы управления свободной памятью;
− способы сжатия данных и т.д.
Этот этап, как правило, в основным скрыт от проектировщика БД, так как
реализуется средствами СУБД.

\subsection{Нормальные формы}

БД, полученной по результатам даталогического проектиро-
вания, должна выполняться проверка на соблюдение свойств нормализации.

Нормализация обеспечивает:
− минимальную избыточность данных (все данные хранятся один раз);
− минимальный физический объем данных;
− ускорение доступа к данным;
− сокращение риска потери данных или их искажения при внесении изме-
нений в БД.
Нормализация выполняется для каждой таблицы, входящей в БД. Выделя-
ется несколько уровней нормализации (нормальных форм): первая, вторая, тре-
тья и т.д. Нормализация каждой таблицы выполняется поэтапно: сначала таб-
лица приводится к первой нормальной форме, затем – ко второй, и. т.д. При пе-
реходе к каждой последующей нормальной форме устраняются недостатки
(аномалии), присущие предыдущей нормальной форме.

Таблица находится в первой нормальной форме (1НФ), если все содер-
жащиеся в ней атрибуты имеют атомарные (одиночные) значения.
Для приведения к 1НФ строки и/или столбцы, содержащие неатомарные
атрибуты, разбиваются на несколько строк и/или столбцов.

Таблица находится во второй нормальной форме (2НФ), если она нахо-
дится в 1НФ, и все ее неключевые атрибуты находятся в полной функциональ-
ной зависимости от ключа.

Атрибут B находится в полной функциональной зависимости от группы
атрибутов A, если он находится в функциональной зависимости от этой группы
атрибутов, но не находится в функциональной зависимости ни от какой из час-
тей этой группы (т.е. ни от каких из атрибутов, входящих в группу атрибу-
тов A).

Таблица приводится к 2НФ в следующем порядке:
− выделяются все атрибуты, находящиеся в полной функциональной зави-
симости от ключа. Эти атрибуты, вместе с ключом, выделяются в отдельную
таблицу;
− выделяются атрибуты, находящиеся в функциональной зависимости от
какой-либо из частей ключа. Эти атрибуты вместе с той частью ключа, от кото-
рой они зависят, выделяются в отдельную таблицу. Такое действие выполняет-
ся для всех неполных функциональных зависимостей, имеющихся в таблице.

Таблица находится в третьей нормальной форме (3НФ), если она нахо-
дится в 2НФ (а значит, и в 1НФ) и не содержит транзитивных зависимостей.

Пусть атрибут B находится в функциональной зависимости от атрибута
или группы атрибутов A (A → B), а атрибут С – в функциональной зависимости
от атрибута B (B → C), причем A - ключ, B и C – неключевые атрибуты. Такая
зависимость (A → B → C) называется транзитивной.

Чтобы привести к 3НФ таблицу, содержащую транзитивную зависимость
A → B → C, требуется выделить атрибуты B и C в отдельную таблицу, где B
становится ключевым атрибутом. Другими словами, для приведения к 3НФ
таблица разбивается на две: в одну из них помещаются все атрибуты, кроме ат-
рибута C, в другую - атрибуты B и C.

Таблица находится в нормальной форме Бойса-Кодда (НФБК), если она
находится в 3НФ, и при этом каждый детерминант в таблице является возмож-
ным ключом.

Атрибут (или группа атрибутов) называется детерминантом, если от него
функционально зависит какой-либо другой атрибут. Другими словами, если
имеется функциональная зависимость A → B, то A – детерминант.
Атрибут (или группа атрибутов) называется возможным ключом, если он
может использоваться в качестве ключа.

Чтобы привести таблицу к НФБК, требуется выделить детерминант, не яв-
ляющийся возможным ключом, и атрибут, находящийся в функциональной за-
висимости от него, в отдельную таблицу. Другими словами, если в таблице
имеется функциональная зависимость A → B, где A не является возможным
ключом, то для приведения к НФБК таблица разбивается на две: в одну из них
помещаются все атрибуты, кроме атрибута B, в другую - атрибуты A и B (атри-
бут A в ней становится ключом).

\subsection{ПРОЕКТИРОВАНИЕ БАЗ ДАННЫХ НА ОСНОВЕ ER-МОДЕЛЕЙ}

ER-модель, или ER-диаграмма (Entity – Relation; в русском переводе - мо-
дель “объект - отношение” или “сущность - связь”) предназначена для форма-
лизованного описания предметной области на этапе инфологического проекти-
рования БД. Модель представляет собой графическое описание предметной об-
ласти с использованием стандартизированного набора обозначений. На основе
ER-модели по определенным правилам строится даталогическая модель для
реализации в конкретной СУБД (для реляционных БД – набор таблиц и связей
между ними). БД, построенная на основе ER-модели, обычно (но не всегда) на-
ходится в 4НФ. Тем не менее, необходимо проверять построенную БД на со-
блюдение правил нормализации.
Основные достоинства ER-моделей:
− наглядность;
− удобство для проектирования БД с большим количеством объектов
(сущностей) и атрибутов;
− широкое применение в CASE-системах - системах автоматизированного
проектирования БД (ERWin, Design/IDEF, Prokit*WORKBENCH и т.д.).
Основные элементы, входящие в состав ER-моделей:
− объекты (сущности);
− атрибуты (свойства) объектов;
− связи между объектами.

Имеются различные виды атрибутов объектов. Для них в ER-моделях ис-
пользуются разные обозначения, и они по-разному влияют на структуру созда-
ваемой БД. На рисунке 2.1 показаны основные виды атрибутов:
− единичные и множественные атрибуты. Если объект может обладать
только одним (точнее, не более чем одним) значением атрибута, то атрибут яв-
ляется единичным. Если объект может иметь одновременно несколько значе-
ний атрибута, то атрибут – множественный. Единичные атрибуты обозначаются
одинарной стрелкой, множественные – двойной.

Как будет
показано ниже, различие между единичными и множественными атрибутами
существенно влияет на дальнейший ход проектирования БД (в частности, на ее
даталогическую модель);
− безусловные (обязательные) и условные (необязательные) атрибу-
ты. Если атрибут имеет некоторое значение для всех объектов, то он является
безусловным, в противном случае – условным. Безусловные атрибуты обозна-
чаются сплошной линией, условные – пунктирной.

Остальные атрибуты - безусловные, так
как, очевидно, любой преподаватель имеет ФИО, год рождения, преподаваемые
курсы (хотя бы один) и т.д. Различие между безусловными и условными атри-
бутами в некоторых случаях влияет на дальнейшее проектирование БД;
− статические и динамические атрибуты. Атрибуты, значения которых,
как правило, не изменяются со временем, являются статическими, в противном
случае – динамическими. Статические атрибуты обозначаются буквой S, дина-
мические – буквой D.

Различие между статическими и ди-
намическими атрибутами обычно не влияет на проектирование структуры БД,
поэтому во многих случаях обозначения S и D в ER-моделях не указывают.

На основе построенной ER-модели (т.е. инфологической модели) датало-
гическая модель (набор таблиц) строится следующим образом:
− все единичные атрибуты сводятся в одну таблицу. Ключом в такой таб-
лице является атрибут-идентификатор;
− каждый множественный атрибут вместе с атрибутом-идентификатором
выделяется в отдельную таблицу. Ключ в таких таблицах составной и состоит
из обоих атрибутов (идентификатора и множественного атрибута).