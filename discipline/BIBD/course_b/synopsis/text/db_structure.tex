\section[Разработка структуры БД]{РАЗРАБОТКА СТРУКТУРЫ БАЗЫ ДАННЫХ}

\subsection{Нормальные формы}

Рассказать про аномалии, перечислить их виды.

Рассказать про нормальные формы как формализованные характерные 
признаки структуры БД.

Приведение структуры БД к НФ высших порядков позволяют устранить аномалии.

Перечислить определения НФ. Сослаться на тот факт, что ER-модели позволяют автоматически
получать БД в подходящей НФ.

\subsection{Этапы проектирования базы данных}

\subsection{Проектирование баз данных на основе ER-моделей}

\subsection{Структура базы данных сервиса nagrady.by}