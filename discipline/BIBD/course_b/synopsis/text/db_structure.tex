\section[Разработка структуры БД]{РАЗРАБОТКА СТРУКТУРЫ БАЗЫ ДАННЫХ}
\label{sub:db_structure}

База данных (БД) --- это некоторый набор постоянно хранимых данных,
используемых прикладными программными системами.

Задача проектирования базы данных в целом формулируется
следующим образом: выбрать подходящую логическую структуру для заданного
массива данных, которые требуется поместить в базу данных~\cite{date05}.

Перечислим основные цели проектирования баз данных:
\begin{itemize}
\item
  обеспечение хранения в БД всех необходимых данных;
\item
  обеспечение получения данных по всем необходимым запросам;
\item
  сокращение избыточности и дублирования данных;
\item
  обеспечение целостности данных: исключение потери данных, противоречий в
  содержании БД, нарушений смысла данных;
\item
  сокращение времени обработки запросов к данным.
\end{itemize}

В этом разделе рассматривается процесс проектирования базы данных разрабатываемого
веб-сервиса.
В подразделе~\ref{ssub:db_structure_stages} описываются основные этапы проектирования
баз данных в целом.
Подразделы~\ref{ssub:db_info_stage},~\ref{ssub:db_data_stage},~\ref{ssub:db_physical_stage}
подробно описывают каждый из этапов проектирования базы данных разрабатываемого веб-сервиса.

\subsection{Этапы проектирования базы данных}
\label{ssub:db_structure_stages}

В процессе проектирования баз данных выделяют три основных этапа:
\begin{itemize}
\item инфологическое проектирование;
\item даталогическое проектирование;
\item физическое проектирование;
\end{itemize}

Рассмотрим каждый из этих этапов более подробно.

\textit{Инфологическое (концептуальное) проектирование} --- анализ
предметной области и ее описание. Этот этап осуществляется без ориентации
на какие-либо конкретные программные или технические средства реализации.
Результатом инфологического проектирования является \textit{инфологическая модель} ---
формализованная модель предметной области, построенная с использованием
специальных языковых средств (обычно графических).

Инфологическая модель включает следующие основные элементы:
\begin{itemize}
\item
  описание объектов предметной области;
\item
  описание атрибутов (свойств) объектов;
\item
  описание связей между объектами.
\end{itemize}

Для описания объектов предметной области, их атрибутов и связей между
ними обычно применяются стандартизированные системы графических обозначений.
Кроме того, инфологическая модель может включать
описание основных запросов к проектируемой БД,
описание документов, используемых в качестве
источников данных для БД или составляемых на основе БД,
описание алгоритмических связей между данными,
описание ограничений целостности, т.е. правил, обеспечивающих
актуальность и непротиворечивость данных.

\textit{Даталогическое (логическое) проектирование} --- описание
логической структуры данных средствами системы управления базами данных
(СУБД), для которой проектируется БД.
Такое описание (\textit{даталогическая модель}) строится на основе инфологической модели
по определенным правилам.
Для реляционных БД даталогическая модель состоит из следующих частей:
\begin{itemize}
\item
  описания таблиц;
\item
  описания связей между таблицами;
\item
  описания атрибутов.
\end{itemize}

\textit{Физическое проектирование} --- на этом этапе выполняется описание физической
структуры БД, то есть ее размещения на запоминающем устройстве.
Такое описание называется \textit{физической моделью}, которая включает:

\begin{itemize}
\item
  тип носителя;
\item
  способы организации данных;
\item
  способы управления свободной памятью;
\item
  способы сжатия данных и т.д.
\end{itemize}

\pagebreak

\subsection{Инфологическое проектирование}
\label{ssub:db_info_stage}

Разрабатываемый сервис должен хранить и предоставлять в наглядной форме 
информацию о государственных наградах РБ и лицах, которые были ими награждены.

Проектируемый сервис должен по запросу пользователей выполнять следующие операции:
\begin{itemize}
\item
  предоставление доступа к хранимой информации;
\item
  показ графиков показателей на основании хранимой информации;
\item
  выгрузка информации о награжденных в машиночитаемом формате. 
\end{itemize}

Обзорная схема работы сервиса представлена на рисунке~\ref{fig:work_scheme}.

\begin{figure}[h!]
  \centering
  
  \caption{Обзорная схема работы проектируемого веб-сервиса}
  \label{fig:work_scheme}
\end{figure}

В качестве источника данных используются файлы в машиночитаемом формате,
которые получаются после разбора соответствующих файлов формата pdf
(portable document format),
предоставленных национальным статистическим комитетом Республики Беларусь.

Файлы-источники данных имеют формат csv (comma-separated values) c заранее определенным
набором cтолбцов, который не может подвергаться изменениям.
Рассмотрим наборы данных, предоставляемых этими файлами.

Файл с информацией о наградах состоит из следующих столбцов:
\begin{itemize}
\item
  award\_title --- название награды;
\item
  type --- тип награды;
\item
  description --- краткое описание награды;
\end{itemize}

Атрибут award\_title является уникальным, следовательно, его можно использовать
в качестве идентификатора награды.

Каждому типу награды может соответствовать несколько названий и описаний наград;
каждому названию или описанию --- единственный тип.

Содержимое полей атрибута description может отстутствовать.

Файл с информацией о награжденных состоит из:
\begin{itemize}
\item
  fio --- фамилия, имя, отчество награжденного;
\item
  dolzhnost --- занимаемая должность;
\item
  pol --- пол награжденного;
\item
  zvanie --- звание;
\item
  mesto\_raboty --- место работы;
\item 
  tip\_nagrady --- тип награды;
\item
  nagrada --- название награды;
\item
  osnovanie --- основание для награждения;
\item
  dokument --- документ, потдвержающий факт награждения;
\item
  nomer\_dokument --- номер официального документа;
\item
  data\_prinjatija --- дата вступления в силу документа о награждении;
\item
  RN --- регистрационный номер награды;
\end{itemize}

В исходном файле могут встречаться люди, которые являются полными тезками,
таким образом, атрибут fio не является идентификатором.

В качестве идентификатора можно использовать следующий набор атрибутов:
fio, nagrada, nomer\_document.

Каждому документу, подтверждающему факт награждения, может соответствовать одна
или несколько наград.

Содержимое полей pol, dolzhnost, zvanie, mesto\_raboty, osnovanie, RN может отсутствовать.

Исходя из специфики предметной области, а также состава входных файлов, можно выделить
следующие объекты предметной области: <<факт награжденния>> и <<награда>>.
Объект <<награжденный>> хранит информацию о факте награждения конкретного лица наградой.
Объект <<награда>> хранит информацию о конкретной награде.

% Между объектами <<факт награждения>> и <<награда>> имеется связь: каждый <<факт награждения>>
% cвязан с определенной <<наградой>> и каждая <<награда>> может быть связана с несколькими
% <<фактами награждения>>.

\subsection{Даталогическое проектирование}
\label{ssub:db_data_stage}

\subsection{Физическое проектирование}
\label{ssub:db_physical_stage}


% \subsection{Нормализация базы данных}
% \label{ssub:db_structure_forms}


% Переменная отношения находится в \textit{первой нормальной форме} (1НФ) тогда
% и только тогда, когда в любом допустимом значении этой переменной отношения каждый
% ее кортеж содержит только одно значение для каждого из атрибутов.

% Переменная отношения находится во \textit{второй нормальной форме} тогда и только тогда,
% когда она находится в первой нормальной форме и каждый неключевой атрибут
% неприводимо зависит от ее первичного ключа.

% Переменная отношения находится в \textit{третьей нормальной форме} тогда и только тогда,
% когда она находится во второй нормальной форме и ни один неключевой атрибут не является
% транзитивно зависимым от ее первичного ключа.

% Перечислить определения НФ. Сослаться на тот факт, что ER-модели позволяют автоматически
% получать БД в подходящей НФ.

% \subsection{Проектирование баз данных на основе ER-моделей}
% \label{ssub:db_structure_er_models}


% \subsection{Преобразование ER-моделей в структуру базы данных}
% \label{ssub:db_structure_convert_models_to_structure}


% \subsection{Проектирование базы данных сервиса nagrady.by}
% \label{ssec:db_structure_nagrady}

% В этом подразделе необходимо описать процесс проектирования БД:
% описание содержимого таблиц и связей между ними.