\section{Ход работы}
\addcontentsline{toc}{section}{Ход работы}	% Добавляем его в оглавление

\begin{enumerate}

\item{Произведем сборку схемы, как показано на рис.~\ref{sch1}}

\begin{multicols}{2}
\item{Схема зависимости $ U_R(t), I_R(t) $}
\vspace{50mm}

\item{Схема зависимости $ U_C(t) $}
\vspace{40mm}
\begin{center}
  $ a_1 = 16 $ мм, $ a_2 = 7 $ мм. 
\end{center}

\end{multicols}

\item{Установим влияние величин R, C на характер протекания переходного процесса}

При увеличении емкости конденсатора уменьшается амплитуда переходного процесса; изменение сопротивления приводит к изменению формы сигнала.

\item{Исследуем дифференцирующую цепь}
\begin{multicols}{2}
\begin{center}

$ C = 0,5 $ мкФ

$ T = 36 $ мс

\vspace{40mm}

$ \tau = \frac{5}{3} $ мс

$ \omega = 2 \pi f = 314 $ (Гц)

$ R = \dfrac{1}{11\omega C} = 579 $ (Ом)

\vspace{40mm}

$\tau = 6 $ мс

\end{center}
\end{multicols}

\item{Исследуем интегрирующую цепь}

\begin{multicols}{3}
\begin{center}

$ C = \dfrac{10}{\omega R} = 5,5 \cdot 10^{-6} $ Ф

$ R = 5800 $ Ом

$ C = 0,5 \cdot 10^{-6} $ Ф

$ R = 5800 $ Ом

$ C = 5 \cdot 10^{-6} $ Ф

$ R = 5800 $ Ом

\end{center}
\end{multicols}

\vspace{30mm}

\begin{multicols}{3}
\begin{center}

$ \tau = 5 $ мс

\end{center}
\end{multicols}



\end{enumerate}

\clearpage
