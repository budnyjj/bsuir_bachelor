\section[Расчёт производственной площади и планировка участка]
{РАСЧЁТ ПРОИЗВОДСТВЕННОЙ ПЛОЩАДИ И \\
ПЛАНИРОВКА УЧАСТКА}
\label{sec:placement}

\subsection{Планировка участка производства}

Поточная линия имеет П-образную форму во избежание перекрещивающихся и
обратных потоков материалов и предметов труда.
Планировка производственного участка расположена в приложении~Б.
Комплектующие изделия поступают на верстаки,
расположенные вдоль ленты конвейера,
проходят через указанные в таблице~\ref{tbl:assembly_process} стадии
технологического процесса и завершается на испытательном стенде,
который обозначен в правом верхнем углу плана.

\subsection{Расчёт производственной площади участка}

Расчёт производственной площади участка, занимаемой технологическим оборудованием
(рабочими местами) и транспортными средствами,
приведен в таблице~\ref{tbl:prod_placement}.
Расчёт общей площади участка приведен в таблице~\ref{tbl:common_placement}.

\begin{table} [h!]
  \caption{
    Расчёт производственной площади участка
  }\label{tbl:prod_placement}
    \begin{tabular}{
      | m{5cm} | c | c | c | c | c |}
      \hline
      \parbox{4.3cm}{\centering Наименование \\ оборудования}
      & \rotatebox[origin=c]{90}{\hspace{2mm}\parbox{4.5cm}{Модель (марка)}}
      & \rotatebox[origin=c]{90}{\hspace{2mm}\parbox{4.5cm}{Габаритные размеры, мм}}
      & \rotatebox[origin=c]{90}{\hspace{2mm}\parbox{4.5cm}{Количество единиц \\ оборудования (\( C_{\text{пр}} \))}}
      & \rotatebox[origin=c]{90}{\hspace{2mm}\parbox{4.5cm}{Коэффициент дополнительной площади (\( K_{\text{дп}} \))}}
      & \rotatebox[origin=c]{90}{\hspace{2mm}\parbox{4.5cm}{Производственная \\ площадь участка \\ (\( S \)), \( \text{м}^2 \)}} \\
      \hline

      Верстак             & НДР-1064  & 1200$\times$700  & 14 & 4,0 & 47,04 \\
      \hline

      Конвейер            & ЭП-201    & 10425$\times$400 & 1  & 2,0 & 8,34 \\
      \hline

      Установка <<Волна>> & --        & 1350$\times$1250 & 3  & 3,5 & 17,72 \\
      \hline

      Испытательный стенд & --        & 1400$\times$1300 & 1  & 3,5 & 6,37 \\
      \hline

      \textbf{Итого} & -- & -- & \textbf{17} & -- & \textbf{79,47} \\
      \hline
    \end{tabular}
\end{table}

\begin{table} [h!]
  \caption{
    Расчёт общей площади участка
  }\label{tbl:common_placement}
    \begin{tabular}{| m{5.8cm} | m{5.75cm} | c |}
      \hline
      \parbox{5.8cm}{
        \smallskip
        \centering Вид  площади
        \smallskip
      }
      & \parbox{5.75cm}{
          \smallskip
          \centering Источник \\ или методика расчёта
          \smallskip
      }
      & Площадь (\( S \)), \( \text{м}^2 \) \\
      \hline

      1. Производственная \newline площадь
      & \centering См.~таблицу~\ref{tbl:prod_placement}
      & 79,47 \\
      \hline

      2. Вспомогательная \newline площадь
      & \centering Принимаем 25\% \newline от производственной
      & 19,87 \\
      \hline


      \textbf{Итого} & \centering -- & \textbf{99,33} \\
      \hline
    \end{tabular}
\end{table}

\newpage

По результатам расчётов, общую площадь цеха примем равной
100 \( \text{м}^2 \).

\subsection{Обоснование выбора типа здания}

Продукция, производимая на данном промышленном предприятии,
отличается небольшой массой и относительно небольшим количеством средств труда.
 В связи с этим целесообразно размещать такое производство в многоэтажном здании;
допустимо размещение указанного цеха на любом этаже.

Помещение цеха имеет форму прямоугольника, габаритные размеры 8$\times$12,5 м.
% Ширина пролета принята равной \( L = 9 \: \text{м} \),
% т.~е. имеются два пролета, образованные железобетонными колоннами.
% Шаг колонн равен \( t = 6 \: \text{м} \), с учетом этого в каждом ряду находится
% по две колонны.
% Таким образом, для поддержания перекрытий цеха следует использовать шесть колонн.
Стены здания выполняются из железобетонных панелей высотой 1{,}2 м и
имеют высоту 3{,}6 м.
