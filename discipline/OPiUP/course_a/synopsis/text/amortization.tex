\section[Расчёт стоимости и амортизации основных производственных фондов]{%
  РАСЧЁТ СТОИМОСТИ И АМОРТИЗАЦИИ \\
  ОСНОВНЫХ ПРОИЗВОДСТВЕННЫХ ФОНДОВ
}
\label{sec:amortization}

\subsection[%
Расчёт стоимости здания, занимаемого участком производства]
{Расчёт стоимости здания, занимаемого участком производства}

Расчёт стоимости здания, занимаемого производственным участком,
и амортизационных отчислений приведен в таблице~\ref{tbl:placement_cost}.

\begin{table} [h!]
  \caption{
    Расчёт стоимости здания и амортизационных отчислений
  }\label{tbl:placement_cost}
    \begin{tabular}{| m{6.6cm} | c | c | c | c | c |}
      \hline
      \parbox{6.6cm}{
        \smallskip
        \centering Элементы расчета
        \smallskip
      }
      & \rotatebox[origin=c]{90}{
          \parbox{4.5cm}{
            Стоимость 1 \( \text{м}^2 \) \\ здания, у.~е./\( \text{м}^2 \)
          }
        }
      & \rotatebox[origin=c]{90}{
          \parbox{4.5cm}{
            Площадь, \\ занимаемая \\ зданием, \( \text{м}^2 \)
          }
        }
      & \rotatebox[origin=c]{90}{
          \parbox{4.5cm}{
            Стоимость \\ здания, у.~е.
          }
        }
      & \rotatebox[origin=c]{90}{
          \parbox{4.5cm}{
            Норма амортизации,~\%
          }
        }
      & \rotatebox[origin=c]{90}{
          \parbox{4.5cm}{
            Сумма амортизационных отчислений, у.~е.
          }
        } \\
      \hline

      1. Производственная площадь & 170 & 79,47 & 13~509,9
      & 2{,}7 & 364,8 \\
      \hline

      2. Вспомогательная площадь & 250 & 19,87 & 4~967,5
      & 3{,}1 & 154{,}0 \\
      \hline

      \textbf{Итого} & \textbf{--} & \textbf{99{,}33}
      & \textbf{18~477,4} & \textbf{--} & \textbf{518,8} \\
      \hline
    \end{tabular}
\end{table}

\vspace{-5mm}

\subsection{Расчёт затрат на оборудование и транспортные средства}

Расчёт затрат на оборудование и транспортные средства
приведен в таблице~\ref{tbl:tech_cost}. Величина затрат на упаковку,
транспортировку, монтаж и пусконаладочные работы принята равной 10\%
от цены оборудования.

Расчёт стоимости конвейера производится исходя из его рабочей длины ($ L_n $),
стоимости одного погонного метра пролётной части (31,77~у.е. для ширины
транспортного средства, равной 400~мм) и стоимости электродвигателя~(принимаем
35\% от стоимости конвейера):
\begin{align*}
  \text{ПС}_{\text{конв}} = (1 + 0{,}35) \cdot 31{,}77 \cdot 32{,}06 = 1~374{,}9~\text{у.е.}.
\end{align*}

\begin{table} [h!]
  \caption{
    Расчёт стоимости транспортного и технологического оборудования
  }\label{tbl:tech_cost}
  {\small
    \begin{tabular}{| m{2.2cm} | c | c | c | c | c | c | c | c |}
      \hline
      \multirow{2}{*}{
        \rotatebox[origin=c]{90}{
          \parbox{6cm}{
            Наименование технологического \\
            оборудования и \\
            транспортных средств
          }
        }
      }
      & \multirow{2}{*}{
          \rotatebox[origin=c]{90}{
            \parbox{6cm}{
              Модель (марка)
            }
          }
        }
      & \multirow{2}{*}{
          \rotatebox[origin=c]{90}{
            \parbox{6cm}{
              Количество единиц оборудова- \\
              ния, транспортных средств, шт.
            }
          }
        }
      & \multicolumn{2}{c|}{Оптовая цена}
      & \multirow{2}{*}{
          \rotatebox[origin=c]{90}{
            \parbox{6cm}{
              Затраты на упаковку, \\
              транспортировку, монтаж, \\
              пуск, наладку, у.~е.
            }
          }
        }
      & \multirow{2}{*}{
          \rotatebox[origin=c]{90}{
            \parbox{6cm}{
              Балансовая (первоначальная) \\
              стоимость техники, у.~е.
            }
          }
        }
      & \multirow{2}{*}{
          \rotatebox[origin=c]{90}{
            \parbox{6cm}{
              Норма амортизации, у.~е.
            }
          }
        }
      & \multirow{2}{*}{
          \rotatebox[origin=c]{90}{
            \parbox{6cm}{
              Сумма амортизационных \\
              отчислений, у.~е.
            }
          }
        } \\ \cline{4-5}

      & &
      & \rotatebox[origin=c]{90}{
          \parbox{5.3cm}{
            единицы, у.~е.
          }
        }
      & \rotatebox[origin=c]{90}{
          \parbox{5.3cm}{
            принятого кол-ва, у.~е.
          }
        }
      & & & & \\
      \hline

      1. Верстак & НДР-1064
      & 14
      & 360 & 5~040 & 504 & 5~544
      & 8,2 & 454,6 \\
      \hline

      2. Конвейер & --
      & 1
      & 1~374,9 & 1~374,9 & 137,5 & 1~512,4
      & 14,8 & 223,8 \\
      \hline

      3. Установка \newline <<Волна>> & --
      & 3
      & 510 & 1~530 & 153 & 1~683
      & 15,7 & 264,2 \\
      \hline

      4. Испыта- \newline тельный стенд & --
      & 1
      & 380 & 380 & 38 & 418
      & 8,1 & 33,9 \\
      \hline

      5. Паяльник & --
      & 1
      & 150 & 150 & 15 & 165
      & 14,3 & 23,6 \\
      \hline

      \textbf{Итого} & \textbf{--}
      & \textbf{20}
      & \textbf{--} & \textbf{8~474,9} & \textbf{847,5} & \textbf{9~322{,}4}
      & \textbf{--} & \textbf{1~000{,}1} \\
      \hline
    \end{tabular}
  }
\end{table}


\subsection{Расчёт затрат на энергетическое оборудование}

Величина затрат на силовое энергетическое оборудование, его монтаж,
упаковку, транспортировку, определяемая исходя из норматива 45 у.е.
на 1 кВт установленной мощности технологического и транспортного
оборудования, составляет:
\begin{equation*}
  \text{ПС}_{\text{э}} = 45 \cdot 56{,}62 = 2~547{,}9~(\text{у.е.}).
\end{equation*}


\subsection{Расчёт затрат на комплект дорогостоящей оснастки, УСПО и \\
инструмента}

Величина затрат на дорогостоящую оснастку, УСПО,
инструмент (первоначальный фонд), принимаемая в размере 10\% от
балансовой стоимости технологического оборудования, составляет:
\begin{equation*}
  \text{ПС}_{\text{ос}} = 0{,}1 \cdot 9~322{,}4 = 932{,}2 ~(\text{у.е.}).
\end{equation*}


\subsection{Расчёт затрат на измерительные и регулирующие приборы}

Величина затрат на измерительные и регулирующие приборы,
принимаемая в размере $1{,}5-2{,}0\%$ от
оптовой цены оборудования:
\begin{equation*}
  \text{ПС}_{\text{из}} = 0{,}0175 \cdot 8~474{,}9 = 148{,}3 ~(\text{у.е.}).
\end{equation*}


\subsection{Расчёт затрат на производственный и хозяйственный инвентарь}

Величина затрат на производственный инвентарь,
принимаемая в размере $1{,}5-2{,}0~\%$ от
стоимости технологического оборудования, а также
затрат на хозяйственный инвентарь
(по $15{,}4$~у.е. на одного работающего), составляет:
\begin{equation*}
  \text{ПС}_{\text{ин}} =
  0{,}0175 \cdot 9~322{,}4 + 15{,}4 \cdot 34 = 686{,}74~(\text{у.~е.}).
\end{equation*}

\subsection{Расчёт стоимости основных производственных фондов и \\
амортизационных отчислений}

Расчёт затрат на оборудование и транспортные средства
приведен в таблице~\ref{tbl:common_cost}.

\begin{table} [h!]
  \caption{
    Расчёт стоимости основных производственных фондов и
    амортизационных отчислений
  }\label{tbl:common_cost}
  {\small
    \begin{tabular}{| m{6.3cm} | c | c | c | c |}
      \hline
        \parbox{6.3cm}{
          \smallskip
          \centering Наименование групп основных производственных фондов
          \smallskip
        }
      & \parbox{1cm}{
          \smallskip
          \centering Усл. \\ обозн.
        }
      & \parbox{2.3cm}{
        \smallskip
          \centering Стоимость производ- ственных фондов, у.~е.
        }
      & \parbox{2.3cm}{
          \centering Норма амортизации, \%
        }
      & \parbox{2.4cm}{
          \centering Сумма аморт. отчислений, у.~е., в мес
        } \\
      \hline

      1. Здание, занимаемое участком
      & \( K_{\text{зд}} \)
      & 18~477,4 & Таблица~\ref{tbl:placement_cost} & 43,23 \\
      \hline

      2. Технологическое оборудование \newline и транспортные средства
      & \( K_{\text{об}} \)
      & 9~322,4 & Таблица~\ref{tbl:tech_cost} & 83,34 \\
      \hline

      3. Энергетическое оборудование
      & \( K_{\text{э}} \)
      & 2~547,9 & 8,2 & 17,41 \\
      \hline

      4. Дорогостоящая оснастка
      & \( K_{\text{ос}} \)
      & 932,2 & 4,5 & 3,49 \\
      \hline

      5. Измерительные \newline и регулирующие приборы
      & \( K_{\text{из}} \)
      & 148,3 & 11,5 & 1,42 \\
      \hline

      6. Производственный \newline и хозяйственный инвентарь
      & \( K_{\text{ин}} \)
      & 686,7 & 18,5 & 127,05 \\
      \hline

      \textbf{Итого}
      & \textbf{--}
      & \textbf{32~114,98} & \textbf{--} & \textbf{275,95} \\
      \hline
    \end{tabular}
  }
\end{table}