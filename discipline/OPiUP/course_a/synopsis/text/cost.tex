\section[%
Расчёт себестоимости и цены единицы продукции с учетом \\
косвенных налогов
]{%
РАСЧЁТ СЕБЕСТОИМОСТИ И ЦЕНЫ ЕДИНИЦЫ \\
ПРОДУКЦИИ С УЧЕТОМ КОСВЕННЫХ \\
НАЛОГОВ
}
\label{sec:cost}

Себестоимость единицы продукции --- это выраженная в денежной форме
сумма затрат на её производство и реализацию.
В качестве калькуляционной единицы примем программу выпуска изделий.

Расчёт затрат на материалы произведём в таблице~\ref{tbl:materials_cost}.
\begin{table} [h!]
  \caption{
    Расчёт затрат на материалы
  }\label{tbl:materials_cost}
  {\small
    \begin{tabular}{| m{2.8cm} | c | c | c | c | c |}
      \hline
      \parbox{2.8cm}{
          \smallskip
          \centering Наименование
          \smallskip
        }
      & \parbox{1.8cm}{
          \smallskip
          \centering Марка, \\ профиль
          \smallskip
        }
      & \parbox{1.8cm}{
          \smallskip
          \centering Единица измерения
          \smallskip
        }
      & \parbox{2.5cm}{
          \smallskip
          \centering Норма расхода на комплект
          \smallskip
        }
      & \parbox{2.5cm}{
          \smallskip
          \centering Оптовая цена за единицу, у.е.
          \smallskip
        }
      & \parbox{2.5cm}{
          \smallskip
          \centering Сумма затрат на изделие, у.е.
          \smallskip
        }
      \\
      \hline

      1.  Провод            & ПЭЛ-041 & м  & 0,0800 & 0,225 & 0,0180 \\ \hline
      2.  Провод            & ПЭЛ-031 & м  & 0,0900 & 0,225 & 0,0203 \\ \hline
      3.  Провод            & ПЭЛ-063 & м  & 0,1000 & 0,225 & 0,0225 \\ \hline
      4.  Бумага            & КТ-05   & кг & 0,0008 & 0,350 & 0,0003 \\ \hline
      5.  Бумага            & КТ-120  & кг & 0,0033 & 0,350 & 0,0012 \\ \hline
      6.  Нитки             & к/б     & кг & 5,0000 & 0,150 & 0,7500 \\ \hline
      7.  Труба             & ТЛВ-1   & м  & 1,5000 & 0,950 & 1,4250 \\ \hline
      8.  Припой            & ПОС-61  & кг & 0,0500 & 1,362 & 0,0681 \\ \hline
      9.  Канифоль          & -       & кг & 0,0200 & 0,320 & 0,0064 \\ \hline
      10. Флюс \newline спиртовой    & -       & кг & 0,0037 & 0,350 & 0,0013 \\ \hline

      \multicolumn{5}{|l|}{\textbf{Итого}} & \textbf{2,3130} \\ \hline
      \multicolumn{5}{|l|}{Транспортно-заготовительные расходы, $4\%$} & 0,0925 \\ \hline
      \multicolumn{5}{|l|}{\textbf{Итого с учётом транспортно-заготовительных расходов}} & \textbf{2,4055}  \\ \hline
      \multicolumn{5}{|l|}{Возвратные отходы, $1\%$} & 0,0241  \\ \hline
      \multicolumn{5}{|l|}{\textbf{Итого с учётом возвратных отходов}} & \textbf{2,3814}  \\ \hline

    \end{tabular}
  }
\end{table}

Таким образом, окончательный расчёт статьи затрат <<Сырьё, материалы и другие
материальные ценности за вычетом реализуемых отходов>>:
\begin{equation*}
  \text{Р}_\text{м} = 2{,}3814 \cdot 30~597 = 72~865{,}05~(\text{у.~е.}).
\end{equation*}

\newpage

\begin{table} [h!]
  \caption{
    Расчёт затрат на комплектующие изделия
  }\label{tbl:components_cost}
  {\small
    \begin{tabular}{| m{4.4cm} | c | c | c | c |}
      \hline
      \parbox{4.0cm}{
          \smallskip
          \centering Наименование
          \smallskip
        }
      & \parbox{1.8cm}{
          \smallskip
          \centering ГОСТ, \\ марка
          \smallskip
        }
      & \parbox{2.5cm}{
          \smallskip
          \centering Количество на комплект, шт.
          \smallskip
        }
      & \parbox{2.5cm}{
          \smallskip
          \centering Оптовая цена за единицу, у.е.
          \smallskip
        }
      & \parbox{2.5cm}{
          \smallskip
          \centering Сумма затрат на изделие, у.е.
          \smallskip
        }
      \\
      \hline

      1. Переключатель       & 2ПК-182     & 1 & 1,365 & 1,365 \\ \hline
      2. Транзистор          & КТ315Г      & 1 & 2,115 & 2,115 \\ \hline
      3. Диод                & Д183        & 1 & 1,365 & 1,365 \\ \hline
      4. Резистор            & IR10        & 2 & 0,910 & 1,820 \\ \hline
      5. Резистор            & IR11        & 1 & 0,920 & 0,920 \\ \hline
      6. Конденсатор         & R50-12      & 1 & 1,100 & 1,100 \\ \hline
      7. Плата гетинаксовая  & ЮК66.72.111 & 1 & 0,950 & 0,950 \\ \hline
      8. Заклёпка пустотелая & -           & 5 & 0,315 & 1,575 \\ \hline

      \multicolumn{4}{|l|}{\textbf{Итого}} & \textbf{11,210} \\ \hline
      \multicolumn{4}{|l|}{Транспортно-заготовительные расходы, $4\%$} & 0,336 \\ \hline
      \multicolumn{4}{|l|}{\textbf{Итого с учётом транспортно-заготовительных расходов}} & \textbf{11,546} \\ \hline

    \end{tabular}
  }
\end{table}

На основании произведенного в таблице~\ref{tbl:components_cost} расчёта, можем
определить статью затрат <<Покупные комплектующие изделия, полуфабрикаты, услуги
производственного характера>>:
\begin{equation*}
  \text{Р}_{\text{к}} = 11{,}546 \cdot 30~597 = 353~282{,}14~(\text{у.~е.}).
\end{equation*}

Расчёт затрат на заработную плату основных производственных рабочих приведен
в таблице~\ref{tbl:zp_workers}. На основе данных таблицы можем произвести расчёт
статьи затрат <<Основная заработная плата основных
производственных рабочих>>:
\begin{equation}
  \text{Р}_{\text{з.о}} = 0{,}1892 \cdot 30~597 = 5~788{,}62
\end{equation}

\begin{table} [h!]
  \caption{
    Основная заработная плата производственных рабочих
  }\label{tbl:zp_workers}
  {\small
  \begin{tabular}{| m{4cm} | c | c | c | c |}
      \hline
        \parbox{4cm}{
          \smallskip
          \centering Наименование операции
          \smallskip
        }
      & \parbox{1.8cm}{
          \smallskip
          \centering Разряд \\ работ
          \smallskip
        }
      & \parbox{2.8cm}{
          \smallskip
          \centering Норма времени (\( t_{\text{шт}_i} \)), мин.
          \smallskip
        }
      & \parbox{2.8cm}{
          \smallskip
          \centering Часовая тарифная ставка, у.~е.
          \smallskip
        }
      & \parbox{2.8cm}{
          \smallskip
          \centering Заработная плата, у.~е.
          \smallskip
        } \\
      \hline

      1. Сборочная & III & 1{,}42 & 0{,}891 & 0,0211 \\
      \hline
      2. Сборочная & III & 0{,}70 & 0{,}891 & 0,0104 \\
      \hline

      3. Сборочная & III & 0{,}68 & 0{,}891 & 0,0101 \\
      \hline

      4. Сборочная & III & 0{,}70 & 0{,}891 & 0,0104 \\
      \hline

      5. Сборочная & III & 2{,}11 & 0{,}891 & 0,0313 \\
      \hline

      6. Сборочная & III & 0{,}72 & 0{,}891 & 0,0107 \\
      \hline

      7. Сборочная & III & 0{,}67 & 0{,}891 & 0,0099 \\
      \hline

      8. Сборочная & III & 2{,}10 & 0{,}891 & 0,0312 \\
      \hline

      9. Сборочная & III & 0{,}70 & 0{,}891 & 0,0104 \\
      \hline

      \multicolumn{4}{|l|}{\textbf{Итого прямой фонд заработной платы}}
      & \textbf{0,1455} \\
      \hline

      \multicolumn{4}{|l|}{Премия за выполнение плана, $30\%$}
      & 0,0437 \\
      \hline

      \multicolumn{4}{|l|}{\textbf{Всего основная заработная плата}}
      & \textbf{0,1892} \\
      \hline
    \end{tabular}
  }
\end{table}


Расчёт статьи затрат
<<Дополнительная заработная плата основных производственных рабочих>>:
\begin{equation*}
\text{Р}_{\text{з.д}} =
\dfrac{\text{Р}_{\text{з.о}} \cdot \text{Н}_{\text{д.з}}}{100} =
\dfrac{5~788{,}62 \cdot 30}{100} = 1~736{,}59~(\text{у.~е.}).
\end{equation*}

Расчёт основной и дополнительной заработной платы вспомогательных рабочих
произведём в таблице~\ref{tbl:zp_additional_workers}.
\begin{table} [h!]
  \caption{
    Заработная плата вспомогательных рабочих
  }\label{tbl:zp_additional_workers}
  {\small
  \begin{tabular}{| m{4cm} | c | c | c | c |}
      \hline
        \parbox{4cm}{
          \smallskip
          \centering Наименование профессии
          \smallskip
        }
      & \parbox{1.8cm}{
          \smallskip
          \centering Разряд \\ работ
          \smallskip
        }
      & \parbox{2.8cm}{
          \smallskip
          \centering Численность, чел.
          \smallskip
        }
      & \parbox{2.8cm}{
          \smallskip
          \centering Часовая тарифная ставка, у.~е.
          \smallskip
        }
      & \parbox{2.8cm}{
          \smallskip
          \centering Заработная плата, у.~е.
          \smallskip
        } \\
      \hline

      Контролёры & V & 2 & 1,127 & 2,254 \\
      \hline

      Кладовщики & I & 2 & 0,697 & 1,394 \\
      \hline

      Уборщики & I & 1 & 0,697 & 0,697 \\
      \hline

      Подсобные рабочие & I & 1 & 0,697 & 0,697 \\
      \hline

      Ремонтные рабочие & V & 2 & 1,127 & 2,254 \\
      \hline

      Рабочие по \newline межремонтному обслуживанию & IV & 1 & 0,971 & 0,971 \\
      \hline

      \multicolumn{4}{|l|}{\textbf{Итого прямой фонд заработной платы}}
      & \textbf{8,267} \\
      \hline

      \multicolumn{4}{|l|}{Дополнительная заработная плата, $30\%$}
      & 2,480 \\
      \hline

      \multicolumn{4}{|l|}{\textbf{Итого основная и дополнительная заработная плата}}
      & \textbf{10,747} \\
      \hline

      \multicolumn{4}{|l|}{Эффективный фонд времени одного рабочего за плановый период, ч}
      & 184,49 \\
      \hline

      \multicolumn{4}{|l|}{\textbf{Итого}}
      & \textbf{1~982,7} \\
      \hline
    \end{tabular}
  }
\end{table}

Рассчитаем основную и дополнительную заработную плату ИТР и управленческого
персонала:
\begin{align*}
  \text{Р}_{\text{з.с.}} &= \text{К}_{\text{прем.}} \cdot \sum\limits_{i=1}^n \text{Ч}_{c.i} \cdot O_i, \\
  \text{Р}_{\text{з.с.}} &= 1{,}3 \cdot 2 \cdot 180 = 468{,}0~(\text{у.~е.}).
\end{align*}

Таким образом, расчёт статьи затрат <<Основная и дополнительная заработная плата
прочего ППП>>:
\begin{equation*}
\text{Р}_{\text{з.ппп}} = \text{Р}_{\text{з.в.р.}} + \text{Р}_{\text{з.с}} =
1~982{,}7+468{,}0 = 2~450{,}73~(\text{у.~е.}).
\end{equation*}


Расчёт статьи затрат
<<Отчисления в государственный фонд социальной защиты населения РБ>>:
\begin{align*}
\text{Р}_{\text{с.з}} &=
\dfrac{
  (\text{Р}_{\text{з.о}} + \text{Р}_{\text{з.д}}  + \text{Р}_{\text{з.ппп}}) \cdot \text{Н}_{\text{с.з}}
}{
  100
} = \\
&= \dfrac{( 5~788{,}62 + 1~736{,}59 + 2~450{,}73 ) \cdot 35}{100} =
3~491{,}58~(\text{у.~е.}).
\end{align*}

Расчёт статьи затрат
<<Единый платеж налогов>>:
\begin{equation*}
\text{Р}_{\text{е.п}} =
\dfrac{
  (\text{Р}_{\text{з.о}} + \text{Р}_{\text{з.д}}  + \text{Р}_{\text{з.ппп}}) \cdot \text{Н}_{\text{е.п}}
}{
  100
} =
\dfrac{9~975{,}93 \cdot 5}{100} =
498{,}80 \: (\text{у.~е.}).
\end{equation*}

Расчёт статьи затрат
<<Топливо и электроэнергия для технологических целей>>:
\begin{align*}
\text{Р}_{\text{э}} &=
W_{\text{у}} \cdot F_{\text{э}} \cdot \text{Ц}_{\text{э}} \cdot
\text{К}_{\text{см}} \cdot \text{К}_{\text{э.в}} \cdot
\text{К}_{\text{э.м}} \cdot \text{К}_{\text{з.о}} \cdot
\dfrac{J}{\eta} = \\
 &=
56{,}62 \cdot 184{,}49 \cdot 0{,}035 \cdot
2 \cdot 0{,}65 \cdot 0{,}45 \cdot 0{,}95 \cdot
\dfrac{1{,}15}{0{,}75} = 311{,}55~(\text{у.~е.}).
\end{align*}

Расчёт статьи затрат
<<Расходы на подготовку и освоение производства>>:
\begin{equation*}
\text{Р}_{\text{п.о}} =
\dfrac{
  \text{Р}_{\text{з.о}} \cdot \text{Н}_{\text{осв}}
}{
  100
} =
\dfrac{5~788{,}62 \cdot 10}{100} =
578{,}86~(\text{у.~е.}).
\end{equation*}

Расчёт статьи затрат
<<Износ инструментов и приспособлений целевого назначения>>:
\begin{equation*}
\text{Р}_{\text{из}} =
\dfrac{
  \text{Р}_{\text{з.о}} \cdot \text{Н}_{\text{из}}
}{
  100
} =
\dfrac{5~788{,}62 \cdot 15}{100} =
868{,}29~(\text{у.~е.}).
\end{equation*}

Расчёт статьи затрат
<<Амортизационные отчисления основных производственных фондов>>
приведен в таблице~\ref{tbl:common_cost}.


Расчёт статьи затрат
<<Общепроизводственные расходы>>:
\begin{equation*}
\text{Р}_{\text{оп}} =
\dfrac{
  \text{Р}_{\text{з.о}} \cdot \text{Н}_{\text{оп}}
}{
  100
} =
\dfrac{5~788{,}62 \cdot 90}{100} =
5~209{,}75 ~ (\text{у.~е.}).
\end{equation*}


Расчёт статьи затрат
<<Общехозяйственные расходы>>:
\begin{equation*}
\text{Р}_{\text{ох}} =
\dfrac{
  \text{Р}_{\text{з.о}} \cdot \text{Н}_{\text{ох}}
}{
  100
} =
\dfrac{5~788{,}62 \cdot 70}{100} =
4~052{,}031 \: (\text{у.~е.}).
\end{equation*}

Потери от брака примем равными нулю.

Расчёт статьи затрат
<<Прочие производственные расходы>>:
\begin{equation*}
\text{Р}_{\text{пр}} =
\dfrac{
  \text{С}_{\text{пр}} \cdot \text{Н}_{\text{пр}}
}{
  100
} =
\dfrac{451~464{,}06 \cdot 1{,}25}{100} = 5~643{,}30 ~ (\text{у.~е.}).
\end{equation*}

Расчёт статьи затрат
<<Коммерческие расходы>>:
\begin{equation*}
\text{Р}_{\text{ком}} =
\dfrac{
  \text{С}_{\text{пр}} \cdot \text{Н}_{\text{ком}}
}{
  100
} =
\dfrac{457~107{,}37 \cdot 1}{100} = 4~571{,}07 \: (\text{у.~е.}).
\end{equation*}

Расчёт нормативной прибыли на весь плановый выпуск:
\begin{equation*}
\text{П}_{\text{н}} =
\dfrac{
  \text{С}_{\text{п}} \cdot \text{У}_{\text{ри}}
}{
  100
} =
\dfrac{461~678{,}44 \cdot 40}{100} = 184~671{,}37 ~ (\text{у.~е.}).
\end{equation*}

Расчёт цены предприятия:
\begin{equation*}
\text{Ц}_{\text{п}} =
\text{С}_{\text{п}} + \text{П}_{\text{н}} =
461~678{,}44 + 184~671{,}37 = 646~349{,}81 ~ (\text{у.~е.}).
\end{equation*}

Расчёт статьи затрат
<<Отчисления в местные целевые бюджетные фонды>>:
\begin{equation*}
\text{Р}_{\text{м.б}} =
\dfrac{
  \text{Ц}_{\text{п}} \cdot \text{Н}_{\text{м.б}}
}{
  100 - \text{Н}_{\text{м.б}}
} =
\dfrac{646~272{,}32 \cdot 2{,}5 }{100 - 2{,}5} = 16~573{,}07 ~ (\text{у.~е.}).
\end{equation*}

Расчёт статьи затрат
<<Отчисления в республиканский фонд поддержки производителей
сельскохозяйственной продукции и дорожный фонд>>:
\begin{equation*}
\text{Р}_{\text{р.б}} =
\dfrac{
  (\text{Ц}_{\text{п}} + \text{Р}_{\text{м.б}}) \cdot \text{Н}_{\text{р.б}}
}{
  100 - \text{Н}_{\text{р.б}}
} =
\dfrac{(646~272{,}32 + 16~571{,}085) \cdot 2}{100 - 2} = 13~529{,}04 ~ (\text{у.~е.}).
\end{equation*}

Расчёт цены без учета НДС:
\begin{equation*}
\text{Ц}_{\text{о.ц}} =
\text{Ц}_{\text{п}} + \text{Р}_{\text{м.б}} + \text{Р}_{\text{р.б}} =
646~272{,}32 + 16~571{,}085 + 13~527{,}416 = 676~451{,}91~(\text{у.~е.}).
\end{equation*}

Расчёт НДС:
\begin{equation*}
\text{Р}_{\text{ндс}} =
\dfrac{
  \text{Ц}_{\text{о.ц}} \cdot \text{Н}_{\text{ндс}}
}{
  100
} =
\dfrac{676~451{,}91 \cdot 20}{100} = 135~290{,}38 \: (\text{у.~е.}).
\end{equation*}

Расчёт цены реализации с учетом косвенных налогов:
\begin{equation*}
\text{Ц}_{\text{р}} =
\text{Ц}_{\text{о.ц}} + \text{Р}_{\text{ндс}} =
676~370{,}818 + 135~274{,}164 = 811~742{,}31 ~ (\text{у.~е.}).
\end{equation*}

Результаты расчётов представлены в таблице~\ref{tbl:calculation}.

{\small
\begin{longtable}{| m{10.3cm} | c | c | c |}
  \caption{
    Калькуляция себестоимости и отпускной цены единицы продукции
  }\label{tbl:calculation} \\
      \hline
      \centering Наименование статей затрат
      & \rotatebox[origin=c]{90}{\parbox{3.5cm}{Условное обозначение}}
      & \rotatebox[origin=c]{90}{
        \parbox{4.0cm}{
          Сумма затрат на \\ плановый выпуск \\ продукции, у.~е.
        }
      }
      & \rotatebox[origin=c]{90}{
        \parbox{3.5cm}{
          Сумма затрат на \\ выпуск единицы \\ продукции, у.~е.
        }
      } \\

      \hline
      \centering 1 & 2 & 3 & 4 \\
      \hline
      \endfirsthead

      \multicolumn{4}{l}{\normalsize Продолжение таблицы \thetable{}} \\
      \hline
      \centering 1 & 2 & 3 & 4 \\
      \hline
      \endhead

      1. Сырьё, материалы и другие материальные ценности \newline
      за вычетом реализуемых отходов
      & \( \text{Р}_{\text{м}} \) & 72~865,05 & 2,3814 \\
      \hline

      2. Покупные комплектующие изделия, полуфабрикаты и \newline
      услуги производственного характера
      & \( \text{Р}_{\text{к}} \) & 353~282,14 & 11,5460 \\
      \hline

      3. Основная з/п основных производственных рабочих
      & \( \text{Р}_{\text{з.о}} \) & 5~788,62 & 0,1892 \\
      \hline

      4. Дополнительная з/п основных производственных \newline рабочих
      & \( \text{Р}_{\text{з.д}} \) & 1~736,59 & 0,0568 \\
      \hline

      5. Основная и дополнительная з/п прочего ППП
      & \( \text{Р}_{\text{з.ппп}} \) & 2~450{,}73 & 0,0801 \\
      \hline

      6. Отчисления в государственный фонд социальной \newline
      защиты населения РБ (35\% от ФЗП)
      & \( \text{Р}_{\text{с.з}} \) & 3~491,58 & 0,1141 \\
      \hline

      7. Единый платёж налогов, норматив 5\% от ФЗП
      & \( \text{Р}_{\text{е.п}} \) & 498,80 & 0,0163 \\
      \hline

      8. Топливо и электроэнергия для технологических целей
      & \( \text{Р}_{\text{э}} \) & 311,55 & 0,0102 \\
      \hline

      9. Расходы на подготовку и освоение производства
      & \( \text{Р}_{\text{п.о}} \) & 578,86 & 0,0189 \\
      \hline

      10. Износ инструментов и приспособлений \newline
      целевого назначения
      & \( \text{Р}_{\text{из}} \) & 868,29 & 0,0284 \\
      \hline

      11. Амортизационные отчисления основных \newline
      производственных фондов
      & \( \text{Р}_{\text{а}} \) & 330,08 & 0,0108 \\
      \hline

      12. Общепроизводственные расходы
      & \( \text{Р}_{\text{оп}} \) & 5~209,75 & 0,1703 \\
      \hline

      13. Общехозяйственные расходы
      & \( \text{Р}_{\text{ох}} \) & 4~052,03 & 0,1324 \\
      \hline

      14. Потери от брака
      & \( \text{Р}_{\text{бр}} \) & 0 & 0 \\
      \hline

      15. Прочие производственные расходы
      & \( \text{Р}_{\text{пр}} \) & 5~643,30 & 0,1844 \\
      \hline

      \textbf{Итого \newline производственная себестоимость продукции}
      & \( \mathbf{\text{С}_{\text{пр}}} \) & \textbf{457~107,36} & \textbf{14,9396} \\
      \hline

      16. Коммерческие расходы
      & \( \text{Р}_{\text{ком}} \) & 4~571,07 & 0,1494 \\
      \hline

      \textbf{Итого \newline полная себестоимость продукции}
      & \( \mathbf{\text{С}_{\text{п}}} \) & \textbf{461~678,44} & \textbf{15,0890} \\
      \hline

      17. Нормативная прибыль на единицу продукции
      & \( \text{П}_{\text{н}} \) & 184~671,38 & 6,0356 \\
      \hline

      18. Цена предприятия
      & \( \mathbf{\text{Ц}_{\text{п}}} \) & 646~349,81 & 21,1246 \\
      \hline

      19. Отчисления в местные целевые бюджетные фонды, \newline
      норматив 2{,}5\%
      & \( \text{Р}_{\text{м.б}} \) & 16~573,072 & 0,5417 \\
      \hline

      20. Отчисления в республиканский фонд поддержки \newline
      (с/х, ДФ), норматив 2\%
      & \( \text{Р}_{\text{р.б}} \) & 13~529,04 & 0,4422 \\
      \hline

      21. Отпускная цена без учета НДС
      & \( \text{Ц}_{\text{оц}} \) & 676~451,92 & 22,1084 \\
      \hline

      22. НДС
      & \( \text{Р}_{\text{ндс}} \) & 135~290,38 & 4,4217 \\
      \hline

      23. Цена реализации с учетом косвенных налогов
      & \( \text{Ц}_{\text{р}} \) & 811~742,31 & 26,5301 \\
      \hline
\end{longtable}
}
