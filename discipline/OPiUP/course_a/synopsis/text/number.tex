\section[Расчёт численности промышленно-производственного персонала]
{РАСЧЕТ ЧИСЛЕННОСТИ \\ ПРОМЫШЛЕННО-ПРОИЗВОДСТВЕННОГО \\ ПЕРСОНАЛА}
\label{sec:number}

\subsection[Расчёт численности основных производственных рабочих]
{Расчёт численности основных производственных рабочих}

Явочная численность производственных рабочих на основе расчётов
подраздела~\ref{subsec:number_of_employees_calculation} составляет 14 человек.
Списочный состав основных производственных рабочих:
\begin{equation*}
  \text{Ч}_{\text{оп.с}} =
    \dfrac{C_{\text{пр}}K_{см}}{1-K_{\text{сп}}} =
    \dfrac{14 \cdot 2}{1-0{,}1} =
    31{,}11 \rightarrow 31~(\text{чел.}).
\end{equation*}


\subsection[Расчёт численности вспомогательных рабочих, ИТР и УП]
{Расчёт численности вспомогательных рабочих, ИТР и \\
управленческого персонала}

При укрупненных расчетах число контролеров можно принять исходя из нормы
обслуживания одним контролером 10--12 рабочих мест в сборочных цехах.
Численность комплектовщиков и кладовщиков можно принять по одному человеку
на участок. Численность уборщиков производственных
помещений определяется исходя из нормы обслуживания ($550~\text{м}^2$ в смену
на одного рабочего). Численность подсобных и вспомогательных рабочих можно
принять~$1-1{,}3\%$ от общей численности рабочих.
Численность ИТР и управленческого персонала на участке с серийным производством
не должна превышать~$4-5\%$~от общей численности производственных рабочих.

В результате примем следующее число вспомогательных рабочих,
ИТР и управленческого персонала:

\begin{itemize}
  \item 2 контролёра (V разряда);
  \item 2 кладовщика (I разряда);
  \item 1 уборщик (I разряда);
  \item 1 подсобный рабочий (I разряда);
  \item 2 ИТР и управленческого персонала;
  \item 2 ремонтных рабочих (V разряда);
  \item 1 рабочий по межремонтному обслуживанию (IV разряда).
\end{itemize}

Общее число рабочих по ремонту и обслуживанию с учетом
двухсменного режима работы составляет 11 человек.

Результаты расчёта состава промышленно-производственного персонала
приведены в таблице~\ref{tbl:number}.

\begin{table} [h!]
  \caption{
    Состав ППП на предприятии
  }\label{tbl:number}
    \begin{tabular}{| m{9cm} | c | c |}
      \hline
        \parbox{9cm}{
          \smallskip
          \centering Категория работающих
          \smallskip
        }
      & \parbox{2.8cm}{
          \smallskip
          \centering Количество человек
          \smallskip
        }
      & \parbox{3.3cm}{
          \smallskip
          \centering \% от общего количества
          \smallskip
        } \\
      \hline

      1. Основные производственные рабочие & 14 & 41,18 \\
      \hline

      2. Вспомогательные рабочие & 9 & 26,47 \\

      В том числе: & & \\
      -- обслуживающие оборудование, & 3 & 8,82 \\
      -- не обслуживающие оборудование & 6 & 17,64 \\
      \hline

      3. ИТР и управленческий персонал & 2 & 5,89 \\
      \hline

      \textbf{Итого} & \textbf{34} & \textbf{100,00} \\
      \hline
    \end{tabular}
\end{table}