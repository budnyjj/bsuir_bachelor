\section[%
Расчёт технико-экономических показателей работы участка
]{%
РАСЧЁТ ТЕХНИКО-ЭКОНОМИЧЕСКИХ \\
ПОКАЗАТЕЛЕЙ РАБОТЫ УЧАСТКА
}
\label{sec:tep}

Стоимость нормируемых оборотных средств примем равной 50\% стоимости
основных производственных фондов:
\begin{equation*}
  \text{О}_{\text{ос}} = 0{,}5 \cdot \text{С}_{\text{пр.ф}} =
  0{,}5 \cdot 32~114{,}98 = 16~203{,}79~(\text{у.~е.}).
\end{equation*}

Полная себестоимость планового объема продукции рассчитана в
разделе~\ref{sec:cost} и равна $461~678{,}44$~у.~е, объём реализованной
продукции за плановый период --- $811~742{,}31$~у.~е.

Расчёт затрат на одну условную единицу реализуемой продукции:
\begin{equation*}
  \text{З}_{\text{р.п}} = \dfrac{\text{С}_{\text{п}}}{\text{Т}_{\text{р}}} =
  \dfrac{461~678{,}44}{811~742{,}31} = 0{,}569 ~ (\text{у.~е.}).
\end{equation*}

Расчёт общей суммы прибыли от реализации продукции:
\begin{align*}
  \text{П}_{\text{р.п}} &=
  \text{T}_{\text{р}} - \text{С}_{\text{п}} - \text{Р}_{\text{м.б}} -
  \text{Р}_{\text{р.б}} - \text{Р}_{\text{ндс}} = \\
  &= 811~742{,}31 - 461~678{,}44 - 16~573{,}072 - \\
  &- 13~529{,}038 - 135~290{,}38 = 184~671{,}38 ~ (\text{у.~е.}), \\
  \text{П}_{\text{пр.р}} &= \text{Н}_{\text{пр.р}} \cdot \text{П}_{\text{р.п}} =
  0{,}15 \cdot 184~671{,}38 = 27~700{,}71 ~ (\text{у.~е.}), \\
  \text{П}_{\text{р}} &= \text{П}_{\text{р.п}} + \text{П}_{\text{пр.р}} =
  184~671{,}38 + 27~700{,}71 = 212~372{,}08 ~ (\text{у.~е.}).
\end{align*}

Расчёт балансовой прибыли предприятия:
\begin{equation*}
  \text{П}_{\text{б}} = \text{П}_{\text{р}} + \text{П}_{\text{в}} - \text{У}_{\text{в}} =
  212~346{,}62 + 0 - 0 = 212~346{,}62~(\text{у.~е.})
\end{equation*}

Расчёт налога на нормируемые оборотные средства:
\begin{equation*}
  \text{Р}_{\text{н.ос}} =
  \dfrac{\text{О}_{\text{ос}} \text{Н}_{\text{ндв}}}{12 \cdot 100} =
  \dfrac{16~203{,}79 \cdot 1}{12 \cdot 100} = 13{,}50 ~ (\text{у.~е.}).
\end{equation*}

Расчёт налога на недвижимость:
\begin{align*}
  \text{О}_{\text{пр}} &= \text{О}_{\text{пр.ф}} - \text{И}_{\text{з}} =
  32~407{,}58 - 330{,}08 = 32~077{,}50 ~ (\text{у.~е.}), \\
  \text{Р}_{\text{н.пр}} &=
  \dfrac{\text{О}_{\text{пр}} \text{Н}_{\text{ндв}}}{12 \cdot 100} =
  \dfrac{32~077{,}50 \cdot 1}{12 \cdot 100} =
  26{,}73~(\text{у.~е.}).
\end{align*}

Расчёт общей суммы налога на недвижимость:
\begin{equation*}
  \text{Р}_{\text{ндв}} = \text{Р}_{\text{н.пр}} + \text{Р}_{\text{н.ос}} =
  26{,}73 + 13{,}50 = 40{,}23~(\text{у.~е.}).
\end{equation*}

Расчёт налогооблагаемой прибыли:
\begin{align*}
  \text{П}_{\text{н.о}} &= \text{П}_{\text{б}} -
  \text{П}_{\text{н.до}} - \text{П}_{\text{лн}} - \text{Р}_{\text{н.пр}} = \\
  &= 212~372{,}08 - 0 - 0 - 40{,}23 = 212~331{,}85~(\text{у.~е.}).
\end{align*}

Расчёт налога на прибыль:
\begin{equation*}
  \text{Р}_{\text{пр}} =
  \dfrac{\text{П}_{\text{н.о}} \text{Н}_{\text{пр}}}{100} =
  \dfrac{212~331{,}85 \cdot 24}{100} =
  50~959{,}64~(\text{у.~е.}).
\end{equation*}

Расчёт транспортного налога:
\begin{align*}
  \text{Р}_{\text{тр}} &=
  \dfrac{
    (\text{П}_{\text{б}} - \text{П}_{\text{н.до}} - \text{П}_{\text{лн}} -
    \text{Р}_{\text{ндв}} - \text{Р}_{\text{пр}}) \cdot \text{Н}_{\text{тр}}
  }{100} = \\
  &= \dfrac{(212~372{,}08 - 0 - 0 - 40{,}23 - 50~959{,}64) \cdot 5}{100} =
  8~068{,}61~(\text{у.~е.}).
\end{align*}

Расчёт чистой прибыли:
\begin{align*}
  \text{П}_{\text{ч}} &= \text{П}_{\text{б}} -
  \text{Р}_{\text{ндв}} - \text{Р}_{\text{пр}}  - \text{Р}_{\text{тр}} = \\
  &= 212~372{,}08 - 40{,}23 - 50~959{,}64 - 8~068{,}61 = 153~303{,}59~(\text{у.~е.}).
\end{align*}

Расчёт уровня рентабельности изделия:
\begin{equation*}
  \text{У}_{\text{изд}} =
  \dfrac{\text{Ц}_{\text{п}} - \text{С}_{\text{п}}}{\text{С}_{\text{п}}} \cdot 100 =
  \dfrac{646~349{,}81 - 461~678{,}44}{461~678{,}44} \cdot 100 = 40{,}00 ~ \%.
\end{equation*}

Расчёт уровня рентабельности производства:
\begin{equation*}
  \text{У}_{\text{р.п}} =
  \dfrac{\text{П}_{\text{ч}}}{\text{О}_{\text{пр.ф}} + \text{О}_{\text{ос}}} \cdot 100 =
  \dfrac{153~303{,}59}{32~407{,}58 + 16~203{,}79} \cdot 100 = 315{,}37~\%.
\end{equation*}

Расчёт фондоотдачи:
\begin{equation*}
  \text{Ф}_{\text{о}} =
  \dfrac{\text{Т}_{\text{р}}}{\text{О}_{\text{пр.ф}}} =
  \dfrac{811~742{,}31}{32~407{,}58} = 25{,}05~(\text{у.~е.}).
\end{equation*}

Результаты расчётов представлены в таблице~\ref{tbl:tep}.

{\small
\begin{longtable}{| m{10.2cm} | c | c |}
  \caption{
    Основные ТЭП работы участка (цеха)
  }\label{tbl:tep} \\
      \hline
      \centering Показатель
      & \parbox{2.5cm}{
        \smallskip
        \centering Условное обозначение
        \smallskip
      }
      & \parbox{2.5cm}{
        \smallskip
        \centering Значение показателя
        \smallskip
      } \\

      \hline
      \centering 1 & 2 & 3 \\
      \hline
      \endfirsthead

      \multicolumn{3}{l}{\normalsize Продолжение таблицы \thetable{}} \\
      \hline
      \centering 1 & 2 & 3 \\
      \hline
      \endhead

      1. Плановый объём производства & шт. & 30~597 \\
      \hline

      2. Объем реализованной продукции & у.~е. & 811~742,31 \\
      \hline

      3. Полная себестоимость реализуемой продукции & у.~е. & 461~678,44 \\
      \hline

      4. Затраты на условную единицу продукции & у.~е. & 0,569 \\
      \hline

      5. Полная себестоимость единицы продукции & у.~е. & 15,089 \\
      \hline

      6. Цена предприятия единицы продукции & у.~е. & 21,124 \\
      \hline

      7. Цена реализации продукции с учетом \newline косвенных налогов
      & у.~е. & 26,53 \\
      \hline

      8. Прибыль от реализации продукции & у.~е. & 184~671,38 \\
      \hline

      9. Чистая прибыль предприятия & у.~е. & 153~303,59 \\
      \hline

      10. Уровень рентабельности производства & \% & 315,37 \\
      \hline

      11. Уровень рентабельности изделия & \% & 40 \\
      \hline

      12. Фондоотдача выпускаемой продукции & у.~е. & 25,05 \\
      \hline

      13. Численность ППП --- всего & \multirow{5}{*}{чел.} & 53 \\
      В том числе: & & \\
      -- основных производственных рабочих & & 31 \\
      -- вспомогательных производственных рабочих & & 10 \\ \hline
      -- ИТР и управленческого персонала & чел. & 2 \\
      \hline

      14. Производительность труда одного \newline
      основного производственного рабочего
      & у.~е./чел. & 242,75 \\
      \hline

      15. Производительность труда работающих
      & у.~е./чел. & 188,23 \\
      \hline

      16. Размер отчислений в фонд СЗН РБ
      & у.~е. & 3~491,58 \\
      \hline

      17. Размер единого платежа налога в бюджет
      & у.~е. & 498,80 \\
      \hline

      18. Размер отчислений в местный целевой бюджет
      & у.~е. & 16~571,09 \\
      \hline

      19. Размер отчислений в республиканский целевой фонд \newline (с/х, ДФ)
      & у.~е. & 13~527,42 \\
      \hline

      20. НДС
      & у.~е. & 135~290,38 \\
      \hline

      21. Размер налога на прибыль
      & у.~е. & 50~959,64 \\
      \hline

      22. Размер налога на недвижимость
      & у.~е. & 40,234 \\
      \hline

      23. Стоимость основных производственных фондов
      & у.~е. & 32~407,58 \\
      \hline

      24. Среднегодовая стоимость оборотного капитала
      & у.~е. & 16~203,79 \\
      \hline

      25. Общий фонд заработной платы ППП
      & у.~е. & 9~975,94 \\
      \hline

      26. Среднемесячная заработная плата одного \newline работающего
      & у.~е. & 188,23 \\
      \hline
\end{longtable}
}
