\section[Обоснование типа производства и вида поточной линии]{
  ОБОСНОВАНИЕ ТИПА ПРОИЗВОДСТВА И ВИДА \\
  ПОТОЧНОЙ ЛИНИИ
}
\label{sec:choice}

\subsection{Краткое описание объекта производства и технологического процесса}

Объектом производства является плата АРУ, применяемая в производстве
телевизоров и приёмников. Габаритные размеры платы --- $45 \times 70$ мм,
вес --- $0{,}18$ кг. Используемые материалы, комплектующие изделия и полуфабрикаты
приведены в таблицах~\ref{tbl:materials_consumption}~и~\ref{tbl:components_consumption},
технологический процесс сборки изделия --- в таблице~\ref{tbl:assembly_process}.

\begin{table} [h!]
  \caption{
    Цена и норма расхода материалов для платы АЛУ
  }\label{tbl:materials_consumption}
  {\small
    \begin{tabular}{| m{3.8cm} | m{2.0cm} | m{2.1cm} | m{3.2cm} | m{3.2cm} |}
      \hline
      Наименование & Марка, профиль & Единица измерения & Норма расхода на комплект &
      Оптовая цена за единицу, у.~е. \\ \hline

      1.  Провод            & ПЭЛ-041 & м  & 0,0800 & 0,225 \\ \hline
      2.  Провод            & ПЭЛ-031 & м  & 0,0900 & 0,225 \\ \hline
      3.  Провод            & ПЭЛ-063 & м  & 0,1000 & 0,225 \\ \hline
      4.  Бумага            & КТ-05   & кг & 0,0008 & 0,350 \\ \hline
      5.  Бумага            & КТ-120  & кг & 0,0033 & 0,350 \\ \hline
      6.  Нитки             & к/б     & кг & 5,0000 & 0,150 \\ \hline
      7.  Труба             & ТЛВ-1   & м  & 1,5000 & 0,950 \\ \hline
      8.  Припой            & ПОС-61  & кг & 0,0500 & 1,362 \\ \hline
      9.  Канифоль          & -       & кг & 0,0200 & 0,320 \\ \hline
      10. Флюс спиртовой    & -       & кг & 0,0037 & 0,350 \\ \hline

    \end{tabular}
  }
\end{table}

Месячная программа выпуска платы составляет 30 597 штук,
режим работы двухсменный, количество рабочих дней --- 23,
процент потерь рабочего времени --- 3\%, продолжительность смены --- 8ч.

\newpage

\begin{table} [h!]
  \caption{
    Технологический процесс сборки платы АЛУ
  }\label{tbl:assembly_process}
  {\small
    \begin{tabular}{| m{10cm} | m{0.4cm} | m{3cm} | m{1.3cm} |}
      \hline
      Содержание операции &
      \rotatebox[origin=c]{90}{\hspace{2mm}\parbox{3.3cm}{Разряд работ}} &
      \rotatebox[origin=c]{90}{\hspace{2mm}\parbox{3.3cm}{Приспособление, инструмент, оборудование}} &
      \rotatebox[origin=c]{90}{\hspace{2mm}\parbox{3.3cm}{Норма времени \\ на операцию, \\ мин}} \\ \hline

      1. Вставить в приспособление $5$ заклёпок $2{,}5 \times 3$. Надеть на
      заклёпки плату. Распальцевать заклёпки до неподвижного состояния &
      3 & вручную & 1,42 \\ \hline

      2. Установить на плату модульный переключатель 2ПК-182, проверив
      предварительно его срабатывание &
      3 & вручную & 0,70 \\ \hline

      3. Вставить резисторы IR10, IR11 на плату согласно чертежу &
      3 & пинцет, вручную & 0,68 \\ \hline

      4. Вставить транзистор КТ315Г, конденсатор К50-12 и диод Д813, соблюдая
      полярность, согласно чертежу &
      3 & пинцет, вручную & 0,70 \\ \hline

      5. Произвести пайку плат на установке <<Волна>>. \newline
      При обслуживании установки следить: \newline
      а) за флюсованием и наличием флюса; \newline
      б) за режимом пайки согласно инструкции; \newline
      в) за промывкой плат; \newline
      г) за сушкой плат &
      3 & установка <<Волна>>, вручную & 2,11 \\ \hline

      6. Очистить платы от припоя, пользуясь приспособлениями. Периодически очищать
      поверхность расплавленного припоя ванны установки <<Волна>> от шлака &
      3 & вручную, паяльник 50~Вт, \newline 36~В, совок-скребок & 0,72 \\ \hline

      7. Провести проверку качества паек допаять вручную до 10 паек. Проверить
      прочность паевых соединений путём натягивания вывода детали &
      3 & вручную, паяльник 50~Вт, \newline 36~В, пинцет с изоляцией & 0,67 \\ \hline

      8. Очистить плату от флюса. Протереть плату обтирочной ветошью. Протереть
      плату со стороны печатного слоя кистью, смоченной в спирте. Протереть
      сухой щёткой поверхность платы &
      3 & вручную & 2,10 \\ \hline

      9. Осуществить контроль платы на стенде &
      3 & испытательный стенд & 0,70 \\ \hline

    \end{tabular}
  }
\end{table}

\newpage

\begin{table} [h!]
  \caption{
    Цена и норма расхода комплектующих изделий и полуфабрикатов для платы АЛУ
  }\label{tbl:components_consumption}
  {\small
    \begin{tabular}{| m{4.8cm} | m{3.1cm} | m{3.2cm} | m{3.2cm} |}
      \hline
      Наименование & ГОСТ, марка & Количество на комплект, шт. &
      Оптовая цена за единицу, у.~е. \\ \hline

      1. Переключатель       & 2ПК-182     & 1 & 1,365 \\ \hline
      2. Транзистор          & КТ315Г      & 1 & 2,115 \\ \hline
      3. Диод                & Д183        & 1 & 1,365 \\ \hline
      4. Резистор            & IR10        & 2 & 0,910 \\ \hline
      5. Резистор            & IR11        & 1 & 0,920 \\ \hline
      6. Конденсатор         & R50-12      & 1 & 1,100 \\ \hline
      7. Плата гетинаксовая  & ЮК66.72.111 & 1 & 0,950 \\ \hline
      8. Заклёпка пустотелая & -           & 5 & 0,315 \\ \hline

    \end{tabular}
  }
\end{table}

\subsection{Выбор и обоснование типа производства и вида поточной линии (участка)}

\newpage

%Вычислим коэффициент специализации \( K_{\text{сп}} \),
%такт выпуска изделий \( r_{\text{н.п.}}\) и
%коэффициент массовости \( K_\text{м} \):

%\begin{align*}
%K_{\text{сп}} &= \dfrac{m}{C_{\text{пр}}} = \dfrac{8}{5} = 1{,}6, \\
%r_{\text{н.п.}} &= \dfrac{60 F_{\text{э}}}{N_{\text{з}}} =
%  \frac{60 F_{\text{н}} F_{\text{п.о}}}{N_{\text{з}}} =
%  \dfrac{60 \cdot 23 \cdot 8 \cdot 2 \cdot 0{,}95}{5000} =
%  4{,}20 \: \text{(мин/шт.)}, \\
%K_{\text{м}} &=
%\dfrac{\sum^m_{i=1} t_{\text{шт}_{i}}}{m \cdot r_{\text{н.п.}}} =
%\dfrac{
%  \frac{6{,}4}{1{,}1} + \frac{8{,}2}{1{,}1} + \frac{9{,}2}{1{,}1} +
%  \frac{4{,}0}{1{,}1} + \frac{7{,}6}{1{,}1} + \frac{5{,}0}{1{,}1} +
%  \frac{6{,}8}{1{,}1} + \frac{7{,}0}{1{,}1}
%}{
%  8 \cdot 4{,}2
%} = 1{,}47.
%\end{align*}