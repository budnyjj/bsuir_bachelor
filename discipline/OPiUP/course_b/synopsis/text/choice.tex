\section[Обоснование типа производства и вида поточной линии]{
  ОБОСНОВАНИЕ ТИПА ПРОИЗВОДСТВА И ВИДА \\
  ПОТОЧНОЙ ЛИНИИ
}
\label{sec:choice}

\subsection{
  Краткое описание объекта производства и 
  технологического процесса
}

Объектом производства является кронштейн.
Кронштейн применяется при изготовлении радиоэлектронных изделий.
Исходные данные для расчета приведены в таблице~\ref{tbl:piece_description}.

\begin{table} [h!]
  \caption{
    Краткая характеристика объекта производства
  }\label{tbl:piece_description}
  {\small
    \begin{tabular}{| m{2.7cm} | m{1.7cm} | m{2.1cm} | m{1.9cm} | m{1.4cm} | m{1.8cm} | m{1.8cm} |}
      \hline
      Наименование детали & Вид заготовки & Материал, марка
      & Вес заготовки, кг & Чистый вес детали, кг
      & Оптовая цена 1 кг металла, у.~е. & Оптовая цена 1 кг отходов, у.~е. \\
      \hline
      Кронштейн & Прокат & Ст. А12-ТВ
      & 0{,}20 & 0{,}12 
      & 0{,}16 & 0{,}05 \\
      \hline
    \end{tabular}
  }
\end{table}

Месячная программа выпуска кронштейна составляет 5000 шт.,
режим работы двухсменный, количество рабочих дней --- 23,
процент потерь рабочего времени --- 3\%, продолжительность смены --- 8 ч.
Цены и нормы расхода материала для технологического 
процесса изготовления кронштейна приведены в таблице~\ref{tbl:tech_process}.

\begin{table} [h!]
  \caption{
    Технологический процесс изготовления детали
  }\label{tbl:tech_process}
  {\small
    \begin{tabular}{
      | m{2.5cm} | c | m{2.5cm} 
      | c | c | c
      | c | c | c |}
      \hline
      \rotatebox[origin=c]{90}{\hspace{0.01mm}
      \parbox{3.5cm}{Наименование \\ операции}}
      & \rotatebox[origin=c]{90}{\parbox{3.5cm}{Разряд работы}}
      & \rotatebox[origin=c]{90}{\parbox{3.5cm}{Наименование оборудования}}
      & \rotatebox[origin=c]{90}{\parbox{3.5cm}{Модель оборудования или марка}}
      & \rotatebox[origin=c]{90}{\parbox{3.5cm}{Габариты  оборудования}}
      & \rotatebox[origin=c]{90}{\parbox{3.5cm}{Мощность, кВт}}
      & \rotatebox[origin=c]{90}{\parbox{3.5cm}{Оптовая цена, у.~е.}}
      & \rotatebox[origin=c]{90}{\parbox{3.5cm}{Коэффициент выполнения норм}} 
      & \rotatebox[origin=c]{90}{\parbox{3.5cm}{
        Норма времени \\ ( \( t_{\text{шт}} \) ), мин}} \\
      \hline
      1. Фрезерная & 3 & Универсаль- ный фрезерный станок 
      & 6Р82Ш & 2470x1250 & 8{,}0
      & 2400 & 1{,}1 & 6{,}4 \\
      \hline
      2. Шлифова- льная & 4 & Плоскошли- фовальный станок
      & 3Б71м1 & 2600x1550 & 7{,}0
      & 3800 & 1{,}1 & 8{,}2 \\
      \hline
      3. Слесарная & 3 & Верстак
      & НДР-1064 & 1200x700 & --
      & 360 & 1{,}1 & 9{,}2 \\
      \hline
      4. Токарная & 4 & Токарно-винторезный станок
      & 1А616П & 2135x1225 & 10{,}0
      & 4425 & 1{,}1 & 4{,}0 \\
      \hline
      5. Фрезерная & 4 & Универса- льный фрезерный станок 
      & 6Р82Ш & 2470x1950 & 8{,}0
      & 2400 & 1{,}1 & 7{,}6 \\
      \hline
      6. Слесарная & 3 & Верстак
      & НДР-1064 & 1200x700 & --
      & 360 & 1{,}1 & 5{,}0 \\
      \hline
      7. Сверлиль- ная & 3 & Настольно-сверлильный станок
      & НС-12А & 710x360 & 3{,}5
      & 630 & 1{,}1 & 6{,}8 \\
      \hline
      8. Токарная & 4 & Токарно-винторезный станок
      & 1А616П & 2135x1225 & 10{,}0
      & 4425 & 1{,}1 & 7{,}0 \\
      \hline
    \end{tabular}
  }
\end{table}

\subsection{
  Выбор и обоснование типа производства и
  вида поточной линии (участка)
}

Вычислим коэффициент специализации \( K_{\text{сп}} \), 
такт выпуска изделий \( r_{\text{н.п.}}\) и
коэффициент массовости \( K_\text{м} \):

\begin{align*}
K_{\text{сп}} &= \dfrac{m}{C_{\text{пр}}} = \dfrac{8}{5} = 1{,}6, \\
r_{\text{н.п.}} &= \dfrac{60 F_{\text{э}}}{N_{\text{з}}} = 
  \frac{60 F_{\text{н}} F_{\text{п.о}}}{N_{\text{з}}} =
  \dfrac{60 \cdot 23 \cdot 8 \cdot 2 \cdot 0{,}95}{5000} =
  4{,}20 \: \text{(мин/шт.)}, \\
K_{\text{м}} &=
\dfrac{\sum^m_{i=1} t_{\text{шт}_{i}}}{m \cdot r_{\text{н.п.}}} = 
\dfrac{
  \frac{6{,}4}{1{,}1} + \frac{8{,}2}{1{,}1} + \frac{9{,}2}{1{,}1} + 
  \frac{4{,}0}{1{,}1} + \frac{7{,}6}{1{,}1} + \frac{5{,}0}{1{,}13} +
  \frac{6{,}8}{1{,}1} + \frac{7{,}0}{1{,}1}
}{
  8 \cdot 4{,}2
} = 1{,}46.
\end{align*}

Поскольку \( K_{\text{сп}} < 2 \) и \( K_{\text{м}} > 1 \),
следовательно, тип производства массовый.
Кроме этого, нормы времени выполнения операций не кратны такту, 
поэтому целесообразна организация поточного производства на основе
однопредметной прерывно-поточной линии.