\section*{ВВЕДЕНИЕ}
\addcontentsline{toc}{section}{Введение}

Планирование промышленного производства является важной составляющей 
экономических расчётов, имеющих целью организацию высокоэффективного 
и прибыльного производства.
Планирование позволяет на ранних этапах предусмотреть все возможные затраты и
риски, которые могут возникнуть при организации производства,
оценить примерную цену изделия,
рассчитать основные технико-экономические показатели функционирования предприятия.

Целью данной курсовой работы является расчёт календарно-плановых нормативов
и технико-экономическое обоснование производства изделия для предприятия.
Объектом производства является кронштейн, применяемый при выпуске
радиоэлектронной аппаратуры. 
Расчёт будет выполняться в соответствии с методикой расчёта и 
исходными данными, приведенными в методических пособиях~\cite{opiup_1,opiup_2}.
