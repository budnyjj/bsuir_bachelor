\section[
Расчет производственной площади и планировка участка]{
РАСЧЕТ ПРОИЗВОДСТВЕННОЙ ПЛОЩАДИ И \\
ПЛАНИРОВКА УЧАСТКА
}
\label{sec:placement}

\subsection{Планировка производственного участка}

Поточная линия имеет П-образную форму во избежание перекрещивающихся и
обратных потоков материалов и предметов труда.
Планировка производственного участка расположена в приложении Б.
Сырьё поступает на универсальный фрезерный станок, 
расположенный в  левом нижнем углу схемы, 
проходит через указанные в таблице~\ref{tbl:tech_process} стадии и
завершается на контрольном столе, который обозначен в правом нижнем углу плана.

\subsection{Расчет производственной площади участка}

Расчет производственной площади участка, занимаемой технологическим оборудованием
(рабочими местами) и транспортными средствами, 
приведен в таблице~\ref{tbl:prod_placement}.
Расчет общей площади участка приведен в таблице~\ref{tbl:common_placement}.

По результатам расчетов, общую площадь цеха примем равной
180 \( \text{м}^2 \) (18x10 м).

\begin{table} [h!]
  \caption{
    Расчет производственной площади участка
  }\label{tbl:prod_placement}
    \begin{tabular}{
      | m{4.3cm} | c | c | c | c | c |}
      \hline
      \parbox{4.3cm}{\centering Наименование \\ оборудования}
      & \rotatebox[origin=c]{90}{\parbox{4.5cm}{Модель (марка)}}
      & \rotatebox[origin=c]{90}{\parbox{4.5cm}{Габаритные размеры, мм}}
      & \rotatebox[origin=c]{90}{
        \parbox{4.5cm}{
            Количество единиц \\ оборудования (\( C_{\text{пр}} \))
          }
        }
      & \rotatebox[origin=c]{90}{
        \parbox{4.5cm}{
            Коэффициент дополнительной площади (\( K_{\text{дп}} \))
          }
        }
      & \rotatebox[origin=c]{90}{
        \parbox{4.5cm}{
            Производственная \\ площадь участка \\ (\( S \)), \( \text{м}^2 \)
          }
        } \\
      \hline

      Универсальный фрезерный станок & 6Р82Ш & 2470x1250 & 4 & 3{,}5 & 43{,}23 \\
      \hline

      Плоско- шлифовальный станок & 3Б71м1 & 2600x1550 & 2 & 3{,}0 & 24{,}18 \\
      \hline 

      Верстак & НДР-1064 & 1200x700 & 3 & 4{,}0 & 10{,}08 \\
      \hline 

      Токарно-винторезный станок & 1А616П & 2135x1225 & 3 & 3{,}5 & 27{,}46 \\
      \hline 

      Настольно-сверлильный станок & НС12А & 710x360 & 2 & 4{,}0 & 2{,}04 \\
      \hline 

      Контрольный стол & -- & 1200x700 & 1 & 4{,}0 & 3{,}36 \\
      \hline 

      Электрокар & ЭП201 
      & \parbox{3cm}{
        \smallskip
        \centering Трасса \\ 12000x1500
        \smallskip
      } & 1 & 1{,}0 & 18{,}00 \\
      \hline 

      Итого & -- & -- & 16 & -- & 128{,}35 \\
      \hline 
    \end{tabular}
\end{table}

\begin{table} [h!]
  \caption{
    Расчет общей площади участка
  }\label{tbl:common_placement}
    \begin{tabular}{| m{5.8cm} | m{5.75cm} | c |}
      \hline
      \parbox{5.8cm}{
        \smallskip
        \centering Вид  площади
        \smallskip
      }
      & \parbox{5.75cm}{
          \smallskip
          \centering Источник \\ или методика расчета
          \smallskip
      }
      & Площадь (\( S \)), \( \text{м}^2 \) \\
      \hline

      1. Производственная \newline площадь 
      & \centering См.~таблицу~\ref{tbl:prod_placement}
      & 128{,}35 \\
      \hline

      2. Вспомогательная \newline площадь 
      & \centering Принимаем 40\% \newline от производственной
      & 51{,}34 \\
      \hline
      
      Итого & & 179{,}69 \\
      \hline
    \end{tabular}
\end{table}

\subsection{Обоснование выбора типа здания}

Продукция, производимая на данном промышленном предприятии,
отличается небольшой массой и относительно небольшим количеством средств труда.
В связи с этим целесообразно размещать такое производство в двух- или трёхэтажном
здании; допустимо размещение указанного цеха на любом этаже.

Здание имеет форму прямоугольника, габаритные размеры 18x10 м, 
площадь 180 \( \text{м}^2 \).
Ширина пролета принята равной \( L = 9 \: \text{м} \),
т.~е. имеются два пролета, образованные железобетонными колоннами. 
Шаг колонн равен \( t = 6 \: \text{м} \), с учетом этого в каждом ряду находится
по две колонны.
Таким образом, для поддержания перекрытий цеха следует использовать шесть колонн.
Стены здания выполняются из железобетонных панелей высотой 1,2 м и
имеют высоту 4,8 м (на один этаж расходуется по четыре панели).
