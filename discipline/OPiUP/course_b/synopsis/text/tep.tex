\section[%
Расчёт технико-экономических показателей работы участка
]{%
РАСЧЁТ ТЕХНИКО-ЭКОНОМИЧЕСКИХ \\
ПОКАЗАТЕЛЕЙ РАБОТЫ УЧАСТКА
}
\label{sec:tep}

Стоимость нормируемых оборотных средств примем равной 50\% стоимости
основных производственных фондов:
\begin{equation*}
  \text{О}_{\text{ос}} = 0{,}5 \cdot \text{С}_{\text{пр.ф}} =
  0{,}5 \cdot 85213{,}49 =
  42606{,}74 \: (\text{у.~е.}).
\end{equation*}

Полная себестоимость выпуска планового объема продукции:
\begin{equation*}
  \text{С}_{\text{п}} = \sum^{H}_{j=1} N_j \text{С}_{\text{п}_j} =
  5000 \cdot 7{,}833 =
  39167{,}05 \: (\text{у.~е.}).
\end{equation*}

Объем реализованной продукции за плановый период:
\begin{equation*}
  \text{Т}_{\text{р}} = \sum^{H}_{j=1} N_j \text{Ц}_{\text{р}_j} =
  5000 \cdot 13{,}773 =
  68865{,}14 \: (\text{у.~е.}).
\end{equation*}

Расчёт затрат на одну условную единицу реализуемой продукции:
\begin{equation*}
  \text{З}_{\text{р.п}} = \dfrac{\text{С}_{\text{п}}}{\text{Т}_{\text{р}}} =
  \dfrac{39167{,}05}{68865{,}14} =
  0{,}57 \: (\text{у.~е.}).
\end{equation*}

Расчёт общей суммы прибыли от реализации продукции:
\begin{align*}
  \text{П}_{\text{р.п}} &= 
  \text{T}_{\text{р}} - \text{С}_{\text{п}} - \text{Р}_{\text{м.б}} - 
  \text{Р}_{\text{р.б}} - \text{Р}_{\text{ндс}} = \\
  &= 68865{,}14 - 39167{,}05 - 1405{,}00 - 1147{,}75 - 11477{,}52 = 
  15666{,}82 \: (\text{у.~е.}), \\
  \text{П}_{\text{пр.р}} &= \text{Н}_{\text{пр.р}} \cdot \text{П}_{\text{р.п}} =
  0{,}15 \cdot 15666{,}82 = 
  2350{,}02 \: (\text{у.~е.}), \\
  \text{П}_{\text{р}} &= \text{П}_{\text{р.п}} + \text{П}_{\text{пр.р}} =
  15666{,}82 + 2350{,}02 = 
  18016{,}84 \: (\text{у.~е.}).
\end{align*}

Расчёт балансовой прибыли предприятия:
\begin{equation*}
  \text{П}_{\text{б}} = \text{П}_{\text{р}} + \text{П}_{\text{в}} - \text{У}_{\text{в}} =
  18016{,}84 + 0 - 0 =
  18016{,}84 \: (\text{у.~е.}) 
\end{equation*}

Расчёт налога на нормируемые оборотные средства:
\begin{equation*}
  \text{Р}_{\text{н.ос}} = 
  \dfrac{\text{О}_{\text{ос}} \text{Н}_{\text{ндв}}}{12 \cdot 100} =
  \dfrac{42606{,}74 \cdot 1}{12 \cdot 100} =
  35{,}51 \: (\text{у.~е.}).
\end{equation*}

Расчёт налога на недвижимость:
\begin{align*}
  \text{О}_{\text{пр}} &= \text{О}_{\text{пр.ф}} - \text{И}_{\text{з}} =
  85213{,}49 - 665{,}91 = 84547{,}58 (\text{у.~е.}), \\
  \text{Р}_{\text{н.пр}} &= 
  \dfrac{\text{О}_{\text{пр}} \text{Н}_{\text{ндв}}}{12 \cdot 100} =
  \dfrac{84547{,}58 \cdot 1}{12 \cdot 100} = 
  70{,}46 \: (\text{у.~е.}).
\end{align*}

Расчёт общей суммы налога на недвижимость:
\begin{equation*}
  \text{Р}_{\text{ндв}} = \text{Р}_{\text{н.пр}} + \text{Р}_{\text{н.ос}} =
  70{,}46 + 35{,}51 =
  105{,}96 \: (\text{у.~е.}).
\end{equation*}

Расчёт налогооблагаемой прибыли:
\begin{align*}
  \text{П}_{\text{н.о}} &= \text{П}_{\text{б}} - 
  \text{П}_{\text{н.до}} - \text{П}_{\text{лн}}  - \text{Р}_{\text{н.пр}} = \\
  &= 18016{,}84 - 0 - 0 - 70{,}46 =
  17946{,}39 \: (\text{у.~е.}).
\end{align*}

Расчёт налога на прибыль:
\begin{equation*}
  \text{Р}_{\text{пр}} = 
  \dfrac{\text{П}_{\text{н.о}} \text{Н}_{\text{пр}}}{100} =
  \dfrac{17946{,}39 \cdot 24}{100} =
  4307{,}13 \: (\text{у.~е.}).
\end{equation*}

Расчёт транспортного налога:
\begin{align*}
  \text{Р}_{\text{тр}} &= 
  \dfrac{
    (\text{П}_{\text{б}} - \text{П}_{\text{н.до}} - \text{П}_{\text{лн}} - 
    \text{Р}_{\text{ндв}} - \text{Р}_{\text{пр}}) \cdot \text{Н}_{\text{тр}}
  }{100} = \\
  &= \dfrac{(18016{,}84 - 0 - 0 - 105{,}96 - 4307{,}13) \cdot 5}{100} =
  680{,}19 \: (\text{у.~е.}).
\end{align*}

Расчёт чистой прибыли:
\begin{align*}
  \text{П}_{\text{ч}} &= \text{П}_{\text{б}} - 
  \text{Р}_{\text{ндв}} - \text{Р}_{\text{пр}}  - \text{Р}_{\text{тр}} = \\
  &= 18016{,}84 - 105{,}96 - 4307{,}13 - 680{,}19 =
  12923{,}56 \: (\text{у.~е.}).
\end{align*}

Расчёт уровня рентабельности изделия:
\begin{equation*}
  \text{У}_{\text{изд}} = 
  \dfrac{\text{Ц}_{\text{п}} - \text{С}_{\text{п}}}{\text{С}_{\text{п}}} \cdot 100 =
  \dfrac{54833{,}87 - 39167{,}05}{54833{,}871} \cdot 100 =
  40{,}00 \: \%. 
\end{equation*}

Расчёт уровня рентабельности производства:
\begin{equation*}
  \text{У}_{\text{р.п}} = 
  \dfrac{\text{П}_{\text{ч}}}{\text{О}_{\text{пр.ф}} + \text{О}_{\text{ос}}} \cdot 100 =
  \dfrac{12923{,}56}{85213{,}49 + 42606{,}74} \cdot 100 =
  10{,}11 \: \%. 
\end{equation*}

Расчёт фондоотдачи:
\begin{equation*}
  \text{Ф}_{\text{о}} = 
  \dfrac{\text{Т}_{\text{р}}}{\text{О}_{\text{пр.ф}}} =
  \dfrac{68865{,}14}{85213{,}49} =
  0{,}81 \: (\text{у.~е.}). 
\end{equation*}

Результаты расчётов представлены в таблице~\ref{tbl:tep}.

{\small
\begin{longtable}{| m{10.2cm} | c | c |}
  \caption{
    Основные ТЭП работы участка (цеха)
  }\label{tbl:tep} \\
      \hline
      \centering Показатель
      & \parbox{2.5cm}{
        \smallskip
        \centering Условное обозначение
        \smallskip
      }
      & \parbox{2.5cm}{
        \smallskip
        \centering Значение показателя
        \smallskip
      } \\

      \hline
      \centering 1 & 2 & 3 \\
      \hline
      \endfirsthead 

      \multicolumn{3}{l}{\normalsize Продолжение таблицы \thetable{}} \\
      \hline
      \centering 1 & 2 & 3 \\
      \hline
      \endhead

      1. Плановый объём производства & шт. & 5000 \\
      \hline

      2. Объем реализованной продукции & у.~е. & 68865{,}14 \\
      \hline

      3. Полная себестоимость реализуемой продукции & у.~е. & 39167{,}05 \\
      \hline

      4. Затраты на условную единицу продукции & у.~е. & 0{,}57 \\
      \hline

      5. Полная себестоимость единицы продукции & у.~е. & 7{,}833 \\
      \hline

      6. Цена предприятия единицы продукции & у.~е. & 10{,}967 \\
      \hline

      7. Цена реализации продукции с учетом \newline косвенных налогов 
      & у.~е. & 13{,}773 \\
      \hline

      8. Прибыль от реализации продукции & у.~е. & 18016{,}84 \\
      \hline

      9. Чистая прибыль предприятия & у.~е. & 12923{,}56 \\
      \hline

      10. Уровень рентабельности производства & \% & 10{,}11 \\
      \hline

      11. Уровень рентабельности изделия & \% & 40 \\
      \hline

      12. Фондоотдача выпускаемой продукции & у.~е. & 0{,}81 \\
      \hline

      13. Численность ППП --- всего & \multirow{5}{*}{у.~е.} & 66 \\
      В том числе: & & \\
      -- основных производственных рабочих & & 27 \\
      -- вспомогательных производственных рабочих & & 36 \\ \hline
      -- ИТР и управленческого персонала & & 3 \\ 
      \hline

      14. Производительность труда одного \newline 
      основного производственного рабочего 
      & у.~е./чел. & 248{,}18 \\
      \hline

      15. Производительность труда работающих
      & у.~е./чел. & 304{,}59 \\
      \hline

      16. Размер отчислений в фонд СЗН РБ
      & у.~е. & 7036{,}03 \\
      \hline

      17. Размер единого платежа налога в бюджет
      & у.~е. & 1005{,}15 \\
      \hline

      18. Размер отчислений в местный целевой бюджет
      & у.~е. & 1406{,}00 \\
      \hline

      19. Размер отчислений в республиканский целевой фонд \newline (с/х, ДФ)
      & у.~е. & 1147{,}75 \\
      \hline

      20. НДС
      & у.~е. & 11477{,}52 \\
      \hline

      21. Размер налога на прибыль
      & у.~е. & 4307{,}13 \\
      \hline

      22. Размер налога на недвижимость
      & у.~е. & 105{,}96 \\
      \hline

      23. Стоимость основных производственных фондов
      & у.~е. & 85213{,}49 \\
      \hline

      24. Среднегодовая стоимость оборотного капитала
      & у.~е. & 42606{,}74 \\
      \hline

      25. Общий фонд заработной платы ППП
      & у.~е. & 20102{,}93 \\
      \hline

      26. Среднемесячная заработная плата одного \newline работающего
      & у.~е. & 304{,}59 \\
      \hline
\end{longtable}
}
