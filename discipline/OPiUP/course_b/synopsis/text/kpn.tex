\section[
Расчет календарно-плановых нормативов и построение стандарт-плана]{
РАСЧЕТ КАЛЕНДАРНО-ПЛАНОВЫХ \\ 
НОРМАТИВОВ И ПОСТРОЕНИЕ \\
СТАНДАРТ-ПЛАНА
}
\label{sec:kpn}

Основной состав календарно-плановых нормативов ОППЛ:
\begin{itemize}
\item укрупнённый такт;
\item количество рабочих мест по операциям и по всей поточной линии;
\item стандарт-план работы линии;
\item размер и динамика движения межоперационных оборотных заделов; 
\item длительность производственного цикла.
\end{itemize}

Все КПН рассчитываются на период оборота \( T_{o} \), который принимается равным
одной смене:
\begin{equation*}
  T_{o} = 8 \cdot 60 = 480 \: \text{(мин)}.
\end{equation*}

Так как брак на операциях отсутствует, 
то программа запуска за период оборота равна программе выпуска:
\begin{equation*}
  N_{\text{з}} = N_{\text{в}} = 
  \dfrac{5000}{23 \cdot 8} = 108{,}7 = 109 \: \text{(шт./смена)}.
\end{equation*}

Такт потока:
\begin{equation*}
  r_{\text{пр}} = \dfrac{F_{\text{э}}}{N_{\text{в}}} = 
  \dfrac{480 \cdot 0{,}95}{108{,}7} = 4{,}42 \: \text{(мин/шт.)}.
\end{equation*}

Расчетное количество рабочих мест составляет 11{,}16 ед.,
принятое --- 14 ед.
Оборудование на рабочих местах \textnumero \textnumero 1--5,7,8 
загружено не полностью.
Средний коэффициент загрузки рабочих мест составляет 0{,}79, что соотвествует
требованиям для организации ОППЛ (\( K_{\text{з.ср}} \ge 0{,}75 \)).
Расчѐтная численность производственных рабочих составляет 14 чел.,
однако после построения графика регламентации труда
(подбора работ и совмещения профессий) выявлено, что достаточно
иметь на линии 12 чел. в смену.
Из них двое рабочих будет работать на двух рабочих местах:
один будет выполнять работу на 2-м и 9-м, а другой – на 12-м и
14-м рабочих местах.

График и порядок обслуживания рабочих мест представлен на стандарт-плане,
расположенном в приложении А.

Размер оборотного задела между каждой парой смежных операций определяется 
по формуле:
\begin{equation*}
  Z_{\text{об}} = 
  \dfrac{T_j \cdot C_i}{t_{\text{шт}_i}} - \dfrac{T_j \cdot C_{i+1}}{t_{\text{шт}_{i+1}}} 
  \: \text{(шт.)},
\end{equation*}

\noindent где \( \: T_j \) --- продолжительность j-го частного периода между смежными 
операциями при неизменном числе работающих единиц оборудования, мин;

\( C_i, C_{i+1} \) --- число единиц оборудования соответственно на i-й и 
(i+1)-й операциях в течение частного периода времени \( T_j \);

\( t_{\text{шт}_i}, t_{\text{шт}_{i+1}}\) --- нормы штучного времени соответственно на
i-й и на (i+1)-й операциях технологического процесса, мин.

Результат расчета межоперационных оборотных заделов в соответствии со 
стандарт-планом приведен в таблице~\ref{tbl:oper_zadel}.

\begin{table} [h!]
  \caption{
    Расчет межоперационных оборотных заделов
  }\label{tbl:oper_zadel}
  {\scriptsize
    \begin{tabular}{| c | c | c | c |}
      \hline
      \parbox{2cm}{
        \centering 
        Частные \\ периоды
      }
      & \parbox{3cm}{
        \centering
        \smallskip
        Длительность \\ частного \\ периода, мин
        \smallskip
      }
      & \parbox{7.7cm}{
        \centering
        Расчет заделов по частным \\ периодам \( T_j\), шт.
      }
      & \parbox{2cm}{
        \centering
        Площадь \\ эпюр, \\ дет/мин
      } \\ 
      \hline

      \multicolumn{4}{|c|}{Между 1-й и 2-й операциями} \\ 
      \hline

      \( T_1 \)
      & 152
      & \parbox{7cm}{
          \centering
          \smallskip
          \( z^{'}_{1,2} =
             \frac{152 \cdot 2}{5{,}82} - \frac{152 \cdot 2}{7{,}45} =
             +12
          \)
          \smallskip
        }
      & v \\
      \hline

      \( T_2 \)
      & 178
      & \parbox{7cm}{
          \centering
          \smallskip
          \( z^{''}_{1,2} = 
             \frac{178 \cdot 1}{5{,}82} - \frac{178 \cdot 2}{7{,}45} =
             -18
          \)
          \smallskip
        }
      & v \\
      \hline

      \( T_3 \)
      & 150
      & \parbox{7cm}{
          \centering
          \smallskip
          \( z^{'''}_{1,2} =
             \frac{150 \cdot 1}{5{,}82} - \frac{150 \cdot 1}{7{,}45} =
             +6
          \)
          \smallskip
        }
      & v \\
      \hline

      Итого & & & v \\
      \hline

      \multicolumn{4}{|c|}{Между 2-й и 3-й операциями} \\ 
      \hline

      \( T_1 \)
      & 330
      & \parbox{7cm}{
          \centering
          \smallskip
          \( z^{'}_{2,3} = 
             \frac{330 \cdot 2}{7{,}45} - \frac{330 \cdot 2}{8{,}36} =
             +10
          \)
          \smallskip
        }
      & v \\
      \hline

      \( T_2 \)
      & 99
      & \parbox{7cm}{
          \centering
          \smallskip
          \( z^{''}_{2,3} = 
             \frac{99 \cdot 1}{7{,}45} - \frac{99 \cdot 2}{8{,}36} =
             -11
          \)
          \smallskip
        }
      & v \\
      \hline

      \( T_3 \)
      & 51
      & \parbox{7cm}{
          \centering
          \smallskip
          \( z^{'''}_{2,3} =
             \frac{51 \cdot 1}{7{,}45} - \frac{51 \cdot 1}{8{,}36} =
             +1
          \)
          \smallskip
        }
      & v \\
      \hline

      Итого & & & v \\
      \hline

      \multicolumn{4}{|c|}{Между 3-й и 4-й операциями} \\ 
      \hline

      \( T_1 \)
      & 395
      & \parbox{7cm}{
          \centering
          \smallskip
          \( z^{'}_{3,4} = 
             \frac{395 \cdot 2}{8{,}36} - \frac{395 \cdot 1}{3{,}64} = 
             -15
          \)
          \smallskip
        }
      & v \\
      \hline

      \( T_2 \)
      & 34
      & \parbox{7cm}{
          \centering
          \smallskip
          \( z^{''}_{3,4} = 
             \frac{34 \cdot 2}{8{,}36} - \frac{34 \cdot 0}{3{,}64} =
             +9
          \)
          \smallskip
        }
      & v \\
      \hline

      \( T_3 \)
      & 51
      & \parbox{7cm}{
          \centering
          \smallskip
          \( z^{'''}_{3,4} = 
             \frac{51 \cdot 1}{8{,}36} - \frac{51 \cdot 0}{3{,}64} =
             +6
          \)
          \smallskip
        }
      & v \\
      \hline

      Итого & & & v \\
      \hline

      \multicolumn{4}{|c|}{Между 4-й и 5-й операциями} \\ 
      \hline

      \( T_1 \)
      & 152
      & \parbox{7cm}{
          \centering
          \smallskip
          \( z^{'}_{4,5} =
             \frac{152 \cdot 1}{3{,}64} - \frac{152 \cdot 1}{6{,}91} =
             +20
          \)
          \smallskip
        }
      & v \\
      \hline

      \( T_2 \)
      & 243
      & \parbox{7cm}{
          \centering
          \smallskip
          \( z^{''}_{4,5} = 
             \frac{243 \cdot 1}{3{,}64} - \frac{243 \cdot 2}{6{,}91} =
             -4
          \)
          \smallskip
        }
      & v \\
      \hline

      \( T_3 \)
      & 28
      & \parbox{7cm}{
          \centering
          \smallskip
          \( z^{'''}_{4,5} =
             \frac{28 \cdot 0}{3{,}64} - \frac{28 \cdot 2}{6{,}91} = 
             -8
          \)
          \smallskip
        }
      & v \\
      \hline

      \( T_4 \)
      & 57
      & \parbox{7cm}{
          \centering
          \smallskip
          \( z^{''''}_{4,5} =
            \frac{57 \cdot 0}{3{,}64} - \frac{57 \cdot 1}{6{,}91} = 
            -8
          \)
          \smallskip
        }
      & v \\
      \hline

      Итого & & & v \\
      \hline

      \multicolumn{4}{|c|}{Между 5-й и 6-й операциями} \\ 
      \hline

      \( T_1 \)
      & 152
      & \parbox{7cm}{
          \centering
          \smallskip
          \( z^{'}_{5,6} = 
             \frac{152 \cdot 1}{6{,}91} - \frac{152 \cdot 1}{4{,}55} =
             -12
          \)
          \smallskip
        }
      & v \\
      \hline

      \( T_2 \)
      & 271
      & \parbox{7cm}{
          \centering
          \smallskip
          \( z^{''}_{5,6} = 
             \frac{271 \cdot 2}{6{,}91} - \frac{271 \cdot 1}{4{,}55} = 
             +19
          \)
          \smallskip
        }
      & v \\
      \hline

      \( T_3 \)
      & 57
      & \parbox{7cm}{
          \centering
          \smallskip
          \( z^{'''}_{5,6} = 
             \frac{57 \cdot 1}{6{,}91} - \frac{57 \cdot 1}{4{,}55} = 
             -7
          \)
          \smallskip
        }
      & v \\
      \hline

      Итого & & & v \\
      \hline

      \multicolumn{4}{|c|}{Между 6-й и 7-й операциями} \\ 
      \hline

      \( T_1 \)
      & 192
      & \parbox{7cm}{
          \centering
          \smallskip
          \( z^{'}_{6,7} =
             \frac{192 \cdot 1}{4{,}55} - \frac{192 \cdot 2}{6{,}18} = 
             -20
          \)
          \smallskip
        }
      & v \\
      \hline

      \( T_2 \)
      & 288
      & \parbox{7cm}{
          \centering
          \smallskip
          \( z^{''}_{6,7} = 
             \frac{288 \cdot 1}{4{,}55} - \frac{288 \cdot 1}{6{,}18} = 
             -20
          \)
          \smallskip
        }
      & v \\
      \hline

      \multicolumn{4}{|c|}{Между 7-й и 8-й операциями} \\ 
      \hline

      \( T_1 \)
      & 192
      & \parbox{7cm}{
          \centering
          \smallskip
          \( z^{'}_{7,8} =
             \frac{192 \cdot 2}{6{,}18} - \frac{192 \cdot 1}{6{,}36} = 
             +32
          \)
          \smallskip
        }
      & v \\
      \hline

      \( T_2 \)
      & 212
      & \parbox{7cm}{
          \centering
          \smallskip
          \( z^{''}_{7,8} =
             \frac{212 \cdot 1}{6{,}18} - \frac{212 \cdot 2}{6{,}36} =
             -33
          \)
          \smallskip
        }
      & v \\
      \hline

      \( T_3 \)
      & 76
      & \parbox{7cm}{
          \centering
          \smallskip
          \( z^{'''}_{7,8} = 
             \frac{76 \cdot 1}{6{,}18} - \frac{76 \cdot 1}{6{,}36} =
             +1
          \)
          \smallskip
        }
      & v \\
      \hline

    \end{tabular}
  }
\end{table}