\section[%
Расчёт себестоимости и цены единицы продукции с учетом \\
косвенных налогов
]{%
РАСЧЁТ СЕБЕСТОИМОСТИ И ЦЕНЫ ЕДИНИЦЫ \\
ПРОДУКЦИИ С УЧЕТОМ КОСВЕННЫХ \\
НАЛОГОВ
}
\label{sec:cost}

Себестоимость единицы продукции --- это выраженная в денежной форме
сумма затрат на её производство и реализацию.
В качестве калькуляционной единицы примем одно изделие.

Расчёт статьи затрат 
<<Сырьё, материалы и другие материальные ценности
за вычетом реализуемых отходов>>:
\begin{equation*}
\text{Р}_{\text{м}} = 
\sum^{H}_{j=1} \text{Н}_{\text{м}_j} \text{Ц}_{\text{м}_j} \text{K}_{\text{т.з}} 
- \text{О}_{\text{м}_j} \text{Ц}_{\text{о}_j} = 
0{,}2 \cdot 0{,}16 \cdot (1 + \dfrac{4}{100}) - 0{,}08 \cdot 0{,}05 =
0{,}029 \: (\text{у.~е.}).
\end{equation*}

Затраты на покупные комлпектующие изделия, 
полуфабрикаты и услуги производственного характера отсутствуют.

Расчёт статьи затрат
<<Основная заработная плата основных производственных рабочих>>
приведен в таблице~\ref{tbl:zp_workers}.

\begin{table} [h!]
  \caption{
    Основная заработная плата производственных рабочих
  }\label{tbl:zp_workers}
  {\small
  \begin{tabular}{| m{4cm} | c | c | c | c |}
      \hline
        \parbox{4cm}{
          \smallskip
          \centering Наименование операции
          \smallskip
        }
      & \parbox{1.8cm}{
          \smallskip
          \centering Разряд \\ работ
          \smallskip
        }
      & \parbox{2.8cm}{
          \smallskip
          \centering Норма времени (\( t_{\text{шт}_i} \)), мин.
          \smallskip
        } 
      & \parbox{2.8cm}{
          \smallskip
          \centering Часовая тарифная ставка, у.~е.
          \smallskip
        }
      & \parbox{2.8cm}{
          \smallskip
          \centering Заработная плата, у.~е.
          \smallskip
        } \\
      \hline

      1. Фрезерная & 3 & 5{,}82 & 0{,}891 & 0{,}086 \\
      \hline
      2. Шлифовальная & 4 & 7{,}45 & 1{,}042 & 0{,}129 \\
      \hline

      3. Слесарная & 3 & 8{,}36 & 0{,}891 & 0{,}124 \\ 
      \hline

      4. Токарная & 4 & 3{,}64 & 1{,}042 & 0{,}063 \\
      \hline

      5. Фрезерная & 4 & 6{,}91 & 1{,}042 & 0{,}120 \\
      \hline

      6. Слесарная & 3 & 4{,}55 & 0{,}891 & 0{,}068 \\
      \hline

      7. Сверлильная & 3 & 6{,}18 & 0{,}891 & 0{,}092 \\
      \hline

      8. Токарная & 4 & 6{,}36 & 1{,}042 & 0{,}111 \\
      \hline

      \multicolumn{4}{|r|}{\textbf{Итого прямой фонд заработной платы}} 
      & \textbf{0{,}793} \\
      \hline

      \multicolumn{4}{|r|}{\textbf{Премия за выполнение плана}} 
      & \textbf{30\%} \\
      \hline                                                               

      \multicolumn{4}{|r|}{\textbf{Всего основная заработная плата}} 
      & \textbf{1{,}031} \\
      \hline                                                              
    \end{tabular}
  }
\end{table}

Расчёт статьи затрат 
<<Дополнительная заработная плата основных производственных рабочих>>:
\begin{equation*}
\text{Р}_{\text{з.д}} = 
\dfrac{\text{Р}_{\text{з.о}} \cdot \text{Н}_{\text{д.з}}}{100} = 
\dfrac{1{,}031 \cdot 30}{100} = 
0{,}309 \: (\text{у.~е.}).
\end{equation*}

Расчёт статьи затрат 
<<Основная и дополнительная заработная плата прочего ППП>>:
\begin{equation*}
\text{Р}_{\text{з.ппп}} = 
( \text{Р}_{\text{з.о}} + \text{Р}_{\text{з.д}} ) \cdot \text{К}_{\text{з.п}} =
( 1{,}031 + 0{,}309 ) \cdot 2{,}00 =
2{,}68 \: (\text{у.~е.}).
% \text{Р}_{\text{з.с}} &= 
% \text{К}_{\text{прем}} \sum^n_{i} \text{Ч}_{\text{с}_i} \text{О}_i =
% 1{,}3 \cdot ( 3 \cdot 160 ) = 
% 624 \: (\text{у.~е.}).
\end{equation*}

Расчёт статьи затрат 
<<Отчисления в государственный фонд социальной защиты населения РБ>>:
\begin{equation*}
\text{Р}_{\text{с.з}} = 
\dfrac{
  (\text{Р}_{\text{з.о}} + \text{Р}_{\text{з.д}}  + \text{Р}_{\text{з.ппп}}) \cdot \text{Н}_{\text{с.з}}
}{
  100
} =
\dfrac{( 1{,}031 + 0{,}309 + 2{,}68 ) \cdot 35}{100} =
1{,}407 \: (\text{у.~е.}).
\end{equation*}

Расчёт статьи затрат 
<<Единый платеж налогов>>:
\begin{equation*}
\text{Р}_{\text{е.п}} = 
\dfrac{
  (\text{Р}_{\text{з.о}} + \text{Р}_{\text{з.д}}  + \text{Р}_{\text{з.ппп}}) \cdot \text{Н}_{\text{е.п}}
}{
  100
} =
\dfrac{( 1{,}031 + 0{,}309 + 2{,}68 ) \cdot 5}{100} =
0{,}201 \: (\text{у.~е.}).
\end{equation*}

Расчёт статьи затрат 
<<Топливо и электроэнергия для технологических целей>>:
\begin{align*}
\text{Р}_{\text{э}} &= 
W_{\text{у}} \cdot F_{\text{э}} \cdot \text{Ц}_{\text{э}} \cdot
\text{К}_{\text{см}} \cdot \text{К}_{\text{э.в}} \cdot 
\text{К}_{\text{э.м}} \cdot \text{К}_{\text{з.о}} \cdot
\dfrac{J}{\eta} = \\
 &=
86{,}5 \cdot 8 \cdot 0{,}035 \cdot
2 \cdot 0{,}65 \cdot 0{,}45 \cdot 
0{,}79 \cdot 
\dfrac{1{,}15}{5000 \cdot 0{,}75} =
0{,}003 \: (\text{у.~е.}).
\end{align*}

Расчёт статьи затрат 
<<Расходы на подготовку и освоение производства>>:
\begin{equation*}
\text{Р}_{\text{п.о}} = 
\dfrac{
  \text{Р}_{\text{з.о}} \cdot \text{Н}_{\text{осв}}
}{
  100
} =
\dfrac{1{,}031 \cdot 10}{100} =
0{,}103 \: (\text{у.~е.}).
\end{equation*}

Расчёт статьи затрат 
<<Износ инструментов и приспособлений целевого назначения>>:
\begin{equation*}
\text{Р}_{\text{из}} = 
\dfrac{
  \text{Р}_{\text{з.о}} \cdot \text{Н}_{\text{из}}
}{
  100
} =
\dfrac{1{,}031 \cdot 15}{100} =
0{,}155 \: (\text{у.~е.}).
\end{equation*}

Расчёт статьи затрат 
<<Амортизационные отчисления основных производственных фондов>> 
приведен в таблице~\ref{tbl:common_cost}.
Величина амортизационных отчислений основных производственных фондов,
приходящизся на одно выпущенное изделие, составляет: 
\begin{equation*}
\text{Р}_{\text{а}} = 
\dfrac{665{,}906}{5000} =
0{,}133 \: (\text{у.~е.}).
\end{equation*}

Расчёт статьи затрат 
<<Общепроизводственные расходы>>:
\begin{equation*}
\text{Р}_{\text{оп}} = 
\dfrac{
  \text{Р}_{\text{з.о}} \cdot \text{Н}_{\text{оп}}
}{
  100
} =
\dfrac{1{,}031 \cdot 90}{100} =
0{,}928 \: (\text{у.~е.}).
\end{equation*}

Расчёт статьи затрат 
<<Общехозяйственные расходы>>:
\begin{equation*}
\text{Р}_{\text{ох}} = 
\dfrac{
  \text{Р}_{\text{з.о}} \cdot \text{Н}_{\text{ох}}
}{
  100
} =
\dfrac{1{,}031 \cdot 70}{100} =
0{,}722 \: (\text{у.~е.}).
\end{equation*}

Потери от брака примем равными нулю.

Расчёт статьи затрат 
<<Прочие производственные расходы>>:
\begin{equation*}
\text{Р}_{\text{пр}} = 
\dfrac{
  C_{\text{пр}} \cdot \text{Н}_{\text{пр}}
}{
  100
} =
\dfrac{7{,}702 \cdot 0{,}7}{100} =
0{,}054 \: (\text{у.~е.}).
\end{equation*}

Расчёт статьи затрат
<<Коммерческие расходы>>:
\begin{equation*}
\text{Р}_{\text{ком}} = 
\dfrac{
  C_{\text{пр}} \cdot \text{Н}_{\text{ком}}
}{
  100
} =
\dfrac{7{,}756 \cdot 1}{100} =
0{,}078 \: (\text{у.~е.}).
\end{equation*}

Расчёт нормативной прибыли на единицу продукции:
\begin{equation*}
\text{П}_{\text{н}} = 
\dfrac{
  C_{\text{п}} \cdot \text{У}_{\text{ри}}
}{
  100
} =
\dfrac{7{,}833 \cdot 40}{100} =
3{,}133 \: (\text{у.~е.}).
\end{equation*}

Расчёт цены предприятия:
\begin{equation*}
\text{Ц}_{\text{п}} = 
C_{\text{п}} + \text{П}_{\text{н}} =
7{,}833 + 3{,}133 =
10{,}967 \: (\text{у.~е.}).
\end{equation*}

Расчёт статьи затрат 
<<Отчисления в местные целевые бюджетные фонды>>:
\begin{equation*}
\text{Р}_{\text{м.б}} = 
\dfrac{
  \text{Ц}_{\text{п}} \cdot \text{Н}_{\text{м.б}}
}{
  100 - \text{Н}_{\text{м.б}}
} =
\dfrac{10{,}967 \cdot 2{,}5 }{100 - 2{,}5}  =
0{,}281 \: (\text{у.~е.}).
\end{equation*}

Расчёт статьи затрат 
<<Отчисления в республиканский фонд поддержки производителей
сельскохозяйственной продукции и дорожный фонд>>:
\begin{equation*}
\text{Р}_{\text{р.б}} = 
\dfrac{
  (\text{Ц}_{\text{п}} + \text{Р}_{\text{м.б}}) \cdot \text{Н}_{\text{р.б}}
}{
  100 - \text{Н}_{\text{р.б}}
} =
\dfrac{(10{,}967 + 0{,}281) \cdot 2}{100 - 2} =
0{,}230 \: (\text{у.~е.}).
\end{equation*}

Расчёт цены без учета НДС:
\begin{equation*}
\text{Ц}_{\text{о.ц}} = 
\text{Ц}_{\text{п}} + \text{Р}_{\text{м.б}} + \text{Р}_{\text{р.б}} = 
10{,}967 + 0{,}281 + 0{,}230 =
11{,}478 \: (\text{у.~е.}).
\end{equation*}

Расчёт НДС:
\begin{equation*}
\text{Р}_{\text{ндс}} = 
\dfrac{
  \text{Ц}_{\text{о.ц}} \cdot \text{Н}_{\text{ндс}}
}{
  100
} =
\dfrac{11{,}478 \cdot 20}{100}  =
2{,}296 \: (\text{у.~е.}).
\end{equation*}

Расчёт цены реализации с учетом косвенных налогов:
\begin{equation*}
\text{Ц}_{\text{р}} = 
\text{Ц}_{\text{о.ц}} + \text{Р}_{\text{ндс}} =
11{,}478 + 2{,}296 = 
13{,}773 \: (\text{у.~е.}).
\end{equation*}

Результаты расчётов представлены в таблице~\ref{tbl:calculation}.

{\small
\begin{longtable}{| m{10.7cm} | c | c | c |}
  \caption{
    Калькуляция себестоимости и отпускной цены единицы продукции
  }\label{tbl:calculation} \\
      \hline
      \centering Наименование статей затрат
      & \rotatebox[origin=c]{90}{\parbox{3.5cm}{Условное обозначение}}
      & \rotatebox[origin=c]{90}{
        \parbox{3.5cm}{
          Сумма затрат на \\ плановый выпуск \\ продукции, у.~е.
        }
      }
      & \rotatebox[origin=c]{90}{
        \parbox{3.5cm}{
          Сумма затрат на \\ выпуск единицы \\ продукции, у.~е.
        }
      } \\

      \hline
      \centering 1 & 2 & 3 & 4 \\
      \hline
      \endfirsthead 

      \multicolumn{4}{l}{\normalsize Продолжение таблицы \thetable{}} \\
      \hline
      \centering 1 & 2 & 3 & 4 \\
      \hline
      \endhead

      1. Сырьё, материалы и другие материальные ценности \newline
      за вычетом реализуемых отходов
      & \( \text{Р}_{\text{м}} \) & 146{,}400 & 0{,}029 \\
      \hline

      2. Покупные комплектующие изделия, полуфабрикаты и \newline
      услуги производственного характера
      & \( \text{Р}_{\text{к}} \) & -- & -- \\
      \hline

      3. Основная з/п основных производственных рабочих
      & \( \text{Р}_{\text{з.о}} \) & 5154{,}598 & 1{,}031 \\
      \hline

      4. Дополнительная з/п основных производственных \newline рабочих
      & \( \text{Р}_{\text{з.д}} \) & 1546{,}380 & 0{,}309 \\
      \hline

      5. Основная и дополнительная з/п прочего ППП
      & \( \text{Р}_{\text{з.ппп}} \) & 13401{,}956 & 2{,}68 \\
      \hline

      6. Отчисления в государственный фонд социальной \newline
      защиты населения РБ (35\% от ФЗП)
      & \( \text{Р}_{\text{с.з}} \) & 7036{,}027 & 1{,}407 \\
      \hline

      7. Единый платёж налогов, норматив 5\% от ФЗП
      & \( \text{Р}_{\text{е.п}} \) & 1005{,}147 & 0{,}201 \\
      \hline

      8. Топливо и электроэнергия для технологических целей
      & \( \text{Р}_{\text{э}} \) & 17{,}269 & 0{,}003 \\
      \hline

      9. Расходы на подготовку и освоение производства
      & \( \text{Р}_{\text{п.о}} \) & 515{,}460 & 0{,}103 \\
      \hline

      10. Износ инструментов и приспособлений \newline
      целевого назначения
      & \( \text{Р}_{\text{из}} \) & 773{,}190 & 0{,}155 \\
      \hline

      11. Амортизационные отчисления основных \newline 
      производственных фондов
      & \( \text{Р}_{\text{а}} \) & 665{,}906 & 0{,}133 \\
      \hline

      12. Общепроизводственные расходы
      & \( \text{Р}_{\text{оп}} \) & 4639{,}139 & 0{,}928 \\
      \hline

      13. Общехозяйственные расходы
      & \( \text{Р}_{\text{ох}} \) & 3608{,}219 & 0{,}722 \\
      \hline

      14. Потери от брака
      & \( \text{Р}_{\text{бр}} \) & -- & -- \\
      \hline

      15. Прочие производственные расходы
      & \( \text{Р}_{\text{пр}} \) & 269{,}568 & 0{,}054 \\
      \hline

      \textbf{Итого \newline производственная себестоимость продукции}
      & \( \mathbf{C_{\text{пр}}} \) & \textbf{38779{,}258} & \textbf{7{,}756} \\
      \hline

      16. Коммерческие расходы
      & \( \text{Р}_{\text{ком}} \) & 387{,}793 & 0{,}078 \\
      \hline

      \textbf{Итого \newline полная себестоимость продукции}
      & \( \mathbf{C_{\text{п}}} \) & \textbf{39167{,}050} & \textbf{7{,}833} \\
      \hline

      17. Нормативная прибыль на единицу продукции
      & \( \text{П}_{\text{н}} \) & 15666{,}820 & 3{,}133 \\
      \hline

      18. Цена предприятия
      & \( \mathbf{\text{Ц}_{\text{п}}} \) & 54833{,}871 & 10{,}967 \\
      \hline

      19. Отчисления в местные целевые бюджетные фонды, \newline
      норматив 2{,}5\%
      & \( \text{Р}_{\text{м.б}} \) & 1405{,}997 & 0{,}281 \\
      \hline

      20. Отчисления в республиканский фонд поддержки \newline
      (с/х, ДФ), норматив 2\%
      & \( \text{Р}_{\text{р.б}} \) & 1147{,}752 & 0{,}230 \\
      \hline

      21. Отпускная цена без учета НДС
      & \( \text{Ц}_{\text{оц}} \) & 57387{,}620 & 11{,}478 \\
      \hline

      22. НДС
      & \( \text{Р}_{\text{ндс}} \) & 11477{,}524 & 2{,}296 \\
      \hline

      23. Цена реализации с учетом косвенных налогов
      & \( \text{Ц}_{\text{р}} \) & 68865{,}144 & 13{,}773 \\
      \hline 
\end{longtable}
}
