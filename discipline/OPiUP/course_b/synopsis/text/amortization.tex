\section[Расчет стоимости и амортизации основных производственных фондов]{%
  РАСЧЕТ СТОИМОСТИ И АМОРТИЗАЦИИ \\ 
  ОСНОВНЫХ ПРОИЗВОДСТВЕННЫХ ФОНДОВ
}
\label{sec:amortization}

\subsection[%
Расчет стоимости здания,
занимаемого производственным \\
участком
]{%
  Расчет стоимости здания,
  занимаемого производственным участком
}

Расчет стоимости здания, занимаемого производственным участком, 
и амортизационных отчислений приведен в таблице~\ref{tbl:placement_cost}.

\begin{table} [h!]
  \caption{
    Расчет стоимости здания и амортизационных отчислений
  }\label{tbl:placement_cost}
    \begin{tabular}{| m{6.6cm} | c | c | c | c | c |}
      \hline
      \parbox{6.6cm}{
        \smallskip
        \centering Элементы расчета
        \smallskip
      }
      & \rotatebox[origin=c]{90}{
          \parbox{4.5cm}{
            Стоимость 1 \( \text{м}^2 \) \\ здания, у.~е./\( \text{м}^2 \)
          }
        }
      & \rotatebox[origin=c]{90}{
          \parbox{4.5cm}{
            Площадь, \\ занимаемая \\ зданием, \( \text{м}^2 \)
          }
        }
      & \rotatebox[origin=c]{90}{
          \parbox{4.5cm}{
            Стоимость \\ здания, у.~е.
          }
        }
      & \rotatebox[origin=c]{90}{
          \parbox{4.5cm}{
            Норма амортизации,~\%
          }
        }
      & \rotatebox[origin=c]{90}{
          \parbox{4.5cm}{
            Сумма амортизационных отчислений, у.~е.
          }
        } \\ 
      \hline

      1. Производственная площадь & 170 & 128{,}35 & 21819{,}71 
      & 2{,}7 & 589{,}13 \\ 
      \hline

      2. Вспомогательная площадь & 250 & 51{,}34 & 12835{,}12 
      & 3{,}1 & 397{,}89 \\ 
      \hline

      \raggedleft \textbf{Итого} & \textbf{--} & \textbf{179{,}69} 
      & \textbf{34654{,}83} & \textbf{--} & \textbf{987{,}02} \\
      \hline
    \end{tabular}
\end{table}

\vspace{-5mm}

\subsection{Расчет затрат на оборудование и транспортные средства}

Расчет затрат на оборудование и транспортные средства
приведен в таблице~\ref{tbl:tech_cost}. Величина затрат на упаковку,
транспортировку, монтаж и пусконаладочные работы принята равной 10\%
от цены оборудования.

\begin{table} [h!]
  \caption{
    Расчет стоимости транспортного и технологического оборудования
  }\label{tbl:tech_cost}
  {\small
    \begin{tabular}{| m{2.8cm} | c | c | c | c | c | c | c | c |}
      \hline
      \multirow{2}{*}{
        \rotatebox[origin=c]{90}{
          \parbox{6cm}{
            Наименование технологического \\
            оборудования и \\
            транспортных средств
          }
        }
      } 
      & \multirow{2}{*}{
          \rotatebox[origin=c]{90}{
            \parbox{6cm}{
              Модель (марка)
            }
          }
        }
      & \multirow{2}{*}{
          \rotatebox[origin=c]{90}{
            \parbox{6cm}{
              Количество единиц оборудова- \\
              ния, транспортных средств, шт.
            }
          }
        }
      & \multicolumn{2}{c|}{Оптовая цена}
      & \multirow{2}{*}{
          \rotatebox[origin=c]{90}{
            \parbox{6cm}{
              Затраты на упаковку, \\
              транспортировку, монтаж, \\
              пуск, наладку, у.~е.
            }
          }
        }
      & \multirow{2}{*}{
          \rotatebox[origin=c]{90}{
            \parbox{6cm}{
              Балансовая (первоначальная) \\
              стоимость техники, у.~е.
            }
          }
        }
      & \multirow{2}{*}{
          \rotatebox[origin=c]{90}{
            \parbox{6cm}{
              Норма амортизации, у.~е.
            }
          }
        }
      & \multirow{2}{*}{
          \rotatebox[origin=c]{90}{
            \parbox{6cm}{
              Сумма амортизационных \\
              отчислений, у.~е.
            }
          }
        } \\ \cline{4-5}

      & & 
      & \rotatebox[origin=c]{90}{
          \parbox{5.3cm}{
            единицы, у.~е.
          }
        }
      & \rotatebox[origin=c]{90}{
          \parbox{5.3cm}{
            принятого кол-ва, у.~е.
          }
        }
      & & & & \\
      \hline

      1. Универсаль- \newline ный фрезер- ный станок & 6Р82Ш 
      & 4                                          
      & 2400 & 9600 & 960 & 10560 
      & 14{,}2 & 1499{,}52 \\
      \hline

      2. Плоскошли- \newline фовальный станок & 3Б71м1 
      & 2
      & 3800 & 7600 & 760 & 8360 
      & 16{,}4 & 1371{,}04 \\
      \hline

      3. Токарно- \newline винторезный станок & 1А616П 
      & 3
      & 4425 & 13275 & 1327{,}5 & 14602{,}5 
      & 16{,}2 & 2365{,}61 \\
      \hline

      4. Верстак & НДР-1064 
      & 3
      & 360 & 1080 & 108 & 1188 
      & 7{,}7 & 91{,}48 \\
      \hline

      5. Настольно- \newline сверлильный станок & НС12А 
      & 2
      & 630 & 1260 & 126 & 1386 
      & 10{,}7 & 148{,}30 \\
      \hline

      6. Электрокар & ЭП201
      & 1
      & 3800 & 3800 & 380 & 4180 
      & 15{,}2 & 635{,}36 \\
      \hline

      % Погонный скат & -- 
      % & 1 & 45 & 45 & 4{,}5 & 49{,}5 
      % & 7 & 3{,}47 \\
      % \hline 

      \raggedleft \textbf{Итого} & \textbf{--}
      & \textbf{16} 
      & \textbf{--} & \textbf{36615} & \textbf{3661{,}5} & \textbf{40276{,}5} 
      & \textbf{--} & \textbf{6111{,}30} \\
      \hline
    \end{tabular}
  }
\end{table}

\subsection{Расчет затрат на энергетическое оборудование}

Величина затрат на силовое энергетическое оборудование, его монтаж,
упаковку, транспортировку, определяемая исходя из норматива 45 у.~е.
на 1 кВт установленной мощности технологического и транспортного 
оборудования, составляет:
\begin{equation*}
  \text{ПС}_{\text{э}} = 45 \cdot 86{,}5 = 3892{,}5 \: (\text{у.~е.}).
\end{equation*}

\subsection{%
  Расчет затрат на комплект дорогостоящей оснастки, УСПО и \\
  инструмента
}

Величина затрат на дорогостоящую оснастку, УСПО, 
инструмент (первоначальный фонд), принимаемая в размере 10\% от 
балансовой стоимости технологического оборудования, составляет:
\begin{equation*}
  \text{ПС}_{\text{ос}} = 0{,}1 \cdot 40276{,}5 = 4027{,}65 \: (\text{у.~е.}).
\end{equation*}

\subsection{Расчет затрат на измерительные и
  регулирующие приборы}

Величина затрат на измерительные и регулирующие приборы,
принимаемая в размере 1{,}5-2{,}0\% от 
оптовой цены оборудования, составляет:
\begin{equation*}
  \text{ПС}_{\text{из}} = 0{,}0175 \cdot 36615 = 640{,}76 \: (\text{у.~е.}).
\end{equation*}

\subsection{Расчет затрат на производственный и
  хозяйственный инвертарь}

Величина затрат на производственный инвертарь,
принимаемая в размере 1{,}5-2{,}0\% от 
стоимости технологического оборудования, а также 
затрат на хозяйственный инвертарь 
(по 15{,}4 у.~е. на одного работающего), составляет:
\begin{equation*}
  \text{ПС}_{\text{ин}} = 
  0{,}0175 \cdot 40276{,}5 + 15{,}4 \cdot 64 = 1690{,}44 \: (\text{у.~е.}).
\end{equation*}

\subsection{Расчет стоимости основных производственных фондов и \\
амортизационных отчислений}

Расчет затрат на оборудование и транспортные средства
приведен в таблице~\ref{tbl:common_cost}.

\begin{table} [h!]
  \caption{
    Расчет стоимости основных производственных фондов и
    амортизационных отчислений
  }\label{tbl:common_cost}
  {\small
    \begin{tabular}{| m{6.3cm} | c | c | c | c |}
      \hline
        \parbox{6.3cm}{
          \smallskip
          \centering Наименование групп основных производственных фондов
          \smallskip
        }
      & \parbox{1cm}{
          \smallskip
          \centering Усл. \\ обозн.
        }
      & \parbox{2.3cm}{
        \smallskip
          \centering Стоимость производ- ственных фондов, у.~е.
        }
      & \parbox{2.3cm}{
          \centering Норма амортизации, \%
        }
      & \parbox{2.4cm}{
          \centering Сумма аморт. отчислений, у.~е.
        } \\
      \hline

      1. Здание, занимаемое участком 
      & \( K_{\text{зд}} \) 
      & 34654{,}83 & Таблица~\ref{tbl:placement_cost} & 987{,}02 \\
      \hline

      2. Технологическое оборудование \newline и транспортные средства 
      & \( K_{\text{об}} \) 
      & 40276{,}5 & Таблица~\ref{tbl:tech_cost} & 6111{,}30 \\
      \hline

      3. Энергетическое оборудование
      & \( K_{\text{э}} \) 
      & 3892{,}5 & 8{,}2 & 319{,}19 \\
      \hline

      4. Дорогостоящая оснастка
      & \( K_{\text{ос}} \) 
      & 4027{,}65 & 4{,}5 & 181{,}24 \\
      \hline

      5. Измерительные \newline и регулирующие приборы
      & \( K_{\text{из}} \) 
      & 640{,}76 & 11{,}5 & 76{,}69 \\
      \hline

      6. Производственный \newline и хозяйственный инвертарь
      & \( K_{\text{ин}} \) 
      & 1690{,}44 & 18{,}5 & 312{,}73 \\
      \hline

      \raggedleft \textbf{Итого} 
      & \textbf{--}
      & \textbf{85182{,}69} & \textbf{--} & \textbf{7985{,}17} \\
      \hline
    \end{tabular}
  }
\end{table}