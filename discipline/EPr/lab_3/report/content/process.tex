\section{ХОД РАБОТЫ}

\subsection{Постановка задачи}

Необходимо рассчитать количественные оценки среднегодовых производственных запасов
и всех нормируемых оборотных средств, а также показателей использования
нормируемых оборотных средств.

Исходные данные --- нормируемые оборотные средства по плану за отчетный год:
\begin{itemize}
\item элементы производственных запасов:
  \begin{itemize}
  \item основные материалы: 94{,}0 тыс.~р.;
  \item покупные полуфабрикаты: 389{,}0 тыс.~р.;
  \item итого основные материалы и покупные п/ф: 483{,}0 тыс.~р.;
  \item вспомогательные материалы: 3{,}2 тыс.~р.;
  \item топливо: 1{,}5 тыс.~р.;
  \item тара: 2{,}5 тыс.~р.;
  \item запасные части: 3{,}0 тыс.~р.;
  \item инструменты и другие малоценные и быстроизнашивающиеся предметы (МБП): 
    96{,}0 тыс.~р.;  
  \end{itemize}

  \item всего производственных запасов: 589{,}2 тыс.~р.;
  \item среднегодовые запасы незавершенного производства и
    п/ф собственного изготовления: 315{,}6 тыс.~р.;
  \item среднегодовые запасы готовой продукции на складах: 88{,}2 тыс.~р.;
  \item итого запасы нормируемых оборотных средств: 993{,}0 тыс.~р..
\end{itemize}

Исходные данные --- структура нормируемых оборотных средств
(норматив по плану отчетного года):

\begin{itemize}
\item производственные запасы:
  \begin{itemize}
    \item всего: 100{,}0 \%;
    \item в том числе основные и вспомогательные материалы, 
      покупные п/ф: 82{,}52 \%;
  \end{itemize}
\item нормируемые оборотные средства:
  \begin{itemize}
    \item всего: 100{,}0 \%;
    \item в том числе производственные запасы: 59{,}34 \%;
    \item в том числе незавершенное производство и
      п/ф собственного изготовления: 31{,}78 \%;
    \item готовая продукция, находящаяся на складе: 8{,}88 \%.
  \end{itemize}
\end{itemize}

Инвертаризационная ведомость учета фактического состояния элементов 
производственных запасов приведена в таблице~\ref{tbl:source_real_state}.

\begin{table}[h!]
  \caption{Инвертаризационная ведомость учета фактического состояния элементов 
    производственных запасов}
  \label{tbl:source_real_state}
    \centering
    \small{
    \begin{tabular}{| p{0.26\textwidth} | p{0.08\textwidth} |
                      p{0.08\textwidth} | p{0.1\textwidth} | p{0.05\textwidth} |
                      p{0.05\textwidth} | p{0.08\textwidth} | p{0.08\textwidth} |}
      \hline
      Период \newline наблюдения 
      & \multicolumn{7}{|p{0.65\textwidth}|}{
        Зафиксированные отклонения фактической \newline
        среднегодовой стоимости произведенных запасов \newline
        от норматива на начало месяца, \%} \\ \cline{2-8}
      & основ\-ные материалы
      & покуп\-ные п/ф
      & вспомо\-гательные материалы
      & топ\-ливо
      & тара
      & запас\-ные части
      & инстру\-мент МБП \\ \hline

      Общая величина зафиксированных отклонений от норматива за год, \%
      & +145
      & +90
      & +200
      & +265
      & +170
      & +40
      & 70 \\ \hline

      Количество месяцев с отклонениями от нормы
      & 10
      & 11
      & 10
      & 11
      & 8
      & 8
      & 8 \\ \hline
    \end{tabular}
    }
\end{table}

Исходные данные --- технико-экономические показатели:
\begin{itemize}
\item объем реализованной продукции (план): 3186{,}0 тыс.~р.;
\item объем реализованной продукции (факт): 3192{,}0 тыс.~р.;
\item фактические среднегодовые запасы незавершенного производства
  и полуфабрикаты собственного изготовления в отчетном году: 316{,}0 тыс.~р.;
\item фактические среднегодовые запасы готовой продукции,
  находящейся на складе предприятия в отчетном году: 90{,}0 тыс.~р.
  
\end{itemize}

\newpage

\subsection{Решение}

Произведем расчет фактической среднегодовой стоимости основных материалов
по формуле~\ref{eq:avg_year_cost}:
\begin{align}
\label{eq:avg_year_cost}
\text{ОС}^\text{О}_{\text{ОМ}_{i}} &= \Bigg( 
  \dfrac{\sum^{n}_{j=1}{\eta_{ij}}}{\sum^{n}_{j=1}{m_{ij} \cdot 100}} + 1                        
  \Bigg) \text{OC}^\text{Н}_{\text{ОМ}_i}, \\
\text{ОС}^\text{О}_{\text{ОМ}_1} &= \Bigg( 
  \dfrac{145}{10 \cdot 100} + 1                        
  \Bigg) 94 = 107{,}63 \: \text{(тыс.~р.)}. \nonumber 
\end{align}

Аналогичным образом произведем расчет для всех остальных элементов
производственных запасов, результаты вычислений занесем в 
таблицу~\ref{tbl:result_reserve}.

Рассчитаем фактические среднегодовые запасы всех нормируемых оборотных 
средств за отчетный год по формуле~\ref{eq:sum_year_cost}.
\begin{align}
\label{eq:sum_year_cost}
\text{ОС}^\text{О}_\text{Н} &= 
  \text{ОС}^\text{О}_\text{ПЗ} + \text{ОС}^\text{О}_\text{НЗП} + \text{ОС}^\text{О}_\text{ГП}, \\
\text{ОС}^\text{О}_\text{Н} &= 
  644{,}74 + 316 + 90 = 1050{,}74 \: \text{(тыс.~р.)}. \nonumber
\end{align}

Занесем полученные значения в таблицу~\ref{tbl:result_reserve}.

\begin{table}[h!]
  \caption{Расчет фактических отклонений в среднегодовых запасах
  нормируемых оборотных средств предприятия, тыс.~р.}
  \label{tbl:result_reserve}
    \centering
    \small{
    \begin{tabular}{| p{0.33\textwidth} | p{0.22\textwidth} |
                      p{0.15\textwidth} | p{0.2\textwidth} |}
      \hline
      Элементы нормируемых оборотных средств
      & Норматив по плану \newline отчетного года
      & Фактичесие среднегодовые запасы
      & Отклонения \newline (+ увеличение, \newline - снижение) \\
      \hline

      1. Производственные \newline запасы, всего \newline
      В том числе:
      & 589{,}2
      & 644{,}74
      & +55{,}54 \\
      \hline

      1.1. Основные материалы
      & 94
      & 107{,}63
      & +13{,}63 \\
      \hline

      1.2. Покупные п/ф
      & 389
      & 420{,}83
      & +31{,}83 \\
      \hline

      Итого основные материалы и полуфабрикаты
      & 483
      & 528{,}5
      & +45{,}5 \\
      \hline

      1.3. Вспомогательные \newline материалы
      & 3{,}2
      & 3{,}84
      & +0{,}64 \\
      \hline

      1.4. Топливо
      & 1{,}5
      & 1{,}86
      & +0{,}36 \\
      \hline

      1.5. Тара
      & 2{,}5
      & 3{,}03
      & +0{,}53 \\
      \hline

      1.6. Запасные части
      & 3
      & 3{,}15
      & +0{,}15 \\
      \hline

      1.7. Инструменты и другие МБП
      & 96
      & 104{,}4
      & +8{,}4 \\
      \hline

      2. Незавершенное производство и полуфабрикаты собственного изготовления
      & 315{,}6
      & 316
      & +0{,}4 \\
      \hline

      3. Готовая продукция \newline на складе
      & 88{,}2
      & 90
      & +0{,}8 \\
      \hline

      Итого нормируемые \newline оборотные средства
      & 993
      & 1050{,}74
      & +57{,}74 \\
      \hline
    \end{tabular}
    }
\end{table}

Рассчитаем фактическое значение удельного веса материалов в среднегодовых
производственных запасах в соответствии с выражением~\ref{eq:materials_part}:
\begin{align}
\label{eq:materials_part}
\text{У}^\text{ПЗО}_\text{М} &=
  \dfrac{\text{ОС}^\text{О}_\text{М}}{\text{ОС}^\text{О}_\text{ПЗ}} \cdot 100, \\
\text{У}^\text{ПЗО}_\text{М} &=
  \dfrac{528{,}5}{644{,}74} \cdot 100 = 81{,}96 \%. \nonumber
\end{align}

Рассчитаем фактическое значение удельного веса среднегодовых производственных
запасов в среднегодовых запасах всех нормируемых оборотных средств 
в соответствии с выражением~\ref{eq:reserve_part}:
\begin{align}
\label{eq:reserve_part}
\text{У}^\text{ОСО}_\text{ПЗ} &=
  \dfrac{\text{ОС}^\text{О}_\text{ПЗ}}{\text{ОС}^\text{О}_\text{Н}} \cdot 100, \\
\text{У}^\text{ОСО}_\text{ПЗ} &=
  \dfrac{644{,}74}{1050{,}74} \cdot 100 = 61{,}36 \%. \nonumber
\end{align}

Рассчитаем фактическое значение удельного веса незавершенного производства и 
п/ф собственного изготовления в среднегодовых запасах всех нормируемых оборотных средств в соответствии с выражением~\ref{eq:not_finished_part}:
\begin{align}
\label{eq:not_finished_part}
\text{У}^\text{ОСО}_\text{НЗП} &=
  \dfrac{\text{ОС}^\text{О}_\text{НЗП}}{\text{ОС}^\text{О}_\text{Н}} \cdot 100, \\
\text{У}^\text{ОСО}_\text{НЗП} &=
  \dfrac{316}{1050{,}74} \cdot 100 = 30{,}07 \%. \nonumber
\end{align}

Рассчитаем фактическое значение удельного веса готовой продукции,
находящейся на складе предприятия, в среднегодовых запасах всех нормируемых
средств в соответствии с выражением~\ref{eq:finished_part}:
\begin{align}
\label{eq:finished_part}
\text{У}^\text{ОСО}_\text{ГП} &=
  \dfrac{\text{ОС}^\text{О}_\text{ГП}}{\text{ОС}^\text{О}_\text{Н}} \cdot 100, \\
\text{У}^\text{ОСО}_\text{ГП} &=
  \dfrac{90}{1050{,}74} \cdot 100  = 8{,}56 \%. \nonumber
\end{align}

Занесем полученные значения в таблицу~\ref{tbl:result_structure}.

\begin{table}[h!]
  \caption{Расчет фактических отклонений в среднегодовых запасах
  нормируемых оборотных средств предприятия, тыс.~р.}
  \label{tbl:result_structure}
    \centering
    \begin{tabular}{| p{0.35\textwidth} | p{0.15\textwidth} |
                      p{0.18\textwidth} | p{0.2\textwidth} |}
      \hline
      Элементы нормируемых оборотных средств
      & Норматив по плану \newline отчетного года
      & Фактические данные
      & Отклонение \newline (+ увеличение, \newline - уменьшение) \\ 
      \hline
      
      Производственные \newline запасы, всего
      & 100
      & 100
      & \\
      \hline 

      В том числе: основные и вспомогательные материалы,
      покупные п/ф
      & 82{,}52
      & 81{,}96
      & -0{,}56 \\
      \hline 

      Нормируемые оборотные средства, всего \newline
      В том числе:
      & 100
      & 100
      & \\
      \hline 

      производственные запасы
      & 59{,}34
      & 61{,}36
      & +2{,}02 \\
      \hline 

      незавершенное производство и п/ф собственного изготовления
      & 31{,}78
      & 30{,}07
      & -1{,}7 \\
      \hline 

      готовая продукция, \newline находящаяся на складе предприятия
      & 8{,}88
      & 8{,}56
      & -0{,}32 \\
      \hline 
    \end{tabular}    
\end{table}

\newpage

По формуле~\ref{eq:coef_turnover} произведем расчет коэффициента оборачиваемости 
нормируемых оборотных средств по плану и по отчету:
\begin{equation}
\label{eq:coef_turnover}
\text{К}_\text{об} = \dfrac{\text{РП}}{\text{ОС}_\text{c}},
\end{equation}
\vspace{-1.8cm}
\begin{equation*}
\text{К}^\text{п}_\text{об} = \dfrac{3186}{993} = 3{,}21, \quad
\text{К}^\text{о}_\text{об} = \dfrac{3192}{1050{,}74} = 3{,}03. 
\end{equation*}

По формуле~\ref{eq:time_turn} произведем расчет длительности одного полного оборота
оборотных средств по плану и по отчету:
\begin{equation}
\label{eq:time_turn}
\text{Д}_\text{об} = \dfrac{\text{Т}}{\text{К}_\text{об}},
\end{equation}
\vspace{-1.8cm}
\begin{equation*}
\text{Д}^\text{п}_\text{об} = \dfrac{360}{3{,}21} = 112{,}2, \quad
\text{Д}^\text{о}_\text{об} = \dfrac{360}{3{,}03} = 118{,}5. 
\end{equation*}

\newpage
