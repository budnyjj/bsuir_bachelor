\section{ХОД РАБОТЫ}

\subsection{Исходные данные}

Исходные данные для расчёта плановой себестоимости и отпускной цены
печатной платы сведены в
таблицах~\ref{tbl:technological_process}--\ref{tbl:indirect_costs}.

\begin{table}[h!]
  \caption{Технологический процесс производства}
  \label{tbl:technological_process}
  \centering
  \small{
    \begin{tabular}{| p{0.56\textwidth} | p{0.15\textwidth} | p{0.2\textwidth} |}
      \hline

      Виды работ (операции) & Разряд работ & Нормы времени, ч \\ \hline

      1. Штамповка печатной платы                                     & 3 & 0,06 \\ \hline
      2. Покрытие печатной платы \newline фотоустойчивым материалом   & 4 & 0,02 \\ \hline
      3. Травление печатных проводников \newline и печатных площадок  & 5 & 0,06 \\ \hline
      4. Сверление и металлизация отверстий                           & 4 & 0,02 \\ \cline{2-3}
                                                                      & 4 & 0,01 \\ \hline
      5. Формовка и обрезание выводов радиоэлементов                  & 3 & 0,04 \\ \hline
      6. Установка элементов на печатную плату                        & 3 & 0,06 \\ \hline
      7. Установка печатной платы в кассету                           & 3 & 0,03 \\ \hline
      8. Размещение кассеты \newline на конвейере паяльной установки  & 3 & 0,05 \\ \hline
      9. Флюсование печатной платы и мойка                            & 4 & 0,12 \\ \hline
      10. Контроль и пайка                                            & 2 & 0,11 \\ \hline
      11. Устранение недостатков                                      & 5 & 0,17 \\ \hline
      12. Выходной контроль                                           & 6 & 0,2  \\ \hline

    \end{tabular}
  }
\end{table}

\begin{table}[h!]
  \caption{Единая тарифная сетка работников Республики Беларусь}
  \label{tbl:tariffs_grid}
  \centering
  \small{
    \begin{tabular}{| p{0.28\textwidth} | p{0.11\textwidth} |
                      p{0.11\textwidth} | p{0.11\textwidth} |
                      p{0.11\textwidth} | p{0.11\textwidth} |}
      \hline
      Разряды & 2 & 3 & 4 & 5 & 6 \\ \hline

      Тарифный коэффициент & 1,16 & 1,35 & 1,57 & 1,74 & 1,9 \\ \hline

    \end{tabular}
  }
\end{table}

\begin{table}[h!]
  \caption{Нормы расхода материалов на одно изделие}
  \label{tbl:material_costs}
  \centering
  \small{
    \begin{tabular}{| p{0.28\textwidth} | p{0.22\textwidth} |
                      p{0.17\textwidth} | p{0.21\textwidth} |}
      \hline
      Наименование материала & Единица измерения &
      Норма расхода & Цена за единицу, р. \\ \hline

      1. Стеклотекстолит   & кг & 0,36  & 46590 \\ \hline
      2. Провод            & м  & 0,08  & 2400  \\ \hline
      3. Припой ПОС-40     & кг & 0,05  & 18018 \\ \hline
      4. Флюс канифольный  & л  & 0,02  & 10770 \\ \hline
      5. Трубка            & м  & 0,1   & 3600  \\ \hline
      6. Спирто-бензиновая & л  & 0,01  & 14850 \\ \hline
      7. Бумага обёрточная & кг & 0,02  & 10395 \\ \hline

    \end{tabular}
  }
\end{table}

\begin{table}[h!]
  \caption{Нормы расхода комплектующих изделий}
  \label{tbl:set_costs}
  \centering
  \small{
    \begin{tabular}{| p{0.34\textwidth} | p{0.14\textwidth} |
                      p{0.2\textwidth} | p{0.2\textwidth} |}
      \hline
      Наименование ПКИ & Единица \newline измерения &
      Количество \newline на изделие & Цена за единицу, \newline р. \\ \hline

      1. Резисторы                & шт. & 38 & 70   \\ \hline
      2. Транзисторы              & шт. & 12 & 640  \\ \hline
      3. Конденсаторы             & шт. & 36 & 130  \\ \hline
      4. Разъёмы                  & шт. & 2  & 440  \\ \hline

    \end{tabular}
  }
\end{table}

\begin{table}[h!]
  \caption{Нормативы косвенных расходов, налогов и отчислений}
  \label{tbl:indirect_costs}
  \centering
  \small{
    \begin{tabular}{| p{0.6\textwidth} | p{0.15\textwidth} | p{0.15\textwidth} |}
      \hline
      Показатель & Условное \newline обозначение & Величина, \% \\ \hline

      Норматив дополнительной зарплаты &
      $ \text{Н}_{\text{д}} $ & 15 \\ \hline

      Норматив возмещения износа &
      $ \text{Н}_{\text{из}} $ & 20 \\ \hline

      Норматив общепроизводственных расходов &
      $ \text{Н}_{\text{обп}} $ & 180 \\ \hline

      Норматив общехозяйственных расходов &
      $ \text{Н}_{\text{обх}} $ & 200 \\ \hline

      Норматив прочих производственных расходов &
      $ \text{Н}_{\text{пр}} $ & 3 \\ \hline

      Норматив расходов на реализацию &
      $ \text{Н}_{\text{ком}} $ & 4 \\ \hline

      Норматив рентабельности единицы продукции &
      $ \text{Р}_{\text{ед}} $ & 30 \\ \hline

      Норматив отчислений в фонд социальной \newline защиты населения &
      $ \text{Н}_{\text{соц}} $ & 34 \\ \hline

      Норматив отчислений на обязательное страхование &
      $ \text{Н}_{\text{стр}} $ & 0,6 \\ \hline

      Ставка налога на добавленную стоимость &
      $ \text{Н}_{\text{дс}} $ & 20 \\ \hline

    \end{tabular}
  }
\end{table}



\subsection{Решение}

Произведём расчёт затрат на материалы путём умножения нормы расхода материала
на цену за единицу конкретного материала. Полученные данные представим в
таблице~\ref{tbl:material_costs_result}.

\begin{table}[h!]
  \caption{Расчёт затрат на материалы}
  \label{tbl:material_costs_result}
  \centering
  \small{
    \begin{tabular}{| p{0.31\textwidth} | p{0.12\textwidth} |
                      p{0.17\textwidth} | p{0.13\textwidth} | p{0.12\textwidth} |}
      \hline
      Наименование материала & Единица \newline измерения &
      Норма расхода & Цена за \newline единицу, р. & Сумма, р. \\ \hline

      1. Стеклотекстолит          & кг & 0,36  & 46590 & 16772,4  \\ \hline
      2. Провод                   & м  & 0,08  & 2400  & 192      \\ \hline
      3. Припой ПОС-40            & кг & 0,05  & 18018 & 900,9    \\ \hline
      4. Флюс канифольный         & л  & 0,02  & 10770 & 215,4    \\ \hline
      5. Трубка                   & м  & 0,1   & 3600  & 360      \\ \hline
      6. Спирто-бензиновая смесь  & л  & 0,01  & 14850 & 148,5    \\ \hline
      7. Бумага обёрточная        & кг & 0,02  & 10395 & 207,9    \\ \hline
      Итого                       &    &       &       & 18 797,1 \\ \hline
      Итого с учётом \newline
      транспортно-заготовительных расходов & & & & 20 676,8 \\ \hline

    \end{tabular}
  }
\end{table}

Расчёт затрат по статье <<Возвратные отходы>>, которые принимаются в размере
1 \% от стоимости материалов с учётом транспортно-заготовительных расходов:
\begin{align}
  \text{O}_{\text{В}} = 20676{,}8 \cdot 0{,}01 = 206{,}8 \: \text{р.}  \nonumber
\end{align}

\newpage

Расчёт затрат на комплектующие изделия произведём путём умножения цены за единицу
материала на норму расхода данного материала. Полученные данные проиллюстрируем
в таблице~\ref{tbl:set_costs_result}.

\begin{table}[h!]
  \caption{Расчёт затрат на комплектующие изделия}
  \label{tbl:set_costs_result}
  \centering
  \small{
    \begin{tabular}{| p{0.34\textwidth} | p{0.12\textwidth} |
                      p{0.14\textwidth} | p{0.13\textwidth} | p{0.13\textwidth} |}
      \hline
      Наименование ПКИ & Единица \newline измерения &
      Количество \newline на изделие & Цена за \newline единицу, р. &
      Сумма, р. \\ \hline

      1. Резисторы                & шт. & 38 & 70   & 2660   \\ \hline
      2. Транзисторы              & шт. & 12 & 640  & 7680   \\ \hline
      3. Конденсаторы             & шт. & 36 & 130  & 4680   \\ \hline
      4. Разъёмы                  & шт. & 2  & 440  & 880    \\ \hline
      Итого                       &    &       &    & 15 900 \\ \hline
      Итого с учётом \newline
      транспортно-заготовительных расходов & & & & 17490     \\ \hline

    \end{tabular}
  }
\end{table}

\newpage

Для расчёта заработной платы необходимо рассчитать часовые тарифные ставки по
видам работ, которые представлены в таблице~\ref{tbl:technological_process}
с учётом тарифных коэффициентов, приведенных в таблице~\ref{tbl:tariffs_grid}.
Тарифная ставка первого разряда установлена в размере $ 9821 $ рублей.

\begin{table}[h!]
  \caption{Расчёт основной заработной платы производственных рабочих}
  \label{tbl:technological_process_result}
  \centering
  \small{
    \begin{tabular}{| p{0.36\textwidth} | p{0.14\textwidth} | p{0.13\textwidth} |
                      p{0.13\textwidth} | p{0.13\textwidth} |}
      \hline
      Виды работ (операции) &
      Разряд работ &
      Часовая \newline тарифная \newline ставка, р/ч &
      Норма \newline времени, ч &
      Расценка, р \\ \hline

      1. Штамповка печатной платы                                    & 3 & 13258 & 0,06 & 795,5   \\ \hline
      2. Покрытие печатной платы \newline фотоустойчивым материалом  & 4 & 15419 & 0,02 & 308,4   \\ \hline
      3. Травление печатных проводников и печатных площадок          & 5 & 17089 & 0,06 & 1025,3  \\ \hline
      4. Сверление и металлизация \newline отверстий                 & 4 & 15419 & 0,02 & 308,4   \\ \cline{2-5}
                                                                     & 4 & 15419 & 0,01 & 154,2   \\ \hline
      5. Формовка и обрезание \newline выводов радиоэлементов        & 3 & 13258 & 0,04 & 530,3   \\ \hline
      6. Установка элементов \newline на печатную плату              & 3 & 13258 & 0,06 & 795,5   \\ \hline
      7. Установка печатной платы \newline в кассету                 & 3 & 13258 & 0,03 & 397,8   \\ \hline
      8. Размещение кассеты на \newline конвейере паяльной установки & 3 & 13258 & 0,05 & 662,9   \\ \hline
      9. Флюсование печатной платы и мойка                           & 4 & 15419 & 0,12 & 1850,3  \\ \hline
      10. Контроль и пайка                                           & 2 & 11392 & 0,11 & 1253,2  \\ \hline
      11. Устранение недостатков                                     & 5 & 17089 & 0,17 & 2905,1  \\ \hline
      12. Выходной контроль                                          & 6 & 18660 & 0,2  & 3732,0  \\ \hline

      Итого               & & & & 14 718,7            \\ \hline
      Премия $30\%$       & & & & 4 415,6             \\ \hline
      Итого с премией     & & & & 19 134,4            \\ \hline

    \end{tabular}
  }
\end{table}

\newpage

Расчёт затрат по статье <<Дополнительная заработная плата производственных
рабочих>> произведём в соответствии с выражением~\ref{eq:extra_salary}.
\begin{align}
  \label{eq:extra_salary}
  \text{З}_{\text{д}} &= \dfrac{\text{З}_{\text{о}} \cdot
    \text{Н}_{\text{д}}}{100}, \\
  \text{З}_{\text{д}} &= 19134{,}4 \cdot 0{,}15 =
    2~870{,}2 \: \text{р.} \nonumber
\end{align}

Расчёт затрат по статье <<Отчисления в фонд социальной защиты населения>>
произведём в соответствии с выражением~\ref{eq:social_fund}.
\begin{align}
  \label{eq:social_fund}
  \text{Р}_{\text{соц}} &= \dfrac{(\text{З}_{\text{о}} + \text{З}_{\text{д}}) \cdot
    \text{Н}_{\text{соц}}}{100}, \\
  \text{Р}_{\text{соц}} &= (19134{,}4 + 2870{,}2) \cdot 0{,}34 =
    7~481{,}5 \: \text{р.} \nonumber
\end{align}

Расчёт затрат по статье <<Обязательное страхование от несчастных случаев
на производстве>> произведём в соответствии с выражением~\ref{eq:insurance}.
\begin{align}
  \label{eq:insurance}
  \text{Р}_{\text{стр}} &= \dfrac{(\text{З}_{\text{о}} + \text{З}_{\text{д}}) \cdot
    \text{Н}_{\text{стр}}}{100}, \\
  \text{Р}_{\text{стр}} &= (19134{,}4 + 2870{,}2) \cdot 0{,}006 =
    132{,}0 \: \text{р.} \nonumber
\end{align}

Расчёт затрат по статье <<Погашение стоимости инструментов и приспособлений
целевого назначения и прочие специальные расходы>>
произведём в соответствии с выражением~\ref{eq:repay}.
\begin{align}
  \label{eq:repay}
  \text{Р}_{\text{из}} &= \dfrac{\text{З}_{\text{о}} \cdot
    \text{Н}_{\text{из}}}{100}, \\
  \text{Р}_{\text{из}} &= 19134{,}4 \cdot 0{,}2 =
    3~826{,}7 \: \text{р.} \nonumber
\end{align}

Расчёт затрат по статье <<Общепроизводственные расходы>>
произведём в соответствии с выражением~\ref{eq:general_production}.
\begin{align}
  \label{eq:general_production}
  \text{Р}_{\text{обп}} &= \dfrac{\text{З}_{\text{о}} \cdot
    \text{Н}_{\text{обп}}}{100}, \\
  \text{Р}_{\text{обп}} &= 19134{,}4 \cdot 1{,}8 =
    34~441{,}8 \: \text{р.} \nonumber
\end{align}

Расчёт затрат по статье <<Общехозяйственные расходы>>
произведём в соответствии с выражением~\ref{eq:general_economic}.
\begin{align}
  \label{eq:general_economic}
  \text{Р}_{\text{обх}} &= \dfrac{\text{З}_{\text{о}} \cdot
    \text{Н}_{\text{обх}}}{100}, \\
  \text{Р}_{\text{обх}} &= 19134{,}4 \cdot 2 =
    38~268{,}7 \: \text{р.} \nonumber
\end{align}

Расчёт затрат по статье <<Прочие производственные расходы>>
произведём в соответствии с выражением~\ref{eq:other}.
\begin{align}
  \label{eq:other}
  \text{Р}_{\text{пр}} &= \dfrac{\text{З}_{\text{о}} \cdot
    \text{Н}_{\text{пр}}}{100}, \\
  \text{Р}_{\text{пр}} &= 19134{,}4 \cdot 0{,}03 =
    574{,}0 \: \text{р.} \nonumber
\end{align}

Производственная себестоимость представляет собой сумму приведенных статей, за
вычетом возвратных отходов:
\begin{align}
  \text{С}_{\text{пр}} =&~20676{,}8 + 17490 - 206{,}8 + 19134{,}4~ + \\ \nonumber
    &+ 2870{,}2 + 7481{,}5 + 132{,}0 + 3826{,}7 +             \\ \nonumber
    &+ 34441{,}8 + 38268{,}7 + 7352{,}3 = 144~689{,}5 \: \text{р.} \nonumber
\end{align}

Расчёт расходов на реализацию произведём в соответствии с
выражением~\ref{eq:realization}.
\begin{align}
  \label{eq:realization}
  \text{Р}_{\text{реа}} &= \dfrac{\text{С}_{\text{пр}} \cdot
    \text{Н}_{\text{реа}}}{100}, \\
  \text{Р}_{\text{реа}} &= 144689{,}5 \cdot 0{,}04 =
    5~787{,}6 \: \text{р.} \nonumber
\end{align}

Полную себестоимость единицы продукции рассчитаем в соответствии с
выражением~\ref{eq:production_cost}.
\begin{align}
  \label{eq:production_cost}
  \text{С}_{\text{п}} &= \text{С}_{\text{пр}} + \text{Р}_{\text{реа}}, \\
  \text{С}_{\text{п}} &= 144689{,}5 + 5787{,}6 =
    150~477{,}1 \: \text{р.} \nonumber
\end{align}

\newpage

Результаты калькуляции себестоимости единицы продукции
сведены в таблице~\ref{tbl:results}.
\begin{table}[h!]
  \caption{Калькуляция себестоимости единицы продукции}
  \label{tbl:results}
  \centering
  \small{
    \begin{tabular}{| p{0.34\textwidth} | p{0.14\textwidth} |
                      p{0.26\textwidth} | p{0.14\textwidth} |}
      \hline

      Наименование статьи & Условное \newline обозначение &
      Характеристика статьи & Результат, р.
      \\ \hline

      Сырьё и материалы & $ \text{Р}_{\text{м}} $ &
      простая, определяется прямым счётом &
      20 676,8 \\ \hline

      Отходы возвратные & $ \text{О}_{\text{в}} $ &
      простая, определяется прямым счётом &
      206,8 \\ \hline

      Покупные комплектующие, изделия, полуфабрикаты, работы и услуги
      производственного характера & $ \text{Р}_{\text{к}} $ &
      простая, определяется прямым счётом &
      17 490 \\ \hline

      Топливо и энергия на технологические цели & $ \text{Р}_{\text{э}} $ &
      простая, определяется прямым счётом только в энергоёмких отраслях &
      - \\ \hline

      Основная заработная плата производственных рабочих & $ \text{З}_{\text{о}} $ &
      простая, определяется прямым счётом &
      19 134,4 \\ \hline

      Дополнительная \newline заработная плата производственных рабочих
      & $ \text{З}_{\text{д}} $ & простая &
      2 870,2 \\ \hline

      Отчисления на социальные нужды & $ \text{Р}_{\text{соц}} $ &
      простая &
      7 481,5 \\ \hline

      Отчисления на страхование от несчастных случаев на \newline производстве &
      $ \text{Р}_{\text{стр}} $ & простая &
      132,0 \\ \hline

      Возмещение износа специнструмента и спецоснастки & $ \text{Р}_{\text{из}} $ &
      комплексная, по смете &
      3 826,7 \\ \hline

      Общепроизводственные \newline расходы & $ \text{Р}_{\text{обп}} $ &
      комплексная, по смете &
      34 441,8 \\ \hline

      Общехозяйственные расходы & $ \text{Р}_{\text{обх}} $ &
      комплексная, по смете &
      38 268,7 \\ \hline

      Прочие производственные \newline расходы & $ \text{Р}_{\text{пр}} $ &
      комплексная, по смете &
      574,0 \\ \hline

      Расходы на реализацию & $ \text{Р}_{\text{реа}} $ &
      комплексная, по смете &
      5 787,6 \\ \hline

    \end{tabular}
  }
\end{table}

\newpage

Расчёт нормативной прибыли произведём в соответствии с
выражением~\ref{eq:norm_profit}.
\begin{align}
  \label{eq:norm_profit}
  \text{П}_{\text{ед}} &= \dfrac{\text{С}_{\text{п}} \cdot
    \text{Р}_{\text{ед}}}{100}, \\
  \text{П}_{\text{ед}} &= 150477{,}1 \cdot 0{,}3 =
    45~143{,}1 \: \text{р.} \nonumber
\end{align}

Расчёт цены предприятия произведём в соответствии с
выражением~\ref{eq:venture_price}.
\begin{align}
  \label{eq:venture_price}
  \text{Ц}_{\text{пред}} &= \text{С}_{\text{п}} + \text{П}_{\text{ед}}, \\
  \text{Ц}_{\text{пред}} &= 150477{,}1 + 45143{,}1 =
    195~620{,}3 \: \text{р.} \nonumber
\end{align}

Налог на добавленную стоимость рассчитаем в соответствии с
выражением~\ref{eq:VAT}.
\begin{align}
  \label{eq:VAT}
  \text{НДС} &= \dfrac{\text{Ц}_{\text{пред}} \cdot \text{Н}_{\text{дс}}}{100}, \\
  \text{НДС} &= 195620{,}3 \cdot 0{,}2 = 39~124{,}1 \: \text{р.} \nonumber
\end{align}

Расчёт отпускной цены единицы продукции произведём в соответствии с
выражением~\ref{eq:final_price}.
\begin{align}
  \label{eq:final_price}
  \text{Ц}_{\text{отп}} &= \text{Ц}_{\text{пред}} + \text{НДС}, \\
  \text{Ц}_{\text{отп}} &= 195620{,}3 + 39124{,}1 =
    234~744{,}3 \: \text{р.} \nonumber
\end{align}
