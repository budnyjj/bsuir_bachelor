\section{ХОД РАБОТЫ}

\subsection{Исходные данные}

Исходные данные для расчёта плановой себестоимости и отпускной цены пожарного
дымового извещателя ИП-212 сведены в
таблицах~\ref{tbl:technological_process}--\ref{tbl:indirect_costs}.

\begin{table}[h!]
  \caption{Технологический процесс производства}
  \label{tbl:technological_process}
  \centering
  \small{
    \begin{tabular}{| p{0.31\textwidth} | p{0.3\textwidth} | p{0.3\textwidth} |}
      \hline

      Виды работ (операции) & Разряд работ & Нормы времени, ч \\ \hline

      \multicolumn{3}{|c|}{Изготовление печатной платы} \\ \hline
      1. Подготовительная & 3 & 0,5  \\ \hline
      2. Монтажная        & 5 & 0,2  \\ \hline
      3. Контрольная      & 4 & 0,01 \\ \hline
      4. Сборочная        & 3 & 0,02 \\ \hline
      5. Маркировочная    & 3 & 0,03 \\ \hline
      6. Лакировочная     & 3 & 0,15 \\ \hline
      7. Контрольная      & 5 & 0,3  \\ \hline

      \multicolumn{3}{|c|}{Изготовление розетки} \\ \hline
      1. Подготовительная & 3 & 0,15 \\ \hline
      2. Сборочная        & 4 & 0,2  \\ \hline
      3. Контрольная      & 4 & 0,03 \\ \hline

      \multicolumn{3}{|c|}{Изготовление извещателя} \\ \hline
      1. Подготовительная & 3 & 0,3  \\ \hline
      2. Сборочная        & 4 & 0,5  \\ \hline
      3. Монтажная        & 5 & 0,3  \\ \hline
      4. Сборочная        & 4 & 0,25 \\ \hline
      5. Регулировочная   & 6 & 1,7  \\ \hline
      6. Сборочная        & 4 & 0,3  \\ \hline
      7. Испытательная    & 6 & 4,5  \\ \hline
      8. Контрольная      & 5 & 0,9  \\ \hline
      9. Сдача ОТК        & 6 & 0,24 \\ \hline

    \end{tabular}
  }
\end{table}

\begin{table}[h!]
  \caption{Нормы расхода материалов на одно изделие}
  \label{tbl:material_costs}
  \centering
  \small{
    \begin{tabular}{| p{0.28\textwidth} | p{0.22\textwidth} |
                      p{0.17\textwidth} | p{0.21\textwidth} |}
      \hline
      Наименование материала & Единица измерения &
      Норма расхода & Цена за единицу, р. \\ \hline

      1. Припой             & кг & 0,07  & 15000 \\ \hline
      2. Провод             & м  & 0,5   & 4600  \\ \hline
      3. Флюс канифольный   & л  & 0,01  & 9600  \\ \hline
      4. Лак                & л  & 0,06  & 16000 \\ \hline
      5. Спирт технический  & л  & 0,025 & 9600  \\ \hline
      6. Клей               & кг & 0,08  & 4500  \\ \hline
      7. Масло              & л  & 0,01  & 9340  \\ \hline
      8. Труба              & м  & 0,5   & 2950  \\ \hline

    \end{tabular}
  }
\end{table}

\begin{table}[h!]
  \caption{Нормы расхода комплектующих изделий}
  \label{tbl:set_costs}
  \centering
  \small{
    \begin{tabular}{| p{0.34\textwidth} | p{0.14\textwidth} |
                      p{0.2\textwidth} | p{0.2\textwidth} |}
      \hline
      Наименование ПКИ & Единица \newline измерения &
      Количество \newline на изделие & Цена за единицу, \newline р. \\ \hline

      1. Резисторы                & шт. & 9,08 & 90   \\ \hline
      2. Конденсаторы             & шт. & 3,03 & 240  \\ \hline
      3. Микросхема К561ЛЕ5       & шт. & 1,01 & 360  \\ \hline
      4. Микросхема К561НЕ16      & шт. & 1,01 & 450  \\ \hline
      5. Транзистор КТ3102БМ      & шт. & 2,02 & 150  \\ \hline
      6. Транзисторы              & шт. & 3,03 & 180  \\ \hline
      7. Полупроводниковый прибор & шт. & 2,02 & 600  \\ \hline
      8. Оповещатель <<Свирель>>  & шт. & 1,01 & 1050 \\ \hline
      9. Колодка КМ-2             & шт. & 5,05 & 240  \\ \hline
      10. Микропереключатель      & шт. & 2    & 285  \\ \hline
      11. Печатная плата          & шт. & 1,01 & 1050 \\ \hline
      12. Липкая аппликация       & шт. & 1    & 30   \\ \hline
      13. Крышка                  & шт. & 1    & 360  \\ \hline
      14. Основание               & шт. & 1    & 450  \\ \hline
      15. Прочие комплектующие    &     & -    & 1020 \\ \hline

    \end{tabular}
  }
\end{table}

\begin{table}[h!]
  \caption{Единая тарифная сетка работников Республики Беларусь}
  \label{tbl:tariffs_grid}
  \centering
  \small{
    \begin{tabular}{| p{0.28\textwidth} | p{0.11\textwidth} |
                      p{0.11\textwidth} | p{0.11\textwidth} |
                      p{0.11\textwidth} | p{0.11\textwidth} |}
      \hline
      Разряды & 2 & 3 & 4 & 5 & 6 \\ \hline

      Тарифный коэффициент & 1,16 & 1,35 & 1,57 & 1,74 & 1,9 \\ \hline

    \end{tabular}
  }
\end{table}

\begin{table}[h!]
  \caption{Нормативы косвенных расходов, налогов и отчислений}
  \label{tbl:indirect_costs}
  \centering
  \small{
    \begin{tabular}{| p{0.6\textwidth} | p{0.15\textwidth} | p{0.15\textwidth} |}
      \hline
      Показатель & Условное \newline обозначение & Величина, \% \\ \hline

      Норматив дополнительной зарплаты &
      $ \text{Н}_{\text{д}} $ & 15 \\ \hline

      Норматив возмещения износа &
      $ \text{Н}_{\text{из}} $ & 10 \\ \hline

      Норматив общепроизводственных расходов &
      $ \text{Н}_{\text{обп}} $ & 200 \\ \hline

      Норматив общехозяйственных расходов &
      $ \text{Н}_{\text{обх}} $ & 150 \\ \hline

      Норматив прочих производственных расходов &
      $ \text{Н}_{\text{пр}} $ & 3 \\ \hline

      Норматив расходов на реализацию &
      $ \text{Н}_{\text{ком}} $ & 4 \\ \hline

      Норматив рентабельности единицы продукции &
      $ \text{Р}_{\text{ед}} $ & 20 \\ \hline

      Норматив отчислений в фонд социальной защиты населения &
      $ \text{Н}_{\text{соц}} $ & 34 \\ \hline

      Норматив отчислений на обязательное страхование &
      $ \text{Н}_{\text{стр}} $ & 0,6 \\ \hline

      Ставка налога на добавленную стоимость &
      $ \text{Н}_{\text{дс}} $ & 20 \\ \hline

    \end{tabular}
  }
\end{table}


\subsection{Решение}

Произведём расчёт затрат на материалы путём умножения нормы расхода материала
на цену за единицу конкретного материала. Полученные данные представим в
таблице~\ref{tbl:material_costs_result}.

\begin{table}[h!]
  \caption{Расчёт затрат на материалы}
  \label{tbl:material_costs_result}
  \centering
  \small{
    \begin{tabular}{| p{0.31\textwidth} | p{0.12\textwidth} |
                      p{0.17\textwidth} | p{0.13\textwidth} | p{0.12\textwidth} |}
      \hline
      Наименование материала & Единица \newline измерения &
      Норма расхода & Цена за \newline единицу, р. & Сумма, р. \\ \hline

      1. Припой             & кг & 0,07  & 15000 & 1050   \\ \hline
      2. Провод             & м  & 0,5   & 4600  & 2300   \\ \hline
      3. Флюс канифольный   & л  & 0,01  & 9600  & 96     \\ \hline
      4. Лак                & л  & 0,06  & 16000 & 960    \\ \hline
      5. Спирт технический  & л  & 0,025 & 9600  & 240    \\ \hline
      6. Клей               & кг & 0,08  & 4500  & 360    \\ \hline
      7. Масло              & л  & 0,01  & 9340  & 93,4   \\ \hline
      8. Труба              & м  & 0,5   & 2950  & 1475   \\ \hline
      Итого                 &    &       &       & 6 574,4 \\ \hline
      Итого с учётом \newline
      транспортно-заготовительных расходов & & & & 7 231,8 \\ \hline

    \end{tabular}
  }
\end{table}

Расчёт затрат на комплектующие изделия произведём путём умножения цены за единицу
материала на норму расхода данного материала. Полученные данные проиллюстрируем
в таблице~\ref{tbl:set_costs_result}.

\begin{table}[h!]
  \caption{Расчёт затрат на комплектующие изделия}
  \label{tbl:set_costs_result}
  \centering
  \small{
    \begin{tabular}{| p{0.34\textwidth} | p{0.12\textwidth} |
                      p{0.14\textwidth} | p{0.13\textwidth} | p{0.13\textwidth} |}
      \hline
      Наименование ПКИ & Единица \newline измерения &
      Количество \newline на изделие & Цена за \newline единицу, р. &
      Сумма, р. \\ \hline

      1. Резисторы                & шт. & 9,08 & 90   & 817,2   \\ \hline
      2. Конденсаторы             & шт. & 3,03 & 240  & 727,2   \\ \hline
      3. Микросхема К561ЛЕ5       & шт. & 1,01 & 360  & 363,6   \\ \hline
      4. Микросхема К561НЕ16      & шт. & 1,01 & 450  & 454,5   \\ \hline
      5. Транзистор КТ3102БМ      & шт. & 2,02 & 150  & 303     \\ \hline
      6. Транзисторы              & шт. & 3,03 & 180  & 545,4   \\ \hline
      7. Полупроводниковый прибор & шт. & 2,02 & 600  & 1212    \\ \hline
      8. Оповещатель <<Свирель>>  & шт. & 1,01 & 1050 & 1060,5  \\ \hline
      9. Колодка КМ-2             & шт. & 5,05 & 240  & 1212    \\ \hline
      10. Микропереключатель      & шт. & 2    & 285  & 570     \\ \hline
      11. Печатная плата          & шт. & 1,01 & 1050 & 1060,5  \\ \hline
      12. Липкая аппликация       & шт. & 1    & 30   & 30      \\ \hline
      13. Крышка                  & шт. & 1    & 360  & 360     \\ \hline
      14. Основание               & шт. & 1    & 450  & 450     \\ \hline
      15. Прочие комплектующие    &     & -    & 1020 & 1020    \\ \hline
      Итого                       &    &       &     & 10 185,9 \\ \hline
      Итого с учётом \newline
      транспортно-заготовительных расходов & & & & 11 204,5     \\ \hline

    \end{tabular}
  }
\end{table}

Расчёт затрат по статье <<Возвратные отходы>>, которые принимаются в размере
1 \% от стоимости материалов с учётом транспортно-заготовительных расходов:
\begin{align}
  \text{O}_{\text{В}} = 7231{,}8 \cdot 0{,}01 = 72{,}2 \: \text{р.}  \nonumber
\end{align}

\newpage

Для расчёта заработной платы необходимо рассчитать часовые тарифные ставки по
видам работ, которые представлены в таблице~\ref{tbl:technological_process}
с учётом тарифных коэффициентов, приведенных в таблице~\ref{tbl:tariffs_grid}.
Тарифная ставка первого разряда установлена в размере $ 9821 $ рублей.

\begin{table}[h!]
  \caption{Расчёт основной заработной платы производственных рабочих}
  \label{tbl:technological_process_result}
  \centering
  \small{
    \begin{tabular}{| p{0.27\textwidth} | p{0.14\textwidth} | p{0.15\textwidth} |
                      p{0.15\textwidth} | p{0.15\textwidth} |}
      \hline
      Виды работ (операции) &
      Разряд работ &
      Часовая \newline тарифная \newline ставка, р/ч &
      Нормы \newline времени, ч &
      Расценка, р \\ \hline

      \multicolumn{5}{|c|}{Изготовление печатной платы} \\ \hline
      1. Подготовительная & 3 & 13258 & 0,5  & 6629,2  \\ \hline
      2. Монтажная        & 5 & 17089 & 0,2  & 3417,7  \\ \hline
      3. Контрольная      & 4 & 15419 & 0,01 & 154,2   \\ \hline
      4. Сборочная        & 3 & 13258 & 0,02 & 265,2   \\ \hline
      5. Маркировочная    & 3 & 13258 & 0,03 & 397,8   \\ \hline
      6. Лакировочная     & 3 & 13258 & 0,15 & 1988,8  \\ \hline
      7. Контрольная      & 5 & 17089 & 0,3  & 5126,6  \\ \hline

      \multicolumn{5}{|c|}{Изготовление розетки} \\ \hline
      1. Подготовительная & 3 & 13258 & 0,15 & 1988,8  \\ \hline
      2. Сборочная        & 4 & 15419 & 0,2  & 3083,8  \\ \hline
      3. Контрольная      & 4 & 15419 & 0,03 & 4625,7  \\ \hline

      \multicolumn{5}{|c|}{Изготовление извещателя} \\ \hline
      1. Подготовительная & 3 & 13258 & 0,3  & 3977,5  \\ \hline
      2. Сборочная        & 4 & 15419 & 0,5  & 7709,5  \\ \hline
      3. Монтажная        & 5 & 17089 & 0,3  & 5126,6  \\ \hline
      4. Сборочная        & 4 & 15419 & 0,25 & 3854,7  \\ \hline
      5. Регулировочная   & 6 & 18660 & 1,7  & 31721,8 \\ \hline
      6. Сборочная        & 4 & 15419 & 0,3  & 4625,7  \\ \hline
      7. Испытательная    & 6 & 18660 & 4,5  & 83969,6 \\ \hline
      8. Контрольная      & 5 & 17089 & 0,9  & 15379,7 \\ \hline
      9. Сдача ОТК        & 6 & 18660 & 0,24 & 4478,4  \\ \hline

      Итого               & & & & 188 521,0            \\ \hline
      Премия $30\%$       & & & & 56 556,3             \\ \hline
      Итого с премией     & & & & 245 077,3            \\ \hline

    \end{tabular}
  }
\end{table}

\newpage

Расчёт затрат по статье <<Дополнительная заработная плата производственных
рабочих>> произведём в соответствии с выражением~\ref{eq:extra_salary}.
\begin{align}
  \label{eq:extra_salary}
  \text{З}_{\text{д}} &= \dfrac{\text{З}_{\text{о}} \cdot
    \text{Н}_{\text{д}}}{100}, \\
  \text{З}_{\text{д}} &= 245077{,}3 \cdot 0{,}15 =
    36~761{,}6 \: \text{р.} \nonumber
\end{align}

Расчёт затрат по статье <<Отчисления в фонд социальной защиты населения>>
произведём в соответствии с выражением~\ref{eq:social_fund}.
\begin{align}
  \label{eq:social_fund}
  \text{Р}_{\text{соц}} &= \dfrac{(\text{З}_{\text{о}} + \text{З}_{\text{д}}) \cdot
    \text{Н}_{\text{соц}}}{100}, \\
  \text{Р}_{\text{соц}} &= (245077{,}3 + 36761{,}6) \cdot 0{,}34 =
    95~825{,}2 \: \text{р.} \nonumber
\end{align}

Расчёт затрат по статье <<Обязательное страхование от несчастных случаев
на производстве>> произведём в соответствии с выражением~\ref{eq:insurance}.
\begin{align}
  \label{eq:insurance}
  \text{Р}_{\text{стр}} &= \dfrac{(\text{З}_{\text{о}} + \text{З}_{\text{д}}) \cdot
    \text{Н}_{\text{стр}}}{100}, \\
  \text{Р}_{\text{стр}} &= (245077{,}3 + 36761{,}6) \cdot 0{,}006 =
    1~691{,}0 \: \text{р.} \nonumber
\end{align}

Расчёт затрат по статье <<Погашение стоимости инструментов и приспособлений
целевого назначения и прочие специальные расходы>>
произведём в соответствии с выражением~\ref{eq:repay}.
\begin{align}
  \label{eq:repay}
  \text{Р}_{\text{из}} &= \dfrac{\text{З}_{\text{о}} \cdot
    \text{Н}_{\text{из}}}{100}, \\
  \text{Р}_{\text{из}} &= 245077{,}3 \cdot 0{,}1 =
    24~507{,}7 \: \text{р.} \nonumber
\end{align}

Расчёт затрат по статье <<Общепроизводственные расходы>>
произведём в соответствии с выражением~\ref{eq:general_production}.
\begin{align}
  \label{eq:general_production}
  \text{Р}_{\text{обп}} &= \dfrac{\text{З}_{\text{о}} \cdot
    \text{Н}_{\text{обп}}}{100}, \\
  \text{Р}_{\text{обп}} &= 245077{,}3 \cdot 2 =
    490~154{,}5 \: \text{р.} \nonumber
\end{align}

Расчёт затрат по статье <<Общехозяйственные расходы>>
произведём в соответствии с выражением~\ref{eq:general_economic}.
\begin{align}
  \label{eq:general_economic}
  \text{Р}_{\text{обх}} &= \dfrac{\text{З}_{\text{о}} \cdot
    \text{Н}_{\text{обх}}}{100}, \\
  \text{Р}_{\text{обх}} &= 245077{,}3 \cdot 1{,}5 =
    367~615{,}9 \: \text{р.} \nonumber
\end{align}

Расчёт затрат по статье <<Прочие производственные расходы>>
произведём в соответствии с выражением~\ref{eq:other}.
\begin{align}
  \label{eq:other}
  \text{Р}_{\text{пр}} &= \dfrac{\text{З}_{\text{о}} \cdot
    \text{Н}_{\text{пр}}}{100}, \\
  \text{Р}_{\text{пр}} &= 245077{,}3 \cdot 0{,}03 =
    7~352{,}3 \: \text{р.} \nonumber
\end{align}

Производственная себестоимость представляет собой сумму приведенных статей, за
вычетом возвратных отходов:
\begin{align}
  \text{С}_{\text{пр}} =&~7231{,}8 + 11204{,}5 - 72{,}2 + 245077{,}3~ + \\ \nonumber
    &+ 36761{,}6 + 95825{,}2 + 1691{,}0 + 24507{,}7 +             \\ \nonumber
    &+ 490154{,}5 + 367615{,}9 + 7352{,}3 = 1~287~349{,}6 \: \text{р.} \nonumber
\end{align}

Расчёт расходов на реализацию произведём в соответствии с
выражением~\ref{eq:realization}.
\begin{align}
  \label{eq:realization}
  \text{Р}_{\text{реа}} &= \dfrac{\text{С}_{\text{пр}} \cdot
    \text{Н}_{\text{реа}}}{100}, \\
  \text{Р}_{\text{реа}} &= 1287349{,}6 \cdot 0{,}04 =
    51~494{,}0 \: \text{р.} \nonumber
\end{align}

Полную себестоимость единицы продукции рассчитаем в соответствии с
выражением~\ref{eq:production_cost}.
\begin{align}
  \label{eq:production_cost}
  \text{С}_{\text{п}} &= \text{С}_{\text{пр}} + \text{Р}_{\text{реа}}, \\
  \text{С}_{\text{п}} &= 1287349{,}6 + 51494{,}0 =
    1~338~843{,}5 \: \text{р.} \nonumber
\end{align}

\newpage

Результаты калькуляции себестоимости единицы продукции
сведены в таблице~\ref{tbl:results}.
\begin{table}[h!]
  \caption{Калькуляция себестоимости единицы продукции}
  \label{tbl:results}
  \centering
  \small{
    \begin{tabular}{| p{0.34\textwidth} | p{0.14\textwidth} |
                      p{0.26\textwidth} | p{0.14\textwidth} |}
      \hline

      Наименование статьи & Условное \newline обозначение &
      Характеристика статьи & Результат, р.
      \\ \hline

      Сырьё и материалы & $ \text{Р}_{\text{м}} $ &
      простая, определяется прямым счётом &
      7 231,8 \\ \hline

      Отходы возвратные & $ \text{О}_{\text{в}} $ &
      простая, определяется прямым счётом &
      72,2 \\ \hline

      Покупные комплектующие, изделия, полуфабрикаты, работы и услуги
      производственного характера & $ \text{Р}_{\text{к}} $ &
      простая, определяется прямым счётом &
      11 204,5 \\ \hline

      Топливо и энергия на технологические цели & $ \text{Р}_{\text{э}} $ &
      простая, определяется прямым счётом только в энергоёмких отраслях &
      - \\ \hline

      Основная заработная плата производственных рабочих & $ \text{З}_{\text{о}} $ &
      простая, определяется прямым счётом &
      245 077,3 \\ \hline

      Дополнительная \newline заработная плата производственных рабочих
      & $ \text{З}_{\text{д}} $ & простая &
      36 761,6 \\ \hline

      Отчисления на социальные нужды & $ \text{Р}_{\text{соц}} $ &
      простая &
      95 825,2 \\ \hline

      Отчисления на страхование от несчастных случаев на \newline производстве &
      $ \text{Р}_{\text{стр}} $ & простая &
      1 691,0 \\ \hline

      Возмещение износа специнструмента и спецоснастки & $ \text{Р}_{\text{из}} $ &
      комплексная, по смете &
      24 507,7 \\ \hline

      Общепроизводственные \newline расходы & $ \text{Р}_{\text{обп}} $ &
      комплексная, по смете &
      490 154,5 \\ \hline

      Общехозяйственные расходы & $ \text{Р}_{\text{обх}} $ &
      комплексная, по смете &
      367 615,9 \\ \hline

      Прочие производственные \newline расходы & $ \text{Р}_{\text{пр}} $ &
      комплексная, по смете &
      7 352,3 \\ \hline

      Расходы на реализацию & $ \text{Р}_{\text{реа}} $ &
      комплексная, по смете &
      51 494,0 \\ \hline

    \end{tabular}
  }
\end{table}

\newpage

Расчёт нормативной прибыли произведём в соответствии с
выражением~\ref{eq:norm_profit}.
\begin{align}
  \label{eq:norm_profit}
  \text{П}_{\text{ед}} &= \dfrac{\text{С}_{\text{п}} \cdot
    \text{Р}_{\text{ед}}}{100}, \\
  \text{П}_{\text{ед}} &= 1338843{,}5 \cdot 0{,}2 =
    267~768{,}7 \: \text{р.} \nonumber
\end{align}

Расчёт цены предприятия произведём в соответствии с
выражением~\ref{eq:venture_price}.
\begin{align}
  \label{eq:venture_price}
  \text{Ц}_{\text{пред}} &= \text{С}_{\text{п}} + \text{П}_{\text{ед}}, \\
  \text{Ц}_{\text{пред}} &= 1338843{,}5 + 267768{,}7 =
    1~606~612{,}3 \: \text{р.} \nonumber
\end{align}

Налог на добавленную стоимость рассчитаем в соответствии с
выражением~\ref{eq:VAT}.
\begin{align}
  \label{eq:VAT}
  \text{НДС} &= \dfrac{\text{Ц}_{\text{пред}} \cdot \text{Н}_{\text{дс}}}{100}, \\
  \text{НДС} &= 1606612{,}3 \cdot 0{,}2 = 321~322{,}5 \: \text{р.} \nonumber
\end{align}

Расчёт отпускной цены единицы продукции произведём в соответствии с
выражением~\ref{eq:final_price}.
\begin{align}
  \label{eq:final_price}
  \text{Ц}_{\text{отп}} &= \text{Ц}_{\text{пред}} + \text{НДС}, \\
  \text{Ц}_{\text{отп}} &= 1606612{,}3 + 321322{,}5 =
    1~927~934{,}7 \: \text{р.} \nonumber
\end{align}
