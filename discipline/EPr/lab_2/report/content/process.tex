\section{ХОД РАБОТЫ}

\subsection{Задание}

Необходимо рассчитать количественные оценки факторов, влияющие
на относительное изменение фондоотдачи в цехе за исследуемый период.

Для решения данной задачи имеем следующие исходные данные:

\begin{itemize}
  \item число видов оборудования цеха, охваченных наблюдением~--- 10~($ j~=~10 $);
  \item количество единиц оборудования по каждому виду~---~1~($ n_j = 1$);
  \item число дней работы одного станка по плану в период наблюдения~---~2~($ \text{Д}_j^{\text{п}} = 2 $);
  \item число смен работы одного станка по плану в период наблюдения~---~2~($ S^\text{п}_j = 2 $);
  \item продолжительность работы одного станка за смену по плану~---~8~($ t^\text{п}_{\text{д}j} = 8 $);
  \item плановый процент потерь рабочего времени одного станка на ремонт и
    переналадку~---~4~($ \text{Н}^\text{п}_{\text{р}j} = 4 $).
\end{itemize}

\begin{table}[h!]
  \caption{Исходные данные для решения задачи}
  \label{tbl:bicycles_scheme}
    \centering
    \begin{tabular}{| p{0.46\textwidth} | p{0.49\textwidth} |}
      \hline
      Норма времени на изготовление \newline единицы продукции на $ j $-м станке \newline по плану $ t^{\text{п}}_{\text{м}j} $, мин/опер. &
      Фактическое увеличение времени \newline на изготовление единицы продукции на $ j $-м станке под влиянием \newline $ k $-ого фактора $ \Delta t^{\text{п}}_{\text{м}j} $, мин/опер. \\ \hline

      $ t^{\text{п}}_{\text{м}1}  = 12 $ \hskip20pt $ t^{\text{п}}_{\text{м}2}  = 15 $ \hskip20pt $ t^{\text{п}}_{\text{м}3}  = 18 $ &
      $ \Delta t^{\text{п}}_{\text{м}1}  = 1{,}0 $ \hskip8pt $ \Delta t^{\text{п}}_{\text{м}2}  = 1{,}2 $ \hskip8pt $ \Delta t^{\text{п}}_{\text{м}3}  = 1{,}2 $ \\

      $ t^{\text{п}}_{\text{м}4}  = 19 $ \hskip20pt $ t^{\text{п}}_{\text{м}5}  = 10 $ \hskip20pt $ t^{\text{п}}_{\text{м}6}  = 20 $ &
      $ \Delta t^{\text{п}}_{\text{м}4}  = 1{,}2 $ \hskip8pt $ \Delta t^{\text{п}}_{\text{м}5}  = 1{,}1 $ \hskip8pt $ \Delta t^{\text{п}}_{\text{м}6}  = 1{,}5 $ \\

      $ t^{\text{п}}_{\text{м}7}  = 16 $ \hskip20pt $ t^{\text{п}}_{\text{м}8}  = 18 $ \hskip20pt $ t^{\text{п}}_{\text{м}9}  = 12 $ &
      $ \Delta t^{\text{п}}_{\text{м}7}  = 1{,}2 $ \hskip8pt $ \Delta t^{\text{п}}_{\text{м}8}  = 1{,}4 $ \hskip8pt $ \Delta t^{\text{п}}_{\text{м}9}  = 1{,}6 $ \\

      $ t^{\text{п}}_{\text{м}10}  = 12 $ & $ \Delta t^{\text{п}}_{\text{м}10}  = 1{,}0 $ \\

      \hline
    \end{tabular}
\end{table}

\newpage

\subsection{Решение задачи}

Решение задачи сводится к последовательному нахождению показателей,
влияющих на относительное изменение фондоотдачи в цехе за исследуемый период.

Далее приведём эти показатели в той последовательности, в которой они будут вычисляться.

\begin{enumerate}

  \item $ \text{Ф}^{\text{п}}_{\text{э}j} = \text{Д}^{\text{п}}_j \cdot S^{\text{п}}_j
    \cdot t^{\text{п}}_{\text{д}j} \cdot \Big(1 - \dfrac{\text{Н}^{\text{п}}_{\text{р}j}}{100} \Big) $
    --- эффективный фонд времени работы $j$-ого станка за исследуемый период \textit{по плану}, ч.

  \item $ \text{Ф}^{\text{п}}_{\text{э}} = \sum\limits_{j=1}^{m}n_j \text{Ф}^{\text{п}}_{\text{э}j} $
    --- эффективный фонд времени работы исследуемого парка оборудования за период наблюдения
    \textit{по плану}, ч.

  \item $ \text{В}^{\text{п}}_{\text{ч}} = \dfrac{60}{\sum\limits_{j=1}^{n} t^{\text{п}}_{\text{м}j}} $
    --- среднечасовая производительность исследуемого парка оборудования \textit{по плану}, шт/ч.

  \item $ N^{\text{п}} = \text{Ф}^{\text{п}}_{\text{э}} \cdot \text{В}^{\text{п}}_{\text{ч}} $
    --- объём выпуска продукции \textit{по плану} за исследуемый период, шт.

  \item $ \text{В}^{\text{ф}}_{\text{ч}} = \dfrac{60}{\sum\limits_{j=1}^{n} t^{\text{п}}_{\text{м}j}
    + \Delta t_{\text{м}kj}} $
    --- \textit{фактическая} среднечасовая производительность исследуемого парка оборудования, шт/ч.

  \item $ \text{Ф}^{\text{ф}}_{\text{э}} = \text{Ф}^{\text{п}}_{\text{э}} -
    \sum\limits_{i=1}^{n}\sum\limits_{j=1}^{m} \text{П}_{\text{вр}ij} $
    --- \textit{фактический} эффективный фонд времени работы исследуемого парка оборудования за период наблюдения, ч.

  \item $ N^{\text{ф}} = \text{Ф}^{\text{ф}}_{\text{э}} \cdot \text{В}^{\text{ф}}_{\text{ч}} $
    --- \textit{фактический} объём выпуска продукции за исследуемый период, шт.

  \item $ \Delta N = N^{\text{ф}} - N^{\text{п}} $
    --- общее абсолютное изменение выпуска продукции за период наблюдения под влиянием исследуемых факторов, шт.

    $ \Delta N_{\text{фэ}} = (\text{Ф}^{\text{ф}}_{\text{э}} - \text{Ф}^{\text{п}}_{\text{э}}) \cdot \text{В}^{\text{п}}_{\text{ч}} $
    --- изменение объёма выпуска продукции в исследуемом периоде за счёт изменения количества отработанного
    оборудованием времени, шт.

    $ \Delta N_{\text{вч}} = (\text{В}^{\text{ф}}_{\text{ч}} - \text{В}^{\text{п}}_{\text{ч}}) \cdot \text{Ф}^{\text{ф}}_{\text{э}} $
    --- изменение объёма выпуска продукции в исследуемом периоде за счёт изменения среднечасовой производительности оборудования, шт.

  \item $ \alpha_i = \dfrac{\text{П}_{\text{вр}i}}{\text{Ф}^{\text{п}}_{\text{э}}} \cdot 100 $
    --- удельный вес потерь рабочего времени исследуемого парка оборудования от действия $i$-ого фактора, \%.

    $ \alpha = \sum\limits^{n}_{i=1}\alpha_i $
    --- удельный вес потерь рабочего времени исследуемого парка оборудования от действия всех $i$-ых факторов, \%.

  \item $ \text{К}_{\text{э}} = \dfrac{\text{Ф}^{\text{ф}}_{\text{э}}}{\text{Ф}^{\text{п}}_{\text{э}}} $
    --- фактический коэффициент экстенсивного использования парка оборудования.

    $ \text{К}_{\text{ин}} = \dfrac{\text{В}^{\text{ф}}_{\text{ч}}}{\text{В}^{\text{п}}_{\text{ч}}} $
    --- фактический коэффициент интенсивного использования парка оборудования.

    $ \text{К}_{\text{и}} = \text{К}_{\text{ин}} \cdot \text{К}_{\text{э}} $
    --- фактический коэффициент интегрального использования парка оборудования.

  \item $ \Delta \% \text{Ф}_{\text{о}} = \dfrac{\Delta N}{N^{\text{п}}} \cdot 100 $
    --- общее относительное изменение фондоотдачи в анализируемом периоде, \%.

    $ \Delta \% \text{Ф}_{\text{офэ}} = \dfrac{\Delta N_{\text{фэ}}}{N^{\text{п}}} \cdot 100 $
    --- относительное изменение фондоотдачи в исследуемом периоде за счёт изменения эффективного
    фонда рабочего времени оборудования, \%.

    $ \Delta \% \text{Ф}_{\text{овч}} = \dfrac{\Delta N_{\text{вч}}}{N^{\text{п}}} \cdot 100 $
    --- относительное изменение фондоотдачи в исследуемом периоде за счёт изменения среднечасовой
    производительности оборудования, \%.

\end{enumerate}

Последовательно рассчитаем все показатели, влияющие на относительное изменение фондоотдачи
в цехе за исследуемый период. Результаты представим в таблице~\ref{tbl:results}.

\begin{table}[h!]
  \caption{Сводная таблица результатов исследования фондоотдачи}
  \label{tbl:results}
  \small{
    \centering
    \begin{tabular}{| p{0.35\textwidth} | p{0.12\textwidth} | p{0.1\textwidth} | p{0.09\textwidth} | p{0.09\textwidth} | p{0.09\textwidth} |}
      \hline

      Показатель & Обозначе- ние & Ед. измерения & План & Факт & Откло- нение \\ \hline

      \multicolumn{6}{|l|}{1. Эффективный фонд времени} \\ \hline

      1.1. Одного станка & $ \text{Ф}^{\text{п}}_{\text{э}j} $, $ \text{Ф}^{\text{ф}}_{\text{э}j} $ &
      \multirow{2}{*}{ч} & 30,72 & 20,55 & $-$10,17 \\ \cline{1-2}\cline{4-6}

      1.2. Исследуемого парка & $ \text{Ф}^{\text{п}}_{\text{э}} $, $ \text{Ф}^{\text{ф}}_{\text{э}} $ & & 307,2 & 205,5 & $-$101,7 \\ \hline

      2. Среднечасовая производительность исследуемого парка & $ \text{В}^{\text{п}}_{\text{ч}} $, $ \text{В}^{\text{ф}}_{\text{ч}} $ & шт/ч & 0,395 & 0,365 & $-$0,03 \\ \hline

      3. Среднедневной объём выпуска продукции & $ N^{\text{п}} $, $ N^{\text{ф}} $ & шт & 121,26 & 75,01 & $-$46,25 \\ \hline

      \multicolumn{6}{ | p{0.95\textwidth} | }{4. Общее абсолютное изменение выпуска продукции за период наблюдения за счёт изменения показателей:} \\ \hline

      4.1. Эффективного фонда рабочего времени исследуемого парка оборудования & $ \Delta N_{\text{фэ}} $ & \multirow{5}{*}{шт} & - & - & $-$40,13 \\ \cline{1-2}\cline{4-6}

      4.2. Среднечасовой производительности исследуемого парка оборудования & $ \Delta N_{\text{вч}} $ & & - & - & $-$6,12 \\ \hline

      \multirow{5}{0.35\textwidth}{5. Удельный вес потерь рабочего времени исследуемого парка оборудования, --- всего и, в том числе, за счёт $i$-ого фактора} & $ \alpha $   & \multirow{5}{*}{\%} & - & - & 33,09 \\
                                               & $ \alpha_1 $ & & - & - & 0,65 \\
                                               & $ \alpha_2 $ & & - & - & 23,3 \\
                                               & $ \alpha_3 $ & & - & - & 8,14 \\
                                               & $ \alpha_4 $ & & - & - & 0,98 \\ \hline

      6. Коэффициент экстенсивного использования оборудования & $ \text{К}_{\text{э}} $ & \multirow{5}{*}{безразм.} & - & - & 0,67 \\ \cline{1-2}\cline{4-6}

      7. Коэффициент интенсивного использования оборудования & $ \text{К}_{\text{ин}} $ & & - & - & 0,92 \\ \cline{1-2}\cline{4-6}

      8. Коэффициент интегрального использования оборудования & $ \text{К}_{\text{ит}} $ & & - & - & 0,62 \\ \hline

      9. Общее относительное изменение фондоотдачи, --- всего и, в том числе, за счёт изменения показателей: & $ \Delta \text{Ф}_o $ & \multirow{6}{*}{\%} & - & - & $-$38,14 \\ \cline{1-2}\cline{4-6}

      9.1. Эффективного фонда рабочего времени оборудования & $ \Delta \text{Ф}_{\text{офэ}} $ & & - & - & $-$33,09 \\ \cline{1-2}\cline{4-6}

      9.2. Среднечасовой производительности оборудования & $ \Delta \text{Ф}_{\text{овч}} $ & & - & - & $-$5,05 \\ \hline
    \end{tabular}
  }
\end{table}

\newpage
