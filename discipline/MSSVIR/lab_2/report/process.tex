\section{Результаты измерений}
\addcontentsline{toc}{section}{Результаты измерений}	% Добавляем его в оглавление

\begin{enumerate}
\item 
Измерить переменные напряжения с помощью вольтметра В7-28. Оценить инструментальные погрешности измерения переменных напряжений.

\hspace{4mm}

\begin{table} [htbp]
  \centering
  \begin{tabular}{| p{2cm} | p{2cm} | p{2cm} | p{2cm} | p{2cm} | p{2cm} | p{2cm}l |}
    \hline
    \centering № п/п &\centering $ f $, кГц &\centering $ U $, В &\centering $ U_{PR} $, В &\centering $ U_{V} $, В &\centering $ \delta_{U} $, \% &\centering $ \delta_{U} $, \% & \\
    \hline
    \centering 1 &\centering &\centering &\centering &\centering &\centering &\centering & \\
    \hline
    \centering 2 &\centering &\centering &\centering &\centering &\centering &\centering & \\
    \hline
    \centering 3 &\centering &\centering &\centering &\centering &\centering &\centering & \\
    \hline
    \centering 4 &\centering &\centering &\centering &\centering &\centering &\centering & \\
    \hline
  \end{tabular}
  \caption{Результаты измерений}
\end{table}

\clearpage

\item
Определить входное сопротивление $ R_{V} $ и входную емкость $ C_{V} $ вольтметра В7-28.

\begin{table} [htbp]
  \centering
  \begin{tabular}{| p{0.7cm} | p{0.7cm} | p{1cm} | p{1cm} | p{1cm} | p{0.7cm} | p{1cm} | p{1cm} | p{1cm} | p{0.7cm} | p{0.7cm} | p{0.7cm} | p{0.7cm}l |}
    \hline
    \centering $ U_{G} $, B &\centering $ f_{H} $, Гц &\centering $ U_{GH} $, B &\centering $ R_{0} $, кОм &\centering $ U_{RV} $, B &\centering $ R_{V} $, B &\centering $ \langle R_{V} \rangle $, кОм &\centering $ f_{B} $, кГц &\centering $ U_{GB} $, B &\centering $ C_{0} $, пФ &\centering $ U_{CV} $, B &\centering $ C_{V} $, пФ &\centering $ \langle C_{V} \rangle $, пФ & \\
    \hline
    & & & & & & & & & & & & & \\ \cline{4-6} \cline{10-12}
    & & & & & & & & & & & & & \\ \cline{4-6} \cline{10-12}
    & & & & & & & & & & & & & \\ \cline{4-6} \cline{10-12}
    \hline
  \end{tabular}
  \caption{Результаты измерений}
\end{table}


\clearpage

\item
Определить для заданных сигналов различной формы пиковое $ U_{m} $, среднеквадратическое $ U_{CK} $, средневыпрямленное $ U_{CB} $ значения напряжения, коэффициент амплитуды $ K_{a} $ и коэффициент формы $ K_{f} $.


\begin{table} [htbp]
  \centering
  \begin{tabular}{| p{4cm} | p{2cm} | p{2cm} | p{2cm} | p{2cm} | p{2cm}l |}
    \hline
    \centering Номер точки &\centering 1 &\centering 2 &\centering 3 &\centering 4 &\centering 5 & \\ 
    \hline
    \centering $ U_{B4-12} $, мВ\par & & & & & & \\ 
    \hline
    \centering $ U_{PR_{B4-12}} $, мВ\par & & & & & & \\ 
    \hline
    \centering $ U_{B3-40} $, мВ\par & & & & & & \\ 
    \hline
    \centering $ U_{PR_{B3-40}} $, мВ\par & & & & & & \\ 
    \hline
    \centering $ U_{B3-38} $, мВ\par & & & & & & \\ 
    \hline
    \centering $ U_{PR_{B3-38}} $, мВ\par & & & & & & \\ 
    \hline
    \centering $ U_{m} $, мВ\par & & & & & & \\ 
    \hline
    \centering $ U_{CK} $, мВ\par & & & & & & \\ 
    \hline
    \centering $ U_{CB} $, мВ\par & & & & & & \\ 
    \hline
    \centering $ K_{a} $\par & & & & & & \\ 
    \hline
    \centering $ K_{f} $ \par & & & & & & \\ 
    \hline
    \centering $ \delta_{U_{B4-12}} $, \% \par & & & & & & \\ 
    \hline
    \centering $ \delta_{U_{B3-40}} $, \% \par & & & & & & \\ 
    \hline
    \centering $ \delta_{U_{B3-38}} $, \% \par & & & & & & \\ 
    \hline
  \end{tabular}
  \caption{Результаты измерений}
\end{table}



\end{enumerate}
\clearpage
