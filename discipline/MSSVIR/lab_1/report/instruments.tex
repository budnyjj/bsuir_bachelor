\section{Приборы, используемые в работе}
\addcontentsline{toc}{section}{Приборы, используемые в работе}	% Добавляем его в оглавление

\begin{table} [htbp]
  \centering
  \begin{tabular}{| p{0.5cm} | p{5cm} | p{1.5cm} | p{9cm}l |}
  \hline
  \centering № & \centering Наименование &\centering Тип &\centering Основные технические характеристики & \\
  \hline
  \centering 1 &\centering Магазин сопротивлений &\centering МСР-63 &\centering Воспроизводимое сопротивление: $ 0 \ldots 99999,99 $ Ом \par Класс точности: $ 0,05/4*10^{-6} $ &  \\
  \hline
  \centering 2 &\centering Прибор электроизмерительный комбинированный &\centering Ц4353 &\centering Класс точности: 1,5 \par Приведенная погрешность: $ \pm  1,5\% $ &  \\
  \hline
  \centering 3 &\centering Цифровой комбинированный прибор  &\centering М92А &\centering 
  $ \Delta_{I} = \pm $ ( $ 0,008 * I_{i} + 1 $  ед. мл. разр.) \par
  
  $ \Delta_{R} = \pm $ ( $ 0,008 * R_{i} + 1 $  ед. мл. разр.) \par
  Цена единицы младшего разряда: \par 1 Ом при диапазоне измерений $ 0\ldots2$ кОм, \par 10 Ом при диапазоне измерений $ 0\ldots20$ кОм, \par 100 Ом при диапазоне измерений $ 0\ldots200$ кОм, \par 1мкА при диапазоне измерений $ 0\ldots2$ мА, \par 10 мкА при диапазоне измерений $ 0\ldots20$ мА & \\
  \hline
  \centering 4 &\centering Вольтметр универсальный цифровой  &\centering В7-34  &\centering 
  Диапазон измерения напряжения постоянного тока: $ 10^{-6} \ldots 1000$ В \par
  Класс точности измерения напряжения постоянного тока: $ 0,015/0,002 $ & \\
  \hline
  \end{tabular}
\end{table}

\clearpage
