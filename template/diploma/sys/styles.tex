%%% Изображения %%%
\graphicspath{{images/}} % Пути к изображениям

%%% Язык текста %%%
\selectlanguage{russian}

%%% Кодировки и шрифты %%%
\renewcommand{\rmdefault}{ftm} % Включаем Times New Roman

%%% Макет страницы %%%
\geometry{a4paper,top=20mm,left=31mm,right=15mm,bottom=27mm}
\setstretch{1.05}

%%% Выравнивание и переносы %%%
\sloppy				% Избавляемся от переполнений
\clubpenalty=10000		% Запрещаем разрыв страницы после первой строки абзаца
\widowpenalty=10000		% Запрещаем разрыв страницы после последней строки абзаца
\interfootnotelinepenalty=10000 % Запрет разрывов сносок

%%% Нумерация страниц %%%
\fancypagestyle{empty}{%
\fancyhf{} % clear all header and footer fields
\renewcommand{\headrulewidth}{0pt}
\renewcommand{\footrulewidth}{0pt}
% \setlength{\footskip}{9mm}
% \setlength{\headheight}{5mm}
}

\fancypagestyle{plain}{%
\fancyhf{} % clear all header and footer fields
\fancyfoot[R]{\thepage}
\renewcommand{\headrulewidth}{0pt}
\renewcommand{\footrulewidth}{0pt}
\setlength{\footskip}{9mm}
% \setlength{\headheight}{5mm}
}

\pagestyle{plain}

%%% Оформление текста

\setlength{\parskip}{0mm}
\setlength{\parindent}{1.25cm}
\raggedbottom{}

%%% Оформление заголовков
\newcommand{\sectionbreak}{\clearpage}

\titleformat{\section}{\large\bfseries}{\thesection}{\wordsep}{}
\titlespacing*{\section}{\parindent}{\baselineskip}{\baselineskip}

\titleformat{name=\section,numberless}{\large\bfseries\filcenter}{}{0mm}{}
\titlespacing*{name=\section,numberless}{0mm}{\baselineskip}{\baselineskip}

\titleformat{name=\subsection}{\normalsize\bfseries}{\thesubsection}{\wordsep}{}
\titlespacing*{\subsection}{\parindent}{\baselineskip}{\baselineskip}

\titleformat{name=\subsection,numberless}{\normalsize\bfseries}{}{0mm}{}
\titlespacing*{name=\subsection,numberless}{0mm}{\baselineskip}{\baselineskip}

\counterwithout{paragraph}{subsubsection}
\counterwithin{paragraph}{subsection}
\renewcommand{\theparagraph}{\thesubsection.\arabic{paragraph}}
\setcounter{secnumdepth}{4}

\titleformat{name=\paragraph}[runin]{\normalsize\bfseries}{\theparagraph}{\wordsep}{}
\titlespacing*{\paragraph}{\parindent}{\baselineskip}{\wordsep}

%%% Оформление списков
\setlist[1]{itemindent=1.85cm,leftmargin=0mm,itemsep=0mm,topsep=0mm,parsep=0mm}
\setlist[itemize,1]{label=---}
\setlist[enumerate,1]{label=\arabic*)}

\setlist[2]{itemindent=3.1cm,leftmargin=0mm,itemsep=0mm,topsep=0mm,parsep=0mm}

% Cтиль для списков, на которые есть ссылки в тексте
\AddEnumerateCounter{\asbuk}{\@asbuk}{\cyrm}
\newlist{reflist}{enumerate}{1}
\setlist*[reflist,1]{label=\asbuk*)}
\setlist*[reflist,2]{label=\arabic*)}

%%% Оформление сносок

\deffootnote[1.65cm]{0mm}{1.25cm}{\textsuperscript{\thefootnotemark) }}
\renewcommand{\footnotesize}{\normalsize\selectfont}
\setlength{\footnotesep}{\parsep}

%%% Оформление ссылок
\urlstyle{same}

%%% Размеры текста формул %%%

\DeclareMathSizes{12}{12}{6}{4}

%%% Расстояние между формулами

\AtBeginDocument{%
  \setlength\abovedisplayskip{14pt}%
  \setlength\belowdisplayskip{14pt}%
  \setlength\abovedisplayshortskip{14pt}%
  \setlength\belowdisplayshortskip{14pt}%
}

%%% Расстояние между плавающими элементами

\setlength{\floatsep}{14pt}     % between top floats
\setlength{\textfloatsep}{14pt} % between top/bottom floats and text
\setlength{\intextsep}{20pt}    % between text and float
\setlength{\dbltextfloatsep}{14pt}
\setlength{\dblfloatsep}{14pt}

% костыль для того, чтобы убрать расстояние от картинки до текста
\setlength{\abovecaptionskip}{0pt}
\setlength{\belowcaptionskip}{0pt}

%% Нумерация плавающих элементов

\counterwithin{figure}{section}
\counterwithin{table}{section}

\makeatletter
\AtBeginDocument{%
\renewcommand{\thetable}{\thesection.\arabic{table}}
\renewcommand{\thelstlisting}{\thesection.\arabic{lstlisting}}
\renewcommand{\thefigure}{\thesection.\arabic{figure}}
\let\c@lstlisting\c@figure}
\makeatother

%% Подписи плавающих элементов

\captionsetup[figure]{
  labelsep=endash,
  justification=centering,
  singlelinecheck=false,
  position=bottom,
  belowskip=-8pt,
  parskip=0pt,
  skip=20pt}

\captionsetup[table]{
  labelsep=endash,
  justification=raggedright,
  singlelinecheck=false,
  position=top,
  skip=0pt}

\captionsetup[lstlisting]{
  labelsep=endash
}

\lstset{
basicstyle=\scriptsize\ttfamily,
numberstyle=\scriptsize\ttfamily,
keywordstyle=\bfseries,
commentstyle=\itshape,
numbers=left,
stepnumber=1,
frame=single,
resetmargins=true,
xleftmargin=7mm,
xrightmargin=2mm,
captionpos=b,
keepspaces=true,
breaklines=true,
aboveskip=22pt,
belowskip=10pt,
abovecaptionskip=16pt}

\renewcommand{\arraystretch}{1.5}

%%% Настройка размеров вертикальных отступов

\renewcommand{\smallskip}{\vspace{0.3\baselineskip}}
\renewcommand{\bigskip}{\vspace{0.8\baselineskip}}

%%% Содержание %%%
\renewcommand{\cfttoctitlefont}{\hfil \large\bfseries}

\setlength{\cftparskip}{0mm}
\setlength{\cftbeforesecskip}{0mm}
\setlength{\cftaftertoctitleskip}{14pt}
\cftsetpnumwidth{4mm}

\renewcommand{\cftsecfont}{}
\renewcommand{\cftsecpagefont}{\normalsize}
\renewcommand{\cftsecleader}{\cftdotfill{\cftdotsep}}

\setlength{\cftsecindent}{0mm}
\setlength{\cftsecnumwidth}{4mm}

\setlength{\cftsubsecindent}{4mm}
\setlength{\cftsubsecnumwidth}{8mm}

%%% Библиография %%%

\makeatletter
\bibliographystyle{ugost2003s} % Оформляем библиографию в соответствии с ГОСТ 7.1 2003

\let\oldthebibliography=\thebibliography
\let\endoldthebibliography=\endthebibliography
\renewenvironment{thebibliography}[1]{
  \begin{oldthebibliography}{#1}
    \setlength{\parskip}{0mm}
    \setlength{\itemsep}{0mm}
}
{
\end{oldthebibliography}
}
